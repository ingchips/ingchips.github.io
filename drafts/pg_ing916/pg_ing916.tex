% Options for packages loaded elsewhere
\PassOptionsToPackage{unicode}{hyperref}
\PassOptionsToPackage{hyphens}{url}
%
\documentclass[
  12pt,
]{book}
\usepackage{lmodern}
\usepackage{amssymb,amsmath}
\usepackage{ifxetex,ifluatex}
\ifnum 0\ifxetex 1\fi\ifluatex 1\fi=0 % if pdftex
  \usepackage[T1]{fontenc}
  \usepackage[utf8]{inputenc}
  \usepackage{textcomp} % provide euro and other symbols
\else % if luatex or xetex
  \usepackage{unicode-math}
  \defaultfontfeatures{Scale=MatchLowercase}
  \defaultfontfeatures[\rmfamily]{Ligatures=TeX,Scale=1}
\fi
% Use upquote if available, for straight quotes in verbatim environments
\IfFileExists{upquote.sty}{\usepackage{upquote}}{}
\IfFileExists{microtype.sty}{% use microtype if available
  \usepackage[]{microtype}
  \UseMicrotypeSet[protrusion]{basicmath} % disable protrusion for tt fonts
}{}
\makeatletter
\@ifundefined{KOMAClassName}{% if non-KOMA class
  \IfFileExists{parskip.sty}{%
    \usepackage{parskip}
  }{% else
    \setlength{\parindent}{0pt}
    \setlength{\parskip}{6pt plus 2pt minus 1pt}}
}{% if KOMA class
  \KOMAoptions{parskip=half}}
\makeatother
\usepackage{xcolor}
\IfFileExists{xurl.sty}{\usepackage{xurl}}{} % add URL line breaks if available
\IfFileExists{bookmark.sty}{\usepackage{bookmark}}{\usepackage{hyperref}}
\hypersetup{
  pdftitle={ING916XX 系列芯片外设开发者手册},
  pdfauthor={Ingchips Technology Co., Ltd.},
  hidelinks,
  pdfcreator={LaTeX via pandoc}}
\urlstyle{same} % disable monospaced font for URLs
\usepackage{color}
\usepackage{fancyvrb}
\newcommand{\VerbBar}{|}
\newcommand{\VERB}{\Verb[commandchars=\\\{\}]}
\DefineVerbatimEnvironment{Highlighting}{Verbatim}{commandchars=\\\{\}}
% Add ',fontsize=\small' for more characters per line
\usepackage{framed}
\definecolor{shadecolor}{RGB}{248,248,248}
\newenvironment{Shaded}{\begin{snugshade}}{\end{snugshade}}
\newcommand{\AlertTok}[1]{\textcolor[rgb]{0.94,0.16,0.16}{#1}}
\newcommand{\AnnotationTok}[1]{\textcolor[rgb]{0.56,0.35,0.01}{\textbf{\textit{#1}}}}
\newcommand{\AttributeTok}[1]{\textcolor[rgb]{0.77,0.63,0.00}{#1}}
\newcommand{\BaseNTok}[1]{\textcolor[rgb]{0.00,0.00,0.81}{#1}}
\newcommand{\BuiltInTok}[1]{#1}
\newcommand{\CharTok}[1]{\textcolor[rgb]{0.31,0.60,0.02}{#1}}
\newcommand{\CommentTok}[1]{\textcolor[rgb]{0.56,0.35,0.01}{\textit{#1}}}
\newcommand{\CommentVarTok}[1]{\textcolor[rgb]{0.56,0.35,0.01}{\textbf{\textit{#1}}}}
\newcommand{\ConstantTok}[1]{\textcolor[rgb]{0.00,0.00,0.00}{#1}}
\newcommand{\ControlFlowTok}[1]{\textcolor[rgb]{0.13,0.29,0.53}{\textbf{#1}}}
\newcommand{\DataTypeTok}[1]{\textcolor[rgb]{0.13,0.29,0.53}{#1}}
\newcommand{\DecValTok}[1]{\textcolor[rgb]{0.00,0.00,0.81}{#1}}
\newcommand{\DocumentationTok}[1]{\textcolor[rgb]{0.56,0.35,0.01}{\textbf{\textit{#1}}}}
\newcommand{\ErrorTok}[1]{\textcolor[rgb]{0.64,0.00,0.00}{\textbf{#1}}}
\newcommand{\ExtensionTok}[1]{#1}
\newcommand{\FloatTok}[1]{\textcolor[rgb]{0.00,0.00,0.81}{#1}}
\newcommand{\FunctionTok}[1]{\textcolor[rgb]{0.00,0.00,0.00}{#1}}
\newcommand{\ImportTok}[1]{#1}
\newcommand{\InformationTok}[1]{\textcolor[rgb]{0.56,0.35,0.01}{\textbf{\textit{#1}}}}
\newcommand{\KeywordTok}[1]{\textcolor[rgb]{0.13,0.29,0.53}{\textbf{#1}}}
\newcommand{\NormalTok}[1]{#1}
\newcommand{\OperatorTok}[1]{\textcolor[rgb]{0.81,0.36,0.00}{\textbf{#1}}}
\newcommand{\OtherTok}[1]{\textcolor[rgb]{0.56,0.35,0.01}{#1}}
\newcommand{\PreprocessorTok}[1]{\textcolor[rgb]{0.56,0.35,0.01}{\textit{#1}}}
\newcommand{\RegionMarkerTok}[1]{#1}
\newcommand{\SpecialCharTok}[1]{\textcolor[rgb]{0.00,0.00,0.00}{#1}}
\newcommand{\SpecialStringTok}[1]{\textcolor[rgb]{0.31,0.60,0.02}{#1}}
\newcommand{\StringTok}[1]{\textcolor[rgb]{0.31,0.60,0.02}{#1}}
\newcommand{\VariableTok}[1]{\textcolor[rgb]{0.00,0.00,0.00}{#1}}
\newcommand{\VerbatimStringTok}[1]{\textcolor[rgb]{0.31,0.60,0.02}{#1}}
\newcommand{\WarningTok}[1]{\textcolor[rgb]{0.56,0.35,0.01}{\textbf{\textit{#1}}}}
\usepackage{longtable,booktabs}
% Correct order of tables after \paragraph or \subparagraph
\usepackage{etoolbox}
\makeatletter
\patchcmd\longtable{\par}{\if@noskipsec\mbox{}\fi\par}{}{}
\makeatother
% Allow footnotes in longtable head/foot
\IfFileExists{footnotehyper.sty}{\usepackage{footnotehyper}}{\usepackage{footnote}}
\makesavenoteenv{longtable}
\usepackage{graphicx,grffile}
\makeatletter
\def\maxwidth{\ifdim\Gin@nat@width>\linewidth\linewidth\else\Gin@nat@width\fi}
\def\maxheight{\ifdim\Gin@nat@height>\textheight\textheight\else\Gin@nat@height\fi}
\makeatother
% Scale images if necessary, so that they will not overflow the page
% margins by default, and it is still possible to overwrite the defaults
% using explicit options in \includegraphics[width, height, ...]{}
\setkeys{Gin}{width=\maxwidth,height=\maxheight,keepaspectratio}
% Set default figure placement to htbp
\makeatletter
\def\fps@figure{htbp}
\makeatother
\setlength{\emergencystretch}{3em} % prevent overfull lines
\providecommand{\tightlist}{%
  \setlength{\itemsep}{0pt}\setlength{\parskip}{0pt}}
\setcounter{secnumdepth}{5}
\usepackage{booktabs}
\usepackage{longtable}
\usepackage[bf,singlelinecheck=on]{caption}

%\usepackage[bookmarksnumbered]{hyperref}
\hypersetup{bookmarksnumbered=true}

\usepackage[BoldFont,SlantFont,CJKchecksingle]{xeCJK}
\setCJKmonofont{SimSun}% 设置缺省中文字体
%\parindent 2em   %段首缩进

\usepackage[
  top=1cm,
  bottom=1cm,
  left=3cm,
  right=2cm,
  headheight=30pt, % as per the warning by fancyhdr
  includehead,includefoot,
  heightrounded, % to avoid spurious underfull messages
]{geometry}

\usepackage[heading]{ctex}

\usepackage{fancyhdr}
\pagestyle{fancy}
\fancyhead[LO]{\includegraphics[width=4cm]{./img/logo_en.jpg}}
\fancyhead[RE]{\includegraphics[width=4cm]{./img/logo_en.jpg}}
%\rhead{\bfseries Result}


%\setmainfont[UprightFeatures={SmallCapsFont=AlegreyaSC-Regular}]{Alegreya}
%\setmainfont[]{Alegreya}
\setmainfont{Liberation Serif}
\setmonofont{Liberation Mono}

\usepackage{framed,color}
\definecolor{shadecolor}{RGB}{248,248,248}

\renewcommand{\textfraction}{0.05}
\renewcommand{\topfraction}{0.8}
\renewcommand{\bottomfraction}{0.8}
\renewcommand{\floatpagefraction}{0.75}

\renewenvironment{quote}{\begin{VF}}{\end{VF}}
\let\oldhref\href
\renewcommand{\href}[2]{#2\footnote{\url{#1}}}

\ifxetex
  \usepackage{letltxmacro}
  \setlength{\XeTeXLinkMargin}{1pt}
  \LetLtxMacro\SavedIncludeGraphics\includegraphics
  \def\includegraphics#1#{% #1 catches optional stuff (star/opt. arg.)
    \IncludeGraphicsAux{#1}%
  }%
  \newcommand*{\IncludeGraphicsAux}[2]{%
    \XeTeXLinkBox{%
      \SavedIncludeGraphics#1{#2}%
    }%
  }%
\fi

\makeatletter
\newenvironment{kframe}{%
\medskip{}
\setlength{\fboxsep}{.8em}
 \def\at@end@of@kframe{}%
 \ifinner\ifhmode%
  \def\at@end@of@kframe{\end{minipage}}%
  \begin{minipage}{\columnwidth}%
 \fi\fi%
 \def\FrameCommand##1{\hskip\@totalleftmargin \hskip-\fboxsep
 \colorbox{shadecolor}{##1}\hskip-\fboxsep
     % There is no \\@totalrightmargin, so:
     \hskip-\linewidth \hskip-\@totalleftmargin \hskip\columnwidth}%
 \MakeFramed {\advance\hsize-\width
   \@totalleftmargin\z@ \linewidth\hsize
   \@setminipage}}%
 {\par\unskip\endMakeFramed%
 \at@end@of@kframe}
\makeatother

\makeatletter
\@ifundefined{Shaded}{
}{\renewenvironment{Shaded}{\begin{kframe}}{\end{kframe}}}
\makeatother

\newenvironment{rmdblock}[1]
  {
  \begin{itemize}
  \renewcommand{\labelitemi}{
    \raisebox{-.7\height}[0pt][0pt]{
      {\setkeys{Gin}{width=3em,keepaspectratio}\includegraphics{images/#1}}
    }
  }
  \setlength{\fboxsep}{1em}
  \begin{kframe}
  \item
  }
  {
  \end{kframe}
  \end{itemize}
  }
\newenvironment{rmdnote}
  {\begin{rmdblock}{note}}
  {\end{rmdblock}}
\newenvironment{rmdcaution}
  {\begin{rmdblock}{caution}}
  {\end{rmdblock}}
\newenvironment{rmdimportant}
  {\begin{rmdblock}{important}}
  {\end{rmdblock}}
\newenvironment{rmdtip}
  {\begin{rmdblock}{tip}}
  {\end{rmdblock}}
\newenvironment{rmdwarning}
  {\begin{rmdblock}{warning}}
  {\end{rmdblock}}

\usepackage{makeidx}
\makeindex

\urlstyle{tt}

\usepackage{amsthm}
\makeatletter
\def\thm@space@setup{%
  \thm@preskip=8pt plus 2pt minus 4pt
  \thm@postskip=\thm@preskip
}
\makeatother

\frontmatter

\let\oldmaketitle\maketitle
\AtBeginDocument{\let\maketitle\relax}

% % Definition of \maketitle
% \makeatletter
%     \begin{titlepage}
%         \begin{center}
%             \includegraphics[width=0.7\linewidth]{./img/logo_en.jpg}\\[4ex]
%             {\huge \bfseries  \@title }\\[2ex]
%             {\LARGE  \@author}\\[50ex]
%             {\large \@date}
%         \end{center}
%     \end{titlepage}
% \makeatother
\usepackage[]{natbib}
\bibliographystyle{apalike}

\title{ING916XX 系列芯片外设开发者手册}
\author{Ingchips Technology Co., Ltd.}
\date{}

\begin{document}
\maketitle

%\cleardoublepage\newpage\thispagestyle{empty}\null
%\cleardoublepage\newpage\thispagestyle{empty}\null
%\cleardoublepage\newpage
\thispagestyle{empty}
\begin{center}
\end{center}

\setlength{\abovedisplayskip}{-5pt}
\setlength{\abovedisplayshortskip}{-5pt}

\thispagestyle{empty}

\makeatletter
\begin{center}
    \vspace{5ex}
    \includegraphics{./img/logo_en.jpg}\\[15ex]
    {\huge  \@title }
    \noindent\rule{10cm}{0.4pt}\\[68ex]
    %{\LARGE  \@author}\\[60ex] 
    {\large \@author}
\end{center}
\makeatother


\newpage
\thispagestyle{empty}

%\let\maketitle\oldmaketitle
%\maketitle

{
\setcounter{tocdepth}{3}
\tableofcontents
}
\listoftables
\listoffigures
\mainmatter

\hypertarget{revision-history}{%
\chapter{版本历史}\label{revision-history}}

\begin{longtable}[]{@{}lll@{}}
\toprule
版本 & 信息 & 日期\tabularnewline
\midrule
\endhead
0.1 & 初始版本 & 2022-xx-xx\tabularnewline
\bottomrule
\end{longtable}

\hypertarget{ch-overview}{%
\chapter{概览}\label{ch-overview}}

欢迎使用 \emph{INGCHIPS} 918xx/916xx 软件开发工具包 (SDK).

ING916XX 系列芯片支持蓝牙 5.3 规范,内置高性能 32bit RISC MCU(支持 DSP 和 FPU)、Flash、低功耗 PMU,
以及丰富的外设、高性能低功耗 BLE RF 收发机。BLE 发射功率。

本文介绍 SoC 外设及其开发方法。每个章节介绍一种外设,各种外设与芯片数据手册之外设一一对应,
基于 API 的兼容性、避免误解等因素,存在以下例外:

\begin{itemize}
\tightlist
\item
  PINCTRL 对应于数据手册之 IOMUX
\item
  PCAP 对应于数据手册之 PCM
\item
  SYSCTRL 是一个``虚拟''外设,负责管理各种 SoC 功能,组合了几种相关的硬件模块
\end{itemize}

SDK 外设驱动的源代码开放,其中包含很多常数,而且几乎没有注释 ------ 这是有意为之,开发者只需要关注头文件,而不要尝试修改源代码。

\hypertarget{ux7f29ux7565ux8bedux53caux672fux8bed}{%
\section{缩略语及术语}\label{ux7f29ux7565ux8bedux53caux672fux8bed}}

\begin{longtable}[]{@{}cl@{}}
\caption{\label{tab:ch0-abbreviations} 缩略语}\tabularnewline
\toprule
缩略语 & 说明\tabularnewline
\midrule
\endfirsthead
\toprule
缩略语 & 说明\tabularnewline
\midrule
\endhead
ADC & 模数转换器(Analog-to-Digital Converter)\tabularnewline
DMA & 直接存储器访问(Direct Memory Access)\tabularnewline
EFUSE & 电编程熔丝(Electronic Fuses)\tabularnewline
FIFO & 先进先出队列(First In First Out)\tabularnewline
FOTA & 固件空中升级(Firmware Over-The-Air)\tabularnewline
GPIO & 通用输入输出(General-Purpose Input/Output)\tabularnewline
I2C & 集成电路间总线(Inter-Integrated Circuit)\tabularnewline
I2S & 集成电路音频总线(Inter-IC Sound)\tabularnewline
IR & 红外线(Infrared)\tabularnewline
PCAP & 脉冲捕捉(Pulse CAPture)\tabularnewline
PDM & 脉冲密度调制(Pulse Density Modulation)\tabularnewline
PTE & 外设触发引擎(Peripheral Trigger Engine)\tabularnewline
PWM & 脉宽调制信号(Pulse Width Modulation)\tabularnewline
QDEC & 正交解码器(Quadrature Decoder)\tabularnewline
RTC & 实时时钟(Real-time Clock)\tabularnewline
SPI & 串行外设接口(Serial Peripheral Interface)\tabularnewline
UART & 通用异步收发器(Universal Asynchronous Receiver/Transmitter)\tabularnewline
USB & 通用串行总线(Universal Serial Bus)\tabularnewline
\bottomrule
\end{longtable}

\hypertarget{ux53c2ux8003ux6587ux6863}{%
\section{参考文档}\label{ux53c2ux8003ux6587ux6863}}

\begin{enumerate}
\def\labelenumi{\arabic{enumi}.}
\tightlist
\item
  Bluetooth SIG\footnote{\url{https://www.bluetooth.com/}}
\item
  ING916XX 系列芯片数据手册
\end{enumerate}

\hypertarget{ch-adc}{%
\chapter{ADC简介}\label{ch-adc}}

ADC全称Analog-to-Digital Converter,即模数转换器。

其主要作用是通过PIN测量电压,并将采集到的电压模拟信号转换成数字信号。

\hypertarget{ux529fux80fdux63cfux8ff0}{%
\section{功能描述}\label{ux529fux80fdux63cfux8ff0}}

\hypertarget{ux7279ux70b9}{%
\subsection{特点}\label{ux7279ux70b9}}

\begin{itemize}
\item
  最多12个单端输入通道或4个差分输入通道
\item
  14位分辨率
\item
  电压输入范围(0\textasciitilde3.6V)
\item
  支持APB总线
\item
  采样频率可编程
\item
  支持单一转换模式
\item
  支持回路转换模式,每个通道均可启用或禁用
\end{itemize}

\hypertarget{adcux6a21ux5f0f}{%
\subsection{ADC模式}\label{adcux6a21ux5f0f}}

\begin{itemize}
\item
  校准模式(calibration):用于校准ADC采样精度。分为单端模式校准和差分模式校准,两种模式校准互相独立
\item
  转换模式(conversion):用于正常工作状态下的模数转换
\end{itemize}

根据ADC输入模式完成对应的模式校准,之后在转换模式下进行正常模数转换。

\hypertarget{adcux8f93ux5165ux6a21ux5f0f}{%
\subsection{ADC输入模式}\label{adcux8f93ux5165ux6a21ux5f0f}}

\begin{itemize}
\item
  单端输入(single end):使用单个输入引脚,采用ADC内部的参考电压
\item
  差分输入(differential):使用一组输入引脚分别作为参考电压
\end{itemize}

一般来说,差分输入有利于避免共模干扰的影响,结果相对准确。

\hypertarget{adcux8f6cux6362ux6a21ux5f0f}{%
\subsection{ADC转换模式}\label{adcux8f6cux6362ux6a21ux5f0f}}

\begin{itemize}
\item
  单次转换(single):单次转换在使能转换通道,采样和转换后,ADC将停止,数据将被拉入 \texttt{FIFO}
\item
  连续转换(continuous):连续转换在逐个使能通道,采样并转换后,经过 \texttt{loop-delay} 时间后循环进行操作,ADC不会停止直到手动关闭
\end{itemize}

\hypertarget{adcux901aux9053}{%
\subsection{ADC通道}\label{adcux901aux9053}}

ADC共12个channel,即ch0-ch11。

其中ch0-ch7是通用通道,ch8-ch11是内部通道,可以用来采集内部信号。

具体通道的输入连接引脚如下:

\begin{longtable}[]{@{}ll@{}}
\caption{\label{tab:ch-ADC-input-connection} ADC输入连接引脚}\tabularnewline
\toprule
通道 & 连接引脚\tabularnewline
\midrule
\endfirsthead
\toprule
通道 & 连接引脚\tabularnewline
\midrule
\endhead
ch0 & GPIO7\tabularnewline
ch1 & GPIO8\tabularnewline
ch2 & GPIO9\tabularnewline
ch3 & GPIO10\tabularnewline
ch4 & GPIO11\tabularnewline
ch5 & GPIO12\tabularnewline
ch6 & GPIO13\tabularnewline
ch7 & GPIO14\tabularnewline
ch8 & diag\_hv\_soc\tabularnewline
ch9 & VREF12\_ADC\_IN\tabularnewline
ch10 & VCC\tabularnewline
ch11 & V18\_FLASH\tabularnewline
\bottomrule
\end{longtable}

ch0-ch7可以配置成为4对差分输入通道,配置后将对应差分的通道0-3。

\textbf{注意:} 配置ADC输入模式为差分模式下则只有ch0-ch3以及ch8-ch11。

\hypertarget{pga}{%
\subsection{PGA}\label{pga}}

ADC的PGA默认是开启的,目前使用时不支持关闭。

ADC的PGA有8个增益值,其大小为 \texttt{2\^{}n(n=0-7)} ,但是目前增益值n=6,7暂不支持配置。

对于PGA增益值配置,目前ch0-ch7可以配置的增益值为 \texttt{2\^{}n(n=0-5)}

ch8-ch11增益值暂不支持配置,其增益值固定为2.

\hypertarget{ux4f7fux7528ux65b9ux6cd5}{%
\section{使用方法}\label{ux4f7fux7528ux65b9ux6cd5}}

\hypertarget{ux65b9ux6cd5ux6982ux8ff0}{%
\subsection{方法概述}\label{ux65b9ux6cd5ux6982ux8ff0}}

方法概述为:时钟配置,ADC校准,ADC参数配置和中断(DMA中断)注册。

\hypertarget{ux65f6ux949fux914dux7f6e}{%
\subsubsection{时钟配置}\label{ux65f6ux949fux914dux7f6e}}

当前ADC所用时钟源为 \texttt{clock\ slow} 经过分频得到的ADC工作时钟

当前ADC工作时钟可以配置为1M、2M、4M和6M。高于6M的时钟暂不支持配置

\textbf{注意:} 由于ADC工作时钟只可以取以上特定值,ADC时钟必须经过 \texttt{clock\ slow} 分频。故在同时使用和ADC同时钟源模块时,如IR,可能出现时钟配置冲突的现象。开发者在实际使用时需要格外注意,具体请参考916时钟树(图 \ref{fig:916-clk-tree-ADC})。

\begin{figure}

{\centering \includegraphics[width=0.7\linewidth]{./img/ADC/916_clk_tree_ADC} 

}

\caption{916时钟树ADC模块部分截图}\label{fig:916-clk-tree-ADC}
\end{figure}

\hypertarget{adcux6821ux51c6}{%
\subsubsection{ADC校准}\label{adcux6821ux51c6}}

ADC校准包含精度校准和内部参考电压校准

ADC精度校准接口:ADC\_Calibration

内部参考电压校准接口:ADC\_VrefCalibration

在进行ADC转换之前需要进行ADC精度校准,如果需要计算采集电压则需要进行内部参考电压校准

ADC精度校准需要明确ADC输入模式(单端/差分),两种模式需要分别进行精度校准

\hypertarget{adcux53c2ux6570ux914dux7f6e}{%
\subsubsection{ADC参数配置}\label{adcux53c2ux6570ux914dux7f6e}}

ADC参数配置接口:ADC\_ConvCfg

涉及参数有:ADC转换模式、ADC输入模式、PGA增益值、PGA开关(目前不支持关闭)、采样通道、data触发中断数、data触发DMA搬运数和 \texttt{loop-delay}

data触发中断数和data触发DMA搬运数决定了搬运ADC数据的方式。前者用触发中断的方式,后者用触发DMA搬运的方式

对于中断/DMA搬运方式选择上,建议如下:

\begin{itemize}
\item
  一般在小数据量情况下,如定时采集温度,建议采用触发中断并CPU读数的方式
\item
  一般大数据量采样,如模拟麦克风采样,建议采用DMA搬运方式或者乒乓搬运,可以大大提高数据搬运处理效率
\end{itemize}

\hypertarget{adcux6570ux636eux5904ux7406}{%
\subsubsection{ADC数据处理}\label{adcux6570ux636eux5904ux7406}}

ADC数据处理的推荐方法为:

\textbf{1.} 调用ADC\_PopFifoData(或DMA搬运buff)读取FIFO中的ADC原始数据

\textbf{2.} 调用ADC\_GetDataChannel得到原始数据中的数据所属通道(如需要)

\textbf{3.} 调用ADC\_GetData得到原始数据中的ADC数据

\textbf{4.} 调用ADC\_GetVol通过ADC数据计算得到其对应的电压值(如需要)

也可以通过调用ADC\_ReadChannelData接口直接得到指定通道的ADC数据。但这样会丢弃其他通道数据,请谨慎使用。其可以作为辅助接口使用,非主要方式。

\hypertarget{ux6ce8ux610fux70b9}{%
\subsection{注意点}\label{ux6ce8ux610fux70b9}}

\begin{itemize}
\item
  一般来说配置ADC时钟频率越高,其工作效率越好。较为推荐使用4M、6M工作时钟,性能稳定
\item
  ADC时钟选择需要根据实际设备采样率选择,如模拟mic,应参考mic操作手册来选择对应ADC时钟频率
\item
  差分输入电压需要控制范围为0\textasciitilde3.6V
\item
  如果需要计算采集电压建议周期性地进行参考电压校准
\item
  目前暂不支持用户自己输入参考电压值,只能用内部电压值作为参考电压
\item
  PGA增益值的选择需要结合实际设备,可以参考具体设备的使用手册或者测试电压范围计算得到。注意不能超过参考电压阈值
\item
  data触发中断数和data触发DMA搬运数应该一个为0,一个非0,。如果两值都非0则优先选择触发中断的方式
\item
  通过ADC数据计算得到的电压值会被限制在0到正参考电压之间
\end{itemize}

\hypertarget{ux7f16ux7a0bux6307ux5357}{%
\section{编程指南}\label{ux7f16ux7a0bux6307ux5357}}

\hypertarget{ux9a71ux52a8ux63a5ux53e3}{%
\subsection{驱动接口}\label{ux9a71ux52a8ux63a5ux53e3}}

此处列举几个较为常用的接口:

\begin{itemize}
\item
  ADC\_ConvCfg:ADC转换标准配置接口
\item
  ADC\_Calibration:ADC校准标准配置接口
\item
  ADC\_VrefCalibration:内部参考电压校准接口
\item
  ADC\_Reset:ADC复位接口
\item
  ADC\_Start:ADC使能接口
\item
  ADC\_AdcClose:ADC关闭接口
\item
  ADC\_GetFifoEmpty:读取FIFO是否为空接口
\item
  ADC\_PopFifoData:读取FIFO原始数据接口
\item
  ADC\_GetDataChannel:读取原始数据中通道号接口
\item
  ADC\_GetData:读取原始数据中ADC数据接口
\item
  ADC\_ReadChannelData:读取特定通道ADC数据接口
\item
  ADC\_GetVol:读取ADC数据对应电压值接口
\end{itemize}

还有部分接口不推荐直接使用,在此不进行介绍,详见对应驱动程序头文件声明。

\hypertarget{ux4ee3ux7801ux793aux4f8b}{%
\subsection{代码示例}\label{ux4ee3ux7801ux793aux4f8b}}

\hypertarget{ux89e6ux53d1ux4e2dux65adux642cux8fd0}{%
\subsubsection{触发中断搬运}\label{ux89e6ux53d1ux4e2dux65adux642cux8fd0}}

下面展示用ADC中断进行数据搬运的基本用法:

以下ADC设置参数仅供参考,具体参数请结合实际需要进行配置。

\begin{Shaded}
\begin{Highlighting}[]
\PreprocessorTok{#define ADC_CHANNEL    ADC_CH_0}

\DataTypeTok{static} \DataTypeTok{uint32_t}\NormalTok{ ADC_cb_isr(}\DataTypeTok{void}\NormalTok{ *user_data)}
\NormalTok{\{}
    \DataTypeTok{uint32_t}\NormalTok{ data = ADC_PopFifoData();}
\NormalTok{    SADC_channelId channel = ADC_GetDataChannel(data);}
    \ControlFlowTok{if}\NormalTok{ (channel == ADC_CHANNEL) \{}
        \DataTypeTok{uint16_t}\NormalTok{ sample = ADC_GetData(data);}
        \CommentTok{// do something with 'sample'}
\NormalTok{    \}}
    \ControlFlowTok{return} \DecValTok{0}\NormalTok{;}
\NormalTok{\}}

\DataTypeTok{void}\NormalTok{ test(}\DataTypeTok{void}\NormalTok{)}
\NormalTok{\{}
\NormalTok{    SYSCTRL_ClearClkGate(SYSCTRL_ITEM_APB_ADC);}
\NormalTok{    SYSCTRL_SetAdcClkDiv(}\DecValTok{4}\NormalTok{, }\DecValTok{1}\NormalTok{);}
\NormalTok{    SYSCTRL_ReleaseBlock(SYSCTRL_ITEM_APB_ADC);}
\NormalTok{    ADC_Calibration(SINGLE_END_MODE);}
\NormalTok{    ADC_ConvCfg(SINGLE_MODE, PGA_GAIN_4, }\DecValTok{1}\NormalTok{, ADC_CHANNEL, }\DecValTok{1}\NormalTok{, }\DecValTok{0}\NormalTok{, SINGLE_END_MODE, }\DecValTok{0}\NormalTok{);}
\NormalTok{    platform_set_irq_callback(PLATFORM_CB_IRQ_ADC, ADC_cb_isr, }\DecValTok{0}\NormalTok{);}
\NormalTok{    ADC_Start(}\DecValTok{1}\NormalTok{);}
\NormalTok{\}}
\end{Highlighting}
\end{Shaded}

以上代码展示了配置ADC时钟,逻辑复位、ADC校准、ADC转换配置,并在触发的ADC中断程序里获取到最终的 \texttt{sample} 数值。

当然,由于使用单一ADC通道,可以直接获取特定通道的ADC数据,代码如下:

\begin{Shaded}
\begin{Highlighting}[]
\DataTypeTok{static} \DataTypeTok{uint32_t}\NormalTok{ ADC_cb_isr(}\DataTypeTok{void}\NormalTok{ *user_data)}
\NormalTok{\{}
    \DataTypeTok{uint16_t}\NormalTok{ sample = ADC_ReadChannelData(ADC_CHANNEL);}
    \CommentTok{// do something with 'sample'}
    \ControlFlowTok{return} \DecValTok{0}\NormalTok{;}
\NormalTok{\}}
\end{Highlighting}
\end{Shaded}

在只有单个通道数据时或者只需要单一通道数据时可以采用以上方式,多通道采样有丢数据的风险。

\hypertarget{ux8fdeux7eedux4e2dux65adux642cux8fd0}{%
\subsubsection{连续中断搬运}\label{ux8fdeux7eedux4e2dux65adux642cux8fd0}}

以上代码展示的是单次搬运情况,如果是多次搬运,建议采用以下两种读数方案:

\textbf{1.} 读数并结合调用ADC\_GetFifoEmpty接口判断FIFO状态,读数直到FIFO为空为止

\textbf{2.} 每次读取的数据量等于配置的data触发中断数

方案1代码示例:

\begin{Shaded}
\begin{Highlighting}[]
\DataTypeTok{static} \DataTypeTok{uint32_t}\NormalTok{ ADC_cb_isr(}\DataTypeTok{void}\NormalTok{ *user_data)}
\NormalTok{\{}
    \ControlFlowTok{while}\NormalTok{ (!ADC_GetFifoEmpty()) \{}
        \DataTypeTok{uint16_t}\NormalTok{ sample = ADC_ReadChannelData(ADC_CHANNEL);}
        \CommentTok{// do something with 'sample'}
\NormalTok{    \}}
    \ControlFlowTok{return} \DecValTok{0}\NormalTok{;}
\NormalTok{\}}
\end{Highlighting}
\end{Shaded}

方案2代码示例:

\begin{Shaded}
\begin{Highlighting}[]
\PreprocessorTok{#define INT_TRIGGER_NUM    8}
\DataTypeTok{static} \DataTypeTok{uint32_t}\NormalTok{ ADC_cb_isr(}\DataTypeTok{void}\NormalTok{ *user_data)}
\NormalTok{\{}
    \DataTypeTok{uint8_t}\NormalTok{ i = }\DecValTok{0}\NormalTok{;}
    \ControlFlowTok{while}\NormalTok{ (i < INT_TRIGGER_NUM) \{}
        \DataTypeTok{uint16_t}\NormalTok{ sample = ADC_ReadChannelData(ADC_CHANNEL);}
        \CommentTok{// do something with 'sample'}
\NormalTok{        i++;}
\NormalTok{    \}}
    \ControlFlowTok{return} \DecValTok{0}\NormalTok{;}
\NormalTok{\}}
\end{Highlighting}
\end{Shaded}

CPU资源方面,方案2节省了每次读数调用接口的开销,建议优选

实际效果方面以上两种方案均可,开发者可以自行选用。

\hypertarget{ux83b7ux53d6ux7535ux538bux503c}{%
\subsubsection{获取电压值}\label{ux83b7ux53d6ux7535ux538bux503c}}

获取电压值需要在上面例子中加入参考电压校准,并调用接口计算电压值,如下:

\begin{Shaded}
\begin{Highlighting}[]
\PreprocessorTok{#define ADC_CHANNEL    ADC_CH_0}

\DataTypeTok{static} \DataTypeTok{uint32_t}\NormalTok{ ADC_cb_isr(}\DataTypeTok{void}\NormalTok{ *user_data)}
\NormalTok{\{}
    \DataTypeTok{uint16_t}\NormalTok{ sample = ADC_ReadChannelData(ADC_CHANNEL);}
    \DataTypeTok{float}\NormalTok{ voltage = ADC_GetVol(sample);}
    \ControlFlowTok{return} \DecValTok{0}\NormalTok{;}
\NormalTok{\}}

\DataTypeTok{void}\NormalTok{ test(}\DataTypeTok{void}\NormalTok{)}
\NormalTok{\{}
\NormalTok{    SYSCTRL_ClearClkGate(SYSCTRL_ITEM_APB_ADC);}
\NormalTok{    SYSCTRL_SetAdcClkDiv(}\DecValTok{4}\NormalTok{, }\DecValTok{1}\NormalTok{);}
\NormalTok{    SYSCTRL_ReleaseBlock(SYSCTRL_ITEM_APB_ADC);}
\NormalTok{    ADC_Calibration(SINGLE_END_MODE);}
\NormalTok{    ADC_VrefCalibration();      }\CommentTok{// calibrate the referenced voltage}
\NormalTok{    ADC_ConvCfg(SINGLE_MODE, PGA_GAIN_4, }\DecValTok{1}\NormalTok{, ADC_CHANNEL, }\DecValTok{1}\NormalTok{, }\DecValTok{0}\NormalTok{, SINGLE_END_MODE, }\DecValTok{0}\NormalTok{);}
\NormalTok{    platform_set_irq_callback(PLATFORM_CB_IRQ_ADC, ADC_cb_isr, }\DecValTok{0}\NormalTok{);}
\NormalTok{    ADC_Start(}\DecValTok{1}\NormalTok{);}
\NormalTok{\}}
\end{Highlighting}
\end{Shaded}

voltage即为最终计算得到的采样电压值。

\hypertarget{adc-dmaux4e52ux4e53ux642cux8fd0}{%
\subsubsection{ADC \& DMA乒乓搬运}\label{adc-dmaux4e52ux4e53ux642cux8fd0}}

ADC结合DMA乒乓搬运数据的方式是一种推荐的标准用法,其优点主要是节省CPU资源,提高处理数据的效率。

下面展示DMA乒乓搬运ADC数据并转换成电压的实例:

\begin{Shaded}
\begin{Highlighting}[]
\PreprocessorTok{#include }\ImportTok{"pingpong.h"}
\PreprocessorTok{#define ADC_CHANNEL    ADC_CH_0}
\PreprocessorTok{#define DMA_CHANNEL    0}
\DataTypeTok{static}\NormalTok{ DMA_PingPong_t PingPong;}

\DataTypeTok{static} \DataTypeTok{uint32_t}\NormalTok{ DMA_cb_isr(}\DataTypeTok{void}\NormalTok{ *user_data)}
\NormalTok{\{}
    \DataTypeTok{uint32_t}\NormalTok{ state = DMA_GetChannelIntState(DMA_CHANNEL);}
\NormalTok{    DMA_ClearChannelIntState(DMA_CHANNEL, state);}

    \DataTypeTok{uint32_t}\NormalTok{ *buff = DMA_PingPongIntProc(&PingPong, DMA_CHANNEL);}
    \DataTypeTok{uint32_t}\NormalTok{ tranSize = DMA_PingPongGetTransSize(&PingPong);}
    \ControlFlowTok{for}\NormalTok{ (}\DataTypeTok{uint32_t}\NormalTok{ i = }\DecValTok{0}\NormalTok{; i < tranSize; ++i) \{}
        \ControlFlowTok{if}\NormalTok{ (ADC_GetDataChannel(buff[i]) != ADC_CHANNEL) }\ControlFlowTok{continue}\NormalTok{;}
        \DataTypeTok{uint16_t}\NormalTok{ sample = ADC_GetData(buff[i]);}
        \DataTypeTok{float}\NormalTok{ voltage = ADC_GetVol(sample);}
\NormalTok{    \}}
    \ControlFlowTok{return} \DecValTok{0}\NormalTok{;}
\NormalTok{\}}
\DataTypeTok{void}\NormalTok{ test(}\DataTypeTok{void}\NormalTok{)}
\NormalTok{\{}
\NormalTok{    SYSCTRL_ClearClkGateMulti((}\DecValTok{1}\NormalTok{ << SYSCTRL_ITEM_APB_DMA));}

\NormalTok{    SYSCTRL_ClearClkGate(SYSCTRL_ITEM_APB_ADC);}
\NormalTok{    SYSCTRL_SetAdcClkDiv(}\DecValTok{4}\NormalTok{, }\DecValTok{1}\NormalTok{);}
\NormalTok{    SYSCTRL_ReleaseBlock(SYSCTRL_ITEM_APB_ADC);}
\NormalTok{    ADC_Calibration(DIFFERENTAIL_MODE);}
\NormalTok{    ADC_VrefCalibration();}
\NormalTok{    ADC_ConvCfg(CONTINUES_MODE, PGA_GAIN_4, }\DecValTok{1}\NormalTok{, ADC_CHANNEL, }\DecValTok{0}\NormalTok{, }\DecValTok{8}\NormalTok{, DIFFERENTAIL_MODE, }\DecValTok{6000}\NormalTok{);}

\NormalTok{    SYSCTRL_SelectUsedDmaItems(}\DecValTok{1}\NormalTok{ << }\DecValTok{9}\NormalTok{);}
\NormalTok{    DMA_PingPongSetup(&PingPong, SYSCTRL_DMA_ADC, }\DecValTok{80}\NormalTok{, }\DecValTok{8}\NormalTok{);}
\NormalTok{    platform_set_irq_callback(PLATFORM_CB_IRQ_DMA, DMA_cb_isr, }\DecValTok{0}\NormalTok{);}
\NormalTok{    DMA_PingPongEnable(&PingPong, DMA_CHANNEL);}
    
\NormalTok{    ADC_Start(}\DecValTok{1}\NormalTok{);}
\NormalTok{\}}
\end{Highlighting}
\end{Shaded}

\hypertarget{ch-dma}{%
\chapter{DMA简介}\label{ch-dma}}

DMA全称direct memory access,即直接存储器访问。

其主要作用是不占用CPU大量资源,在AMBA AHB总线上的设备之间以硬件方式高速有效地传输数据。

\hypertarget{ux529fux80fdux63cfux8ff0-1}{%
\section{功能描述}\label{ux529fux80fdux63cfux8ff0-1}}

\hypertarget{ux7279ux70b9-1}{%
\subsection{特点}\label{ux7279ux70b9-1}}

\begin{itemize}
\item
  符合AMBA、AHB和APB4标准
\item
  最多8个DMA通道
\item
  最多16个硬件握手请求/确认配对
\item
  支持8/16/32/64位宽的数据传输
\item
  支持24-64位地址宽度
\item
  支持成链传输数据
\end{itemize}

\hypertarget{ux642cux8fd0ux65b9ux5f0f}{%
\subsection{搬运方式}\label{ux642cux8fd0ux65b9ux5f0f}}

\begin{itemize}
\item
  单次数据块搬运:DMA使用单个通道,一次使能将数据从SRC到DST位置搬运一次
\item
  成串多数据块搬运:DMA使用单个通道,一次使能按照DMA链表信息依次将数据从SRC到DST位置搬运多次或循环搬运。
\end{itemize}

其根本区别是有无注册有效的DMA链表。

\hypertarget{ux642cux8fd0ux7c7bux578b}{%
\subsection{搬运类型}\label{ux642cux8fd0ux7c7bux578b}}

\begin{itemize}
\item
  memory到memory搬运
\item
  memory到peripheral搬运
\item
  peripheral到memory搬运
\item
  peripheral到peripheral搬运
\end{itemize}

\hypertarget{ux4e2dux65adux7c7bux578b}{%
\subsection{中断类型}\label{ux4e2dux65adux7c7bux578b}}

\begin{itemize}
\item
  IntErr:错误中断表示DMA传输发生了错误而触发中断,主要包括总线错误、地址没对齐和传输数据宽度没对齐等。
\item
  IntAbt:终止传输中断会在终止DMA通道传输时产生。
\item
  IntTC:TC中断会在没有产生IntErr和IntAbt的情况下完成一次传输时产生。
\end{itemize}

\hypertarget{ux6570ux636eux5730ux5740ux7c7bux578b}{%
\subsection{数据地址类型}\label{ux6570ux636eux5730ux5740ux7c7bux578b}}

\begin{itemize}
\item
  Increment address
\item
  Decrement address
\item
  Fixed address
\end{itemize}

如果Increment则DMA从地址由小到大搬运数据,相反的Decrement则由大到小搬运。fixed地址适用于外设FIFO的寄存器搬运数据。

\hypertarget{ux6570ux636eux65b9ux5f0f}{%
\subsection{数据方式}\label{ux6570ux636eux65b9ux5f0f}}

\begin{itemize}
\item
  normal mode
\item
  handshake mode
\end{itemize}

一般外设寄存器选择用握手方式。

选择握手方式要和外设协商好SrcBurstSize,当前支持2\^{}n(n = 0-7)大小的SrcBurstSize。

\hypertarget{ux6570ux636eux4f4dux5bbd}{%
\subsection{数据位宽}\label{ux6570ux636eux4f4dux5bbd}}

DMA传输要求传输两端的数据类型一致,支持数据类型有:

\begin{itemize}
\item
  Byte transfer
\item
  Half-word transfer
\item
  Word transfer
\item
  Double word transfer
\end{itemize}

覆盖所有常见数据类型。

\hypertarget{ux4f7fux7528ux65b9ux6cd5-1}{%
\section{使用方法}\label{ux4f7fux7528ux65b9ux6cd5-1}}

\hypertarget{ux65b9ux6cd5ux6982ux8ff0-1}{%
\subsection{方法概述}\label{ux65b9ux6cd5ux6982ux8ff0-1}}

首先确认数据搬运需求是单次搬运还是成串搬运,搬运类型memory和peripheral的关系。

\hypertarget{ux5355ux6b21ux642cux8fd0}{%
\subsubsection{单次搬运}\label{ux5355ux6b21ux642cux8fd0}}

\textbf{1.} 注册DMA中断

\textbf{2.} 定义一个DMA\_Descriptor变量用来配置DMA通道寄存器

\textbf{3.} 根据数据搬运类型选择DMA寄存器配置接口,正确调用驱动接口配置DMA寄存器

\textbf{4.} 使能DMA通道开始搬运

\hypertarget{ux6210ux4e32ux642cux8fd0}{%
\subsubsection{成串搬运}\label{ux6210ux4e32ux642cux8fd0}}

\textbf{1.} 注册一个或多个DMA中断

\textbf{2.} 定义多个DMA\_Descriptor变量用来配置DMA通道寄存器

\textbf{3.} 根据数据搬运类型选择DMA寄存器配置接口,正确调用驱动接口配置DMA寄存器

\textbf{4.} 将多个DMA\_Descriptor变量首尾相连成串,类似链表

\textbf{5.} 使能DMA通道开始搬运

\hypertarget{ux6ce8ux610fux70b9-1}{%
\subsection{注意点}\label{ux6ce8ux610fux70b9-1}}

\begin{itemize}
\item
  定义DMA\_Descriptor变量需要8字节对齐,否则DMA搬运不成功
\item
  成串搬运如果配置多个DMA中断则需要在每个中断里使能DMA,直到最后一次搬运完成
\item
  对于从外设搬运需要确认外设是否支持DMA
\item
  建议从外设搬运选择握手方式,并与外设正确协商burstSize
\item
  burstSize尽量取较大值,有利于减少DMA中断次数提高单次中断处理效率。但burstSize太大可能最后一次不能搬运丢弃较多数据
\item
  建议设置从外设搬运总数据量为burstSize的整数倍或采用乒乓搬运的方式
\item
  在DMA从外设搬运的情况下,正确的操作顺序是先配置并使能好DMA,再使能外设开始产生数据
\end{itemize}

\hypertarget{ux7f16ux7a0bux6307ux5357-1}{%
\section{编程指南}\label{ux7f16ux7a0bux6307ux5357-1}}

\hypertarget{ux9a71ux52a8ux63a5ux53e3-1}{%
\subsection{驱动接口}\label{ux9a71ux52a8ux63a5ux53e3-1}}

\begin{itemize}
\item
  DMA\_PrepareMem2Mem:memory到memory搬运标准DMA寄存器配置接口
\item
  DMA\_PreparePeripheral2Mem:Peripheral到memory搬运标准DMA寄存器配置接口
\item
  DMA\_PrepareMem2Peripheral:memory到Peripheral搬运标准DMA寄存器配置接口
\item
  DMA\_PreparePeripheral2Peripheral:Peripheral到Peripheral搬运标准DMA寄存器配置接口
\item
  DMA\_Reset:DMA复位接口
\item
  DMA\_GetChannelIntState:DMA通道中断状态获取接口
\item
  DMA\_ClearChannelIntState:DMA通道清中断接口
\item
  DMA\_EnableChannel:DMA通道使能接口
\item
  DMA\_AbortChannel:DMA通道终止接口
\end{itemize}

\hypertarget{ux4ee3ux7801ux793aux4f8b-1}{%
\subsection{代码示例}\label{ux4ee3ux7801ux793aux4f8b-1}}

\hypertarget{ux5355ux6b21ux642cux8fd0-1}{%
\subsubsection{单次搬运}\label{ux5355ux6b21ux642cux8fd0-1}}

下面以memory到memory单次搬运展示DMA的基本用法:

\begin{Shaded}
\begin{Highlighting}[]
\PreprocessorTok{#define CHANNEL_ID  0}
\DataTypeTok{char}\NormalTok{ src[] = }\StringTok{"hello world!"}\NormalTok{;}
\DataTypeTok{char}\NormalTok{ dst[}\DecValTok{20}\NormalTok{];}
\NormalTok{DMA_Descriptor test __attribute__((aligned (}\DecValTok{8}\NormalTok{)));}
\DataTypeTok{static} \DataTypeTok{uint32_t}\NormalTok{ DMA_cb_isr(}\DataTypeTok{void}\NormalTok{ *user_data)}
\NormalTok{\{}
    \DataTypeTok{uint32_t}\NormalTok{ state = DMA_GetChannelIntState(CHANNEL_ID);}
\NormalTok{    DMA_ClearChannelIntState(CHANNEL_ID, state);}

\NormalTok{    printf(}\StringTok{"dst = %s}\SpecialCharTok{\textbackslash{}n}\StringTok{"}\NormalTok{, dst);}
    \ControlFlowTok{return} \DecValTok{0}\NormalTok{;}
\NormalTok{\}}

\DataTypeTok{void}\NormalTok{ DMA_Test(}\DataTypeTok{void}\NormalTok{)}
\NormalTok{\{}
\NormalTok{    DMA_Reset();}
\NormalTok{    platform_set_irq_callback(PLATFORM_CB_IRQ_DMA, DMA_cb_isr, }\DecValTok{0}\NormalTok{);}
\NormalTok{    DMA_PrepareMem2Mem(&test[}\DecValTok{0}\NormalTok{],}
\NormalTok{                       dst,}
\NormalTok{                       src, strlen(src),}
\NormalTok{                       DMA_ADDRESS_INC, DMA_ADDRESS_INC, }\DecValTok{0}\NormalTok{);}
\NormalTok{    DMA_EnableChannel(CHANNEL_ID, &test);}
\NormalTok{\}}
\end{Highlighting}
\end{Shaded}

最终会在DMA中断程序里面将搬运到dst中的``hello world!''字符串打印出来。

\hypertarget{ux6210ux4e32ux642cux8fd0-1}{%
\subsubsection{成串搬运}\label{ux6210ux4e32ux642cux8fd0-1}}

下面以memory到memory两块数据搬运拼接字符串展示DMA成串搬运的基本用法:

\begin{Shaded}
\begin{Highlighting}[]
\PreprocessorTok{#define CHANNEL_ID  0}
\DataTypeTok{char}\NormalTok{ src[] = }\StringTok{"hello world!"}\NormalTok{;}
\DataTypeTok{char}\NormalTok{ src1[] = }\StringTok{"I am a chinese."}\NormalTok{;}
\DataTypeTok{char}\NormalTok{ dst[}\DecValTok{100}\NormalTok{];}
\NormalTok{DMA_Descriptor test[}\DecValTok{2}\NormalTok{] __attribute__((aligned (}\DecValTok{8}\NormalTok{)));}
\DataTypeTok{static} \DataTypeTok{uint32_t}\NormalTok{ DMA_cb_isr(}\DataTypeTok{void}\NormalTok{ *user_data)}
\NormalTok{\{}
    \DataTypeTok{uint32_t}\NormalTok{ state = DMA_GetChannelIntState(CHANNEL_ID);}
\NormalTok{    DMA_ClearChannelIntState(CHANNEL_ID, state);}

\NormalTok{    printf(}\StringTok{"dst = %s}\SpecialCharTok{\textbackslash{}n}\StringTok{"}\NormalTok{, dst);}
    \ControlFlowTok{return} \DecValTok{0}\NormalTok{;}
\NormalTok{\}}

\DataTypeTok{void}\NormalTok{ DMA_Test(}\DataTypeTok{void}\NormalTok{)}
\NormalTok{\{}
\NormalTok{    DMA_Reset();}
\NormalTok{    platform_set_irq_callback(PLATFORM_CB_IRQ_DMA, DMA_cb_isr, }\DecValTok{0}\NormalTok{);}
\NormalTok{      test[}\DecValTok{0}\NormalTok{].Next = &test[}\DecValTok{1}\NormalTok{];    }\CommentTok{// make a DMA link chain}
\NormalTok{      test[}\DecValTok{1}\NormalTok{].Next = NULL;}
\NormalTok{    DMA_PrepareMem2Mem(&test[}\DecValTok{0}\NormalTok{],}
\NormalTok{                       dst,}
\NormalTok{                       src, strlen(src),}
\NormalTok{                       DMA_ADDRESS_INC, DMA_ADDRESS_INC, }\DecValTok{0}\NormalTok{);}
\NormalTok{    DMA_PrepareMem2Mem(&test[}\DecValTok{1}\NormalTok{],}
\NormalTok{                       dst + strlen(src),}
\NormalTok{                       src1, }\KeywordTok{sizeof}\NormalTok{(src1),}
\NormalTok{                       DMA_ADDRESS_INC, DMA_ADDRESS_INC, }\DecValTok{0}\NormalTok{);}
\NormalTok{    DMA_EnableChannel(CHANNEL_ID, &test[}\DecValTok{0}\NormalTok{]);}
\NormalTok{\}}
\end{Highlighting}
\end{Shaded}

最终将会打印出``hello world!I am a chinese.''字符串。

\hypertarget{dmaux4e52ux4e53ux642cux8fd0}{%
\subsubsection{DMA乒乓搬运}\label{dmaux4e52ux4e53ux642cux8fd0}}

DMA乒乓搬运是一种DMA搬运的特殊用法,其主要应用场景是将外设FIFO中数据循环搬运到memory中并处理。

可实现``搬运''和``数据处理''分离,从而大大提高程序处理数据的效率。

对于大量且连续的数据搬运,如音频,我们推荐选用DMA乒乓搬运的方式。

\hypertarget{dmaux4e52ux4e53ux642cux8fd0ux63a5ux53e3}{%
\paragraph{DMA乒乓搬运接口}\label{dmaux4e52ux4e53ux642cux8fd0ux63a5ux53e3}}

在最新SDK中我们已将DMA乒乓搬运封装成标准接口,方便开发者调用,提高开发效率。

使用时请添加 \texttt{pingpong.c}文件,并包含 \texttt{pingpong.h}文件。

\begin{itemize}
\item
  DMA\_PingPongSetup:DMA乒乓搬运建立接口
\item
  DMA\_PingPongIntProc:DMA乒乓搬运标准中断处理接口
\item
  DMA\_PingPongGetTransSize:获取DMA乒乓搬运数据量接口
\item
  DMA\_PingPongEnable:DMA乒乓搬运使能接口
\item
  DMA\_PingPongDisable:DMA乒乓搬运去使能接口
\end{itemize}

\hypertarget{dmaux4e52ux4e53ux642cux8fd0ux793aux4f8b}{%
\paragraph{DMA乒乓搬运示例}\label{dmaux4e52ux4e53ux642cux8fd0ux793aux4f8b}}

下面将以最常见的DMA乒乓搬运I2s数据为例展示DMA乒乓搬运的用法。

I2s的相关配置不在本文的介绍范围内,默认I2s已经配置好,DMA和I2s协商burstSize=8。

\begin{Shaded}
\begin{Highlighting}[]
\PreprocessorTok{#include }\ImportTok{"pingpong.h"}
\PreprocessorTok{#define CHANNEL_ID  0}
\NormalTok{DMA_PingPong_t PingPong;}

\DataTypeTok{static} \DataTypeTok{uint32_t}\NormalTok{ DMA_cb_isr(}\DataTypeTok{void}\NormalTok{ *user_data)}
\NormalTok{\{}
    \DataTypeTok{uint32_t}\NormalTok{ state = DMA_GetChannelIntState(CHANNEL_ID);}
\NormalTok{    DMA_ClearChannelIntState(CHANNEL_ID, state);}
    
    \CommentTok{// call 'DMA_PingPongIntProc' to get the pointer of data-buff.}
    \DataTypeTok{uint32_t}\NormalTok{ *rr = DMA_PingPongIntProc(&PingPong, CHANNEL_ID);}
    \DataTypeTok{uint32_t}\NormalTok{ i = }\DecValTok{0}\NormalTok{;}
    \CommentTok{// call 'DMA_PingPongGetTransSize' to kwon how much data in data-buff.}
    \DataTypeTok{uint32_t}\NormalTok{ transSize = DMA_PingPongGetTransSize(&PingPong);}
    \ControlFlowTok{while}\NormalTok{ (i < transSize) \{}
        \CommentTok{// do something with data 'rr[i]'}
\NormalTok{        i++;}
\NormalTok{    \}}

    \ControlFlowTok{return} \DecValTok{0}\NormalTok{;}
\NormalTok{\}}

\DataTypeTok{void}\NormalTok{ DMA_Test(}\DataTypeTok{void}\NormalTok{)}
\NormalTok{\{}
    \CommentTok{// call 'DMA_PingPongSetup' to setup ping-pong DMA.}
\NormalTok{    DMA_PingPongSetup(&PingPong, SYSCTRL_DMA_I2S_RX, }\DecValTok{100}\NormalTok{, }\DecValTok{8}\NormalTok{);}
\NormalTok{    platform_set_irq_callback(PLATFORM_CB_IRQ_DMA, DMA_cb_isr, }\DecValTok{0}\NormalTok{);}
    
    \CommentTok{// call 'DMA_PingPongEnable' to start ping-pong DMA transmission.}
\NormalTok{    DMA_PingPongEnable(&PingPong, CHANNEL_ID);}
\NormalTok{    I2S_ClearRxFIFO(APB_I2S);}
\NormalTok{    I2S_DMAEnable(APB_I2S, }\DecValTok{1}\NormalTok{, }\DecValTok{1}\NormalTok{);}
\NormalTok{    I2S_Enable(APB_I2S, }\DecValTok{0}\NormalTok{, }\DecValTok{1}\NormalTok{);   }\CommentTok{// Enable I2s finally}
\NormalTok{\}}
\end{Highlighting}
\end{Shaded}

停止DMA乒乓搬运可以调用以下接口:

\begin{Shaded}
\begin{Highlighting}[]
\DataTypeTok{void}\NormalTok{ Stop(}\DataTypeTok{void}\NormalTok{)}
\NormalTok{\{}
    \CommentTok{// call 'DMA_PingPongEnable' to disable ping-pong DMA transmission.}
\NormalTok{    DMA_PingPongDisable(&PingPong, CHANNEL_ID);}
\NormalTok{    I2S_Enable(APB_I2S, }\DecValTok{0}\NormalTok{, }\DecValTok{0}\NormalTok{);}
\NormalTok{    I2S_DMAEnable(APB_I2S, }\DecValTok{0}\NormalTok{, }\DecValTok{0}\NormalTok{);}
\NormalTok{\}}
\end{Highlighting}
\end{Shaded}

\hypertarget{efuseselectronic-fuses}{%
\chapter{eFuses(electronic fuses)}\label{efuseselectronic-fuses}}

\hypertarget{ux529fux80fdux6982ux8ff0}{%
\section{功能概述}\label{ux529fux80fdux6982ux8ff0}}

Efuse是一种片内一次性可编程存储器,可以在断电后保持数据,并且编程后无法被再次修改。 ING916系列提供128bit EFUSE,支持按bit编程或者按Word编程,bit的默认值是0,通过编程可以写成1

\hypertarget{ux4f7fux7528ux8bf4ux660e}{%
\section{使用说明}\label{ux4f7fux7528ux8bf4ux660e}}

\hypertarget{ux6a21ux5757ux521dux59cbux5316}{%
\subsection{模块初始化}\label{ux6a21ux5757ux521dux59cbux5316}}

\begin{Shaded}
\begin{Highlighting}[]
\DataTypeTok{void}\NormalTok{ setup_peripherals_efuse_module(}\DataTypeTok{void}\NormalTok{)}
\NormalTok{\{}
    \CommentTok{// 打开clock,并且reset模块}
\NormalTok{    SYSCTRL_ClearClkGateMulti( (}\DecValTok{1}\NormalTok{ << SYSCTRL_ITEM_APB_EFUSE));}
\NormalTok{    SYSCTRL_ResetBlock(SYSCTRL_ITEM_APB_EFUSE);}
\NormalTok{    SYSCTRL_ReleaseBlock(SYSCTRL_ITEM_APB_EFUSE);}
\NormalTok{\}}
\end{Highlighting}
\end{Shaded}

\hypertarget{ux6309bitux7f16ux7a0b}{%
\subsection{按bit编程}\label{ux6309bitux7f16ux7a0b}}

\begin{Shaded}
\begin{Highlighting}[]
\DataTypeTok{void}\NormalTok{ peripherals_efuse_write_bit(}\DataTypeTok{void}\NormalTok{)}
\NormalTok{\{}
    \CommentTok{// 提供需要编程的bit位置,0到127之间}
\NormalTok{    EFUSE_UnLock(APB_EFUSE, EFUSE_UNLOCK_FLAG);}
\NormalTok{    EFUSE_WriteEfuseDataBitToOne(APB_EFUSE, pos);}\CommentTok{//pos is bit position from 0 to 127}
    
    \CommentTok{//写操作完成}
    \CommentTok{//如果要读取,需要reset模块}
\NormalTok{    SYSCTRL_ResetBlock(SYSCTRL_ITEM_APB_EFUSE);}
\NormalTok{    SYSCTRL_ReleaseBlock(SYSCTRL_ITEM_APB_EFUSE);}
    
    \CommentTok{//设置读取flag}
\NormalTok{    EFUSE_SetRdFlag(APB_EFUSE);}
    \CommentTok{//等待数据读取标志}
    \ControlFlowTok{while}\NormalTok{(!EFUSE_GetDataValidFlag(APB_EFUSE));}
    \CommentTok{//读取数据}
\NormalTok{    platform_printf(}\StringTok{"word 0: 0x%x }\SpecialCharTok{\textbackslash{}n}\StringTok{"}\NormalTok{,EFUSE_GetEfuseData(APB_EFUSE, }\DecValTok{0}\NormalTok{));}\CommentTok{//bit 0  - 31}
\NormalTok{    platform_printf(}\StringTok{"word 1: 0x%x }\SpecialCharTok{\textbackslash{}n}\StringTok{"}\NormalTok{,EFUSE_GetEfuseData(APB_EFUSE, }\DecValTok{1}\NormalTok{));}\CommentTok{//bit 32 - 63}
\NormalTok{    platform_printf(}\StringTok{"word 2: 0x%x }\SpecialCharTok{\textbackslash{}n}\StringTok{"}\NormalTok{,EFUSE_GetEfuseData(APB_EFUSE, }\DecValTok{2}\NormalTok{));}\CommentTok{//bit 64 - 95}
\NormalTok{    platform_printf(}\StringTok{"word 3: 0x%x }\SpecialCharTok{\textbackslash{}n}\StringTok{"}\NormalTok{,EFUSE_GetEfuseData(APB_EFUSE, }\DecValTok{3}\NormalTok{));}\CommentTok{//bit 96 - 127}
\NormalTok{\}}
\end{Highlighting}
\end{Shaded}

\hypertarget{ux6309wordux7f16ux7a0b}{%
\subsection{按word编程}\label{ux6309wordux7f16ux7a0b}}

\begin{Shaded}
\begin{Highlighting}[]
\DataTypeTok{void}\NormalTok{ peripherals_efuse_write_word(}\DataTypeTok{void}\NormalTok{)}
\NormalTok{\{}
    \CommentTok{// 提供需要编程的word位置,0到3之间,每个word 32bit,一共128bit}
    \CommentTok{// EFUSE_PROGRAMWORDCNT_0代表word 0}
    \CommentTok{// data是要写入的32bit数据}
\NormalTok{    EFUSE_WriteEfuseDataWord(APB_EFUSE, EFUSE_PROGRAMWORDCNT_0, data);}
    
    \CommentTok{//写操作完成}
    \CommentTok{//如果要读取,需要reset模块}
\NormalTok{    SYSCTRL_ResetBlock(SYSCTRL_ITEM_APB_EFUSE);}
\NormalTok{    SYSCTRL_ReleaseBlock(SYSCTRL_ITEM_APB_EFUSE);}
    
    \CommentTok{//设置读取flag}
\NormalTok{    EFUSE_SetRdFlag(APB_EFUSE);}
    \CommentTok{//等待数据读取标志}
    \ControlFlowTok{while}\NormalTok{(!EFUSE_GetDataValidFlag(APB_EFUSE));}
    \CommentTok{//读取数据}
\NormalTok{    platform_printf(}\StringTok{"word 0: 0x%x }\SpecialCharTok{\textbackslash{}n}\StringTok{"}\NormalTok{,EFUSE_GetEfuseData(APB_EFUSE, }\DecValTok{0}\NormalTok{));}\CommentTok{//bit 0  - 31}
\NormalTok{    platform_printf(}\StringTok{"word 1: 0x%x }\SpecialCharTok{\textbackslash{}n}\StringTok{"}\NormalTok{,EFUSE_GetEfuseData(APB_EFUSE, }\DecValTok{1}\NormalTok{));}\CommentTok{//bit 32 - 63}
\NormalTok{    platform_printf(}\StringTok{"word 2: 0x%x }\SpecialCharTok{\textbackslash{}n}\StringTok{"}\NormalTok{,EFUSE_GetEfuseData(APB_EFUSE, }\DecValTok{2}\NormalTok{));}\CommentTok{//bit 64 - 95}
\NormalTok{    platform_printf(}\StringTok{"word 3: 0x%x }\SpecialCharTok{\textbackslash{}n}\StringTok{"}\NormalTok{,EFUSE_GetEfuseData(APB_EFUSE, }\DecValTok{3}\NormalTok{));}\CommentTok{//bit 96 - 127}
\NormalTok{\}}
\end{Highlighting}
\end{Shaded}

\begin{center}\rule{0.5\linewidth}{0.5pt}\end{center}

\hypertarget{ch-gpio}{%
\chapter{通用输入输出(GPIO)}\label{ch-gpio}}

\hypertarget{ux529fux80fdux6982ux8ff0-1}{%
\section{功能概述}\label{ux529fux80fdux6982ux8ff0-1}}

GPIO 模块常用于驱动 LED 或者其它指示器,控制片外设备,感知数字信号输入,检测信号边沿,
或者从低功耗状态唤醒系统。ING916XX 系列芯片内部支持最多 42 个 GPIO,通过 \protect\hyperlink{ch-pinctrl}{PINCTRL}
可将 GPIO \(n\) 引出到芯片 IO 管脚 \(n\)。

特性:

\begin{itemize}
\tightlist
\item
  每个 GPIO 都可单独配置为输入或输出
\item
  每个 GPIO 都可作为中断请求,中断触发方式支持边沿触发(上升、下降单沿触发,或者双沿触发)
  和电平触发(高电平或低电平)
\item
  硬件去抖
\end{itemize}

在硬件上存在 \textbf{两个} GPIO 模块,每个模块包含 21 个 GPIO,相应地定义了两个 \texttt{SYSCTRL\_Item}:

\begin{Shaded}
\begin{Highlighting}[]
\KeywordTok{typedef} \KeywordTok{enum}
\NormalTok{\{}
\NormalTok{    SYSCTRL_ITEM_APB_GPIO0     ,}
\NormalTok{    SYSCTRL_ITEM_APB_GPIO1     ,}
    \CommentTok{// ...}
\NormalTok{\} SYSCTRL_Item;}
\end{Highlighting}
\end{Shaded}

\begin{rmdcaution}
注意按照所使用的 GPIO 管脚打开对应的 GPIO 模块。
\end{rmdcaution}

\hypertarget{ux4f7fux7528ux8bf4ux660e-1}{%
\section{使用说明}\label{ux4f7fux7528ux8bf4ux660e-1}}

\hypertarget{ux8bbeux7f6e-io-ux65b9ux5411}{%
\subsection{设置 IO 方向}\label{ux8bbeux7f6e-io-ux65b9ux5411}}

在使用 GPIO 之前先按需要配置 IO 方向:

\begin{itemize}
\tightlist
\item
  需要用于输出信号时:配置为输出
\item
  需要用于读取信号时:配置为输入
\item
  需要用于生产中断请求时:配置为输入
\item
  需要高阻态时:配置为高阻态
\end{itemize}

使用 \texttt{GIO\_SetDirection} 配置 GPIO 的方向。GPIO 支持四种方向:

\begin{Shaded}
\begin{Highlighting}[]
\KeywordTok{typedef} \KeywordTok{enum}
\NormalTok{\{}
\NormalTok{    GIO_DIR_INPUT,  }\CommentTok{// 输入}
\NormalTok{    GIO_DIR_OUTPUT, }\CommentTok{// 输出}
\NormalTok{    GIO_DIR_BOTH,   }\CommentTok{// 同时支持输入、输出}
\NormalTok{    GIO_DIR_NONE    }\CommentTok{// 高阻态}
\NormalTok{\} GIO_Direction_t;}
\end{Highlighting}
\end{Shaded}

\begin{rmdcaution}
如无必要,不要使用 \texttt{GIO\_DIR\_BOTH}。
\end{rmdcaution}

\hypertarget{ux8bfbux53d6ux8f93ux5165}{%
\subsection{读取输入}\label{ux8bfbux53d6ux8f93ux5165}}

使用 \texttt{GIO\_ReadValue} 读取某个 GPIO 当前输入的电平信号,例如读取 GPIO 0 的输入:

\begin{Shaded}
\begin{Highlighting}[]
\DataTypeTok{uint8_t}\NormalTok{ value = GIO_ReadValue(GIO_GPIO_0);}
\end{Highlighting}
\end{Shaded}

使用 \texttt{GIO\_ReadAll} 可以同时读取所有 GPIO 当前输入的电平信号。其返回值的第 \(n\) 比特
(第 0 比特为最低比特)对应 GPIO \(n\) 的输入;如果 GPIO \(n\) 当前不支持输入,那么第 \(n\)
比特为 0:

\begin{Shaded}
\begin{Highlighting}[]
\DataTypeTok{uint64_t}\NormalTok{ GIO_ReadAll(}\DataTypeTok{void}\NormalTok{);}
\end{Highlighting}
\end{Shaded}

\hypertarget{ux8bbeux7f6eux8f93ux51fa}{%
\subsection{设置输出}\label{ux8bbeux7f6eux8f93ux51fa}}

使用 \texttt{GIO\_WriteValue} 设置某个 GPIO 输出的电平信号,例如使 GPIO 0 输出高电平(1):

\begin{Shaded}
\begin{Highlighting}[]
\NormalTok{GIO_WriteValue(GIO_GPIO_0, }\DecValTok{1}\NormalTok{);}
\end{Highlighting}
\end{Shaded}

\hypertarget{ux914dux7f6eux4e2dux65adux8bf7ux6c42}{%
\subsection{配置中断请求}\label{ux914dux7f6eux4e2dux65adux8bf7ux6c42}}

使用 \texttt{GIO\_ConfigIntSource} 配置 GPIO 生成中断请求。

\begin{Shaded}
\begin{Highlighting}[]
\DataTypeTok{void}\NormalTok{ GIO_ConfigIntSource(}
  \DataTypeTok{const}\NormalTok{ GIO_Index_t io_index,     }\CommentTok{// GPIO 编号}
  \DataTypeTok{const} \DataTypeTok{uint8_t}\NormalTok{ enable,           }\CommentTok{// 使能的边沿或者电平类型组合}
  \DataTypeTok{const}\NormalTok{ GIO_IntTriggerType_t type }\CommentTok{// 触发类型}
\NormalTok{  );}
\end{Highlighting}
\end{Shaded}

其中的 \texttt{enable} 为以下两个值的组合(0 表示禁止产生中断请求):

\begin{Shaded}
\begin{Highlighting}[]
\KeywordTok{typedef} \KeywordTok{enum}
\NormalTok{\{}
\NormalTok{    ...LOGIC_LOW_OR_FALLING_EDGE = ..., }\CommentTok{// 低电平或者下降沿}
\NormalTok{    ...LOGIC_HIGH_OR_RISING_EDGE = ...  }\CommentTok{// 高电平或者上升沿}
\NormalTok{\} GIO_IntTriggerEnable_t;}
\end{Highlighting}
\end{Shaded}

触发类型有两种:

\begin{Shaded}
\begin{Highlighting}[]
\KeywordTok{typedef} \KeywordTok{enum}
\NormalTok{\{}
\NormalTok{    GIO_INT_EDGE,   }\CommentTok{// 边沿触发}
\NormalTok{    GIO_INT_LOGIC   }\CommentTok{// 电平触发}
\NormalTok{\} GIO_IntTriggerType_t;}
\end{Highlighting}
\end{Shaded}

\begin{itemize}
\item
  例如将 GPIO 0 配置为上升沿触发中断

\begin{Shaded}
\begin{Highlighting}[]
\NormalTok{GIO_ConfigIntSource(GIO_GPIO_0,}
\NormalTok{  ...LOGIC_HIGH_OR_RISING_EDGE,}
\NormalTok{  GIO_INT_EDGE);}
\end{Highlighting}
\end{Shaded}
\item
  例如将 GPIO 0 配置为双沿触发中断

\begin{Shaded}
\begin{Highlighting}[]
\NormalTok{GIO_ConfigIntSource(GIO_GPIO_0,}
\NormalTok{  ...LOGIC_HIGH_OR_RISING_EDGE | ..._HIGH_OR_RISING_EDGE,}
\NormalTok{  GIO_INT_EDGE);}
\end{Highlighting}
\end{Shaded}
\item
  例如将 GPIO 0 配置为高电平触发

\begin{Shaded}
\begin{Highlighting}[]
\NormalTok{GIO_ConfigIntSource(GIO_GPIO_0,}
\NormalTok{  ...LOGIC_HIGH_OR_RISING_EDGE,}
\NormalTok{  GIO_INT_LOGIC);}
\end{Highlighting}
\end{Shaded}
\end{itemize}

\hypertarget{ux5904ux7406ux4e2dux65adux72b6ux6001}{%
\subsection{处理中断状态}\label{ux5904ux7406ux4e2dux65adux72b6ux6001}}

用 \texttt{GIO\_GetIntStatus} 获取某个 GPIO 上的中断触发状态,返回非 0 值表示该 GPIO
上产生了中断请求;用 \texttt{GIO\_GetAllIntStatus} 一次性获取所有 GPIO 的中断触发状态,
第 \(n\) 比特(第 0 比特为最低比特)对应 GPIO \(n\) 上的中断触发状态。

GPIO 产生中断后,需要消除中断状态方可再次触发。用 \texttt{GIO\_ClearIntStatus} 消除某个 GPIO
上中断状态,用 \texttt{GIO\_ClearAllIntStatus} 一次性清除所有 GPIO 上可能存在的中断触发状态。

\hypertarget{ux8f93ux5165ux53bbux6296}{%
\subsection{输入去抖}\label{ux8f93ux5165ux53bbux6296}}

使用 \texttt{GIO\_DebounceCtrl} 配置输入去抖参数,每个 GPIO 硬件模块使用单独的参数:

\begin{Shaded}
\begin{Highlighting}[]
\DataTypeTok{void}\NormalTok{ GIO_DebounceCtrl(}
  \DataTypeTok{uint8_t}\NormalTok{ group_mask,     }\CommentTok{// 比特 0 为 1 时配置模块 0}
                          \CommentTok{// 比特 1 为 1 时配置模块 1}
  \DataTypeTok{uint8_t}\NormalTok{ clk_pre_scale,}
\NormalTok{  GIO_DbClk_t clk         }\CommentTok{// 防抖时钟选择}
\NormalTok{  );}
\end{Highlighting}
\end{Shaded}

所谓去抖就是过滤掉长度小于 \texttt{(clk\_pre\_scale\ +\ 1)} 个防抖时钟周期的``毛刺''。

防抖时钟共有 2 种:

\begin{Shaded}
\begin{Highlighting}[]
\KeywordTok{typedef} \KeywordTok{enum}
\NormalTok{\{}
\NormalTok{    GIO_DB_CLK_32K,     }\CommentTok{// 使用 32k 时钟}
\NormalTok{    GIO_DB_CLK_PCLK,    }\CommentTok{// 使用快速 PCLK}
\NormalTok{\} GIO_DbClk_t;}
\end{Highlighting}
\end{Shaded}

快速 PCLK 的具体频率参考 \protect\hyperlink{ch-sysctrl}{SYSCTRL}。

通过 \texttt{GIO\_DebounceEn} 为单个 GPIO 使能去抖。例如要在 GPIO 0 上启用硬件去抖,忽略宽度小于
\(5/32768 \approx 0.15 (ms)\) 的``毛刺'':

\begin{Shaded}
\begin{Highlighting}[]
\NormalTok{GIO_DebounceCtrl(}\DecValTok{1}\NormalTok{, }\DecValTok{4}\NormalTok{, GIO_DB_CLK_32K);}
\NormalTok{GIO_DebounceEn(GIO_GPIO_0, }\DecValTok{1}\NormalTok{);}
\end{Highlighting}
\end{Shaded}

\hypertarget{i2cux529fux80fdux6982ux8ff0}{%
\chapter{I2C功能概述}\label{i2cux529fux80fdux6982ux8ff0}}

\begin{itemize}
\tightlist
\item
  两个I2C模块
\item
  支持Master/Slave模式
\item
  支持7bit/10bit地址
\item
  支持速率调整
\item
  支持DMA
\end{itemize}

\hypertarget{i2cux4f7fux7528ux8bf4ux660e}{%
\section{I2C使用说明}\label{i2cux4f7fux7528ux8bf4ux660e}}

针对916系列,IIC master有两种使用方式可以选择:

\begin{itemize}
\item
  方法1:以blocking的方式操作I2C(读写操作完成后API才会返回),针对I2C Master读取外设的场景。
\item
  方法2:使用I2C中断操作I2C,需要在中断中操作读写的数据。
\end{itemize}

IIC Slave 则需要使用方法2,以中断方式操作。

\hypertarget{ux65b9ux6cd51blocking}{%
\subsection{方法1(blocking)}\label{ux65b9ux6cd51blocking}}

\begin{itemize}
\item
  参考:\texttt{\textbackslash{}ING\_SDK\textbackslash{}sdk\textbackslash{}src\textbackslash{}BSP\textbackslash{}iic.c}
\item
  包含API:
\end{itemize}

\begin{Shaded}
\begin{Highlighting}[]
\CommentTok{/**}
\CommentTok{ * }\AnnotationTok{@brief}\CommentTok{ Init an I2C peripheral}
\CommentTok{ *}
\CommentTok{ * }\AnnotationTok{@param[in]}\CommentTok{ }\CommentVarTok{port}\CommentTok{              I2C peripheral ID}
\CommentTok{ */}
\DataTypeTok{void}\NormalTok{ i2c_init(}\DataTypeTok{const}\NormalTok{ i2c_port_t port);}

\CommentTok{/**}
\CommentTok{ * }\AnnotationTok{@brief}\CommentTok{ Write data to an I2C slave}
\CommentTok{ *}
\CommentTok{ * }\AnnotationTok{@param[in]}\CommentTok{ }\CommentVarTok{port}\CommentTok{              I2C peripheral ID}
\CommentTok{ * }\AnnotationTok{@param[in]}\CommentTok{ }\CommentVarTok{addr}\CommentTok{              address of the slave}
\CommentTok{ * }\AnnotationTok{@param[in]}\CommentTok{ }\CommentVarTok{byte_data}\CommentTok{         data to be written}
\CommentTok{ * }\AnnotationTok{@param[in]}\CommentTok{ }\CommentVarTok{length}\CommentTok{            data length}
\CommentTok{ * }\AnnotationTok{@return}\CommentTok{                      0 if success else non-0 (e.g. time out)}
\CommentTok{ */}
\DataTypeTok{int}\NormalTok{ i2c_write(}\DataTypeTok{const}\NormalTok{ i2c_port_t port, }\DataTypeTok{uint8_t}\NormalTok{ addr, }\DataTypeTok{const} \DataTypeTok{uint8_t}\NormalTok{ *byte_data, }\DataTypeTok{int16_t}\NormalTok{ length);}

\CommentTok{/**}
\CommentTok{ * }\AnnotationTok{@brief}\CommentTok{ Read data from an I2C slave}
\CommentTok{ *}
\CommentTok{ * }\AnnotationTok{@param[in]}\CommentTok{ }\CommentVarTok{port}\CommentTok{              I2C peripheral ID}
\CommentTok{ * }\AnnotationTok{@param[in]}\CommentTok{ }\CommentVarTok{addr}\CommentTok{              address of the slave}
\CommentTok{ * }\AnnotationTok{@param[in]}\CommentTok{ }\CommentVarTok{write_data}\CommentTok{        data to be written before reading}
\CommentTok{ * }\AnnotationTok{@param[in]}\CommentTok{ }\CommentVarTok{write_len}\CommentTok{         data length to be written before reading}
\CommentTok{ * }\AnnotationTok{@param[in]}\CommentTok{ }\CommentVarTok{byte_data}\CommentTok{         data to be read}
\CommentTok{ * }\AnnotationTok{@param[in]}\CommentTok{ }\CommentVarTok{length}\CommentTok{            data length to be read}
\CommentTok{ * }\AnnotationTok{@return}\CommentTok{                      0 if success else non-0 (e.g. time out)}
\CommentTok{ */}
\DataTypeTok{int}\NormalTok{ i2c_read(}\DataTypeTok{const}\NormalTok{ i2c_port_t port, }\DataTypeTok{uint8_t}\NormalTok{ addr, }\DataTypeTok{const} \DataTypeTok{uint8_t}\NormalTok{ *write_data, }\DataTypeTok{int16_t}\NormalTok{ write_len, }
              \DataTypeTok{uint8_t}\NormalTok{ *byte_data, }\DataTypeTok{int16_t}\NormalTok{ length);}
\end{Highlighting}
\end{Shaded}

\begin{itemize}
\item
  使用方法:

  \begin{itemize}
  \tightlist
  \item
    初始化IIC模块
  \end{itemize}

\begin{Shaded}
\begin{Highlighting}[]
\NormalTok{setup_peripherals_i2c_pin();}
\NormalTok{i2c_init(I2C_PORT_0);}
\end{Highlighting}
\end{Shaded}

  \texttt{setup\_peripherals\_i2c\_pin()}的使用请参考该文档中\texttt{配置pin}的描述(不需要配置IIC中断)。

  \begin{itemize}
  \tightlist
  \item
    写数据
  \end{itemize}

\begin{Shaded}
\begin{Highlighting}[]
\NormalTok{i2c_write(I2C_PORT_0, ADDRESS, write_data, DATA_CNT);}
\end{Highlighting}
\end{Shaded}

  当读操作完成后API才会返回,为了避免长时间等待ACK等意外情况,使用I2C\_HW\_TIME\_OUT来控制blocking的时间。

  \begin{itemize}
  \tightlist
  \item
    读数据
  \end{itemize}

\begin{Shaded}
\begin{Highlighting}[]
\NormalTok{i2c_read(I2C_PORT_0, ADDRESS, write_data, DATA_CNT, read_data, DATA_CNT);}
\end{Highlighting}
\end{Shaded}

  如果write\_data不为空,则会首先执行写操作,然后再执行读操作。
\end{itemize}

\hypertarget{ux65b9ux6cd52interrupt}{%
\subsection{方法2(Interrupt)}\label{ux65b9ux6cd52interrupt}}

IIC slave 以及 IIC Master方法2的使用请参考\texttt{场景5}。

以下场景中均以I2C0为例,如果需要I2C1则可以根据情况修改

\hypertarget{ux573aux666f1masterux8bfbslaveux4e0dux4f7fux7528dma}{%
\section{场景1:Master读Slave,不使用DMA}\label{ux573aux666f1masterux8bfbslaveux4e0dux4f7fux7528dma}}

其中I2C配置为Master读操作,Slave收到地址后,将数据返回给Master,CPU操作读写,没有使用DMA。 配置之前需要决定使用的GPIO,请参考datasheet了解可以使用的GPIO

\begin{Shaded}
\begin{Highlighting}[]
\PreprocessorTok{#define I2C_SCL         GIO_GPIO_10}
\PreprocessorTok{#define I2C_SDA         GIO_GPIO_11}
\end{Highlighting}
\end{Shaded}

\hypertarget{i2c-masterux914dux7f6e}{%
\subsection{I2C Master配置}\label{i2c-masterux914dux7f6e}}

// 测试数据,每次传输10个字节(fifo深度是8字节),每个传输单元必须是1字节

\begin{Shaded}
\begin{Highlighting}[]
\PreprocessorTok{#define DATA_CNT (10)}
\DataTypeTok{uint8_t}\NormalTok{ read_data[DATA_CNT] = \{}\DecValTok{0}\NormalTok{,\};}
\DataTypeTok{uint8_t}\NormalTok{ read_data_cnt = }\DecValTok{0}\NormalTok{;}
\end{Highlighting}
\end{Shaded}

\hypertarget{ux914dux7f6epin}{%
\subsubsection{配置Pin}\label{ux914dux7f6epin}}

将GPIO映射成I2C引脚

\begin{Shaded}
\begin{Highlighting}[]
\DataTypeTok{void}\NormalTok{ setup_peripherals_i2c_pin(}\DataTypeTok{void}\NormalTok{)}
\NormalTok{\{}
  \CommentTok{// 打开clock,注意此处使用的是I2C0}
  \CommentTok{// GPIO的clock请根据需要打开GPIO0或者GPIO1}
\NormalTok{  SYSCTRL_ClearClkGateMulti(    (}\DecValTok{1}\NormalTok{ << SYSCTRL_ITEM_APB_I2C0)}
\NormalTok{                                | (}\DecValTok{1}\NormalTok{ << SYSCTRL_ITEM_APB_SysCtrl)}
\NormalTok{                                | (}\DecValTok{1}\NormalTok{ << SYSCTRL_ITEM_APB_PinCtrl)}
\NormalTok{                                | (}\DecValTok{1}\NormalTok{ << SYSCTRL_ITEM_APB_GPIO1)}
\NormalTok{                                | (}\DecValTok{1}\NormalTok{ << SYSCTRL_ITEM_APB_GPIO0));}
  \CommentTok{// 使用默认内部上拉,实际使用中建议使用外部上拉 - 响应速度更快,可以支持更高的时钟}
  \CommentTok{// 如果使用外部上拉,则不需要pull}
\NormalTok{  PINCTRL_Pull(I2C_SCL,PINCTRL_PULL_UP);}
\NormalTok{  PINCTRL_Pull(I2C_SDA,PINCTRL_PULL_UP);}
  
  \CommentTok{// 将GPIO映射成I2C引脚}
\NormalTok{  PINCTRL_SelI2cIn(I2C_PORT_0,I2C_SCL,I2C_SDA);}
\NormalTok{  PINCTRL_SetPadMux(I2C_SCL,IO_SOURCE_I2C0_SCL_OUT);}
\NormalTok{  PINCTRL_SetPadMux(I2C_SDA,IO_SOURCE_I2C0_SDA_OUT);}
  
  \CommentTok{// 打开I2C中断}
\NormalTok{  platform_set_irq_callback(PLATFORM_CB_IRQ_I2C0, peripherals_i2c_isr, NULL);}
  
\NormalTok{\}}
\end{Highlighting}
\end{Shaded}

\hypertarget{ux521dux59cbux5316i2cux6a21ux5757}{%
\subsubsection{初始化I2C模块}\label{ux521dux59cbux5316i2cux6a21ux5757}}

\begin{Shaded}
\begin{Highlighting}[]
\PreprocessorTok{#define ADDRESS (0x71)}
\DataTypeTok{void}\NormalTok{ setup_peripherals_i2c_module(}\DataTypeTok{void}\NormalTok{)}
\NormalTok{\{}
  \CommentTok{// 配置为Master, 7bit地址}
\NormalTok{  I2C_Config(APB_I2C0,I2C_ROLE_MASTER,I2C_ADDRESSING_MODE_07BIT,ADDRESS);}
  \CommentTok{// 配置时钟,可选}
\NormalTok{  I2C_ConfigClkFrequency(APB_I2C0,I2C_CLOCKFREQUENY_STANDARD);}
  
\NormalTok{  I2C_Enable(APB_I2C0,}\DecValTok{1}\NormalTok{);}
  \CommentTok{// 打开传输结束中断和fifo满中断}
\NormalTok{  I2C_IntEnable(APB_I2C0,(}\DecValTok{1}\NormalTok{<<I2C_INT_CMPL)|(}\DecValTok{1}\NormalTok{ << I2C_INT_FIFO_FULL));}

\NormalTok{\}}
\end{Highlighting}
\end{Shaded}

\hypertarget{i2cux4e2dux65adux5b9eux73b0}{%
\subsubsection{I2C中断实现}\label{i2cux4e2dux65adux5b9eux73b0}}

\begin{Shaded}
\begin{Highlighting}[]
\DataTypeTok{static} \DataTypeTok{uint32_t}\NormalTok{ peripherals_i2c_isr(}\DataTypeTok{void}\NormalTok{ *user_data)}
\NormalTok{\{}
  \DataTypeTok{uint8_t}\NormalTok{ i;}
  \DataTypeTok{uint32_t}\NormalTok{ status = I2C_GetIntState(APB_I2C0);}
  
  \CommentTok{// FIFO 满之后,触发中断,此时需要将所有数据都读出来}
  \ControlFlowTok{if}\NormalTok{(status & (}\DecValTok{1}\NormalTok{ << I2C_STATUS_FIFO_FULL))}
\NormalTok{  \{}
    \ControlFlowTok{for}\NormalTok{(; read_data_cnt < DATA_CNT; read_data_cnt++)}
\NormalTok{    \{}
      \ControlFlowTok{if}\NormalTok{(I2C_FifoEmpty(APB_I2C0))\{}\ControlFlowTok{break}\NormalTok{;\}}
\NormalTok{      read_data[read_data_cnt] = I2C_DataRead(APB_I2C0);}
\NormalTok{    \}}
\NormalTok{  \}}
  
  \CommentTok{//传输结束后,触发中断,读取完剩下的数据}
  \ControlFlowTok{if}\NormalTok{(status & (}\DecValTok{1}\NormalTok{ << I2C_STATUS_CMPL))}
\NormalTok{  \{}
    \ControlFlowTok{for}\NormalTok{(; read_data_cnt < DATA_CNT; read_data_cnt++)}
\NormalTok{    \{}
\NormalTok{      read_data[read_data_cnt] = I2C_DataRead(APB_I2C0);}
\NormalTok{    \}}

\NormalTok{    I2C_ClearIntState(APB_I2C0, (}\DecValTok{1}\NormalTok{ << I2C_STATUS_CMPL));}
    
\NormalTok{  \}}
  
  \ControlFlowTok{return} \DecValTok{0}\NormalTok{;}
\NormalTok{\}}
\end{Highlighting}
\end{Shaded}

\hypertarget{i2c-masterux89e6ux53d1ux4f20ux8f93}{%
\subsubsection{I2C master触发传输}\label{i2c-masterux89e6ux53d1ux4f20ux8f93}}

\begin{Shaded}
\begin{Highlighting}[]
\DataTypeTok{void}\NormalTok{ peripheral_i2c_send_data(}\DataTypeTok{void}\NormalTok{)}
\NormalTok{\{}
  \CommentTok{// 设置方向,Master读取}
\NormalTok{  I2C_CtrlUpdateDirection(APB_I2C0,I2C_TRANSACTION_SLAVE2MASTER);}
  \CommentTok{// 设置每次传输的大小}
\NormalTok{  I2C_CtrlUpdateDataCnt(APB_I2C0, DATA_CNT);}
  \CommentTok{// 触发传输}
\NormalTok{  I2C_CommandWrite(APB_I2C0, I2C_COMMAND_ISSUE_DATA_TRANSACTION);}
  
\NormalTok{\}}
\end{Highlighting}
\end{Shaded}

\hypertarget{ux4f7fux7528ux6d41ux7a0b}{%
\subsubsection{使用流程}\label{ux4f7fux7528ux6d41ux7a0b}}

\begin{itemize}
\tightlist
\item
  设置GPIO,setup\_peripherals\_i2c\_pin()
\item
  初始化I2C,setup\_peripherals\_i2c\_module()
\item
  在需要时候触发I2C读取,peripheral\_i2c\_send\_data()
\item
  检查中断状态
\end{itemize}

\hypertarget{i2c-slaveux914dux7f6e}{%
\subsection{I2C Slave配置}\label{i2c-slaveux914dux7f6e}}

// 测试数据,每次传输10个字节(fifo深度是8字节),每个传输单元必须是1字节

\begin{Shaded}
\begin{Highlighting}[]
\PreprocessorTok{#define DATA_CNT (10)}
\DataTypeTok{uint8_t}\NormalTok{ write_data[DATA_CNT] = \{}\DecValTok{0}\NormalTok{,\};}
\DataTypeTok{uint8_t}\NormalTok{ write_data_cnt = }\DecValTok{0}\NormalTok{;}
\end{Highlighting}
\end{Shaded}

\hypertarget{ux914dux7f6epin-1}{%
\subsubsection{配置Pin}\label{ux914dux7f6epin-1}}

将GPIO映射成I2C引脚

\begin{Shaded}
\begin{Highlighting}[]
\DataTypeTok{void}\NormalTok{ setup_peripherals_i2c_pin(}\DataTypeTok{void}\NormalTok{)}
\NormalTok{\{}
  \CommentTok{// 打开clock,注意此处使用的是I2C0}
  \CommentTok{// GPIO的clock请根据需要打开GPIO0或者GPIO1}
\NormalTok{  SYSCTRL_ClearClkGateMulti(    (}\DecValTok{1}\NormalTok{ << SYSCTRL_ITEM_APB_I2C0)}
\NormalTok{                                | (}\DecValTok{1}\NormalTok{ << SYSCTRL_ITEM_APB_SysCtrl)}
\NormalTok{                                | (}\DecValTok{1}\NormalTok{ << SYSCTRL_ITEM_APB_PinCtrl)}
\NormalTok{                                | (}\DecValTok{1}\NormalTok{ << SYSCTRL_ITEM_APB_GPIO1)}
\NormalTok{                                | (}\DecValTok{1}\NormalTok{ << SYSCTRL_ITEM_APB_GPIO0));}
  \CommentTok{// 使用默认内部上拉,实际使用中建议使用外部上拉 - 响应速度更快,可以支持更高的时钟}
  \CommentTok{// 如果使用外部上拉,则不需要pull}
\NormalTok{  PINCTRL_Pull(I2C_SCL,PINCTRL_PULL_UP);}
\NormalTok{  PINCTRL_Pull(I2C_SDA,PINCTRL_PULL_UP);}
  
  \CommentTok{// 将GPIO映射成I2C引脚}
\NormalTok{  PINCTRL_SelI2cIn(I2C_PORT_0,I2C_SCL,I2C_SDA);}
\NormalTok{  PINCTRL_SetPadMux(I2C_SCL,IO_SOURCE_I2C0_SCL_OUT);}
\NormalTok{  PINCTRL_SetPadMux(I2C_SDA,IO_SOURCE_I2C0_SDA_OUT);}
  
  \CommentTok{// 打开I2C中断}
\NormalTok{  platform_set_irq_callback(PLATFORM_CB_IRQ_I2C0, peripherals_i2c_isr, NULL);}
  
\NormalTok{\}}
\end{Highlighting}
\end{Shaded}

\hypertarget{ux521dux59cbux5316i2cux6a21ux5757-1}{%
\subsubsection{初始化I2C模块}\label{ux521dux59cbux5316i2cux6a21ux5757-1}}

\begin{Shaded}
\begin{Highlighting}[]
\PreprocessorTok{#define ADDRESS (0x71)}
\DataTypeTok{void}\NormalTok{ setup_peripherals_i2c_module(}\DataTypeTok{void}\NormalTok{)}
\NormalTok{\{}
  \CommentTok{// 配置为Slave, 7bit地址}
\NormalTok{  I2C_Config(APB_I2C0,I2C_ROLE_SLAVE,I2C_ADDRESSING_MODE_07BIT,ADDRESS);}
  
\NormalTok{  I2C_Enable(APB_I2C0,}\DecValTok{1}\NormalTok{);}
  \CommentTok{// 打开传输结束中断和地址触发中断}
\NormalTok{  I2C_IntEnable(APB_I2C0,(}\DecValTok{1}\NormalTok{<<I2C_INT_ADDR_HIT)|(}\DecValTok{1}\NormalTok{<<I2C_INT_CMPL));}

\NormalTok{\}}
\end{Highlighting}
\end{Shaded}

\hypertarget{i2cux4e2dux65adux5b9eux73b0ux4ee5ux53caux53d1ux9001ux6570ux636e}{%
\subsubsection{I2C中断实现以及发送数据}\label{i2cux4e2dux65adux5b9eux73b0ux4ee5ux53caux53d1ux9001ux6570ux636e}}

\begin{Shaded}
\begin{Highlighting}[]
\DataTypeTok{static} \DataTypeTok{uint32_t}\NormalTok{ peripherals_i2c_isr(}\DataTypeTok{void}\NormalTok{ *user_data)}
\NormalTok{\{}
  \DataTypeTok{uint8_t}\NormalTok{ i;}
  \DataTypeTok{static} \DataTypeTok{uint8_t}\NormalTok{ dir = }\DecValTok{2}\NormalTok{;}
  \DataTypeTok{uint32_t}\NormalTok{ status = I2C_GetIntState(APB_I2C0);}

  \CommentTok{// Slave 收到匹配的地址,触发中断}
  \ControlFlowTok{if}\NormalTok{(status & (}\DecValTok{1}\NormalTok{ << I2C_STATUS_ADDRHIT))}
\NormalTok{  \{}
    \CommentTok{// 判断是读操作还是写操作}
\NormalTok{    dir = I2C_GetTransactionDir(APB_I2C0);}
    \ControlFlowTok{if}\NormalTok{(dir == I2C_TRANSACTION_MASTER2SLAVE)}
\NormalTok{    \{}
\NormalTok{        I2C_IntEnable(APB_I2C0,(}\DecValTok{1}\NormalTok{ << I2C_INT_FIFO_FULL));}
\NormalTok{    \}}
    \ControlFlowTok{else} \ControlFlowTok{if}\NormalTok{(dir == I2C_TRANSACTION_SLAVE2MASTER)}
\NormalTok{    \{}
\NormalTok{        I2C_IntEnable(APB_I2C0,(}\DecValTok{1}\NormalTok{ << I2C_INT_FIFO_EMPTY));}
\NormalTok{    \}}

\NormalTok{    I2C_ClearIntState(APB_I2C0, (}\DecValTok{1}\NormalTok{ << I2C_STATUS_ADDRHIT));}

\NormalTok{  \}}
  
  \CommentTok{// 如果是读操作,则会触发empty中断,此时需要填写需要发送的数据,直到FIFO满}
  \ControlFlowTok{if}\NormalTok{(status & (}\DecValTok{1}\NormalTok{ << I2C_STATUS_FIFO_EMPTY))}
\NormalTok{  \{}
    \CommentTok{// push data until fifo is full}
    \ControlFlowTok{for}\NormalTok{(; write_data_cnt < DATA_CNT; write_data_cnt++)}
\NormalTok{    \{}
      \ControlFlowTok{if}\NormalTok{(I2C_FifoFull(APB_I2C0))\{}\ControlFlowTok{break}\NormalTok{;\}}
\NormalTok{      I2C_DataWrite(APB_I2C0,write_data[write_data_cnt]);}
\NormalTok{    \}}
\NormalTok{  \}}
  
  \CommentTok{// 传输结束,清理打开的中断}
  \ControlFlowTok{if}\NormalTok{(status & (}\DecValTok{1}\NormalTok{ << I2C_STATUS_CMPL))}
\NormalTok{  \{}
    
\NormalTok{    I2C_ClearIntState(APB_I2C0, (}\DecValTok{1}\NormalTok{ << I2C_STATUS_CMPL));}
    
    \CommentTok{// prepare for next}
    \ControlFlowTok{if}\NormalTok{(dir == I2C_TRANSACTION_MASTER2SLAVE)}
\NormalTok{    \{}
\NormalTok{        I2C_IntDisable(APB_I2C0,(}\DecValTok{1}\NormalTok{ << I2C_INT_FIFO_FULL));}
\NormalTok{    \}}
    \ControlFlowTok{else} \ControlFlowTok{if}\NormalTok{(dir == I2C_TRANSACTION_SLAVE2MASTER)}
\NormalTok{    \{}
\NormalTok{        I2C_IntDisable(APB_I2C0,(}\DecValTok{1}\NormalTok{ << I2C_INT_FIFO_EMPTY));}
\NormalTok{    \}}

\NormalTok{  \}}

  \ControlFlowTok{return} \DecValTok{0}\NormalTok{;}
\NormalTok{\}}
\end{Highlighting}
\end{Shaded}

\hypertarget{ux4f7fux7528ux6d41ux7a0b-1}{%
\subsubsection{使用流程}\label{ux4f7fux7528ux6d41ux7a0b-1}}

\begin{itemize}
\tightlist
\item
  设置GPIO,setup\_peripherals\_i2c\_pin()
\item
  初始化I2C,setup\_peripherals\_i2c\_module()
\item
  检查中断状态,在中断中发送数据,I2C\_STATUS\_CMPL中断代表传输结束
\end{itemize}

\hypertarget{ux573aux666f2masterux5199slaveux4e0dux4f7fux7528dma}{%
\section{场景2:Master写Slave,不使用DMA}\label{ux573aux666f2masterux5199slaveux4e0dux4f7fux7528dma}}

其中I2C配置为Master写操作,Slave收到地址后,将从Master读取数据,CPU操作读写,没有使用DMA。 配置之前需要决定使用的GPIO,请参考datasheet了解可以使用的GPIO

\begin{Shaded}
\begin{Highlighting}[]
\PreprocessorTok{#define I2C_SCL         GIO_GPIO_10}
\PreprocessorTok{#define I2C_SDA         GIO_GPIO_11}
\end{Highlighting}
\end{Shaded}

\hypertarget{i2c-masterux914dux7f6e-1}{%
\subsection{I2C Master配置}\label{i2c-masterux914dux7f6e-1}}

// 测试数据,每次传输10个字节(fifo深度是8字节),每个传输单元必须是1字节

\begin{Shaded}
\begin{Highlighting}[]
\PreprocessorTok{#define DATA_CNT (10)}
\DataTypeTok{uint8_t}\NormalTok{ write_data[DATA_CNT] = \{}\DecValTok{0}\NormalTok{,\};}
\DataTypeTok{uint8_t}\NormalTok{ write_data_cnt = }\DecValTok{0}\NormalTok{;}
\end{Highlighting}
\end{Shaded}

\hypertarget{ux914dux7f6epin-2}{%
\subsubsection{配置Pin}\label{ux914dux7f6epin-2}}

将GPIO映射成I2C引脚

\begin{Shaded}
\begin{Highlighting}[]
\DataTypeTok{void}\NormalTok{ setup_peripherals_i2c_pin(}\DataTypeTok{void}\NormalTok{)}
\NormalTok{\{}
  \CommentTok{// 打开clock,注意此处使用的是I2C0}
  \CommentTok{// GPIO的clock请根据需要打开GPIO0或者GPIO1}
\NormalTok{  SYSCTRL_ClearClkGateMulti(    (}\DecValTok{1}\NormalTok{ << SYSCTRL_ITEM_APB_I2C0)}
\NormalTok{                                | (}\DecValTok{1}\NormalTok{ << SYSCTRL_ITEM_APB_SysCtrl)}
\NormalTok{                                | (}\DecValTok{1}\NormalTok{ << SYSCTRL_ITEM_APB_PinCtrl)}
\NormalTok{                                | (}\DecValTok{1}\NormalTok{ << SYSCTRL_ITEM_APB_GPIO1)}
\NormalTok{                                | (}\DecValTok{1}\NormalTok{ << SYSCTRL_ITEM_APB_GPIO0));}
  \CommentTok{// 使用默认内部上拉,实际使用中建议使用外部上拉 - 响应速度更快,可以支持更高的时钟}
  \CommentTok{// 如果使用外部上拉,则不需要pull}
\NormalTok{  PINCTRL_Pull(I2C_SCL,PINCTRL_PULL_UP);}
\NormalTok{  PINCTRL_Pull(I2C_SDA,PINCTRL_PULL_UP);}
  
  \CommentTok{// 将GPIO映射成I2C引脚}
\NormalTok{  PINCTRL_SelI2cIn(I2C_PORT_0,I2C_SCL,I2C_SDA);}
\NormalTok{  PINCTRL_SetPadMux(I2C_SCL,IO_SOURCE_I2C0_SCL_OUT);}
\NormalTok{  PINCTRL_SetPadMux(I2C_SDA,IO_SOURCE_I2C0_SDA_OUT);}
  
  \CommentTok{// 打开I2C中断}
\NormalTok{  platform_set_irq_callback(PLATFORM_CB_IRQ_I2C0, peripherals_i2c_isr, NULL);}
  
\NormalTok{\}}
\end{Highlighting}
\end{Shaded}

\hypertarget{ux521dux59cbux5316i2cux6a21ux5757-2}{%
\subsubsection{初始化I2C模块}\label{ux521dux59cbux5316i2cux6a21ux5757-2}}

\begin{Shaded}
\begin{Highlighting}[]
\PreprocessorTok{#define ADDRESS (0x71)}
\DataTypeTok{void}\NormalTok{ setup_peripherals_i2c_module(}\DataTypeTok{void}\NormalTok{)}
\NormalTok{\{}
  \CommentTok{// 配置为Master, 7bit地址}
\NormalTok{  I2C_Config(APB_I2C0,I2C_ROLE_MASTER,I2C_ADDRESSING_MODE_07BIT,ADDRESS);}
  \CommentTok{// 配置时钟,可选}
\NormalTok{  I2C_ConfigClkFrequency(APB_I2C0,I2C_CLOCKFREQUENY_STANDARD);}
  
\NormalTok{  I2C_Enable(APB_I2C0,}\DecValTok{1}\NormalTok{);}
  \CommentTok{// 打开传输结束中断和fifo空中断}
\NormalTok{  I2C_IntEnable(APB_I2C0,(}\DecValTok{1}\NormalTok{<<I2C_INT_CMPL)|(}\DecValTok{1}\NormalTok{ << I2C_INT_FIFO_EMPTY));}

\NormalTok{\}}
\end{Highlighting}
\end{Shaded}

\hypertarget{i2cux4e2dux65adux5b9eux73b0-1}{%
\subsubsection{I2C中断实现}\label{i2cux4e2dux65adux5b9eux73b0-1}}

\begin{Shaded}
\begin{Highlighting}[]
\DataTypeTok{static} \DataTypeTok{uint32_t}\NormalTok{ peripherals_i2c_isr(}\DataTypeTok{void}\NormalTok{ *user_data)}
\NormalTok{\{}
  \DataTypeTok{uint8_t}\NormalTok{ i;}
  \DataTypeTok{uint32_t}\NormalTok{ status = I2C_GetIntState(APB_I2C0);}
  
  \CommentTok{// 传输结束,触发中断}
  \ControlFlowTok{if}\NormalTok{(status & (}\DecValTok{1}\NormalTok{ << I2C_STATUS_CMPL))}
\NormalTok{  \{}
\NormalTok{    I2C_ClearIntState(APB_I2C0, (}\DecValTok{1}\NormalTok{ << I2C_STATUS_CMPL));}
\NormalTok{  \}}
  
  \CommentTok{// empty FIFO 中断,填写需要发送的数据,数据全部发送之后,关掉I2C_INT_FIFO_EMPTY}
  \ControlFlowTok{if}\NormalTok{(status & (}\DecValTok{1}\NormalTok{ << I2C_STATUS_FIFO_EMPTY))}
\NormalTok{  \{}
    \CommentTok{// push data until fifo is full}
    \ControlFlowTok{for}\NormalTok{(; write_data_cnt < DATA_CNT; write_data_cnt++)}
\NormalTok{    \{}
      \ControlFlowTok{if}\NormalTok{(I2C_FifoFull(APB_I2C0))\{}\ControlFlowTok{break}\NormalTok{;\}}
\NormalTok{      I2C_DataWrite(APB_I2C0,write_data[write_data_cnt]);}
\NormalTok{    \}}
    
    \CommentTok{// if its the last, disable empty int}
    \ControlFlowTok{if}\NormalTok{(write_data_cnt == DATA_CNT)}
\NormalTok{    \{}
\NormalTok{      I2C_IntDisable(APB_I2C0,(}\DecValTok{1}\NormalTok{ << I2C_INT_FIFO_EMPTY));}
\NormalTok{    \}}

\NormalTok{  \}}
  
  \ControlFlowTok{return} \DecValTok{0}\NormalTok{;}
\NormalTok{\}}
\end{Highlighting}
\end{Shaded}

\hypertarget{i2c-masterux89e6ux53d1ux4f20ux8f93-1}{%
\subsubsection{I2C master触发传输}\label{i2c-masterux89e6ux53d1ux4f20ux8f93-1}}

\begin{Shaded}
\begin{Highlighting}[]
\DataTypeTok{void}\NormalTok{ peripheral_i2c_send_data(}\DataTypeTok{void}\NormalTok{)}
\NormalTok{\{}
  \CommentTok{// 设置方向,Master写}
\NormalTok{  I2C_CtrlUpdateDirection(APB_I2C0,I2C_TRANSACTION_MASTER2SLAVE);}
  \CommentTok{// 设置每次传输的大小}
\NormalTok{  I2C_CtrlUpdateDataCnt(APB_I2C0, DATA_CNT);}
  \CommentTok{// 触发传输}
\NormalTok{  I2C_CommandWrite(APB_I2C0, I2C_COMMAND_ISSUE_DATA_TRANSACTION);}
  
\NormalTok{\}}
\end{Highlighting}
\end{Shaded}

\hypertarget{ux4f7fux7528ux6d41ux7a0b-2}{%
\subsubsection{使用流程}\label{ux4f7fux7528ux6d41ux7a0b-2}}

\begin{itemize}
\tightlist
\item
  设置GPIO,setup\_peripherals\_i2c\_pin()
\item
  初始化I2C,setup\_peripherals\_i2c\_module()
\item
  在需要时候发送I2C数据,peripheral\_i2c\_send\_data()
\item
  检查中断状态
\end{itemize}

\hypertarget{i2c-slaveux914dux7f6e-1}{%
\subsection{I2C Slave配置}\label{i2c-slaveux914dux7f6e-1}}

// 测试数据,每次传输10个字节(fifo深度是8字节),每个传输单元必须是1字节

\begin{Shaded}
\begin{Highlighting}[]
\PreprocessorTok{#define DATA_CNT (10)}
\DataTypeTok{uint8_t}\NormalTok{ read_data[DATA_CNT] = \{}\DecValTok{0}\NormalTok{,\};}
\DataTypeTok{uint8_t}\NormalTok{ read_data_cnt = }\DecValTok{0}\NormalTok{;}
\end{Highlighting}
\end{Shaded}

\hypertarget{ux914dux7f6epin-3}{%
\subsubsection{配置Pin}\label{ux914dux7f6epin-3}}

将GPIO映射成I2C引脚

\begin{Shaded}
\begin{Highlighting}[]
\DataTypeTok{void}\NormalTok{ setup_peripherals_i2c_pin(}\DataTypeTok{void}\NormalTok{)}
\NormalTok{\{}
  \CommentTok{// 打开clock,注意此处使用的是I2C0}
  \CommentTok{// GPIO的clock请根据需要打开GPIO0或者GPIO1}
\NormalTok{  SYSCTRL_ClearClkGateMulti(    (}\DecValTok{1}\NormalTok{ << SYSCTRL_ITEM_APB_I2C0)}
\NormalTok{                                | (}\DecValTok{1}\NormalTok{ << SYSCTRL_ITEM_APB_SysCtrl)}
\NormalTok{                                | (}\DecValTok{1}\NormalTok{ << SYSCTRL_ITEM_APB_PinCtrl)}
\NormalTok{                                | (}\DecValTok{1}\NormalTok{ << SYSCTRL_ITEM_APB_GPIO1)}
\NormalTok{                                | (}\DecValTok{1}\NormalTok{ << SYSCTRL_ITEM_APB_GPIO0));}
  \CommentTok{// 使用默认内部上拉,实际使用中建议使用外部上拉 - 响应速度更快,可以支持更高的时钟}
  \CommentTok{// 如果使用外部上拉,则不需要pull}
\NormalTok{  PINCTRL_Pull(I2C_SCL,PINCTRL_PULL_UP);}
\NormalTok{  PINCTRL_Pull(I2C_SDA,PINCTRL_PULL_UP);}
  
  \CommentTok{// 将GPIO映射成I2C引脚}
\NormalTok{  PINCTRL_SelI2cIn(I2C_PORT_0,I2C_SCL,I2C_SDA);}
\NormalTok{  PINCTRL_SetPadMux(I2C_SCL,IO_SOURCE_I2C0_SCL_OUT);}
\NormalTok{  PINCTRL_SetPadMux(I2C_SDA,IO_SOURCE_I2C0_SDA_OUT);}
  
  \CommentTok{// 打开I2C中断}
\NormalTok{  platform_set_irq_callback(PLATFORM_CB_IRQ_I2C0, peripherals_i2c_isr, NULL);}
  
\NormalTok{\}}
\end{Highlighting}
\end{Shaded}

\hypertarget{ux521dux59cbux5316i2cux6a21ux5757-3}{%
\subsubsection{初始化I2C模块}\label{ux521dux59cbux5316i2cux6a21ux5757-3}}

\begin{Shaded}
\begin{Highlighting}[]
\PreprocessorTok{#define ADDRESS (0x71)}
\DataTypeTok{void}\NormalTok{ setup_peripherals_i2c_module(}\DataTypeTok{void}\NormalTok{)}
\NormalTok{\{}
  \CommentTok{// 配置为Slave, 7bit地址}
\NormalTok{  I2C_Config(APB_I2C0,I2C_ROLE_SLAVE,I2C_ADDRESSING_MODE_07BIT,ADDRESS);}
  
\NormalTok{  I2C_Enable(APB_I2C0,}\DecValTok{1}\NormalTok{);}
  \CommentTok{// 打开传输结束中断和地址触发中断}
\NormalTok{  I2C_IntEnable(APB_I2C0,(}\DecValTok{1}\NormalTok{<<I2C_INT_ADDR_HIT)|(}\DecValTok{1}\NormalTok{<<I2C_INT_CMPL));}

\NormalTok{\}}
\end{Highlighting}
\end{Shaded}

\hypertarget{i2cux4e2dux65adux5b9eux73b0ux4ee5ux53caux53d1ux9001ux6570ux636e-1}{%
\subsubsection{I2C中断实现以及发送数据}\label{i2cux4e2dux65adux5b9eux73b0ux4ee5ux53caux53d1ux9001ux6570ux636e-1}}

\begin{Shaded}
\begin{Highlighting}[]
\DataTypeTok{static} \DataTypeTok{uint32_t}\NormalTok{ peripherals_i2c_isr(}\DataTypeTok{void}\NormalTok{ *user_data)}
\NormalTok{\{}
  \DataTypeTok{uint8_t}\NormalTok{ i;}
  \DataTypeTok{static} \DataTypeTok{uint8_t}\NormalTok{ dir = }\DecValTok{2}\NormalTok{;}
  \DataTypeTok{uint32_t}\NormalTok{ status = I2C_GetIntState(APB_I2C0);}

  \CommentTok{// Slave 收到匹配的地址,触发中断}
  \ControlFlowTok{if}\NormalTok{(status & (}\DecValTok{1}\NormalTok{ << I2C_STATUS_ADDRHIT))}
\NormalTok{  \{}
    \CommentTok{// 判断是读操作还是写操作}
\NormalTok{    dir = I2C_GetTransactionDir(APB_I2C0);}
    \ControlFlowTok{if}\NormalTok{(dir == I2C_TRANSACTION_MASTER2SLAVE)}
\NormalTok{    \{}
\NormalTok{        I2C_IntEnable(APB_I2C0,(}\DecValTok{1}\NormalTok{ << I2C_INT_FIFO_FULL));}
\NormalTok{    \}}
    \ControlFlowTok{else} \ControlFlowTok{if}\NormalTok{(dir == I2C_TRANSACTION_SLAVE2MASTER)}
\NormalTok{    \{}
\NormalTok{        I2C_IntEnable(APB_I2C0,(}\DecValTok{1}\NormalTok{ << I2C_INT_FIFO_EMPTY));}
\NormalTok{    \}}

\NormalTok{    I2C_ClearIntState(APB_I2C0, (}\DecValTok{1}\NormalTok{ << I2C_STATUS_ADDRHIT));}

\NormalTok{  \}}
  
  \CommentTok{// 等待FIFO FULL,读取FIFO直到FIFO变空}
  \ControlFlowTok{if}\NormalTok{(status & (}\DecValTok{1}\NormalTok{ << I2C_STATUS_FIFO_FULL))}
\NormalTok{  \{}
    \ControlFlowTok{for}\NormalTok{(; read_data_cnt < DATA_CNT; read_data_cnt++)}
\NormalTok{    \{}
      \ControlFlowTok{if}\NormalTok{(I2C_FifoEmpty(APB_I2C0))\{}\ControlFlowTok{break}\NormalTok{;\}}
\NormalTok{      read_data[read_data_cnt] = I2C_DataRead(APB_I2C0);}
\NormalTok{    \}}
\NormalTok{  \}}
  
  \CommentTok{// 传输结束,读取FIFO中剩余的数据,清除相关中断}
  \ControlFlowTok{if}\NormalTok{(status & (}\DecValTok{1}\NormalTok{ << I2C_STATUS_CMPL))}
\NormalTok{  \{}
    \ControlFlowTok{for}\NormalTok{(;read_data_cnt < DATA_CNT; read_data_cnt++)}
\NormalTok{    \{}
\NormalTok{      read_data[read_data_cnt] = I2C_DataRead(APB_I2C0);}
\NormalTok{    \}}
    
\NormalTok{    I2C_ClearIntState(APB_I2C0, (}\DecValTok{1}\NormalTok{ << I2C_STATUS_CMPL));}
    
    \CommentTok{// prepare for next}
    \ControlFlowTok{if}\NormalTok{(dir == I2C_TRANSACTION_MASTER2SLAVE)}
\NormalTok{    \{}
\NormalTok{        I2C_IntDisable(APB_I2C0,(}\DecValTok{1}\NormalTok{ << I2C_INT_FIFO_FULL));}
\NormalTok{    \}}
    \ControlFlowTok{else} \ControlFlowTok{if}\NormalTok{(dir == I2C_TRANSACTION_SLAVE2MASTER)}
\NormalTok{    \{}
\NormalTok{        I2C_IntDisable(APB_I2C0,(}\DecValTok{1}\NormalTok{ << I2C_INT_FIFO_EMPTY));}
\NormalTok{    \}}

\NormalTok{  \}}


  \ControlFlowTok{return} \DecValTok{0}\NormalTok{;}
\NormalTok{\}}
\end{Highlighting}
\end{Shaded}

\hypertarget{ux4f7fux7528ux6d41ux7a0b-3}{%
\subsubsection{使用流程}\label{ux4f7fux7528ux6d41ux7a0b-3}}

\begin{itemize}
\tightlist
\item
  设置GPIO,setup\_peripherals\_i2c\_pin()
\item
  初始化I2C,setup\_peripherals\_i2c\_module()
\item
  检查中断状态,I2C\_STATUS\_CMPL中断代表传输结束
\end{itemize}

\hypertarget{ux573aux666f3masterux8bfbslaveux4f7fux7528dma}{%
\section{场景3:Master读Slave,使用DMA}\label{ux573aux666f3masterux8bfbslaveux4f7fux7528dma}}

其中I2C配置为Master读操作,Slave收到地址后,将数据返回给Master,DMA操作读写。 配置之前需要决定使用的GPIO,请参考datasheet了解可以使用的GPIO

\begin{Shaded}
\begin{Highlighting}[]
\PreprocessorTok{#define I2C_SCL         GIO_GPIO_10}
\PreprocessorTok{#define I2C_SDA         GIO_GPIO_11}
\end{Highlighting}
\end{Shaded}

\hypertarget{i2c-masterux914dux7f6e-2}{%
\subsection{I2C Master配置}\label{i2c-masterux914dux7f6e-2}}

// 测试数据,每次传输23个字节(fifo深度是8字节),每个传输单元必须是1字节

\begin{Shaded}
\begin{Highlighting}[]
\PreprocessorTok{#define DATA_CNT (23)}
\DataTypeTok{uint8_t}\NormalTok{ read_data[DATA_CNT] = \{}\DecValTok{0}\NormalTok{,\};}
\DataTypeTok{uint8_t}\NormalTok{ read_data_cnt = }\DecValTok{0}\NormalTok{;}
\end{Highlighting}
\end{Shaded}

\hypertarget{ux914dux7f6epin-4}{%
\subsubsection{配置Pin}\label{ux914dux7f6epin-4}}

将GPIO映射成I2C引脚

\begin{Shaded}
\begin{Highlighting}[]
\DataTypeTok{void}\NormalTok{ setup_peripherals_i2c_pin(}\DataTypeTok{void}\NormalTok{)}
\NormalTok{\{}
  \CommentTok{// 打开clock,注意此处使用的是I2C0}
  \CommentTok{// GPIO的clock请根据需要打开GPIO0或者GPIO1}
\NormalTok{  SYSCTRL_ClearClkGateMulti(    (}\DecValTok{1}\NormalTok{ << SYSCTRL_ITEM_APB_I2C0)}
\NormalTok{                                | (}\DecValTok{1}\NormalTok{ << SYSCTRL_ITEM_APB_SysCtrl)}
\NormalTok{                                | (}\DecValTok{1}\NormalTok{ << SYSCTRL_ITEM_APB_PinCtrl)}
\NormalTok{                                | (}\DecValTok{1}\NormalTok{ << SYSCTRL_ITEM_APB_GPIO1)}
\NormalTok{                                | (}\DecValTok{1}\NormalTok{ << SYSCTRL_ITEM_APB_GPIO0));}
  \CommentTok{// 使用默认内部上拉,实际使用中建议使用外部上拉 - 响应速度更快,可以支持更高的时钟}
  \CommentTok{// 如果使用外部上拉,则不需要pull}
\NormalTok{  PINCTRL_Pull(I2C_SCL,PINCTRL_PULL_UP);}
\NormalTok{  PINCTRL_Pull(I2C_SDA,PINCTRL_PULL_UP);}
  
  \CommentTok{// 将GPIO映射成I2C引脚}
\NormalTok{  PINCTRL_SelI2cIn(I2C_PORT_0,I2C_SCL,I2C_SDA);}
\NormalTok{  PINCTRL_SetPadMux(I2C_SCL,IO_SOURCE_I2C0_SCL_OUT);}
\NormalTok{  PINCTRL_SetPadMux(I2C_SDA,IO_SOURCE_I2C0_SDA_OUT);}
  
  \CommentTok{// 打开I2C中断}
\NormalTok{  platform_set_irq_callback(PLATFORM_CB_IRQ_I2C0, peripherals_i2c_isr, NULL);}
  
\NormalTok{\}}
\end{Highlighting}
\end{Shaded}

\hypertarget{ux521dux59cbux5316i2cux6a21ux5757-4}{%
\subsubsection{初始化I2C模块}\label{ux521dux59cbux5316i2cux6a21ux5757-4}}

\begin{Shaded}
\begin{Highlighting}[]
\PreprocessorTok{#define ADDRESS (0x71)}
\DataTypeTok{void}\NormalTok{ setup_peripherals_i2c_module(}\DataTypeTok{void}\NormalTok{)}
\NormalTok{\{}
  \CommentTok{// 配置为Master, 7bit地址}
\NormalTok{  I2C_Config(APB_I2C0,I2C_ROLE_MASTER,I2C_ADDRESSING_MODE_07BIT,ADDRESS);}
  \CommentTok{// 配置时钟,可选}
\NormalTok{  I2C_ConfigClkFrequency(APB_I2C0,I2C_CLOCKFREQUENY_STANDARD);}
  
\NormalTok{  I2C_Enable(APB_I2C0,}\DecValTok{1}\NormalTok{);}
  \CommentTok{// 打开传输结束中断}
\NormalTok{  I2C_IntEnable(APB_I2C0,(}\DecValTok{1}\NormalTok{<<I2C_INT_CMPL));}

\NormalTok{\}}
\end{Highlighting}
\end{Shaded}

\hypertarget{ux521dux59cbux5316dmaux6a21ux5757}{%
\subsubsection{初始化DMA模块}\label{ux521dux59cbux5316dmaux6a21ux5757}}

\begin{Shaded}
\begin{Highlighting}[]
\DataTypeTok{static} \DataTypeTok{void}\NormalTok{ setup_peripherals_dma_module(}\DataTypeTok{void}\NormalTok{)}
\NormalTok{\{}
\NormalTok{    SYSCTRL_ClearClkGateMulti(}\DecValTok{1}\NormalTok{ << SYSCTRL_ClkGate_APB_DMA);}
\NormalTok{    DMA_Reset(}\DecValTok{1}\NormalTok{);}
\NormalTok{    DMA_Reset(}\DecValTok{0}\NormalTok{);}
\NormalTok{\}}
\end{Highlighting}
\end{Shaded}

\hypertarget{i2cux4e2dux65adux5b9eux73b0-2}{%
\subsubsection{I2C中断实现}\label{i2cux4e2dux65adux5b9eux73b0-2}}

\begin{Shaded}
\begin{Highlighting}[]
\DataTypeTok{static} \DataTypeTok{uint32_t}\NormalTok{ peripherals_i2c_isr(}\DataTypeTok{void}\NormalTok{ *user_data)}
\NormalTok{\{}
  \DataTypeTok{uint8_t}\NormalTok{ i;}
  \DataTypeTok{uint32_t}\NormalTok{ status = I2C_GetIntState(APB_I2C0);}
  
  \ControlFlowTok{if}\NormalTok{(status & (}\DecValTok{1}\NormalTok{ << I2C_STATUS_CMPL))}
\NormalTok{  \{}
\NormalTok{    I2C_ClearIntState(APB_I2C0, (}\DecValTok{1}\NormalTok{ << I2C_STATUS_CMPL));}
\NormalTok{  \}}

  \ControlFlowTok{return} \DecValTok{0}\NormalTok{;}
\NormalTok{\}}
\end{Highlighting}
\end{Shaded}

\hypertarget{i2c-master-dmaux8bbeux7f6e}{%
\subsubsection{I2C master DMA设置}\label{i2c-master-dmaux8bbeux7f6e}}

\begin{Shaded}
\begin{Highlighting}[]
\CommentTok{// 注意此处是I2C0}
\DataTypeTok{void}\NormalTok{ peripherals_i2c_rxfifo_to_dma(}\DataTypeTok{int}\NormalTok{ channel_id, }\DataTypeTok{void}\NormalTok{ *dst, }\DataTypeTok{int}\NormalTok{ size)}
\NormalTok{\{}
\NormalTok{  DMA_Descriptor descriptor __attribute__((aligned (}\DecValTok{8}\NormalTok{)));}

\NormalTok{  descriptor.Next = (DMA_Descriptor *)}\DecValTok{0}\NormalTok{;}
\NormalTok{  DMA_PreparePeripheral2Mem(&descriptor,dst,SYSCTRL_DMA_I2C0,size,DMA_ADDRESS_INC,}\DecValTok{0}\NormalTok{);}

\NormalTok{  DMA_EnableChannel(channel_id, &descriptor);}
\NormalTok{\}}
\end{Highlighting}
\end{Shaded}

\hypertarget{i2c-masterux89e6ux53d1ux4f20ux8f93-2}{%
\subsubsection{I2C master触发传输}\label{i2c-masterux89e6ux53d1ux4f20ux8f93-2}}

\begin{Shaded}
\begin{Highlighting}[]
\DataTypeTok{void}\NormalTok{ peripheral_i2c_send_data(}\DataTypeTok{void}\NormalTok{)}
\NormalTok{\{}
  \CommentTok{// I2C DMA 功能打开}
\NormalTok{  I2C_DmaEnable(APB_I2C0,}\DecValTok{1}\NormalTok{);}
  \CommentTok{// 设置传输方向,Master读}
\NormalTok{  I2C_CtrlUpdateDirection(APB_I2C0,I2C_TRANSACTION_SLAVE2MASTER);}
  \CommentTok{// 设置需要传输的数据大小}
\NormalTok{  I2C_CtrlUpdateDataCnt(APB_I2C0, DATA_CNT);}

  \CommentTok{// 配置DMA}
  \PreprocessorTok{#define I2C_DMA_RX_CHANNEL   (0)}\CommentTok{//DMA channel 0}
\NormalTok{  peripherals_i2c_rxfifo_to_dma(I2C_DMA_RX_CHANNEL, read_data, }\KeywordTok{sizeof}\NormalTok{(read_data));}
  
  \CommentTok{// 触发传输}
\NormalTok{  I2C_CommandWrite(APB_I2C0, I2C_COMMAND_ISSUE_DATA_TRANSACTION);}
  
\NormalTok{\}}
\end{Highlighting}
\end{Shaded}

\hypertarget{ux4f7fux7528ux6d41ux7a0b-4}{%
\subsubsection{使用流程}\label{ux4f7fux7528ux6d41ux7a0b-4}}

\begin{itemize}
\tightlist
\item
  设置GPIO,setup\_peripherals\_i2c\_pin()
\item
  初始化I2C,setup\_peripherals\_i2c\_module()
\item
  初始化DMA,setup\_peripherals\_dma\_module()
\item
  在需要时候触发I2C读取,peripheral\_i2c\_send\_data()
\item
  检查中断状态
\end{itemize}

\hypertarget{i2c-slaveux914dux7f6e-2}{%
\subsection{I2C Slave配置}\label{i2c-slaveux914dux7f6e-2}}

// 测试数据,每次传输23个字节(fifo深度是8字节),每个传输单元必须是1字节

\begin{Shaded}
\begin{Highlighting}[]
\PreprocessorTok{#define DATA_CNT (23)}
\DataTypeTok{uint8_t}\NormalTok{ write_data[DATA_CNT] = \{}\DecValTok{0}\NormalTok{,\};}
\DataTypeTok{uint8_t}\NormalTok{ write_data_cnt = }\DecValTok{0}\NormalTok{;}
\end{Highlighting}
\end{Shaded}

\hypertarget{ux914dux7f6epin-5}{%
\subsubsection{配置Pin}\label{ux914dux7f6epin-5}}

将GPIO映射成I2C引脚

\begin{Shaded}
\begin{Highlighting}[]
\DataTypeTok{void}\NormalTok{ setup_peripherals_i2c_pin(}\DataTypeTok{void}\NormalTok{)}
\NormalTok{\{}
  \CommentTok{// 打开clock,注意此处使用的是I2C0}
  \CommentTok{// GPIO的clock请根据需要打开GPIO0或者GPIO1}
\NormalTok{  SYSCTRL_ClearClkGateMulti(    (}\DecValTok{1}\NormalTok{ << SYSCTRL_ITEM_APB_I2C0)}
\NormalTok{                                | (}\DecValTok{1}\NormalTok{ << SYSCTRL_ITEM_APB_SysCtrl)}
\NormalTok{                                | (}\DecValTok{1}\NormalTok{ << SYSCTRL_ITEM_APB_PinCtrl)}
\NormalTok{                                | (}\DecValTok{1}\NormalTok{ << SYSCTRL_ITEM_APB_GPIO1)}
\NormalTok{                                | (}\DecValTok{1}\NormalTok{ << SYSCTRL_ITEM_APB_GPIO0));}
  \CommentTok{// 使用默认内部上拉,实际使用中建议使用外部上拉 - 响应速度更快,可以支持更高的时钟}
  \CommentTok{// 如果使用外部上拉,则不需要pull}
\NormalTok{  PINCTRL_Pull(I2C_SCL,PINCTRL_PULL_UP);}
\NormalTok{  PINCTRL_Pull(I2C_SDA,PINCTRL_PULL_UP);}
  
  \CommentTok{// 将GPIO映射成I2C引脚}
\NormalTok{  PINCTRL_SelI2cIn(I2C_PORT_0,I2C_SCL,I2C_SDA);}
\NormalTok{  PINCTRL_SetPadMux(I2C_SCL,IO_SOURCE_I2C0_SCL_OUT);}
\NormalTok{  PINCTRL_SetPadMux(I2C_SDA,IO_SOURCE_I2C0_SDA_OUT);}
  
  \CommentTok{// 打开I2C中断}
\NormalTok{  platform_set_irq_callback(PLATFORM_CB_IRQ_I2C0, peripherals_i2c_isr, NULL);}
  
\NormalTok{\}}
\end{Highlighting}
\end{Shaded}

\hypertarget{ux521dux59cbux5316i2cux6a21ux5757-5}{%
\subsubsection{初始化I2C模块}\label{ux521dux59cbux5316i2cux6a21ux5757-5}}

\begin{Shaded}
\begin{Highlighting}[]
\PreprocessorTok{#define ADDRESS (0x71)}
\DataTypeTok{void}\NormalTok{ setup_peripherals_i2c_module(}\DataTypeTok{void}\NormalTok{)}
\NormalTok{\{}
  \CommentTok{// 配置为Slave, 7bit地址}
\NormalTok{  I2C_Config(APB_I2C0,I2C_ROLE_SLAVE,I2C_ADDRESSING_MODE_07BIT,ADDRESS);}
  
\NormalTok{  I2C_Enable(APB_I2C0,}\DecValTok{1}\NormalTok{);}
  \CommentTok{// 打开传输结束中断和地址触发中断}
\NormalTok{  I2C_IntEnable(APB_I2C0,(}\DecValTok{1}\NormalTok{<<I2C_INT_ADDR_HIT)|(}\DecValTok{1}\NormalTok{<<I2C_INT_CMPL));}

\NormalTok{\}}
\end{Highlighting}
\end{Shaded}

\hypertarget{ux521dux59cbux5316dmaux6a21ux5757-1}{%
\subsubsection{初始化DMA模块}\label{ux521dux59cbux5316dmaux6a21ux5757-1}}

\begin{Shaded}
\begin{Highlighting}[]
\DataTypeTok{static} \DataTypeTok{void}\NormalTok{ setup_peripherals_dma_module(}\DataTypeTok{void}\NormalTok{)}
\NormalTok{\{}
\NormalTok{    SYSCTRL_ClearClkGateMulti(}\DecValTok{1}\NormalTok{ << SYSCTRL_ClkGate_APB_DMA);}
\NormalTok{    DMA_Reset(}\DecValTok{1}\NormalTok{);}
\NormalTok{    DMA_Reset(}\DecValTok{0}\NormalTok{);}
\NormalTok{\}}
\end{Highlighting}
\end{Shaded}

\hypertarget{i2cux4e2dux65adux5b9eux73b0ux4ee5ux53caux53d1ux9001ux6570ux636e-2}{%
\subsubsection{I2C中断实现以及发送数据}\label{i2cux4e2dux65adux5b9eux73b0ux4ee5ux53caux53d1ux9001ux6570ux636e-2}}

\begin{Shaded}
\begin{Highlighting}[]
\DataTypeTok{static} \DataTypeTok{uint32_t}\NormalTok{ peripherals_i2c_isr(}\DataTypeTok{void}\NormalTok{ *user_data)}
\NormalTok{\{}
  \DataTypeTok{uint8_t}\NormalTok{ i;}
  \DataTypeTok{static} \DataTypeTok{uint8_t}\NormalTok{ dir = }\DecValTok{2}\NormalTok{;}
  \DataTypeTok{uint32_t}\NormalTok{ status = I2C_GetIntState(APB_I2C0);}

  \CommentTok{// Slave收到了匹配的地址,触发中断,判断方向}
  \ControlFlowTok{if}\NormalTok{(status & (}\DecValTok{1}\NormalTok{ << I2C_STATUS_ADDRHIT))}
\NormalTok{  \{}
\NormalTok{    dir = I2C_GetTransactionDir(APB_I2C0);}
    \ControlFlowTok{if}\NormalTok{(dir == I2C_TRANSACTION_MASTER2SLAVE)}
\NormalTok{    \{}

\NormalTok{    \}}
    \ControlFlowTok{else} \ControlFlowTok{if}\NormalTok{(dir == I2C_TRANSACTION_SLAVE2MASTER)}
\NormalTok{    \{}
      \CommentTok{// 设置DMA,发送数据  }
\NormalTok{      peripherals_i2c_write_data_dma_setup();}
\NormalTok{    \}}
\NormalTok{    I2C_ClearIntState(APB_I2C0, (}\DecValTok{1}\NormalTok{ << I2C_STATUS_ADDRHIT));}
\NormalTok{  \}}
  
  \CommentTok{// 传输结束,关闭DMA}
  \ControlFlowTok{if}\NormalTok{(status & (}\DecValTok{1}\NormalTok{ << I2C_STATUS_CMPL))}
\NormalTok{  \{}
\NormalTok{    I2C_DmaEnable(APB_I2C0,}\DecValTok{0}\NormalTok{);}
\NormalTok{    I2C_ClearIntState(APB_I2C0, (}\DecValTok{1}\NormalTok{ << I2C_STATUS_CMPL));}
\NormalTok{  \}}

  \ControlFlowTok{return} \DecValTok{0}\NormalTok{;}
\NormalTok{\}}
\end{Highlighting}
\end{Shaded}

\hypertarget{i2c-slave-ux53d1ux9001ux6570ux636edmaux8bbeux7f6e}{%
\subsubsection{I2C Slave 发送数据DMA设置}\label{i2c-slave-ux53d1ux9001ux6570ux636edmaux8bbeux7f6e}}

\begin{Shaded}
\begin{Highlighting}[]
\CommentTok{// 此处配置的是I2C0}
\DataTypeTok{void}\NormalTok{ peripherals_i2c_dma_to_txfifo(}\DataTypeTok{int}\NormalTok{ channel_id, }\DataTypeTok{void}\NormalTok{ *src, }\DataTypeTok{int}\NormalTok{ size)}
\NormalTok{\{}
\NormalTok{  DMA_Descriptor descriptor __attribute__((aligned (}\DecValTok{8}\NormalTok{)));}

\NormalTok{  descriptor.Next = (DMA_Descriptor *)}\DecValTok{0}\NormalTok{;}
\NormalTok{  DMA_PrepareMem2Peripheral(&descriptor,SYSCTRL_DMA_I2C0,src,size,DMA_ADDRESS_INC,}\DecValTok{0}\NormalTok{);}

\NormalTok{  DMA_EnableChannel(channel_id, &descriptor);}
\NormalTok{\}}

\DataTypeTok{void}\NormalTok{ peripherals_i2c_write_data_dma_setup(}\DataTypeTok{void}\NormalTok{)}
\NormalTok{\{}
  \CommentTok{// 设置DMA}
  \PreprocessorTok{#define I2C_DMA_TX_CHANNEL   (0)}\CommentTok{//DMA channel 0}
\NormalTok{  peripherals_i2c_dma_to_txfifo(I2C_DMA_TX_CHANNEL, write_data, }\KeywordTok{sizeof}\NormalTok{(write_data));}
  \CommentTok{// 更新需要传输的数据}
\NormalTok{  I2C_CtrlUpdateDataCnt(APB_I2C0, DATA_CNT);}
  \CommentTok{// 打开I2C DMA}
\NormalTok{  I2C_DmaEnable(APB_I2C0,}\DecValTok{1}\NormalTok{);}
\NormalTok{\}}
\end{Highlighting}
\end{Shaded}

\hypertarget{ux4f7fux7528ux6d41ux7a0b-5}{%
\subsubsection{使用流程}\label{ux4f7fux7528ux6d41ux7a0b-5}}

\begin{itemize}
\tightlist
\item
  设置GPIO,setup\_peripherals\_i2c\_pin()
\item
  初始化I2C,setup\_peripherals\_i2c\_module()
\item
  初始化DMA,setup\_peripherals\_dma\_module()
\item
  检查中断状态,在中断中设置DMA发送数据,I2C\_STATUS\_CMPL中断代表传输结束
\end{itemize}

\hypertarget{ux573aux666f4masterux5199slaveux4f7fux7528dma}{%
\section{场景4:Master写Slave,使用DMA}\label{ux573aux666f4masterux5199slaveux4f7fux7528dma}}

其中I2C配置为Master写操作,Slave收到地址后,读取Master发送的数据,DMA操作读写。 配置之前需要决定使用的GPIO,请参考datasheet了解可以使用的GPIO

\begin{Shaded}
\begin{Highlighting}[]
\PreprocessorTok{#define I2C_SCL         GIO_GPIO_10}
\PreprocessorTok{#define I2C_SDA         GIO_GPIO_11}
\end{Highlighting}
\end{Shaded}

\hypertarget{i2c-masterux914dux7f6e-3}{%
\subsection{I2C Master配置}\label{i2c-masterux914dux7f6e-3}}

// 测试数据,每次传输23个字节(fifo深度是8字节),每个传输单元必须是1字节

\begin{Shaded}
\begin{Highlighting}[]
\PreprocessorTok{#define DATA_CNT (23)}
\DataTypeTok{uint8_t}\NormalTok{ write_data[DATA_CNT] = \{}\DecValTok{0}\NormalTok{,\};}
\DataTypeTok{uint8_t}\NormalTok{ write_data_cnt = }\DecValTok{0}\NormalTok{;}
\end{Highlighting}
\end{Shaded}

\hypertarget{ux914dux7f6epin-6}{%
\subsubsection{配置Pin}\label{ux914dux7f6epin-6}}

将GPIO映射成I2C引脚

\begin{Shaded}
\begin{Highlighting}[]
\DataTypeTok{void}\NormalTok{ setup_peripherals_i2c_pin(}\DataTypeTok{void}\NormalTok{)}
\NormalTok{\{}
  \CommentTok{// 打开clock,注意此处使用的是I2C0}
  \CommentTok{// GPIO的clock请根据需要打开GPIO0或者GPIO1}
\NormalTok{  SYSCTRL_ClearClkGateMulti(    (}\DecValTok{1}\NormalTok{ << SYSCTRL_ITEM_APB_I2C0)}
\NormalTok{                                | (}\DecValTok{1}\NormalTok{ << SYSCTRL_ITEM_APB_SysCtrl)}
\NormalTok{                                | (}\DecValTok{1}\NormalTok{ << SYSCTRL_ITEM_APB_PinCtrl)}
\NormalTok{                                | (}\DecValTok{1}\NormalTok{ << SYSCTRL_ITEM_APB_GPIO1)}
\NormalTok{                                | (}\DecValTok{1}\NormalTok{ << SYSCTRL_ITEM_APB_GPIO0));}
  \CommentTok{// 使用默认内部上拉,实际使用中建议使用外部上拉 - 响应速度更快,可以支持更高的时钟}
  \CommentTok{// 如果使用外部上拉,则不需要pull}
\NormalTok{  PINCTRL_Pull(I2C_SCL,PINCTRL_PULL_UP);}
\NormalTok{  PINCTRL_Pull(I2C_SDA,PINCTRL_PULL_UP);}
  
  \CommentTok{// 将GPIO映射成I2C引脚}
\NormalTok{  PINCTRL_SelI2cIn(I2C_PORT_0,I2C_SCL,I2C_SDA);}
\NormalTok{  PINCTRL_SetPadMux(I2C_SCL,IO_SOURCE_I2C0_SCL_OUT);}
\NormalTok{  PINCTRL_SetPadMux(I2C_SDA,IO_SOURCE_I2C0_SDA_OUT);}
  
  \CommentTok{// 打开I2C中断}
\NormalTok{  platform_set_irq_callback(PLATFORM_CB_IRQ_I2C0, peripherals_i2c_isr, NULL);}
  
\NormalTok{\}}
\end{Highlighting}
\end{Shaded}

\hypertarget{ux521dux59cbux5316i2cux6a21ux5757-6}{%
\subsubsection{初始化I2C模块}\label{ux521dux59cbux5316i2cux6a21ux5757-6}}

\begin{Shaded}
\begin{Highlighting}[]
\PreprocessorTok{#define ADDRESS (0x71)}
\DataTypeTok{void}\NormalTok{ setup_peripherals_i2c_module(}\DataTypeTok{void}\NormalTok{)}
\NormalTok{\{}
  \CommentTok{// 配置为Master, 7bit地址}
\NormalTok{  I2C_Config(APB_I2C0,I2C_ROLE_MASTER,I2C_ADDRESSING_MODE_07BIT,ADDRESS);}
  \CommentTok{// 配置时钟,可选}
\NormalTok{  I2C_ConfigClkFrequency(APB_I2C0,I2C_CLOCKFREQUENY_STANDARD);}
  
\NormalTok{  I2C_Enable(APB_I2C0,}\DecValTok{1}\NormalTok{);}
  \CommentTok{// 打开传输结束中断}
\NormalTok{  I2C_IntEnable(APB_I2C0,(}\DecValTok{1}\NormalTok{<<I2C_INT_CMPL));}

\NormalTok{\}}
\end{Highlighting}
\end{Shaded}

\hypertarget{ux521dux59cbux5316dmaux6a21ux5757-2}{%
\subsubsection{初始化DMA模块}\label{ux521dux59cbux5316dmaux6a21ux5757-2}}

\begin{Shaded}
\begin{Highlighting}[]
\DataTypeTok{static} \DataTypeTok{void}\NormalTok{ setup_peripherals_dma_module(}\DataTypeTok{void}\NormalTok{)}
\NormalTok{\{}
\NormalTok{    SYSCTRL_ClearClkGateMulti(}\DecValTok{1}\NormalTok{ << SYSCTRL_ClkGate_APB_DMA);}
\NormalTok{    DMA_Reset(}\DecValTok{1}\NormalTok{);}
\NormalTok{    DMA_Reset(}\DecValTok{0}\NormalTok{);}
\NormalTok{\}}
\end{Highlighting}
\end{Shaded}

\hypertarget{i2cux4e2dux65adux5b9eux73b0-3}{%
\subsubsection{I2C中断实现}\label{i2cux4e2dux65adux5b9eux73b0-3}}

\begin{Shaded}
\begin{Highlighting}[]
\DataTypeTok{static} \DataTypeTok{uint32_t}\NormalTok{ peripherals_i2c_isr(}\DataTypeTok{void}\NormalTok{ *user_data)}
\NormalTok{\{}
  \DataTypeTok{uint8_t}\NormalTok{ i;}
  \DataTypeTok{uint32_t}\NormalTok{ status = I2C_GetIntState(APB_I2C0);}
  
  \ControlFlowTok{if}\NormalTok{(status & (}\DecValTok{1}\NormalTok{ << I2C_STATUS_CMPL))}
\NormalTok{  \{}
\NormalTok{    I2C_ClearIntState(APB_I2C0, (}\DecValTok{1}\NormalTok{ << I2C_STATUS_CMPL));}
\NormalTok{  \}}

  \ControlFlowTok{return} \DecValTok{0}\NormalTok{;}
\NormalTok{\}}
\end{Highlighting}
\end{Shaded}

\hypertarget{i2c-master-dmaux8bbeux7f6e-1}{%
\subsubsection{I2C master DMA设置}\label{i2c-master-dmaux8bbeux7f6e-1}}

\begin{Shaded}
\begin{Highlighting}[]
\CommentTok{// 注意此处是I2C0}
\DataTypeTok{void}\NormalTok{ peripherals_i2c_dma_to_txfifo(}\DataTypeTok{int}\NormalTok{ channel_id, }\DataTypeTok{void}\NormalTok{ *src, }\DataTypeTok{int}\NormalTok{ size)}
\NormalTok{\{}
\NormalTok{    DMA_Descriptor descriptor __attribute__((aligned (}\DecValTok{8}\NormalTok{)));}

\NormalTok{    descriptor.Next = (DMA_Descriptor *)}\DecValTok{0}\NormalTok{;}
\NormalTok{    DMA_PrepareMem2Peripheral(&descriptor,SYSCTRL_DMA_I2C0,src,size,DMA_ADDRESS_INC,}\DecValTok{0}\NormalTok{);}

\NormalTok{    DMA_EnableChannel(channel_id, &descriptor);}
\NormalTok{\}}
\end{Highlighting}
\end{Shaded}

\hypertarget{i2c-masterux89e6ux53d1ux4f20ux8f93-3}{%
\subsubsection{I2C master触发传输}\label{i2c-masterux89e6ux53d1ux4f20ux8f93-3}}

\begin{Shaded}
\begin{Highlighting}[]
\DataTypeTok{void}\NormalTok{ peripheral_i2c_send_data(}\DataTypeTok{void}\NormalTok{)}
\NormalTok{\{}
  \CommentTok{// I2C DMA 功能打开}
\NormalTok{  I2C_DmaEnable(APB_I2C0,}\DecValTok{1}\NormalTok{);}
  \CommentTok{// 设置传输方向,Master读}
\NormalTok{  I2C_CtrlUpdateDirection(APB_I2C0,I2C_TRANSACTION_MASTER2SLAVE);}
  \CommentTok{// 设置需要传输的数据大小}
\NormalTok{  I2C_CtrlUpdateDataCnt(APB_I2C0, DATA_CNT);}

  \CommentTok{// 配置DMA}
  \PreprocessorTok{#define I2C_DMA_TX_CHANNEL   (0)}\CommentTok{//DMA channel 0}
\NormalTok{  peripherals_i2c_dma_to_txfifo(I2C_DMA_TX_CHANNEL, write_data, }\KeywordTok{sizeof}\NormalTok{(write_data));}
  
  \CommentTok{// 触发传输}
\NormalTok{  I2C_CommandWrite(APB_I2C0, I2C_COMMAND_ISSUE_DATA_TRANSACTION);}
  
\NormalTok{\}}
\end{Highlighting}
\end{Shaded}

\hypertarget{ux4f7fux7528ux6d41ux7a0b-6}{%
\subsubsection{使用流程}\label{ux4f7fux7528ux6d41ux7a0b-6}}

\begin{itemize}
\tightlist
\item
  设置GPIO,setup\_peripherals\_i2c\_pin()
\item
  初始化I2C,setup\_peripherals\_i2c\_module()
\item
  初始化DMA,setup\_peripherals\_dma\_module()
\item
  在需要时候触发I2C读取,peripheral\_i2c\_send\_data()
\item
  检查中断状态
\end{itemize}

\hypertarget{i2c-slaveux914dux7f6e-3}{%
\subsection{I2C Slave配置}\label{i2c-slaveux914dux7f6e-3}}

// 测试数据,每次传输23个字节(fifo深度是8字节),每个传输单元必须是1字节

\begin{Shaded}
\begin{Highlighting}[]
\PreprocessorTok{#define DATA_CNT (23)}
\DataTypeTok{uint8_t}\NormalTok{ read_data[DATA_CNT] = \{}\DecValTok{0}\NormalTok{,\};}
\DataTypeTok{uint8_t}\NormalTok{ read_data_cnt = }\DecValTok{0}\NormalTok{;}
\end{Highlighting}
\end{Shaded}

\hypertarget{ux914dux7f6epin-7}{%
\subsubsection{配置Pin}\label{ux914dux7f6epin-7}}

将GPIO映射成I2C引脚

\begin{Shaded}
\begin{Highlighting}[]
\DataTypeTok{void}\NormalTok{ setup_peripherals_i2c_pin(}\DataTypeTok{void}\NormalTok{)}
\NormalTok{\{}
  \CommentTok{// 打开clock,注意此处使用的是I2C0}
  \CommentTok{// GPIO的clock请根据需要打开GPIO0或者GPIO1}
\NormalTok{  SYSCTRL_ClearClkGateMulti(    (}\DecValTok{1}\NormalTok{ << SYSCTRL_ITEM_APB_I2C0)}
\NormalTok{                                | (}\DecValTok{1}\NormalTok{ << SYSCTRL_ITEM_APB_SysCtrl)}
\NormalTok{                                | (}\DecValTok{1}\NormalTok{ << SYSCTRL_ITEM_APB_PinCtrl)}
\NormalTok{                                | (}\DecValTok{1}\NormalTok{ << SYSCTRL_ITEM_APB_GPIO1)}
\NormalTok{                                | (}\DecValTok{1}\NormalTok{ << SYSCTRL_ITEM_APB_GPIO0));}
  \CommentTok{// 使用默认内部上拉,实际使用中建议使用外部上拉 - 响应速度更快,可以支持更高的时钟}
  \CommentTok{// 如果使用外部上拉,则不需要pull}
\NormalTok{  PINCTRL_Pull(I2C_SCL,PINCTRL_PULL_UP);}
\NormalTok{  PINCTRL_Pull(I2C_SDA,PINCTRL_PULL_UP);}
  
  \CommentTok{// 将GPIO映射成I2C引脚}
\NormalTok{  PINCTRL_SelI2cIn(I2C_PORT_0,I2C_SCL,I2C_SDA);}
\NormalTok{  PINCTRL_SetPadMux(I2C_SCL,IO_SOURCE_I2C0_SCL_OUT);}
\NormalTok{  PINCTRL_SetPadMux(I2C_SDA,IO_SOURCE_I2C0_SDA_OUT);}
  
  \CommentTok{// 打开I2C中断}
\NormalTok{  platform_set_irq_callback(PLATFORM_CB_IRQ_I2C0, peripherals_i2c_isr, NULL);}
  
\NormalTok{\}}
\end{Highlighting}
\end{Shaded}

\hypertarget{ux521dux59cbux5316i2cux6a21ux5757-7}{%
\subsubsection{初始化I2C模块}\label{ux521dux59cbux5316i2cux6a21ux5757-7}}

\begin{Shaded}
\begin{Highlighting}[]
\PreprocessorTok{#define ADDRESS (0x71)}
\DataTypeTok{void}\NormalTok{ setup_peripherals_i2c_module(}\DataTypeTok{void}\NormalTok{)}
\NormalTok{\{}
  \CommentTok{// 配置为Slave, 7bit地址}
\NormalTok{  I2C_Config(APB_I2C0,I2C_ROLE_SLAVE,I2C_ADDRESSING_MODE_07BIT,ADDRESS);}
  
\NormalTok{  I2C_Enable(APB_I2C0,}\DecValTok{1}\NormalTok{);}
  \CommentTok{// 打开传输结束中断和地址触发中断}
\NormalTok{  I2C_IntEnable(APB_I2C0,(}\DecValTok{1}\NormalTok{<<I2C_INT_ADDR_HIT)|(}\DecValTok{1}\NormalTok{<<I2C_INT_CMPL));}

\NormalTok{\}}
\end{Highlighting}
\end{Shaded}

\hypertarget{ux521dux59cbux5316dmaux6a21ux5757-3}{%
\subsubsection{初始化DMA模块}\label{ux521dux59cbux5316dmaux6a21ux5757-3}}

\begin{Shaded}
\begin{Highlighting}[]
\DataTypeTok{static} \DataTypeTok{void}\NormalTok{ setup_peripherals_dma_module(}\DataTypeTok{void}\NormalTok{)}
\NormalTok{\{}
\NormalTok{    SYSCTRL_ClearClkGateMulti(}\DecValTok{1}\NormalTok{ << SYSCTRL_ClkGate_APB_DMA);}
\NormalTok{    DMA_Reset(}\DecValTok{1}\NormalTok{);}
\NormalTok{    DMA_Reset(}\DecValTok{0}\NormalTok{);}
\NormalTok{\}}
\end{Highlighting}
\end{Shaded}

\hypertarget{i2cux4e2dux65adux5b9eux73b0ux4ee5ux53caux53d1ux9001ux6570ux636e-3}{%
\subsubsection{I2C中断实现以及发送数据}\label{i2cux4e2dux65adux5b9eux73b0ux4ee5ux53caux53d1ux9001ux6570ux636e-3}}

\begin{Shaded}
\begin{Highlighting}[]
\DataTypeTok{static} \DataTypeTok{uint32_t}\NormalTok{ peripherals_i2c_isr(}\DataTypeTok{void}\NormalTok{ *user_data)}
\NormalTok{\{}
  \DataTypeTok{uint8_t}\NormalTok{ i;}
  \DataTypeTok{static} \DataTypeTok{uint8_t}\NormalTok{ dir = }\DecValTok{2}\NormalTok{;}
  \DataTypeTok{uint32_t}\NormalTok{ status = I2C_GetIntState(APB_I2C0);}

  \CommentTok{// Slave收到了匹配的地址,检查传输方向}
  \ControlFlowTok{if}\NormalTok{(status & (}\DecValTok{1}\NormalTok{ << I2C_STATUS_ADDRHIT))}
\NormalTok{  \{}
\NormalTok{    dir = I2C_GetTransactionDir(APB_I2C0);}
    \ControlFlowTok{if}\NormalTok{(dir == I2C_TRANSACTION_MASTER2SLAVE)}
\NormalTok{    \{}
      \CommentTok{// 设置DMA读取数据}
\NormalTok{      peripherals_i2c_read_data_dma_setup();}
\NormalTok{    \}}
    \ControlFlowTok{else} \ControlFlowTok{if}\NormalTok{(dir == I2C_TRANSACTION_SLAVE2MASTER)}
\NormalTok{    \{}

\NormalTok{    \}}
\NormalTok{    I2C_ClearIntState(APB_I2C0, (}\DecValTok{1}\NormalTok{ << I2C_STATUS_ADDRHIT));}
\NormalTok{  \}}
  
  \CommentTok{// 传输结束,关闭DMA}
  \ControlFlowTok{if}\NormalTok{(status & (}\DecValTok{1}\NormalTok{ << I2C_STATUS_CMPL))}
\NormalTok{  \{}
\NormalTok{    I2C_DmaEnable(APB_I2C0,}\DecValTok{0}\NormalTok{);}
\NormalTok{    I2C_ClearIntState(APB_I2C0, (}\DecValTok{1}\NormalTok{ << I2C_STATUS_CMPL));}
\NormalTok{  \}}
  
  \ControlFlowTok{return} \DecValTok{0}\NormalTok{;}
\NormalTok{\}}
\end{Highlighting}
\end{Shaded}

\hypertarget{i2c-slave-ux53d1ux9001ux6570ux636edmaux8bbeux7f6e-1}{%
\subsubsection{I2C Slave 发送数据DMA设置}\label{i2c-slave-ux53d1ux9001ux6570ux636edmaux8bbeux7f6e-1}}

\begin{Shaded}
\begin{Highlighting}[]
\CommentTok{// 此处配置的是I2C0}
\DataTypeTok{void}\NormalTok{ peripherals_i2c_rxfifo_to_dma(}\DataTypeTok{int}\NormalTok{ channel_id, }\DataTypeTok{void}\NormalTok{ *dst, }\DataTypeTok{int}\NormalTok{ size)}
\NormalTok{\{}
\NormalTok{    DMA_Descriptor descriptor __attribute__((aligned (}\DecValTok{8}\NormalTok{)));}

\NormalTok{    descriptor.Next = (DMA_Descriptor *)}\DecValTok{0}\NormalTok{;}
\NormalTok{    DMA_PreparePeripheral2Mem(&descriptor,dst,SYSCTRL_DMA_I2C0,size,DMA_ADDRESS_INC,}\DecValTok{0}\NormalTok{);}

\NormalTok{    DMA_EnableChannel(channel_id, &descriptor);}
\NormalTok{\}}

\DataTypeTok{void}\NormalTok{ peripherals_i2c_read_data_dma_setup(}\DataTypeTok{void}\NormalTok{)}
\NormalTok{\{}
  \CommentTok{// 设置DMA}
  \PreprocessorTok{#define I2C_DMA_RX_CHANNEL   (0)}\CommentTok{//DMA channel 0}
\NormalTok{  peripherals_i2c_rxfifo_to_dma(I2C_DMA_RX_CHANNEL, read_data, }\KeywordTok{sizeof}\NormalTok{(read_data));}
  \CommentTok{// 更新需要传输的数据}
\NormalTok{  I2C_CtrlUpdateDataCnt(APB_I2C0, DATA_CNT);}
  \CommentTok{// 打开I2C DMA}
\NormalTok{  I2C_DmaEnable(APB_I2C0,}\DecValTok{1}\NormalTok{);}
\NormalTok{\}}
\end{Highlighting}
\end{Shaded}

\hypertarget{ux4f7fux7528ux6d41ux7a0b-7}{%
\subsubsection{使用流程}\label{ux4f7fux7528ux6d41ux7a0b-7}}

\begin{itemize}
\tightlist
\item
  设置GPIO,setup\_peripherals\_i2c\_pin()
\item
  初始化I2C,setup\_peripherals\_i2c\_module()
\item
  初始化DMA,setup\_peripherals\_dma\_module()
\item
  检查中断状态,在中断中设置DMA读取数据,I2C\_STATUS\_CMPL中断代表传输结束
\end{itemize}

\hypertarget{ux573aux666f5masterux5199slave-masterux8bfbslave}{%
\section{场景5:Master写Slave + Master读Slave}\label{ux573aux666f5masterux5199slave-masterux8bfbslave}}

其中I2C操作为,首先执行写操作然后再执行读操作,CPU操作读写,没有使用DMA。 配置之前需要决定使用的GPIO,请参考datasheet了解可以使用的GPIO

\begin{Shaded}
\begin{Highlighting}[]
\PreprocessorTok{#define I2C_SCL         GIO_GPIO_10}
\PreprocessorTok{#define I2C_SDA         GIO_GPIO_11}
\end{Highlighting}
\end{Shaded}

\hypertarget{i2c-masterux914dux7f6e-4}{%
\subsection{I2C Master配置}\label{i2c-masterux914dux7f6e-4}}

// 测试数据,每次传输8个字节(fifo深度是8字节),每个传输单元必须是1字节

\begin{Shaded}
\begin{Highlighting}[]
\PreprocessorTok{#define DATA_CNT (8)}
\DataTypeTok{uint8_t}\NormalTok{ write_data[DATA_CNT] = \{}\DecValTok{0}\NormalTok{,\};}
\DataTypeTok{uint8_t}\NormalTok{ write_data_cnt = }\DecValTok{0}\NormalTok{;}
\DataTypeTok{uint8_t}\NormalTok{ read_data[DATA_CNT] = \{}\DecValTok{0}\NormalTok{,\};}
\DataTypeTok{uint8_t}\NormalTok{ read_data_cnt = }\DecValTok{0}\NormalTok{;}
\end{Highlighting}
\end{Shaded}

\hypertarget{ux914dux7f6epin-8}{%
\subsubsection{配置Pin}\label{ux914dux7f6epin-8}}

IO的配置请参考场景1。

\hypertarget{ux521dux59cbux5316i2cux6a21ux5757-8}{%
\subsubsection{初始化I2C模块}\label{ux521dux59cbux5316i2cux6a21ux5757-8}}

\begin{Shaded}
\begin{Highlighting}[]
\PreprocessorTok{#define ADDRESS (0x71)}
\DataTypeTok{void}\NormalTok{ setup_peripherals_i2c_module(}\DataTypeTok{void}\NormalTok{)}
\NormalTok{\{}
  \CommentTok{// 配置为Master, 7bit地址}
\NormalTok{  I2C_Config(APB_I2C0,I2C_ROLE_MASTER,I2C_ADDRESSING_MODE_07BIT,ADDRESS);}
  \CommentTok{// 配置时钟,可选}
\NormalTok{  I2C_ConfigClkFrequency(APB_I2C0,I2C_CLOCKFREQUENY_STANDARD);}
  
\NormalTok{  I2C_Enable(APB_I2C0,}\DecValTok{1}\NormalTok{);}
  \CommentTok{// 打开传输结束中断和addr hit中断}
  \CommentTok{// 对于master来说,如果有slave响应了地址,则会有I2C_INT_ADDR_HIT中断}
\NormalTok{  I2C_IntEnable(APB_I2C0,(}\DecValTok{1}\NormalTok{<<I2C_INT_CMPL)|(}\DecValTok{1}\NormalTok{<<I2C_INT_ADDR_HIT));}

\NormalTok{\}}
\end{Highlighting}
\end{Shaded}

\hypertarget{i2cux4e2dux65adux5b9eux73b0-4}{%
\subsubsection{I2C中断实现}\label{i2cux4e2dux65adux5b9eux73b0-4}}

首先定义一个变量,如果为1,代表是master的写操作,否则为master的读操作。

\begin{Shaded}
\begin{Highlighting}[]
\DataTypeTok{uint8_t}\NormalTok{ master_write_flag = }\DecValTok{0}\NormalTok{;}
\end{Highlighting}
\end{Shaded}

\begin{Shaded}
\begin{Highlighting}[]
\DataTypeTok{static} \DataTypeTok{uint32_t}\NormalTok{ peripherals_i2c_isr(}\DataTypeTok{void}\NormalTok{ *user_data)}
\NormalTok{\{}
  \DataTypeTok{uint8_t}\NormalTok{ i;}
  \DataTypeTok{uint32_t}\NormalTok{ status = I2C_GetIntState(APB_I2C0);}
  
  \CommentTok{// 传输结束中断,代表DATA_CNT个字节接收或者发射完成}
  \ControlFlowTok{if}\NormalTok{(status & (}\DecValTok{1}\NormalTok{ << I2C_STATUS_CMPL))}
\NormalTok{  \{}
    \CommentTok{// master写DATA_CNT个字节成功}
    \ControlFlowTok{if}\NormalTok{(master_write_flag)}
\NormalTok{    \{}
\NormalTok{      platform_printf(}\StringTok{"wr cmp %d}\SpecialCharTok{\textbackslash{}n}\StringTok{"}\NormalTok{, I2C_CtrlGetDataCnt(APB_I2C0));}
\NormalTok{      I2C_ClearIntState(APB_I2C0, (}\DecValTok{1}\NormalTok{ << I2C_STATUS_CMPL));}
\NormalTok{    \}}
    \ControlFlowTok{else}\CommentTok{// master读DATA_CNT个字节成功}
\NormalTok{    \{}
      \CommentTok{// 将剩余的fifo中的数据读出来}
      \ControlFlowTok{for}\NormalTok{(; read_data_cnt < DATA_CNT; read_data_cnt++)}
\NormalTok{      \{}
\NormalTok{        read_data[read_data_cnt] = I2C_DataRead(APB_I2C0);}
\NormalTok{      \}}

\NormalTok{      I2C_ClearIntState(APB_I2C0, (}\DecValTok{1}\NormalTok{ << I2C_STATUS_CMPL));}
      
      \CommentTok{// debug trace}
\NormalTok{      platform_printf(}\StringTok{"rd cmp %d "}\NormalTok{,read_data_cnt);}
      \ControlFlowTok{for}\NormalTok{(i=}\DecValTok{0}\NormalTok{;i<DATA_CNT;i++)\{platform_printf(}\StringTok{" 0x%x -"}\NormalTok{, read_data[i]);\};}
\NormalTok{      printf(}\StringTok{"}\SpecialCharTok{\textbackslash{}n}\StringTok{"}\NormalTok{);}
      
      \CommentTok{// prepare for next}
\NormalTok{      read_data_cnt = }\DecValTok{0}\NormalTok{;}
\NormalTok{    \}}
\NormalTok{  \}}
  
  \CommentTok{// 有slave响应了master的地址}
  \ControlFlowTok{if}\NormalTok{(status & (}\DecValTok{1}\NormalTok{ << I2C_STATUS_ADDRHIT))}
\NormalTok{  \{}
    \CommentTok{// 打开相应中断,主要是用来填写或者读取fifo}
    \ControlFlowTok{if}\NormalTok{(master_write_flag)}
\NormalTok{    \{}
\NormalTok{      I2C_IntDisable(APB_I2C0,(}\DecValTok{1}\NormalTok{ << I2C_INT_FIFO_FULL));}
\NormalTok{      I2C_IntEnable(APB_I2C0,(}\DecValTok{1}\NormalTok{<<I2C_INT_CMPL)|(}\DecValTok{1}\NormalTok{ << I2C_INT_FIFO_EMPTY));}
\NormalTok{    \}}
    \ControlFlowTok{else}
\NormalTok{    \{}
\NormalTok{      I2C_IntDisable(APB_I2C0,(}\DecValTok{1}\NormalTok{ << I2C_INT_FIFO_EMPTY));}
\NormalTok{      I2C_IntEnable(APB_I2C0,(}\DecValTok{1}\NormalTok{<<I2C_INT_CMPL)|(}\DecValTok{1}\NormalTok{ << I2C_INT_FIFO_FULL));}
\NormalTok{    \}}
\NormalTok{    I2C_ClearIntState(APB_I2C0, (}\DecValTok{1}\NormalTok{ << I2C_STATUS_ADDRHIT));}
\NormalTok{    platform_printf(}\StringTok{"addr hit}\SpecialCharTok{\textbackslash{}n}\StringTok{"}\NormalTok{);}
\NormalTok{  \}}
  
  \CommentTok{// 该中断在master_write_flag==1打开,代表fifo为空,需要填充待发送的数据}
  \ControlFlowTok{if}\NormalTok{(status & (}\DecValTok{1}\NormalTok{ << I2C_STATUS_FIFO_EMPTY))}
\NormalTok{  \{}
    \ControlFlowTok{if}\NormalTok{(master_write_flag)}
\NormalTok{    \{}
      \CommentTok{// push data until fifo is full}
      \ControlFlowTok{for}\NormalTok{(; write_data_cnt < DATA_CNT; write_data_cnt++)}
\NormalTok{      \{}
        \CommentTok{//platform_printf("ept: %d \textbackslash{}n",write_data_cnt);}
        \ControlFlowTok{if}\NormalTok{(I2C_FifoFull(APB_I2C0))\{}\ControlFlowTok{break}\NormalTok{;\}}
\NormalTok{        I2C_DataWrite(APB_I2C0,write_data[write_data_cnt]);}
\NormalTok{      \}}
      
      \CommentTok{// if its the last, disable empty int}
      \ControlFlowTok{if}\NormalTok{(write_data_cnt == DATA_CNT)}
\NormalTok{      \{}
\NormalTok{        I2C_IntDisable(APB_I2C0,(}\DecValTok{1}\NormalTok{ << I2C_INT_FIFO_EMPTY));}
\NormalTok{      \}}
      
\NormalTok{    \}}
\NormalTok{  \}}
  
  \CommentTok{// 该中断在master_write_flag==0打开,FIFO 满之后,触发中断,此时需要将所有数据都读出来}
  \ControlFlowTok{if}\NormalTok{(status & (}\DecValTok{1}\NormalTok{ << I2C_STATUS_FIFO_FULL))}
\NormalTok{  \{}
    \ControlFlowTok{if}\NormalTok{(!master_write_flag)}
\NormalTok{    \{}
      \ControlFlowTok{for}\NormalTok{(; read_data_cnt < DATA_CNT; read_data_cnt++)}
\NormalTok{      \{}
        \ControlFlowTok{if}\NormalTok{(I2C_FifoEmpty(APB_I2C0))\{}\ControlFlowTok{break}\NormalTok{;\}}
\NormalTok{        read_data[read_data_cnt] = I2C_DataRead(APB_I2C0);}
\NormalTok{      \}}
\NormalTok{      platform_printf(}\StringTok{"rd full %d }\SpecialCharTok{\textbackslash{}n}\StringTok{"}\NormalTok{, read_data_cnt);}
\NormalTok{    \}}
\NormalTok{  \}}
  
  \ControlFlowTok{return} \DecValTok{0}\NormalTok{;}
\NormalTok{\}}
\end{Highlighting}
\end{Shaded}

\hypertarget{i2c-masterux5199ux4f20ux8f93}{%
\subsubsection{I2C master写传输}\label{i2c-masterux5199ux4f20ux8f93}}

\begin{Shaded}
\begin{Highlighting}[]
\DataTypeTok{void}\NormalTok{ peripheral_i2c_write_data(}\DataTypeTok{void}\NormalTok{)}
\NormalTok{\{}
\NormalTok{  master_write_flag = }\DecValTok{1}\NormalTok{;}
  \CommentTok{// 设置方向,Master写}
\NormalTok{  I2C_CtrlUpdateDirection(APB_I2C0,I2C_TRANSACTION_MASTER2SLAVE);}
  \CommentTok{// 设置每次传输的大小}
\NormalTok{  I2C_CtrlUpdateDataCnt(APB_I2C0, DATA_CNT);}
  \CommentTok{// 触发传输}
\NormalTok{  I2C_CommandWrite(APB_I2C0, I2C_COMMAND_ISSUE_DATA_TRANSACTION);}
  
\NormalTok{\}}
\end{Highlighting}
\end{Shaded}

\hypertarget{i2c-masterux8bfbux4f20ux8f93}{%
\subsubsection{I2C master读传输}\label{i2c-masterux8bfbux4f20ux8f93}}

\begin{Shaded}
\begin{Highlighting}[]
\DataTypeTok{void}\NormalTok{ peripheral_i2c_read_data(}\DataTypeTok{void}\NormalTok{)}
\NormalTok{\{}
\NormalTok{  master_write_flag = }\DecValTok{0}\NormalTok{;}
\NormalTok{  I2C_CtrlUpdateDirection(APB_I2C0,I2C_TRANSACTION_SLAVE2MASTER);}
\NormalTok{  I2C_CtrlUpdateDataCnt(APB_I2C0, DATA_CNT);}
\NormalTok{  I2C_CommandWrite(APB_I2C0, I2C_COMMAND_ISSUE_DATA_TRANSACTION);}
\NormalTok{\}}
\end{Highlighting}
\end{Shaded}

\hypertarget{ux4f7fux7528ux6d41ux7a0b-8}{%
\subsubsection{使用流程}\label{ux4f7fux7528ux6d41ux7a0b-8}}

\begin{itemize}
\tightlist
\item
  设置GPIO,setup\_peripherals\_i2c\_pin()
\item
  初始化I2C,setup\_peripherals\_i2c\_module()
\item
  在需要时候触发I2C写数据,peripheral\_i2c\_write\_data(),I2C\_STATUS\_CMPL代表结束
\item
  当写结束后,可以触发I2C读取,peripheral\_i2c\_read\_data(),I2C\_STATUS\_CMPL代表结束
\item
  检查中断状态
\end{itemize}

\hypertarget{i2c-slaveux914dux7f6e-4}{%
\subsection{I2C Slave配置}\label{i2c-slaveux914dux7f6e-4}}

\hypertarget{ux914dux7f6epin-9}{%
\subsubsection{配置Pin}\label{ux914dux7f6epin-9}}

IO的配置请参考场景1。

\hypertarget{ux521dux59cbux5316i2cux6a21ux5757-9}{%
\subsubsection{初始化I2C模块}\label{ux521dux59cbux5316i2cux6a21ux5757-9}}

\begin{Shaded}
\begin{Highlighting}[]
\PreprocessorTok{#define ADDRESS (0x71)}
\DataTypeTok{void}\NormalTok{ setup_peripherals_i2c_module(}\DataTypeTok{void}\NormalTok{)}
\NormalTok{\{}
  \CommentTok{// 配置为Slave, 7bit地址}
\NormalTok{  I2C_Config(APB_I2C0,I2C_ROLE_SLAVE,I2C_ADDRESSING_MODE_07BIT,ADDRESS);}
  
\NormalTok{  I2C_Enable(APB_I2C0,}\DecValTok{1}\NormalTok{);}
  \CommentTok{// 打开传输结束中断和地址触发中断}
\NormalTok{  I2C_IntEnable(APB_I2C0,(}\DecValTok{1}\NormalTok{<<I2C_INT_ADDR_HIT)|(}\DecValTok{1}\NormalTok{<<I2C_INT_CMPL));}

\NormalTok{\}}
\end{Highlighting}
\end{Shaded}

\hypertarget{i2cux4e2dux65adux5b9eux73b0ux4ee5ux53caux53d1ux9001ux6570ux636e-4}{%
\subsubsection{I2C中断实现以及发送数据}\label{i2cux4e2dux65adux5b9eux73b0ux4ee5ux53caux53d1ux9001ux6570ux636e-4}}

\begin{Shaded}
\begin{Highlighting}[]
\DataTypeTok{static} \DataTypeTok{uint32_t}\NormalTok{ peripherals_i2c_isr(}\DataTypeTok{void}\NormalTok{ *user_data)}
\NormalTok{\{}
  \DataTypeTok{uint8_t}\NormalTok{ i;}
  \DataTypeTok{static} \DataTypeTok{uint8_t}\NormalTok{ dir = }\DecValTok{2}\NormalTok{;}
  \DataTypeTok{uint32_t}\NormalTok{ status = I2C_GetIntState(APB_I2C0);}

  \CommentTok{// Slave 收到匹配的地址,触发中断}
  \ControlFlowTok{if}\NormalTok{(status & (}\DecValTok{1}\NormalTok{ << I2C_STATUS_ADDRHIT))}
\NormalTok{  \{}
    \CommentTok{// 判断是读操作还是写操作}
\NormalTok{    dir = I2C_GetTransactionDir(APB_I2C0);}
    \ControlFlowTok{if}\NormalTok{(dir == I2C_TRANSACTION_MASTER2SLAVE)}
\NormalTok{    \{}
\NormalTok{        master_write_flag = }\DecValTok{1}\NormalTok{;}
\NormalTok{        I2C_IntEnable(APB_I2C0,(}\DecValTok{1}\NormalTok{ << I2C_INT_FIFO_FULL));}
\NormalTok{        platform_printf(}\StringTok{"addr wr 0x%x}\SpecialCharTok{\textbackslash{}n}\StringTok{"}\NormalTok{, APB_I2C0->IntEn);}
\NormalTok{    \}}
    \ControlFlowTok{else} \ControlFlowTok{if}\NormalTok{(dir == I2C_TRANSACTION_SLAVE2MASTER)}
\NormalTok{    \{}
\NormalTok{        master_write_flag = }\DecValTok{0}\NormalTok{;}
\NormalTok{        I2C_IntEnable(APB_I2C0,(}\DecValTok{1}\NormalTok{ << I2C_INT_FIFO_EMPTY));}
        
        \CommentTok{//示例: slave 此时应将要发送的数据准备好}
\NormalTok{        write_data_cnt = }\DecValTok{0}\NormalTok{;}
\NormalTok{        memset(write_data, }\BaseNTok{0x44}\NormalTok{, }\KeywordTok{sizeof}\NormalTok{(write_data));}
\NormalTok{        platform_printf(}\StringTok{"addr rd }\SpecialCharTok{\textbackslash{}n}\StringTok{"}\NormalTok{);}
\NormalTok{    \}}

\NormalTok{    I2C_ClearIntState(APB_I2C0, (}\DecValTok{1}\NormalTok{ << I2C_STATUS_ADDRHIT));}

\NormalTok{  \}}
  
  \CommentTok{// 如果是读操作,则会触发empty中断,此时需要填写需要发送的数据,直到FIFO满}
  \ControlFlowTok{if}\NormalTok{(status & (}\DecValTok{1}\NormalTok{ << I2C_STATUS_FIFO_EMPTY))}
\NormalTok{  \{}
    \CommentTok{// master read}
    \ControlFlowTok{if}\NormalTok{(!master_write_flag)}
\NormalTok{    \{}
      \CommentTok{// push data until fifo is full}
      \ControlFlowTok{for}\NormalTok{(; write_data_cnt < DATA_CNT; write_data_cnt++)}
\NormalTok{      \{}
        \ControlFlowTok{if}\NormalTok{(I2C_FifoFull(APB_I2C0))\{}\ControlFlowTok{break}\NormalTok{;\}}
\NormalTok{        I2C_DataWrite(APB_I2C0,write_data[write_data_cnt]);}
\NormalTok{      \}}
\NormalTok{      platform_printf(}\StringTok{"rd empty %d }\SpecialCharTok{\textbackslash{}n}\StringTok{"}\NormalTok{, write_data_cnt);}
\NormalTok{    \}}
\NormalTok{  \}}
  
  \CommentTok{// 如果是写操作,FIFO满之后会触发该中断,slave应读取fifo数据}
  \ControlFlowTok{if}\NormalTok{(status & (}\DecValTok{1}\NormalTok{ << I2C_STATUS_FIFO_FULL))}
\NormalTok{  \{}
    \CommentTok{// master write}
    \ControlFlowTok{if}\NormalTok{(master_write_flag)}
\NormalTok{    \{}
      \ControlFlowTok{for}\NormalTok{(; read_data_cnt < DATA_CNT; read_data_cnt++)}
\NormalTok{      \{}
        \ControlFlowTok{if}\NormalTok{(I2C_FifoEmpty(APB_I2C0))\{}\ControlFlowTok{break}\NormalTok{;\}}
\NormalTok{        read_data[read_data_cnt] = I2C_DataRead(APB_I2C0);}
\NormalTok{      \}}
\NormalTok{      platform_printf(}\StringTok{"wr full %d }\SpecialCharTok{\textbackslash{}n}\StringTok{"}\NormalTok{, read_data_cnt);}
\NormalTok{    \}}
\NormalTok{  \}}
  
  \CommentTok{// 传输结束,清理打开的中断}
  \ControlFlowTok{if}\NormalTok{(status & (}\DecValTok{1}\NormalTok{ << I2C_STATUS_CMPL))}
\NormalTok{  \{}
    \CommentTok{// master write, 此时应读取fifo中剩余的数据}
    \ControlFlowTok{if}\NormalTok{(master_write_flag)}
\NormalTok{    \{}
      \ControlFlowTok{for}\NormalTok{(;read_data_cnt < DATA_CNT; read_data_cnt++)}
\NormalTok{      \{}
\NormalTok{        read_data[read_data_cnt] = I2C_DataRead(APB_I2C0);}
\NormalTok{      \}}
      
\NormalTok{      I2C_ClearIntState(APB_I2C0, (}\DecValTok{1}\NormalTok{ << I2C_STATUS_CMPL));}
      
      \CommentTok{// prepare for next}
      \ControlFlowTok{if}\NormalTok{(dir == I2C_TRANSACTION_MASTER2SLAVE)}
\NormalTok{      \{}
\NormalTok{          I2C_IntDisable(APB_I2C0,(}\DecValTok{1}\NormalTok{ << I2C_INT_FIFO_FULL));}
\NormalTok{      \}}
      \ControlFlowTok{else} \ControlFlowTok{if}\NormalTok{(dir == I2C_TRANSACTION_SLAVE2MASTER)}
\NormalTok{      \{}
\NormalTok{          I2C_IntDisable(APB_I2C0,(}\DecValTok{1}\NormalTok{ << I2C_INT_FIFO_EMPTY));}
\NormalTok{      \}}
      
      \CommentTok{// debug trace}
\NormalTok{      platform_printf(}\StringTok{"wr cmp %d "}\NormalTok{,read_data_cnt);}
      \ControlFlowTok{for}\NormalTok{(i=}\DecValTok{0}\NormalTok{;i<DATA_CNT;i++)\{platform_printf(}\StringTok{" 0x%x -"}\NormalTok{, read_data[i]);\};}
\NormalTok{      printf(}\StringTok{"}\SpecialCharTok{\textbackslash{}n}\StringTok{"}\NormalTok{);}
      
\NormalTok{      read_data_cnt = }\DecValTok{0}\NormalTok{;}
\NormalTok{    \}}
    \ControlFlowTok{else}\CommentTok{//master写操作,清理中断}
\NormalTok{    \{}
\NormalTok{      I2C_ClearIntState(APB_I2C0, (}\DecValTok{1}\NormalTok{ << I2C_STATUS_CMPL));}
      
      \CommentTok{// prepare for next}
      \ControlFlowTok{if}\NormalTok{(dir == I2C_TRANSACTION_MASTER2SLAVE)}
\NormalTok{      \{}
\NormalTok{          I2C_IntDisable(APB_I2C0,(}\DecValTok{1}\NormalTok{ << I2C_INT_FIFO_FULL));}
\NormalTok{      \}}
      \ControlFlowTok{else} \ControlFlowTok{if}\NormalTok{(dir == I2C_TRANSACTION_SLAVE2MASTER)}
\NormalTok{      \{}
\NormalTok{          I2C_IntDisable(APB_I2C0,(}\DecValTok{1}\NormalTok{ << I2C_INT_FIFO_EMPTY));}
\NormalTok{      \}}
      
\NormalTok{      write_data_cnt = }\DecValTok{0}\NormalTok{;}
\NormalTok{      platform_printf(}\StringTok{"rd cmp }\SpecialCharTok{\textbackslash{}n}\StringTok{ "}\NormalTok{);}
\NormalTok{    \}}

\NormalTok{  \}}

  \ControlFlowTok{return} \DecValTok{0}\NormalTok{;}
\NormalTok{\}}
\end{Highlighting}
\end{Shaded}

\hypertarget{ux4f7fux7528ux6d41ux7a0b-9}{%
\subsubsection{使用流程}\label{ux4f7fux7528ux6d41ux7a0b-9}}

\begin{itemize}
\tightlist
\item
  设置GPIO,setup\_peripherals\_i2c\_pin()
\item
  初始化I2C,setup\_peripherals\_i2c\_module()
\item
  检查中断状态,在中断中发送数据,I2C\_STATUS\_CMPL中断代表传输结束
\item
  如果是读操作,slave应该在master\_write\_flag=0之后准备好数据写到fifo
\end{itemize}

\hypertarget{i2c-ux65f6ux949fux914dux7f6e}{%
\section{I2C 时钟配置}\label{i2c-ux65f6ux949fux914dux7f6e}}

I2C时钟配置使用API:

\begin{Shaded}
\begin{Highlighting}[]
\CommentTok{/**}
\CommentTok{ * }\AnnotationTok{@brief}\CommentTok{ Set clk frequency for controller.}
\CommentTok{ * }\AnnotationTok{@param[in]}\CommentTok{ }\CommentVarTok{I2C_BASE}\CommentTok{              base address}
\CommentTok{ * }\AnnotationTok{@param[in]}\CommentTok{ }\CommentVarTok{option}\CommentTok{                see I2C_ClockFrequenyOptions}
\CommentTok{ */}
\DataTypeTok{void}\NormalTok{ I2C_ConfigClkFrequency(I2C_TypeDef *I2C_BASE, I2C_ClockFrequenyOptions option);}
\end{Highlighting}
\end{Shaded}

其中option中定义了几个可选项

\begin{Shaded}
\begin{Highlighting}[]
\KeywordTok{typedef} \KeywordTok{enum}
\NormalTok{\{}
\NormalTok{    I2C_CLOCKFREQUENY_NULL,}
\NormalTok{    I2C_CLOCKFREQUENY_STANDARD,}\CommentTok{//up to 100kbit/s}
\NormalTok{    I2C_CLOCKFREQUENY_FASTMODE,}\CommentTok{//up to 400kbit/s}
\NormalTok{    I2C_CLOCKFREQUENY_FASTMODE_PLUS,}\CommentTok{//up to 1Mbit/s}
\NormalTok{    I2C_CLOCKFREQUENY_MANUAL}
\NormalTok{\} I2C_ClockFrequenyOptions;}
\end{Highlighting}
\end{Shaded}

如果选择MANUAL,需要手动配置相关寄存器来生成需要的时钟

\begin{Shaded}
\begin{Highlighting}[]
\NormalTok{I2C_BASE->TPM : multiply factor, width }\DecValTok{5}\ErrorTok{bit}\NormalTok{, 所有setup中的时间参数都会被乘以(TPM+}\DecValTok{1}\NormalTok{)}
\NormalTok{I2C_BASE->Setup: 使用I2C_ConfigSCLTiming()配置该寄存器}
\NormalTok{  scl_hi:高电平持续时间,width }\DecValTok{9}\ErrorTok{bit}\NormalTok{,default }\BaseNTok{0x10} 
\NormalTok{  scl_ratio: 低电平持续时间因子,width }\DecValTok{1}\ErrorTok{bit}\NormalTok{,default }\DecValTok{1}
\NormalTok{  hddat:SCL拉低后SDA的保持时间,width }\DecValTok{5}\ErrorTok{bit}\NormalTok{,default }\DecValTok{5}
\NormalTok{  sp: 可以被过滤的脉冲毛刺宽度,width }\DecValTok{3}\ErrorTok{bit}\NormalTok{,default }\DecValTok{1}
\NormalTok{  sudat: 释放SCL之前的数据建立时间,width }\DecValTok{5}\ErrorTok{bit}\NormalTok{,default }\DecValTok{5}
\end{Highlighting}
\end{Shaded}

\begin{itemize}
\tightlist
\item
  高电平持续时间计算: \texttt{SCL\ high\ period\ =\ (2\ *\ pclk)\ +\ (2\ +\ sp\ +\ scl\_hi)\ *\ pclk\ *\ (TPM\ +\ 1)} 其中pclk为I2C模块的系统时钟,默认为24M
\end{itemize}

如果\texttt{sp\ =\ 1,\ pclk\ =\ 42ns,\ TPM\ =\ 3,\ scl\_hi\ =\ 150} :

\texttt{SCL\ high\ period\ =\ (2\ *\ 42)\ +\ (2\ +\ 1\ +\ 150)\ *\ 42\ *\ (3\ +\ 1)\ =\ 25788ns}

\begin{itemize}
\tightlist
\item
  低电平持续时间计算: \texttt{SCL\ low\ period\ =\ (2\ *\ pclk)\ +\ (2\ +\ sp\ +\ scl\_hi\ *\ (scl\_ratio+1))\ *\ pclk\ *\ (TPM\ +\ 1)}
\end{itemize}

如果\texttt{sp\ =\ 1,\ pclk\ =\ 42ns,\ TPM\ =\ 3,\ scl\_hi\ =\ 150,\ scl\_ratio\ =\ 0} :

\texttt{SCL\ low\ period\ =\ (2\ *\ 42)\ +\ (2\ +\ 1\ +\ 150\ *\ 1)\ *\ 42\ *\ (3\ +\ 1)\ =\ 25788ns}

\begin{itemize}
\tightlist
\item
  毛刺抑制宽度: \texttt{spike\ suppression\ width\ =\ sp\ *\ pclk\ *\ (TPM\ +\ 1)}
\end{itemize}

如果\texttt{sp\ =\ 1,\ pclk\ =\ 42ns,\ TPM\ =\ 3}:

\texttt{spike\ suppression\ width\ =\ 1\ *\ 42\ *\ (3\ +\ 1)\ =\ 168ns}

\begin{itemize}
\tightlist
\item
  SCL之前的数据建立时间:\texttt{setup\ time\ =\ (2\ *\ pclk)\ +\ (2\ +\ sp\ +\ sudat)\ *\ pclk\ *\ (TPM\ +\ 1)}
\end{itemize}

如果\texttt{sp\ =\ 1,\ pclk\ =\ 42ns,\ TPM\ =\ 3,\ sudat\ =\ 5}:

\texttt{setup\ time\ =\ (2\ *\ 42)\ +\ (2\ +\ 1\ +\ 5)\ *\ 42\ *\ (3\ +\ 1)\ =\ 1428ns}

协议对SCL之前的数据建立时间要求为:

\texttt{standard\ mode:\ 最小250ns\ -\ fast\ mode:\ 最小100ns\ -\ fast\ mode\ plus:\ 最小50ns}

\begin{itemize}
\tightlist
\item
  SCL拉低后SDA的保持时间 \texttt{hold\ time\ =\ (2\ *\ pclk)\ +\ (2\ +\ sp\ +\ hddat)\ *\ pclk\ *\ (TPM\ +\ 1)}
\end{itemize}

如果\texttt{sp\ =\ 1,\ pclk\ =\ 42ns,\ TPM\ =\ 3,\ hddat\ =\ 5}: \texttt{hold\ time\ =\ (2\ *\ 42)\ +\ (2\ +\ 1\ +\ 5)\ *\ 42\ *\ (3\ +\ 1)\ =\ 1428ns}

协议对SCL拉低后SDA的保持时间要求为: \texttt{standard\ mode:\ 最小300ns\ -\ fast\ mode:\ 最小300ns\ -\ fast\ mode\ plus:\ 最小0ns}

\hypertarget{ch-I2s}{%
\chapter{I2s简介}\label{ch-I2s}}

I2S(inter-IC sound)总线是数字音频专用总线。它有四个引脚,两个数据引脚(DOUT和DIN),一个位率时钟引脚(BCLK)和一个左右通道选择引脚(LRCLK)。

另外,通过ING91682A的MCLK输出,它可用于给外部 DAC/ADC 芯片提供时钟(可选)。

\hypertarget{ux529fux80fdux63cfux8ff0-2}{%
\section{功能描述}\label{ux529fux80fdux63cfux8ff0-2}}

\hypertarget{ux7279ux70b9-2}{%
\subsection{特点}\label{ux7279ux70b9-2}}

\begin{itemize}
\item
  遵从 I2S 协议标准,支持12S标准模式和左对齐模式
\item
  支持PCM(脉冲编码调制)时序
\item
  可编程的主从模式
\item
  可配置的LRCLK和BCLK极性
\item
  可配置数据位宽
\item
  独立发送和接收FIFO
\item
  TX和RX的FIFO深度分别为16*32bit
\item
  支持立体声和单声道模式
\item
  可配置的采样频率
\item
  TX和RX分别支持DMA搬运
\end{itemize}

\hypertarget{i2sux89d2ux8272}{%
\subsection{I2s角色}\label{i2sux89d2ux8272}}

在 I2S 总线上,提供时钟和通道选择信号的器件是 MASTER,另一方则为 SLAVER。

MASTER和SLAVE都可以进行数据收发。

\hypertarget{i2sux5de5ux4f5cux6a21ux5f0f}{%
\subsection{I2s工作模式}\label{i2sux5de5ux4f5cux6a21ux5f0f}}

I2S 有两种工作模式:一种是立体声音频模式,另外一种是话音模式。

\hypertarget{ux4e32ux884cux6570ux636e}{%
\subsection{串行数据}\label{ux4e32ux884cux6570ux636e}}

串行数据是以高位(MSB)在前,低位(LSB)在后的方式进行传送的。

如果音频codec发送的位数多于I2S控制器的接收位数,I2S控制器会将低位多余的位数忽略掉;

如果音频codec发送的位数小于I2S控制器接收位数,I2S控制器将后面的位补零。

\hypertarget{ux65f6ux949fux5206ux9891}{%
\subsection{时钟分频}\label{ux65f6ux949fux5206ux9891}}

916芯片可选用系统24MHz时钟或者PLL作为I2s时钟源。

位率时钟(BCLK)可以通过对功能时钟进行分频得到;

通道选择时钟(LRCLK)即音频数据的采样频率可以通过对BCLK进行分频得到。

音频 Codec 中对采样频率 LRCLK 要求精度比较高,我们在计算分频时应该首先根据不同的采样频率计算得到对应的 MCLK 和 BCLK。

\hypertarget{ux65f6ux949fux5206ux9891ux8ba1ux7b97}{%
\subsubsection{时钟分频计算}\label{ux65f6ux949fux5206ux9891ux8ba1ux7b97}}

计算示例:

假设当前codec采用16K采样频率,mic要求一帧64位(参考具体的mic使用手册)。

有以下关系:

\begin{itemize}
\item
  f\_bclk = clk/(2*b\_div)
\item
  f\_lrclk = f\_bclk/(2*lr\_div)
\end{itemize}

其中clk为codec时钟,f\_bclk、f\_lrclk分别为BCLK和LRCLK,b\_div、lr\_div分别为BCLK和LRCLK的分频系数。

BCLK和LRCLK之间的关系是可变的,但是BCLK必须大于等于LRCLK的48倍。即lr\_div\textgreater=24。

支持lr\_div = 32,DATA\_LEN = 32位的配置,其他情况下lr\_div - DATA\_LEN \textgreater{} 3。

通过f\_lrclk = 16000,lr\_div = 32计算出f\_bclk = 1.024MHz。也就是clk = 2.048*b\_div。

clk通过时钟源分频得到必定是整数,b\_div也同样是整数,通过计算得知在384MHz内只有当b\_div = 125时clk = 256MHz为整数。

故需要将PLL时钟配置为256MHz,b\_div = 125可以得到16K采样率。

\hypertarget{i2sux5b58ux50a8ux5668}{%
\subsection{i2s存储器}\label{i2sux5b58ux50a8ux5668}}

采用两个深度为16,宽度为32bit的FIFO分别存储接收、发送的音频数据。

有如下规则:

\begin{itemize}
\item
  音频数据位宽为16bit时,每32bit存储两个音频数据,高16bit存储左声道数据,低16bit存储右声道数据。
\item
  音频数据位宽大于16bit时,每32bit存储一个音频数据,低地址存储左声道数据,高地址存储右声道数据。
\end{itemize}

\hypertarget{ux4f7fux7528ux65b9ux6cd5-2}{%
\section{使用方法}\label{ux4f7fux7528ux65b9ux6cd5-2}}

\hypertarget{ux65b9ux6cd5ux6982ux8ff0-2}{%
\subsection{方法概述}\label{ux65b9ux6cd5ux6982ux8ff0-2}}

I2s使用方法总结为:时钟配置,I2s配置(包括采样率)和数据处理。

数据发送:

\textbf{1.} I2s引脚GPIO配置

\textbf{2.} 配置外部 codec 芯片,使其处于工作模式

\textbf{3.} 写相应配置寄存器

\textbf{4.} 将数据写入TX\_MEM

\textbf{5.} 使能I2s

\textbf{6.} 等待中断产生

\textbf{7.} 读取状态寄存器,将数据写入TX\_MEM

\textbf{8.} 传输完毕,关闭 I2S

数据接收:

\textbf{1.} I2s引脚GPIO配置

\textbf{2.} 配置外部 codec 芯片,使其处于工作模式

\textbf{3.} 写相应配置寄存器

\textbf{4.} 使能I2s

\textbf{5.} 等待中断产生

\textbf{6.} 读取状态寄存器,读取 RX\_MEM 中数据

\textbf{7.} 传输完毕,关闭 I2S

I2S控制器操作流程图如下:

\begin{figure}

{\centering \includegraphics[width=0.5\linewidth]{./img/I2s/I2s_structure} 

}

\caption{I2S控制器操作流程图}\label{fig:I2S}
\end{figure}

如果需要用到DMA搬运则需要在使能I2s之前配置DMA并使能。

\hypertarget{ux6ce8ux610fux70b9-2}{%
\subsection{注意点}\label{ux6ce8ux610fux70b9-2}}

\begin{itemize}
\item
  I2s时钟源可以选择晶振24M时钟和PLL时钟,要注意是选择哪一个时钟源
\item
  I2s数据可能会进行采样,需要注意具体的数据结构以及对应的数据处理,如是否需要数据移位等
\item
  当前I2s支持的发送/接收数据位宽为16-32bit,需要查阅mic文档或其他使用手册来确定数据位宽,否则不能正常工作
\item
  配置DMA要在使能I2s之前完成,使能I2s一定是最后一步
\item
  建议采用DMA乒乓搬运的方式来传输I2s数据
\end{itemize}

\hypertarget{ux7f16ux7a0bux6307ux5357-2}{%
\section{编程指南}\label{ux7f16ux7a0bux6307ux5357-2}}

\hypertarget{ux9a71ux52a8ux63a5ux53e3-2}{%
\subsection{驱动接口}\label{ux9a71ux52a8ux63a5ux53e3-2}}

\begin{itemize}
\item
  I2S\_ConfigClk:I2s时钟配置接口
\item
  I2S\_Config:I2s配置接口
\item
  I2S\_ConfigIRQ:I2s中断配置接口
\item
  I2S\_DMAEnable:I2s DMA使能接口
\item
  I2S\_Enable:I2s使能接口
\item
  I2S\_PopRxFIFO、I2S\_PushTxFIFO:I2s FIFO读写接口
\item
  I2S\_ClearRxFIFO、I2S\_ClearTxFIFO:I2s清FIFO接口
\item
  I2S\_GetIntState、I2S\_ClearIntState:I2s获取中断、清中断接口
\item
  I2S\_GetRxFIFOCount、I2S\_GetTxFIFOCount:I2s获取FIFO数据数量接口
\item
  I2S\_DataFromPDM:I2s获取PDM数据接口
\end{itemize}

\hypertarget{ux4ee3ux7801ux793aux4f8b-2}{%
\subsection{代码示例}\label{ux4ee3ux7801ux793aux4f8b-2}}

下面将通过实际代码展示I2s的基本配置及使用代码。

\hypertarget{i2sux914dux7f6e}{%
\subsubsection{I2s配置}\label{i2sux914dux7f6e}}

\begin{Shaded}
\begin{Highlighting}[]
\PreprocessorTok{#define I2S_PIN_BCLK        28}
\PreprocessorTok{#define I2S_PIN_IN          29}
\PreprocessorTok{#define I2S_PIN_LRCLK       30}
\DataTypeTok{void}\NormalTok{ I2sSetup(}\DataTypeTok{void}\NormalTok{)}
\NormalTok{\{}
    \CommentTok{// pinctrl & GPIO mux}
\NormalTok{    PINCTRL_SetPadMux(I2S_PIN_BCLK, IO_SOURCE_I2S_BCLK_OUT);}
\NormalTok{    PINCTRL_SetPadMux(I2S_PIN_IN, IO_SOURCE_I2S_DATA_IN);}
\NormalTok{    PINCTRL_SelI2sIn(IO_NOT_A_PIN, IO_NOT_A_PIN, I2S_PIN_IN);}
\NormalTok{    PINCTRL_SetPadMux(I2S_PIN_LRCLK, IO_SOURCE_I2S_LRCLK_OUT);}
\NormalTok{    PINCTRL_Pull(IO_SOURCE_I2S_DATA_IN, PINCTRL_PULL_DOWN);}
    
    \CommentTok{// CLK & Register}
\NormalTok{    SYSCTRL_ConfigPLLClk(}\DecValTok{6}\NormalTok{, }\DecValTok{128}\NormalTok{, }\DecValTok{2}\NormalTok{); }\CommentTok{// sorce clk PLL = 256MHz}
\NormalTok{    SYSCTRL_SelectI2sClk(SYSCTRL_CLK_PLL_DIV_1 + }\DecValTok{4}\NormalTok{); }\CommentTok{// I2s_Clk = 51.2MHz}
\NormalTok{    I2S_ConfigClk(APB_I2S, }\DecValTok{25}\NormalTok{, }\DecValTok{32}\NormalTok{); }\CommentTok{// F_bclk = 1.024MHz, F_lrclk = 16K}
\NormalTok{    I2S_ConfigIRQ(APB_I2S, }\DecValTok{0}\NormalTok{, }\DecValTok{1}\NormalTok{, }\DecValTok{0}\NormalTok{, }\DecValTok{10}\NormalTok{);}
\NormalTok{    I2S_DMAEnable(APB_I2S, }\DecValTok{0}\NormalTok{, }\DecValTok{0}\NormalTok{);    }
\NormalTok{    I2S_Config(APB_I2S, I2S_ROLE_MASTER, I2S_MODE_STANDARD, }\DecValTok{0}\NormalTok{, }\DecValTok{0}\NormalTok{, }\DecValTok{0}\NormalTok{, }\DecValTok{1}\NormalTok{, }\DecValTok{24}\NormalTok{);}

    \CommentTok{// I2s interrupt}
\NormalTok{    platform_set_irq_callback(PLATFORM_CB_IRQ_I2S, cb_isr, }\DecValTok{0}\NormalTok{);}
\NormalTok{\}}
\end{Highlighting}
\end{Shaded}

\hypertarget{i2sux4f7fux80fd}{%
\subsubsection{I2s使能}\label{i2sux4f7fux80fd}}

I2s使能分3种情况:I2s发送、I2s接收、使用DMA搬运

接收:

\begin{Shaded}
\begin{Highlighting}[]
\DataTypeTok{void}\NormalTok{ I2sStart(}\DataTypeTok{void}\NormalTok{)}
\NormalTok{\{}
\NormalTok{    I2S_ClearRxFIFO(APB_I2S);}
\NormalTok{    I2S_Enable(APB_I2S, }\DecValTok{0}\NormalTok{, }\DecValTok{1}\NormalTok{);}
\NormalTok{\}}
\end{Highlighting}
\end{Shaded}

发送:

\begin{Shaded}
\begin{Highlighting}[]
\DataTypeTok{uint8_t}\NormalTok{  sendSize = }\DecValTok{10}\NormalTok{;}
\DataTypeTok{uint32_t}\NormalTok{ sendData[}\DecValTok{10}\NormalTok{];}
\DataTypeTok{void}\NormalTok{ I2sStart(}\DataTypeTok{void}\NormalTok{)}
\NormalTok{\{}
    \DataTypeTok{int}\NormalTok{ i; }
\NormalTok{    I2S_ClearTxFIFO(APB_I2S);}
    \CommentTok{// push data into TX_FIFO first}
    \ControlFlowTok{for}\NormalTok{ (i = }\DecValTok{0}\NormalTok{; i < sendSize; i++) \{}
\NormalTok{        I2S_PushTxFIFO(APB_I2S, sendData[i]);}
\NormalTok{    \}}
\NormalTok{    I2S_Enable(APB_I2S, }\DecValTok{0}\NormalTok{, }\DecValTok{1}\NormalTok{);}
\NormalTok{\}}
\end{Highlighting}
\end{Shaded}

使用DMA(接收):

\begin{Shaded}
\begin{Highlighting}[]
\PreprocessorTok{#define CHANNEL_ID  0}
\NormalTok{DMA_Descriptor test __attribute__((aligned (}\DecValTok{8}\NormalTok{)));                                }
\DataTypeTok{void}\NormalTok{ I2sStart(}\DataTypeTok{uint32_t}\NormalTok{ data)}
\NormalTok{\{}
\NormalTok{   DMA_EnableChannel(CHANNEL_ID, &test);}
\NormalTok{   I2S_ClearRxFIFO(APB_I2S);}
\NormalTok{   I2S_DMAEnable(APB_I2S, }\DecValTok{1}\NormalTok{, }\DecValTok{1}\NormalTok{);    }
\NormalTok{   I2S_Enable(APB_I2S, }\DecValTok{0}\NormalTok{, }\DecValTok{1}\NormalTok{);}
\NormalTok{\}}
\end{Highlighting}
\end{Shaded}

无论哪种情况都必须最后一步使能I2s,否则I2s工作异常。

\hypertarget{i2sux4e2dux65ad}{%
\subsubsection{I2s中断}\label{i2sux4e2dux65ad}}

接收:

\begin{Shaded}
\begin{Highlighting}[]
\DataTypeTok{uint32_t}\NormalTok{ cb_isr(}\DataTypeTok{void}\NormalTok{ *user_data)}
\NormalTok{\{ }
    \DataTypeTok{uint32_t}\NormalTok{ state = I2S_GetIntState(APB_I2S);}
\NormalTok{    I2S_ClearIntState(APB_I2S, state);}

    \DataTypeTok{int}\NormalTok{ i = I2S_GetRxFIFOCount(APB_I2S);}

    \ControlFlowTok{while}\NormalTok{ (i) \{}
        \DataTypeTok{uint32_t}\NormalTok{ data = I2S_PopRxFIFO(APB_I2S);}
\NormalTok{        i--;}
        \CommentTok{// do something with data}
\NormalTok{    \}}

    \ControlFlowTok{return} \DecValTok{0}\NormalTok{;}
\NormalTok{\}}
\end{Highlighting}
\end{Shaded}

发送:

\begin{Shaded}
\begin{Highlighting}[]
\DataTypeTok{uint8_t}\NormalTok{  sendSize = }\DecValTok{10}\NormalTok{;}
\DataTypeTok{uint32_t}\NormalTok{ sendData[}\DecValTok{10}\NormalTok{];}
\DataTypeTok{uint32_t}\NormalTok{ cb_isr(}\DataTypeTok{void}\NormalTok{ *user_data)}
\NormalTok{\{ }
    \DataTypeTok{uint32_t}\NormalTok{ state = I2S_GetIntState(APB_I2S);}
\NormalTok{    I2S_ClearIntState(APB_I2S, state);}

    \DataTypeTok{int}\NormalTok{ i;}
    \ControlFlowTok{for}\NormalTok{ (i = }\DecValTok{0}\NormalTok{; i < sendSize; i++) \{}
\NormalTok{        I2S_PushTxFIFO(APB_I2S, sendData[i]);}
\NormalTok{    \}}
    \ControlFlowTok{return} \DecValTok{0}\NormalTok{;}
\NormalTok{\}}
\end{Highlighting}
\end{Shaded}

\hypertarget{i2s-dmaux4e52ux4e53ux642cux8fd0}{%
\subsubsection{I2s \& DMA乒乓搬运}\label{i2s-dmaux4e52ux4e53ux642cux8fd0}}

下面以经典的DMA乒乓搬运I2s接收数据为例展示I2s实际使用方法。

这里我们采用16K采样率,单个数据帧固定64位,和DMA协商握手、burstSize=8、一次搬运80个数据。

\begin{Shaded}
\begin{Highlighting}[]
\PreprocessorTok{#include }\ImportTok{"pingpong.h"}
\PreprocessorTok{#define I2S_PIN_BCLK        28}
\PreprocessorTok{#define I2S_PIN_IN          29}
\PreprocessorTok{#define I2S_PIN_LRCLK       30}
\PreprocessorTok{#define CHANNEL_ID  0}
\NormalTok{DMA_PingPong_t PingPong;}

\DataTypeTok{static} \DataTypeTok{uint32_t}\NormalTok{ DMA_cb_isr(}\DataTypeTok{void}\NormalTok{ *user_data)}
\NormalTok{\{}
    \DataTypeTok{uint32_t}\NormalTok{ state = DMA_GetChannelIntState(CHANNEL_ID);}
\NormalTok{    DMA_ClearChannelIntState(CHANNEL_ID, state);}

    \DataTypeTok{uint32_t}\NormalTok{ *rr = DMA_PingPongIntProc(&PingPong, CHANNEL_ID);}
    \DataTypeTok{uint32_t}\NormalTok{ i = }\DecValTok{0}\NormalTok{;}
    \DataTypeTok{uint32_t}\NormalTok{ transSize = DMA_PingPongGetTransSize(&PingPong);}
    \ControlFlowTok{while}\NormalTok{ (i < transSize) \{}
        \CommentTok{// do something with data 'rr[i]'}
\NormalTok{        i++;}
\NormalTok{    \}}

    \ControlFlowTok{return} \DecValTok{0}\NormalTok{;}
\NormalTok{\}}
\DataTypeTok{void}\NormalTok{ I2sSetup(}\DataTypeTok{void}\NormalTok{)}
\NormalTok{\{}
    \CommentTok{// pinctrl & GPIO mux}
\NormalTok{    PINCTRL_SetPadMux(I2S_PIN_BCLK, IO_SOURCE_I2S_BCLK_OUT);}
\NormalTok{    PINCTRL_SetPadMux(I2S_PIN_IN, IO_SOURCE_I2S_DATA_IN);}
\NormalTok{    PINCTRL_SelI2sIn(IO_NOT_A_PIN, IO_NOT_A_PIN, I2S_PIN_IN);}
\NormalTok{    PINCTRL_SetPadMux(I2S_PIN_LRCLK, IO_SOURCE_I2S_LRCLK_OUT);}
\NormalTok{    PINCTRL_Pull(IO_SOURCE_I2S_DATA_IN, PINCTRL_PULL_DOWN);}
    
    \CommentTok{// CLK & Register}
\NormalTok{    SYSCTRL_ConfigPLLClk(}\DecValTok{6}\NormalTok{, }\DecValTok{128}\NormalTok{, }\DecValTok{2}\NormalTok{); }\CommentTok{// sorce clk PLL = 256MHz}
\NormalTok{    SYSCTRL_SelectI2sClk(SYSCTRL_CLK_PLL_DIV_1 + }\DecValTok{4}\NormalTok{); }\CommentTok{// I2s_Clk = 51.2MHz}
\NormalTok{    I2S_ConfigClk(APB_I2S, }\DecValTok{25}\NormalTok{, }\DecValTok{32}\NormalTok{); }\CommentTok{// F_bclk = 1.024MHz, F_lrclk = 16K}
\NormalTok{    I2S_ConfigIRQ(APB_I2S, }\DecValTok{0}\NormalTok{, }\DecValTok{1}\NormalTok{, }\DecValTok{0}\NormalTok{, }\DecValTok{8}\NormalTok{);}
\NormalTok{    I2S_DMAEnable(APB_I2S, }\DecValTok{0}\NormalTok{, }\DecValTok{0}\NormalTok{);    }
\NormalTok{    I2S_Config(APB_I2S, I2S_ROLE_MASTER, I2S_MODE_STANDARD, }\DecValTok{0}\NormalTok{, }\DecValTok{0}\NormalTok{, }\DecValTok{0}\NormalTok{, }\DecValTok{1}\NormalTok{, }\DecValTok{24}\NormalTok{);}

    \CommentTok{// setup DMA}
\NormalTok{    DMA_PingPongSetup(&PingPong, SYSCTRL_DMA_I2S_RX, }\DecValTok{100}\NormalTok{, }\DecValTok{8}\NormalTok{);}
\NormalTok{    platform_set_irq_callback(PLATFORM_CB_IRQ_DMA, DMA_cb_isr, }\DecValTok{0}\NormalTok{);}
    
    \CommentTok{// start working}
\NormalTok{    DMA_PingPongEnable(&PingPong, CHANNEL_ID);}
\NormalTok{    I2S_ClearRxFIFO(APB_I2S);}
\NormalTok{    I2S_DMAEnable(APB_I2S, }\DecValTok{1}\NormalTok{, }\DecValTok{1}\NormalTok{);    }
\NormalTok{    I2S_Enable(APB_I2S, }\DecValTok{0}\NormalTok{, }\DecValTok{1}\NormalTok{);}
\NormalTok{\}}
\end{Highlighting}
\end{Shaded}

DMA(乒乓搬运)的具体用法请参见本手册DMA一节。

更加系统化的I2s代码请参考SDK中voice\_remote\_ctrl例程。

\hypertarget{ir-ux7ea2ux5916}{%
\chapter{IR 红外}\label{ir-ux7ea2ux5916}}

\hypertarget{ux529fux80fdux6982ux8ff0-2}{%
\section{功能概述}\label{ux529fux80fdux6982ux8ff0-2}}

\begin{itemize}
\tightlist
\item
  支持红外发射\&接收
\item
  时序可调整,支持多种编码
\end{itemize}

\hypertarget{ux4f7fux7528ux8bf4ux660e-2}{%
\section{使用说明}\label{ux4f7fux7528ux8bf4ux660e-2}}

\hypertarget{ux53c2ux6570ux4e0dux540cux7f16ux7801ux7684ux65f6ux95f4ux53c2ux6570}{%
\subsection{参数(不同编码的时间参数)}\label{ux53c2ux6570ux4e0dux540cux7f16ux7801ux7684ux65f6ux95f4ux53c2ux6570}}

\begin{Shaded}
\begin{Highlighting}[]
\CommentTok{//32KHz 调制时钟}
\PreprocessorTok{#define FCLK            32000}

\CommentTok{//NEC 载波频率38KHz}
\PreprocessorTok{#define NEC_WAVE_FREQ   38000}

\CommentTok{// RC5 载波频率36KHz}
\PreprocessorTok{#define RC5_WAVE_FREQ   36000}

\CommentTok{// 计算载波频率发生器数据,由此产生如38KHz NEC的载波,NEC与TC9012通用38KHz载波}
\PreprocessorTok{#define IR_WAVE_NEC_TC9012_FREQ (OSC_CLK_FREQ/NEC_WAVE_FREQ)    }
\PreprocessorTok{#define IR_WAVE_RC5_FREQ (OSC_CLK_FREQ/RC5_WAVE_FREQ)  }

\CommentTok{//通过FCLK产生各种协议每bit位调制周期最小单位,如NEC协议中,560us为最小调制周期,NEC:bit 0:560us载波+560us不载波,bit 1:560 us载波+560*3 us不载波}
\PreprocessorTok{#define NEC_UINT    (FCLK*560/1000000+1)}
\PreprocessorTok{#define TC9012_UINT (FCLK*260/1000000+1)}
\PreprocessorTok{#define RC5_UINT    (FCLK*889/1000000+1)}

\CommentTok{//不必要的参数}
\PreprocessorTok{#define INESSENTIAL     0}

\KeywordTok{typedef} \KeywordTok{struct}
\NormalTok{\{}
    \DataTypeTok{uint16_t}\NormalTok{ timer1;}\CommentTok{//发送模式下,表示引导码低电平时间:如NEC为16*UNIT = 9ms,接收模式下与timer2组成检测引导码低电平窗口}
    \DataTypeTok{uint16_t}\NormalTok{ timer2;}\CommentTok{//发送模式下,表示重复码低电平时间:如NEC为4*UNIT = 2.25ms,接收模式下与timer1组成检测引导码低窗口}
    \DataTypeTok{uint16_t}\NormalTok{ timer3;}\CommentTok{//发送模式下,表示引导码高电平时间:如NEC为8*UNIT = 4.5ms,接收模式下与timer4组成检测引导码高电平+低电平窗口}
    \DataTypeTok{uint16_t}\NormalTok{ timer4;}\CommentTok{//发送模式下,表示重复码高电平时间:如NEC为UNIT = 560us,接收模式下与timer3组成检测引导码高电平+低电平窗口}
    \DataTypeTok{uint16_t}\NormalTok{ timer5;}\CommentTok{//接收时接收超时定时器,发射不必关注}
    \DataTypeTok{uint16_t}\NormalTok{ btimer1;}\CommentTok{//逻辑 0 的bit时长:如NEC = 2*UINT = 1.12ms.}
    \DataTypeTok{uint16_t}\NormalTok{ btimer2;}\CommentTok{//逻辑 1 的bit时长:如NEC = 4*UINT = 2.25ms.}
    \DataTypeTok{uint16_t}\NormalTok{ bit_cycle;}\CommentTok{//发射模式下:bit调制周期最小单位,如NEC为560us,接收模式下为bit检测超时时间}
    \DataTypeTok{uint16_t}\NormalTok{ carry_low;}\CommentTok{//载波低电平时长,与carry_high组合形成占空比可调的载波波形,如NEC为38KHz 30%占空比载波则carry_low = 2/3*载波周期,carry_high = 1/3*载波周期}
    \DataTypeTok{uint16_t}\NormalTok{ carry_high;}\CommentTok{//载波高电平时长,与carry_low组合形成占空比可调的载波波形,如NEC为38KHz 30%占空比载波则carry_low = 2/3*载波周期,carry_high = 1/3*载波周期}
\NormalTok{\}Ir_mode_param_t;}

\CommentTok{//由于发送和接收初始化不同参数 定义结构体表示接收发送初始化参数。}
\KeywordTok{typedef} \KeywordTok{struct}\NormalTok{\{}
\NormalTok{    Ir_mode_param_t param_tx;}
\NormalTok{    Ir_mode_param_t param_rx;}
\NormalTok{\}Ir_type_param_t;}

\CommentTok{//定义初始化数据,初始化函数体根据不同协议自动从此表适配参数}
\DataTypeTok{const} \DataTypeTok{static}\NormalTok{ Ir_type_param_t t_ir_type_param_table[] = }
\NormalTok{\{}
\NormalTok{    \{}\CommentTok{//NEC param}
\NormalTok{        \{  }\CommentTok{//TX}
            \DecValTok{16}\NormalTok{*NEC_UINT-}\DecValTok{1}\NormalTok{,}\DecValTok{4}\NormalTok{*NEC_UINT-}\DecValTok{1}\NormalTok{,}\DecValTok{8}\NormalTok{*NEC_UINT-}\DecValTok{1}\NormalTok{,}\DecValTok{1}\NormalTok{*NEC_UINT-}\DecValTok{1}\NormalTok{,INESSENTIAL,}
            \DecValTok{2}\NormalTok{*NEC_UINT-}\DecValTok{1}\NormalTok{, }\DecValTok{4}\NormalTok{*NEC_UINT-}\DecValTok{1}\NormalTok{,}\DecValTok{1}\NormalTok{*NEC_UINT-}\DecValTok{1}\NormalTok{,IR_WAVE_NEC_TC9012_FREQ*}\DecValTok{2}\NormalTok{/}\DecValTok{3}\NormalTok{,IR_WAVE_NEC_TC9012_FREQ*}\DecValTok{1}\NormalTok{/}\DecValTok{3}\NormalTok{\},}
\NormalTok{        \{   }\CommentTok{//RX}
            \DecValTok{14}\NormalTok{*NEC_UINT-}\DecValTok{1}\NormalTok{,}\DecValTok{18}\NormalTok{*NEC_UINT-}\DecValTok{1}\NormalTok{,}\DecValTok{22}\NormalTok{*NEC_UINT-}\DecValTok{1}\NormalTok{,}\DecValTok{26}\NormalTok{*NEC_UINT-}\DecValTok{1}\NormalTok{,}\BaseNTok{0xfff}\NormalTok{,}
\NormalTok{            INESSENTIAL,}\DecValTok{2}\NormalTok{*NEC_UINT-}\DecValTok{1}\NormalTok{,}\BaseNTok{0x7f}\NormalTok{,INESSENTIAL,INESSENTIAL\},    }
\NormalTok{    \},}
\NormalTok{    \{}\CommentTok{//TC9012 param}
\NormalTok{        \{   }\CommentTok{//TX}
            \DecValTok{16}\NormalTok{*TC9012_UINT-}\DecValTok{1}\NormalTok{,}\DecValTok{8}\NormalTok{*TC9012_UINT-}\DecValTok{1}\NormalTok{,}\DecValTok{16}\NormalTok{*TC9012_UINT-}\DecValTok{1}\NormalTok{,}\DecValTok{2}\NormalTok{*TC9012_UINT-}\DecValTok{1}\NormalTok{,INESSENTIAL,}
            \DecValTok{4}\NormalTok{*TC9012_UINT-}\DecValTok{1}\NormalTok{,}\DecValTok{6}\NormalTok{*TC9012_UINT-}\DecValTok{1}\NormalTok{,}\DecValTok{2}\NormalTok{*TC9012_UINT-}\DecValTok{1}\NormalTok{,IR_WAVE_NEC_TC9012_FREQ*}\DecValTok{2}\NormalTok{/}\DecValTok{3}\NormalTok{,IR_WAVE_NEC_TC9012_FREQ*}\DecValTok{1}\NormalTok{/}\DecValTok{3}\NormalTok{\},}
\NormalTok{        \{    }\CommentTok{//RX}
            \DecValTok{28}\NormalTok{*TC9012_UINT-}\DecValTok{1}\NormalTok{,}\DecValTok{36}\NormalTok{*TC9012_UINT-}\DecValTok{1}\NormalTok{,}\DecValTok{44}\NormalTok{*TC9012_UINT-}\DecValTok{1}\NormalTok{,}\DecValTok{32}\NormalTok{*TC9012_UINT-}\DecValTok{1}\NormalTok{,}\BaseNTok{0xfff}\NormalTok{,INESSENTIAL,}
            \DecValTok{4}\NormalTok{*TC9012_UINT-}\DecValTok{1}\NormalTok{,}\BaseNTok{0x7f}\NormalTok{,INESSENTIAL,INESSENTIAL\},    }
\NormalTok{    \},    }
\NormalTok{    \{}\CommentTok{//RC5 param}
\NormalTok{        \{   }\CommentTok{//TX}
            \DecValTok{2}\NormalTok{*RC5_UINT-}\DecValTok{1}\NormalTok{,}\DecValTok{2}\NormalTok{*RC5_UINT-}\DecValTok{1}\NormalTok{,INESSENTIAL,INESSENTIAL,INESSENTIAL,}\DecValTok{2}\NormalTok{*RC5_UINT-}\DecValTok{1}\NormalTok{,}
\NormalTok{            INESSENTIAL,}\DecValTok{1}\NormalTok{*RC5_UINT,IR_WAVE_RC5_FREQ*}\DecValTok{2}\NormalTok{/}\DecValTok{3}\NormalTok{,IR_WAVE_RC5_FREQ*}\DecValTok{1}\NormalTok{/}\DecValTok{3}\NormalTok{\},}
        
\NormalTok{        \{   }\CommentTok{//RX}
            \DecValTok{1}\NormalTok{*RC5_UINT-}\DecValTok{2}\NormalTok{,}\DecValTok{1}\NormalTok{*RC5_UINT,}\DecValTok{3}\NormalTok{*RC5_UINT-}\DecValTok{1}\NormalTok{,}\DecValTok{5}\NormalTok{*RC5_UINT-}\DecValTok{1}\NormalTok{,INESSENTIAL,}\DecValTok{1}\NormalTok{*RC5_UINT-}\DecValTok{3}\NormalTok{,}
            \DecValTok{1}\NormalTok{*RC5_UINT-}\DecValTok{1}\NormalTok{,}\DecValTok{2}\NormalTok{*RC5_UINT-}\DecValTok{1}\NormalTok{,INESSENTIAL,INESSENTIAL\},    }
\NormalTok{    \}}
\NormalTok{\};}
\end{Highlighting}
\end{Shaded}

\hypertarget{ux7ea2ux5916ux53d1ux5c04ux63a5ux6536}{%
\subsection{红外发射接收}\label{ux7ea2ux5916ux53d1ux5c04ux63a5ux6536}}

\hypertarget{ux914dux7f6epin-10}{%
\subsubsection{配置pin}\label{ux914dux7f6epin-10}}

\begin{Shaded}
\begin{Highlighting}[]
\PreprocessorTok{#define IR_DOUT GIO_GPIO_10}
\PreprocessorTok{#define IR_DIN  GIO_GPIO_11}

\DataTypeTok{void}\NormalTok{ setup_peripherals_ir_module(}\DataTypeTok{void}\NormalTok{)}
\NormalTok{\{}
    \CommentTok{// 打开时钟}
    \CommentTok{// 大于等于GPIO 18则使用SYSCTRL_ClkGate_APB_GPIO1, 否则是SYSCTRL_ClkGate_APB_GPIO0}
\NormalTok{    SYSCTRL_ClearClkGateMulti(   (}\DecValTok{1}\NormalTok{ << SYSCTRL_ClkGate_APB_GPIO0)}
\NormalTok{                                | (}\DecValTok{1}\NormalTok{ << SYSCTRL_ClkGate_APB_GPIO1)}
\NormalTok{                                | (}\DecValTok{1}\NormalTok{ << SYSCTRL_ClkGate_APB_IR)}
\NormalTok{                                | (}\DecValTok{1}\NormalTok{ << SYSCTRL_ClkGate_APB_PinCtrl));}
                                
\NormalTok{    PINCTRL_SelIrIn(IR_DIN);}
\NormalTok{    PINCTRL_SetPadMux(IR_DOUT,IO_SOURCE_IR_DATA_OUT);}
\NormalTok{    platform_set_irq_callback(PLATFORM_CB_IRQ_IR_INT, IRQHandler_IR_INT, NULL);}
\NormalTok{\}}
\end{Highlighting}
\end{Shaded}

\hypertarget{ux914dux7f6eux6a21ux5757}{%
\subsubsection{配置模块}\label{ux914dux7f6eux6a21ux5757}}

\begin{Shaded}
\begin{Highlighting}[]
\DataTypeTok{static} \DataTypeTok{void}\NormalTok{ user_ir_device_init(IR_IrMode_e mode,IR_TxRxMode_e tx_rx_mode)}
\NormalTok{\{}
\NormalTok{    IR_CtrlSetIrMode(APB_IR,mode);}
\NormalTok{    IR_CtrlSetTxRxMode(APB_IR,tx_rx_mode);}
\NormalTok{    IR_CtrlSetIrIntEn(APB_IR);}
    
    \ControlFlowTok{if}\NormalTok{(IR_TXRX_MODE_TX_MODE == tx_rx_mode)}
\NormalTok{    \{}
\NormalTok{        IR_TxConfigIrTxPol(APB_IR);}
\NormalTok{        IR_TxConfigCarrierCntClr(APB_IR);}
\NormalTok{        IR_TxConfigIrIntEn(APB_IR);}
\NormalTok{        IR_CarryConfigSetIrCarryLow(APB_IR,t_ir_type_param_table[mode].param_tx.carry_low);   }\CommentTok{//30% pwm}
\NormalTok{        IR_CarryConfigSetIrCarryHigh(APB_IR,t_ir_type_param_table[mode].param_tx.carry_high);}
\NormalTok{        IR_TimeSetIrTime1(APB_IR,t_ir_type_param_table[mode].param_tx.timer1);}
\NormalTok{        IR_TimeSetIrTime2(APB_IR,t_ir_type_param_table[mode].param_tx.timer2);}
\NormalTok{        IR_TimeSetIrTime3(APB_IR,t_ir_type_param_table[mode].param_tx.timer3 );}
\NormalTok{        IR_TimeSetIrTime4(APB_IR,t_ir_type_param_table[mode].param_tx.timer4); }
\NormalTok{        IR_CtrlIrSetBitTime1(APB_IR,t_ir_type_param_table[mode].param_tx.btimer1);}
\NormalTok{        IR_CtrlIrSetBitTime2(APB_IR,t_ir_type_param_table[mode].param_tx.btimer2);    }
\NormalTok{        IR_CtrlIrSetIrBitCycle(APB_IR,t_ir_type_param_table[mode].param_tx.bit_cycle);        }
\NormalTok{    \}}
    \ControlFlowTok{else}\NormalTok{\{}
\NormalTok{        IR_CtrlSetIrEndDetectEn(APB_IR);}\CommentTok{//end code detect en}
\NormalTok{        IR_CtrlIrUsercodeVerify(APB_IR);}
\NormalTok{        IR_CtrlIrDatacodeVerify(APB_IR);}
\NormalTok{        IR_TimeSetIrTime1(APB_IR,t_ir_type_param_table[mode].param_rx.timer1);}
\NormalTok{        IR_TimeSetIrTime2(APB_IR,t_ir_type_param_table[mode].param_rx.timer2);}
\NormalTok{        IR_TimeSetIrTime3(APB_IR,t_ir_type_param_table[mode].param_rx.timer3);}
\NormalTok{        IR_TimeSetIrTime4(APB_IR,t_ir_type_param_table[mode].param_rx.timer4); }
\NormalTok{        IR_TimeSetIrTime5(APB_IR,t_ir_type_param_table[mode].param_rx.timer5); }
\NormalTok{        IR_CtrlIrSetBitTime1(APB_IR,t_ir_type_param_table[mode].param_rx.btimer1);}
\NormalTok{        IR_CtrlIrSetBitTime2(APB_IR,t_ir_type_param_table[mode].param_rx.btimer2);    }
\NormalTok{        IR_CtrlIrSetIrBitCycle(APB_IR,t_ir_type_param_table[mode].param_rx.bit_cycle);   }
\NormalTok{    \}}
\NormalTok{    IR_CtrlEnable(APB_IR);}
\NormalTok{\}}
\end{Highlighting}
\end{Shaded}

\hypertarget{ir-ux4e2dux65adux4ee5ux53cairux63a5ux6536}{%
\subsubsection{IR 中断(以及IR接收)}\label{ir-ux4e2dux65adux4ee5ux53cairux63a5ux6536}}

\begin{Shaded}
\begin{Highlighting}[]
\DataTypeTok{static} \DataTypeTok{uint32_t}\NormalTok{ IRQHandler_IR_INT(}\DataTypeTok{void}\NormalTok{ *user_data)}
\NormalTok{\{}
    \ControlFlowTok{if}\NormalTok{(IR_FsmGetIrTransmitOk(APB_IR))}
\NormalTok{        ;}\CommentTok{//platform_printf("int ir send ok\textbackslash{}n");}
    \ControlFlowTok{if}\NormalTok{(IR_FsmGetIrTxRepeat(APB_IR))}
\NormalTok{        ;}\CommentTok{//platform_printf("int ir repeat ok\textbackslash{}n");}
    \ControlFlowTok{if}\NormalTok{(IR_FsmGetIrReceivedOk(APB_IR))}
\NormalTok{    \{}
        \DataTypeTok{uint32_t}\NormalTok{ value = IR_RxCodeGetIrRxUsercode(APB_IR) <<}\DecValTok{16}\NormalTok{ | IR_RxCodeGetIrRxDatacode(APB_IR);}
\NormalTok{        platform_printf(}\StringTok{"Received:0x%x}\SpecialCharTok{\textbackslash{}n}\StringTok{"}\NormalTok{,value);}
\NormalTok{    \}}
    
    \ControlFlowTok{if}\NormalTok{(IR_FsmGetIrRepeat(APB_IR))}
\NormalTok{        ;}
\NormalTok{    IR_FsmClearIrInt(APB_IR);}
    \ControlFlowTok{return} \DecValTok{0}\NormalTok{;}
\NormalTok{\}}
\end{Highlighting}
\end{Shaded}

\hypertarget{ir-ux53d1ux5c04}{%
\subsubsection{IR 发射}\label{ir-ux53d1ux5c04}}

\begin{Shaded}
\begin{Highlighting}[]
\DataTypeTok{static} \DataTypeTok{int}\NormalTok{ ir_transmit_fun(}\DataTypeTok{uint16_t}\NormalTok{ addr,}\DataTypeTok{uint16_t}\NormalTok{ data) }\CommentTok{//ir hard transmit data}
\NormalTok{\{}
\NormalTok{    IR_TxCodeSetIrTxUsercode(APB_IR,addr);}
\NormalTok{    IR_TxCodeSetIrTxDatacode(APB_IR,data);}
\NormalTok{    IR_CleanIrTxRepeatMode(APB_IR);}\CommentTok{//must clearn the repeat mode reg}
\NormalTok{    IR_TxConfigTxStart(APB_IR);}
    \CommentTok{//while(!IR_FsmGetIrTransmitOk(APB_IR));}
    \ControlFlowTok{return} \DecValTok{0}\NormalTok{;}
\NormalTok{\}}

\DataTypeTok{static} \DataTypeTok{int}\NormalTok{ ir_transmit_repeat(}\DataTypeTok{void}\NormalTok{) }\CommentTok{//ir hard transmit repeat}
\NormalTok{\{}
\NormalTok{    IR_CtrlIrTxRepeatMode(APB_IR);   }
\NormalTok{    IR_TxConfigTxStart(APB_IR);}
 \CommentTok{//   while(!IR_FsmGetIrTransmitOk(APB_IR)); }
    \ControlFlowTok{return} \DecValTok{0}\NormalTok{;}
\NormalTok{\}}
\end{Highlighting}
\end{Shaded}

\begin{center}\rule{0.5\linewidth}{0.5pt}\end{center}

\hypertarget{ch-PDM}{%
\chapter{PDM简介}\label{ch-PDM}}

PDM全称pulse density modulation,即脉冲密度调制。

PDM模块处理来自外部音频前端(如数字麦克风)的脉冲密度调制信号的输入。该模块生成PDM时钟,支持双通道数据输入。

\hypertarget{ux529fux80fdux63cfux8ff0-3}{%
\section{功能描述}\label{ux529fux80fdux63cfux8ff0-3}}

\hypertarget{ux7279ux70b9-3}{%
\subsection{特点}\label{ux7279ux70b9-3}}

\begin{itemize}
\item
  支持双通道,数据输入相同
\item
  16kHz输出采样率,24位采样
\item
  HW抽取过滤器
\item
  时钟和输出采样率之间的可选比为64或80
\item
  支持DMA和I2S的样本缓冲
\end{itemize}

\hypertarget{pdm-pcm}{%
\subsection{PDM \& PCM}\label{pdm-pcm}}

PDM和PCM同为用数字信号表示模拟信号的音频数据调制方法,其主要区别是:

\begin{itemize}
\item
  PDM不像PCM等间隔采样
\item
  PDM只有1位非0即1的输出,而PCM采样结果可以是Nbit
\item
  PDM使用远高于PCM采样率的时钟频率进行采样
\item
  PDM逻辑相对PCM复杂
\item
  PDM只需要2根信号线,即时钟线和数据线;PCM需要4根线
\end{itemize}

\hypertarget{ux4f7fux7528ux65b9ux6cd5-3}{%
\section{使用方法}\label{ux4f7fux7528ux65b9ux6cd5-3}}

\hypertarget{ux65b9ux6cd5ux6982ux8ff0-3}{%
\subsection{方法概述}\label{ux65b9ux6cd5ux6982ux8ff0-3}}

PDM使用方法分为PDM结合I2s使用和PDM数据直接DMA搬运两种。

PDM结合I2s:

\textbf{1.} PDM引脚GPIO配置(时钟、数据)

\textbf{2.} 外部时钟配置,使其处于工作模式

\textbf{3.} 写相应配置寄存器

\textbf{4.} 配置I2s数据源为PDM,配置I2s时钟、寄存器、中断

\textbf{5.} 使能PDM,使能I2s

\textbf{6.} 等待I2s中断产生

\textbf{7.} 读取状态寄存器,读取RX\_MEM中数据

\textbf{8.} 传输完毕,关闭PDM和I2S

PDM数据DMA搬运:

\textbf{1.} PDM引脚GPIO配置(时钟、数据)

\textbf{2.} 外部时钟配置,使其处于工作模式

\textbf{3.} 写相应配置寄存器

\textbf{4.} 配置DMA寄存器、中断

\textbf{5.} 使能DMA,使能PDM

\textbf{6.} 等待DMA中断产生

\textbf{7.} 传输完毕,关闭PDM和DMA

其中I2s和DMA相关配置不在本节介绍内容范围内,可参考本手册对应章节。

\hypertarget{ux6ce8ux610fux70b9-3}{%
\subsection{注意点}\label{ux6ce8ux610fux70b9-3}}

\begin{itemize}
\item
  I2s时钟源可以选择晶振24M时钟和PLL时钟,要注意是选择哪一个时钟源
\item
  建议选择晶振24M作为时钟源,这样可以获得较好的准确度和防抖动效果
\item
  需要查阅数字麦克风数据手册了解其时钟要求,并正确配置PDM时钟频率
\item
  结合I2s使用时要先使能PDM最后开启I2s
\item
  结合DMA使用时要先使能DMA最后开启PDM
\item
  建议采用DMA乒乓搬运的方式
\end{itemize}

\hypertarget{ux7f16ux7a0bux6307ux5357-3}{%
\section{编程指南}\label{ux7f16ux7a0bux6307ux5357-3}}

\hypertarget{ux9a71ux52a8ux63a5ux53e3-3}{%
\subsection{驱动接口}\label{ux9a71ux52a8ux63a5ux53e3-3}}

\begin{itemize}
\item
  PDM\_Config:PDM配置接口
\item
  PDM\_Start:PDM使能接口
\item
  PDM\_DmaEnable:PDM使能DMA接口
\end{itemize}

\hypertarget{ux4ee3ux7801ux793aux4f8b-3}{%
\subsection{代码示例}\label{ux4ee3ux7801ux793aux4f8b-3}}

下面以PDM结合I2s使用和PDM数据直接DMA搬运两种方式来展示PDM的具体使用方法。

已知现有mic使用的时钟频率为3M,I2s采样率16K。

\hypertarget{pdmux7ed3ux5408i2s}{%
\subsubsection{PDM结合I2s:}\label{pdmux7ed3ux5408i2s}}

\begin{Shaded}
\begin{Highlighting}[]
\PreprocessorTok{#define PDM_PIN_MCLK        28}
\PreprocessorTok{#define PDM_PIN_IN          29}
\DataTypeTok{static} \DataTypeTok{uint32_t}\NormalTok{ I2s_isr(}\DataTypeTok{void}\NormalTok{ *user_data)}
\NormalTok{\{ }
    \DataTypeTok{uint32_t}\NormalTok{ state = I2S_GetIntState(APB_I2S);}
\NormalTok{    I2S_ClearIntState(APB_I2S, state);}

    \DataTypeTok{int}\NormalTok{ i = I2S_GetRxFIFOCount(APB_I2S);}
    \ControlFlowTok{while}\NormalTok{ (i) \{}
        \CommentTok{// do something with these data}
\NormalTok{        i--;}
\NormalTok{    \}}
\NormalTok{\}}

\DataTypeTok{void}\NormalTok{ audio_input_setup(}\DataTypeTok{void}\NormalTok{)}
\NormalTok{\{}
    \CommentTok{// GPIO & Pin Ctrl}
\NormalTok{    PINCTRL_SetPadMux(PDM_PIN_MCLK, IO_SOURCE_PDM_DMIC_MCLK);}
\NormalTok{    PINCTRL_SetPadMux(PDM_PIN_IN, IO_SOURCE_PDM_DMIC_IN);}
\NormalTok{    PINCTRL_SelPdmIn(PDM_PIN_IN);}

    \CommentTok{// PDM clock configuration, 3M}
\NormalTok{    SYSCTRL_SetPdmClkDiv(}\DecValTok{8}\NormalTok{, }\DecValTok{1}\NormalTok{);}
\NormalTok{    SYSCTRL_SelectTypeAClk(SYSCTRL_ITEM_APB_PDM, SYSCTRL_CLK_ADC_DIV);}
    
    \CommentTok{// PDM register configuration}
\NormalTok{    PDM_Config(APB_PDM, }\DecValTok{0}\NormalTok{, }\DecValTok{1}\NormalTok{, }\DecValTok{1}\NormalTok{, }\DecValTok{1}\NormalTok{, }\DecValTok{0}\NormalTok{);    }

    \CommentTok{// I2s configuration, bclk=2.4M, samplerate=16K, data from PDM}
\NormalTok{    I2S_DataFromPDM(}\DecValTok{1}\NormalTok{);    }
\NormalTok{    I2S_ConfigClk(APB_I2S, }\DecValTok{5}\NormalTok{, }\DecValTok{75}\NormalTok{);}
\NormalTok{    I2S_ConfigIRQ(APB_I2S, }\DecValTok{0}\NormalTok{, }\DecValTok{1}\NormalTok{, }\DecValTok{0}\NormalTok{, }\DecValTok{10}\NormalTok{);}
\NormalTok{    I2S_DMAEnable(APB_I2S, }\DecValTok{0}\NormalTok{, }\DecValTok{0}\NormalTok{);    }
\NormalTok{    I2S_Config(APB_I2S, I2S_ROLE_MASTER, I2S_MODE_STANDARD, }\DecValTok{0}\NormalTok{, }\DecValTok{1}\NormalTok{, }\DecValTok{0}\NormalTok{, }\DecValTok{1}\NormalTok{, }\DecValTok{24}\NormalTok{);}
    
    \CommentTok{// I2s interruption}
\NormalTok{    platform_set_irq_callback(PLATFORM_CB_IRQ_I2S, I2s_isr, }\DecValTok{0}\NormalTok{);}

    \CommentTok{// enable I2s and PDM}
\NormalTok{    I2S_ClearRxFIFO(APB_I2S);}
\NormalTok{    PDM_Start(APB_PDM, }\DecValTok{1}\NormalTok{);}
\NormalTok{    I2S_Enable(APB_I2S, }\DecValTok{0}\NormalTok{, }\DecValTok{1}\NormalTok{);}
\NormalTok{\}}
\end{Highlighting}
\end{Shaded}

上面示例涉及到的关于I2s配置参考手册的I2s章节:

\hypertarget{pdmux6570ux636edmaux642cux8fd0}{%
\subsubsection{PDM数据DMA搬运}\label{pdmux6570ux636edmaux642cux8fd0}}

\begin{Shaded}
\begin{Highlighting}[]
\PreprocessorTok{#define PDM_PIN_MCLK        28}
\PreprocessorTok{#define PDM_PIN_IN          29}
\PreprocessorTok{#define CHANNEL_ID  0}
\DataTypeTok{uint32_t}\NormalTok{ buff[}\DecValTok{80}\NormalTok{];}
\NormalTok{DMA_Descriptor test __attribute__((aligned (}\DecValTok{8}\NormalTok{)));}
\DataTypeTok{static} \DataTypeTok{uint32_t}\NormalTok{ DMA_cb_isr(}\DataTypeTok{void}\NormalTok{ *user_data)}
\NormalTok{\{ }
    \DataTypeTok{uint32_t}\NormalTok{ state = DMA_GetChannelIntState(CHANNEL_ID);}
\NormalTok{    DMA_ClearChannelIntState(CHANNEL_ID, state);}

\NormalTok{    DMA_EnableChannel(CHANNEL_ID, &test);}

    \CommentTok{// do something with data in buff}
\NormalTok{\}}

\DataTypeTok{void}\NormalTok{ DMA_SetUp(}\DataTypeTok{void}\NormalTok{)}
\NormalTok{\{}
\NormalTok{    DMA_Reset();}
\NormalTok{    platform_set_irq_callback(PLATFORM_CB_IRQ_DMA, DMA_cb_isr, }\DecValTok{0}\NormalTok{);  }
\NormalTok{    test.Next = NULL;}
\NormalTok{    DMA_PreparePeripheral2RAM(&test, }
\NormalTok{                              buff, }
\NormalTok{                              SYSCTRL_DMA_PDM, }
                              \DecValTok{80}\NormalTok{, }
\NormalTok{                              DMA_ADDRESS_INC, }
                              \DecValTok{0}\NormalTok{ | }\DecValTok{1}\NormalTok{ << }\DecValTok{24}\NormalTok{);}
\NormalTok{    DMA_EnableChannel(CHANNEL_ID, &test);}
\NormalTok{\}}
\DataTypeTok{void}\NormalTok{ audio_input_setup(}\DataTypeTok{void}\NormalTok{)}
\NormalTok{\{}
    \CommentTok{//GPIO & Pin Ctrl}
\NormalTok{    PINCTRL_SetPadMux(PDM_PIN_MCLK, IO_SOURCE_PDM_DMIC_MCLK);}
\NormalTok{    PINCTRL_SetPadMux(PDM_PIN_IN, IO_SOURCE_PDM_DMIC_IN);}
\NormalTok{    PINCTRL_SelPdmIn(PDM_PIN_IN);}

    \CommentTok{// PDM clock configuration, 3M}
\NormalTok{    SYSCTRL_SetAdcClkDiv(}\DecValTok{8}\NormalTok{, }\DecValTok{1}\NormalTok{);}
\NormalTok{    SYSCTRL_SelectTypeAClk(SYSCTRL_ITEM_APB_PDM, SYSCTRL_CLK_ADC_DIV);}
    
    \CommentTok{// PDM register configuration}
\NormalTok{    PDM_Config(APB_PDM, }\DecValTok{0}\NormalTok{, }\DecValTok{1}\NormalTok{, }\DecValTok{1}\NormalTok{, }\DecValTok{0}\NormalTok{, }\DecValTok{0}\NormalTok{);    }
\NormalTok{    PDM_DmaEnable(APB_PDM, }\DecValTok{1}\NormalTok{, }\DecValTok{0}\NormalTok{);}
    
    \CommentTok{// DMA setup}
\NormalTok{    DMA_SetUp();}
    
    \CommentTok{// enable DMA and PDM}
\NormalTok{    PDM_DmaEnable(APB_PDM, }\DecValTok{1}\NormalTok{, }\DecValTok{1}\NormalTok{);}
\NormalTok{    PDM_Start(APB_PDM, }\DecValTok{1}\NormalTok{);}
\NormalTok{\}}
\end{Highlighting}
\end{Shaded}

建议采用DMA乒乓搬运的方法进行数据搬运,具体讲解参考手册DMA章节。

\hypertarget{ch-pinctrl}{%
\chapter{管脚管理(PINCTRL)}\label{ch-pinctrl}}

\hypertarget{ux529fux80fdux6982ux8ff0-3}{%
\section{功能概述}\label{ux529fux80fdux6982ux8ff0-3}}

PINCTRL 模块管理芯片所有 IO 管脚的功能,包括外设 IO 的映射,上拉、下拉选择,输入模式控制,
输出驱动能力设置等。

每个 IO 管脚都可以配置为数字或模拟模式,当配置为数字模式时,特性如下:

\begin{itemize}
\tightlist
\item
  每个 IO 管脚可以映射多种不同功能的外设
\item
  每个 IO 管脚都支持上拉或下拉
\item
  每个 IO 管脚都支持施密特触发输入方式
\item
  每个 IO 管脚支持四种输出驱动能力
\end{itemize}

鉴于片内外设丰富、IO 管脚多,进行管脚全映射并不现实,为此,PINCTRL 尽量保证灵活性的前提下做了一定取舍、优化。
部分常用外设的输入、输出功能管脚可与 \(\{{0 .. 17, 21, 22, 31, 34, 35\}}\) 这 23 个常用 IO 之间任意连接(全映射),
这部分常用外设功能管脚总结于表 \ref{tab:ch-pinctrl-common-set}。
表 \ref{tab:ch-pinctrl-mapping} 列出了其它外设功能管脚支持映射到哪些 IO 管脚上。
除此以外,所有 IO 管脚都可以配置为 GPIO 或者 DEBUG 模式。GPIO 模式的输入、输出方向由 \texttt{GIO\_SetDirection} 控制。

\begin{longtable}[]{@{}ll@{}}
\caption{\label{tab:ch-pinctrl-common-set} 支持与常用 IO 全映射的常用功能管脚}\tabularnewline
\toprule
\begin{minipage}[b]{0.09\columnwidth}\raggedright
外设\strut
\end{minipage} & \begin{minipage}[b]{0.85\columnwidth}\raggedright
功能管脚\strut
\end{minipage}\tabularnewline
\midrule
\endfirsthead
\toprule
\begin{minipage}[b]{0.09\columnwidth}\raggedright
外设\strut
\end{minipage} & \begin{minipage}[b]{0.85\columnwidth}\raggedright
功能管脚\strut
\end{minipage}\tabularnewline
\midrule
\endhead
\begin{minipage}[t]{0.09\columnwidth}\raggedright
I2C\strut
\end{minipage} & \begin{minipage}[t]{0.85\columnwidth}\raggedright
I2C0\_SCL\_I, I2C0\_SCL\_O, I2C0\_SDA\_I, I2C0\_SDA\_O, I2C1\_SCL\_I, I2C1\_SCL\_O, I2C1\_SDA\_I, I2C1\_SDA\_O\strut
\end{minipage}\tabularnewline
\begin{minipage}[t]{0.09\columnwidth}\raggedright
I2S\strut
\end{minipage} & \begin{minipage}[t]{0.85\columnwidth}\raggedright
I2S\_BCLK\_I, I2S\_BCLK\_O, I2S\_DIN, I2S\_DOUT, I2S\_LRCLK\_I, I2S\_LRCLK\_O\strut
\end{minipage}\tabularnewline
\begin{minipage}[t]{0.09\columnwidth}\raggedright
IR\strut
\end{minipage} & \begin{minipage}[t]{0.85\columnwidth}\raggedright
IR\_DATIN, IR\_DATOUT, IR\_WAKEUP\strut
\end{minipage}\tabularnewline
\begin{minipage}[t]{0.09\columnwidth}\raggedright
PCAP\strut
\end{minipage} & \begin{minipage}[t]{0.85\columnwidth}\raggedright
PCAP0\_IN, PCAP1\_IN, PCAP2\_IN, PCAP3\_IN, PCAP4\_IN, PCAP5\_IN\strut
\end{minipage}\tabularnewline
\begin{minipage}[t]{0.09\columnwidth}\raggedright
PDM\strut
\end{minipage} & \begin{minipage}[t]{0.85\columnwidth}\raggedright
PDM\_DMIC\_IN, PDM\_DMIC\_MCLK\strut
\end{minipage}\tabularnewline
\begin{minipage}[t]{0.09\columnwidth}\raggedright
QDEC\strut
\end{minipage} & \begin{minipage}[t]{0.85\columnwidth}\raggedright
QDEC\_INDEX, QDEC\_PHASEA, QDEC\_PHASEB\strut
\end{minipage}\tabularnewline
\begin{minipage}[t]{0.09\columnwidth}\raggedright
SPI0\strut
\end{minipage} & \begin{minipage}[t]{0.85\columnwidth}\raggedright
SPI0\_CLK\_IN, SPI0\_CLK\_OUT, SPI0\_CSN\_IN, SPI0\_CSN\_OUT, SPI0\_HOLD\_IN, SPI0\_HOLD\_OUT, SPI0\_MISO\_IN, SPI0\_MISO\_OUT, SPI0\_MOSI\_IN, SPI0\_MOSI\_OUT, SPI0\_WP\_IN, SPI0\_WP\_OUT\strut
\end{minipage}\tabularnewline
\begin{minipage}[t]{0.09\columnwidth}\raggedright
SPI\strut
\end{minipage} & \begin{minipage}[t]{0.85\columnwidth}\raggedright
SPI1\_CLK\_IN, SPI1\_CLK\_OUT, SPI1\_CSN\_IN, SPI1\_CSN\_OUT, SPI1\_HOLD\_IN, SPI1\_HOLD\_OUT, SPI1\_MISO\_IN, SPI1\_MISO\_OUT, SPI1\_MOSI\_IN, SPI1\_MOSI\_OUT, SPI1\_WP\_IN, SPI1\_WP\_OUT\strut
\end{minipage}\tabularnewline
\begin{minipage}[t]{0.09\columnwidth}\raggedright
SWD\strut
\end{minipage} & \begin{minipage}[t]{0.85\columnwidth}\raggedright
SWDO, SW\_TCK, SW\_TMS\strut
\end{minipage}\tabularnewline
\begin{minipage}[t]{0.09\columnwidth}\raggedright
UART0\strut
\end{minipage} & \begin{minipage}[t]{0.85\columnwidth}\raggedright
UART0\_CTS, UART0\_RTS, UART0\_RXD, UART0\_TXD\strut
\end{minipage}\tabularnewline
\begin{minipage}[t]{0.09\columnwidth}\raggedright
UART1\strut
\end{minipage} & \begin{minipage}[t]{0.85\columnwidth}\raggedright
UART1\_CTS, UART1\_RTS, UART1\_RXD, UART1\_TXD\strut
\end{minipage}\tabularnewline
\bottomrule
\end{longtable}

\begin{longtable}[]{@{}ll@{}}
\caption{\label{tab:ch-pinctrl-mapping} 其它外设功能管脚的映射关系}\tabularnewline
\toprule
外设功能管脚 & 可连接到的 IO 管脚\tabularnewline
\midrule
\endfirsthead
\toprule
外设功能管脚 & 可连接到的 IO 管脚\tabularnewline
\midrule
\endhead
KEY\_IN\_COL\_0 & 0, 23\tabularnewline
KEY\_IN\_COL\_1 & 1, 24\tabularnewline
KEY\_IN\_COL\_2 & 2, 25\tabularnewline
KEY\_IN\_COL\_3 & 3, 29\tabularnewline
KEY\_IN\_COL\_4 & 4, 30\tabularnewline
KEY\_IN\_COL\_5 & 5, 31\tabularnewline
KEY\_IN\_COL\_6 & 6, 32\tabularnewline
KEY\_IN\_COL\_7 & 7, 33\tabularnewline
KEY\_IN\_COL\_8 & 8, 34\tabularnewline
KEY\_IN\_COL\_9 & 9, 35\tabularnewline
KEY\_IN\_COL\_10 & 10, 36\tabularnewline
KEY\_IN\_COL\_11 & 11, 37\tabularnewline
KEY\_IN\_COL\_12 & 12, 38\tabularnewline
KEY\_IN\_COL\_13 & 13, 39\tabularnewline
KEY\_IN\_COL\_14 & 14, 40\tabularnewline
KEY\_IN\_COL\_15 & 15, 41\tabularnewline
KEY\_IN\_COL\_16 & 16\tabularnewline
KEY\_IN\_COL\_17 & 17\tabularnewline
KEY\_IN\_COL\_18 & 21\tabularnewline
KEY\_IN\_COL\_19 & 22\tabularnewline
KEY\_OUT\_ROW\_0 & 0, 23\tabularnewline
KEY\_OUT\_ROW\_1 & 1, 24\tabularnewline
KEY\_OUT\_ROW\_2 & 2, 25\tabularnewline
KEY\_OUT\_ROW\_3 & 3, 29\tabularnewline
KEY\_OUT\_ROW\_4 & 4, 30\tabularnewline
KEY\_OUT\_ROW\_5 & 5, 31\tabularnewline
KEY\_OUT\_ROW\_6 & 6, 32\tabularnewline
KEY\_OUT\_ROW\_7 & 7, 33\tabularnewline
KEY\_OUT\_ROW\_8 & 8, 34\tabularnewline
KEY\_OUT\_ROW\_9 & 9, 35\tabularnewline
KEY\_OUT\_ROW\_10 & 10, 36\tabularnewline
KEY\_OUT\_ROW\_11 & 11, 37\tabularnewline
KEY\_OUT\_ROW\_12 & 12, 38\tabularnewline
KEY\_OUT\_ROW\_13 & 13, 39\tabularnewline
KEY\_OUT\_ROW\_14 & 14, 40\tabularnewline
KEY\_OUT\_ROW\_15 & 15, 41\tabularnewline
KEY\_OUT\_ROW\_16 & 16\tabularnewline
KEY\_OUT\_ROW\_17 & 17\tabularnewline
KEY\_OUT\_ROW\_18 & 21\tabularnewline
KEY\_OUT\_ROW\_19 & 22\tabularnewline
ANT\_SW0 & 0, 3, 6, 9, 12, 15, 21, 34\tabularnewline
ANT\_SW1 & 1, 4, 7, 10, 13, 16, 22, 35\tabularnewline
ANT\_SW2 & 2, 5, 8, 11, 14, 17, 31\tabularnewline
ANT\_SW3 & 0, 3, 6, 9, 12, 15, 21, 34\tabularnewline
ANT\_SW4 & 1, 4, 7, 10, 13, 16, 22, 35\tabularnewline
ANT\_SW5 & 2, 5, 8, 11, 14, 17, 31\tabularnewline
ANT\_SW6 & 0, 3, 6, 9, 12, 15, 21, 34\tabularnewline
ANT\_SW7 & 1, 4, 7, 10, 13, 16, 22, 35\tabularnewline
PA\_RXEN & 11, 12, 13, 14, 15, 16, 17, 34, 35\tabularnewline
PA\_TXEN & 4, 5, 6, 7, 8, 9, 10, 34, 35\tabularnewline
TIMER0\_PWM\_0A & 0, 2, 4, 6, 8, 10, 12, 14, 16, 21, 31, 35\tabularnewline
TIMER0\_PWM\_0B & 1, 3, 5, 7, 9, 11, 13, 15, 17, 22, 34\tabularnewline
TIMER0\_PWM\_1A & 0, 2, 4, 6, 8, 10, 12, 14, 16, 21, 31, 35\tabularnewline
TIMER0\_PWM\_1B & 1, 3, 5, 7, 9, 11, 13, 15, 17, 22, 34\tabularnewline
TIMER1\_PWM\_0A & 0, 2, 4, 6, 8, 10, 12, 14, 16, 21, 31, 35\tabularnewline
TIMER1\_PWM\_0B & 1, 3, 5, 7, 9, 11, 13, 15, 17, 22, 34\tabularnewline
TIMER1\_PWM\_1A & 0, 2, 4, 6, 8, 10, 12, 14, 16, 21, 31, 35\tabularnewline
TIMER1\_PWM\_1B & 1, 3, 5, 7, 9, 11, 13, 15, 17, 22, 34\tabularnewline
TIMER2\_PWM\_0A & 0, 2, 4, 6, 8, 10, 12, 14, 16, 21, 31, 35\tabularnewline
TIMER2\_PWM\_0B & 1, 3, 5, 7, 9, 11, 13, 15, 17, 22, 34\tabularnewline
TIMER2\_PWM\_1A & 0, 2, 4, 6, 8, 10, 12, 14, 16, 21, 31, 35\tabularnewline
TIMER2\_PWM\_1B & 1, 3, 5, 7, 9, 11, 13, 15, 17, 22, 34\tabularnewline
PWM\_0A & 0, 2, 4, 6, 8, 10, 12, 14, 16, 21, 31, 35\tabularnewline
PWM\_0B & 1, 3, 5, 7, 9, 11, 13, 15, 17, 22, 34\tabularnewline
PWM\_1A & 0, 2, 4, 6, 8, 10, 12, 14, 16, 21, 31, 35\tabularnewline
PWM\_1B & 1, 3, 5, 7, 9, 11, 13, 15, 17, 22, 34\tabularnewline
PWM\_2A & 0, 2, 4, 6, 8, 10, 12, 14, 16, 21, 31, 35\tabularnewline
PWM\_2B & 1, 3, 5, 7, 9, 11, 13, 15, 17, 22, 34\tabularnewline
QDEC\_EXT\_IN\_CLK & 3, 9\tabularnewline
QDEC\_TIMER\_EXT\_IN1\_A & 1, 7\tabularnewline
QDEC\_TIMER\_EXT\_IN2\_A & 2, 8\tabularnewline
QDEC\_TIMER\_EXT\_IN2\_B & 5, 11\tabularnewline
QDEC\_TIMER\_EXT\_OUT0\_A & 0, 6\tabularnewline
QDEC\_TIMER\_EXT\_OUT1\_A & 1, 7\tabularnewline
QDEC\_TIMER\_EXT\_OUT2\_A & 2, 8\tabularnewline
QDEC\_TIMER\_EXT\_OUT0\_B & 3, 9\tabularnewline
QDEC\_TIMER\_EXT\_OUT1\_B & 4, 10\tabularnewline
QDEC\_TIMER\_EXT\_OUT2\_B & 5, 11\tabularnewline
SPI0\_CLK\_IN & 19\tabularnewline
SPI0\_CLK\_OUT & 19\tabularnewline
SPI0\_CSN\_IN & 18\tabularnewline
SPI0\_CSN\_OUT & 18\tabularnewline
SPI0\_HOLD\_IN & 20\tabularnewline
SPI0\_HOLD\_OUT & 20\tabularnewline
SPI0\_MISO\_IN & 27\tabularnewline
SPI0\_MISO\_OUT & 27\tabularnewline
SPI0\_MOSI\_IN & 28\tabularnewline
SPI0\_MOSI\_OUT & 28\tabularnewline
SPI0\_WP\_IN & 26\tabularnewline
SPI0\_WP\_OUT & 26\tabularnewline
SPI2AHB\_CS & 16\tabularnewline
SPI2AHB\_DI & 17\tabularnewline
SPI2AHB\_DO & 17\tabularnewline
SPI2AHB\_SCLK & 15\tabularnewline
\bottomrule
\end{longtable}

\hypertarget{ux4f7fux7528ux8bf4ux660e-3}{%
\section{使用说明}\label{ux4f7fux7528ux8bf4ux660e-3}}

\hypertarget{ux4e3aux5916ux8bbeux914dux7f6e-io-ux7ba1ux811a}{%
\subsection{为外设配置 IO 管脚}\label{ux4e3aux5916ux8bbeux914dux7f6e-io-ux7ba1ux811a}}

\begin{enumerate}
\def\labelenumi{\arabic{enumi}.}
\item
  将外设输出连接到 IO 管脚

  通过 \texttt{PINCTRL\_SetPadMux} 将外设输出连接到 IO 管脚。
  注意按照表 \ref{tab:ch-pinctrl-common-set} 和 表 \ref{tab:ch-pinctrl-mapping}
  确认硬件是否支持。对于不支持的配置,显然无法生效,函数将返回非 0 值。

\begin{Shaded}
\begin{Highlighting}[]
\DataTypeTok{int}\NormalTok{ PINCTRL_SetPadMux(}
  \DataTypeTok{const} \DataTypeTok{uint8_t}\NormalTok{ io_pin_index, }\CommentTok{// IO 序号 (0 .. IO_PIN_NUMBER - 1)}
  \DataTypeTok{const}\NormalTok{ io_source_t source    }\CommentTok{// IO 源}
\NormalTok{);}
\end{Highlighting}
\end{Shaded}

  例如将 IO 管脚 10 配置为 GPIO 模式:

  \texttt{PINCTRL\_SetPadMux(10,\ IO\_SOURCE\_GPIO);}
\item
  将 IO 管脚连接到外设的输入

  对于有些外设的输入同样通过 \texttt{PINCTRL\_SetPadMux} 配置。对于另一些输入,
  PINCTRL 为不同的外设分别提供了 API 用以配置输入。比如 UART 的数据输入 RXD 和用于硬件流控的
  CTS,需要通过 \texttt{PINCTRL\_SelUartIn} 配置 :

\begin{Shaded}
\begin{Highlighting}[]
\DataTypeTok{int}\NormalTok{ PINCTRL_SelUartIn(}
\NormalTok{  uart_port_t port,     }\CommentTok{// UART 序号}
  \DataTypeTok{uint8_t}\NormalTok{ io_pin_rxd,   }\CommentTok{// 连接到 RXD 输入的 IO 管脚}
  \DataTypeTok{uint8_t}\NormalTok{ io_pin_cts);  }\CommentTok{// 连接到 CTS 输入的 IO 管脚}
\end{Highlighting}
\end{Shaded}

  对于不需要配置的输入,可在对应的参数上填入值 \texttt{IO\_NOT\_A\_PIN}。

  表 \ref{tab:ch-pinctrl-sel-in} 罗列了为各外设提供的输入配置函数。

  \begin{longtable}[]{@{}ll@{}}
  \caption{\label{tab:ch-pinctrl-sel-in} 其它外设功能管脚的映射关系}\tabularnewline
  \toprule
  外设功能管脚 & 可连接到的 IO 管脚\tabularnewline
  \midrule
  \endfirsthead
  \toprule
  外设功能管脚 & 可连接到的 IO 管脚\tabularnewline
  \midrule
  \endhead
  KeyScan & \texttt{PINCTRL\_SelKeyScanColIn}\tabularnewline
  I2C & \texttt{PINCTRL\_SelI2cIn}\tabularnewline
  I2S & \texttt{PINCTRL\_SelI2sIn}\tabularnewline
  IR & \texttt{PINCTRL\_SelIrIn}\tabularnewline
  PDM & \texttt{PINCTRL\_SelPdmIn}\tabularnewline
  PCAP & \texttt{PINCTRL\_SelPCAPIn}\tabularnewline
  QDEC & \texttt{PINCTRL\_SelQDECIn}\tabularnewline
  SWD & \texttt{PINCTRL\_SelSwIn}\tabularnewline
  SPI & \texttt{PINCTRL\_SelSpiIn}\tabularnewline
  UART & \texttt{PINCTRL\_SelUartIn}\tabularnewline
  \bottomrule
  \end{longtable}
\end{enumerate}

\hypertarget{ux914dux7f6eux4e0bux62c9ux4e0bux62c9}{%
\subsection{配置下拉、下拉}\label{ux914dux7f6eux4e0bux62c9ux4e0bux62c9}}

IO 管脚的上拉、下拉模式通过 \texttt{PINCTRL\_Pull} 配置:

\begin{Shaded}
\begin{Highlighting}[]
\DataTypeTok{void}\NormalTok{ PINCTRL_Pull(}
  \DataTypeTok{const} \DataTypeTok{uint8_t}\NormalTok{ io_pin_index,     }\CommentTok{// IO 管脚序号}
  \DataTypeTok{const}\NormalTok{ pinctrl_pull_mode_t mode  }\CommentTok{// 模式}
\NormalTok{  );}
\end{Highlighting}
\end{Shaded}

\hypertarget{ux914dux7f6eux9a71ux52a8ux80fdux529b}{%
\subsection{配置驱动能力}\label{ux914dux7f6eux9a71ux52a8ux80fdux529b}}

通过 \texttt{PINCTRL\_SetDriveStrength} 配置 IO 管脚的驱动能力:

\begin{Shaded}
\begin{Highlighting}[]
\DataTypeTok{void}\NormalTok{ PINCTRL_SetDriveStrength(}
  \DataTypeTok{const} \DataTypeTok{uint8_t}\NormalTok{ io_pin_index,}
  \DataTypeTok{const}\NormalTok{ pinctrl_drive_strength_t strength);}
\end{Highlighting}
\end{Shaded}

\hypertarget{pteux7b80ux4ecb}{%
\chapter{PTE简介}\label{pteux7b80ux4ecb}}

PTE全称Peripheral trigger engine,即外设触发引擎。

其主要作用是使外围设备可以通过其他外围设备或事件独立于CPU进行自主交互。PTE允许外围设备之间可以精确触发。

\hypertarget{ux529fux80fdux63cfux8ff0-4}{%
\section{功能描述}\label{ux529fux80fdux63cfux8ff0-4}}

\hypertarget{ux7279ux70b9-4}{%
\subsection{特点}\label{ux7279ux70b9-4}}

\begin{itemize}
\item
  支持APB总线触发
\item
  支持4通道PTE
\item
  支持复用触发源或复用触发地址
\item
  支持产生CPU中断
\end{itemize}

\hypertarget{pteux539fux7406ux56fe}{%
\subsection{PTE原理图}\label{pteux539fux7406ux56fe}}

\begin{figure}

{\centering \includegraphics[width=1\linewidth]{./img/PTE/PTE_structure} 

}

\caption{PTE原理图}\label{fig:PTE}
\end{figure}

\hypertarget{ux529fux80fd}{%
\subsection{功能}\label{ux529fux80fd}}

PTE具有不同外设之间的可编程内部通道,可以从src外设触发dst外设。PTE可以不依赖CPU而通过硬件的方式触发任务,因此任务可以在同步DFF所占用的周期内启动。

src外设通过pte\_in\_mask配置,dst外设通过pte\_out\_mask配置。在SOC中集成了4个PTE通道,每个通道可以通过通道使能信号来启用/禁用。

当DFF为高时PTE中断将挂起。在清除PTE中断之前,src外设中断必须被清除,否则另一个启动脉冲将发送到dst外设,这可能会产生未知的错误。

\hypertarget{ux4f7fux7528ux65b9ux6cd5-4}{%
\section{使用方法}\label{ux4f7fux7528ux65b9ux6cd5-4}}

\hypertarget{ux65b9ux6cd5ux6982ux8ff0-4}{%
\subsection{方法概述}\label{ux65b9ux6cd5ux6982ux8ff0-4}}

PTE使用方法总结为:建议不使用PTE中断,在dst外设中断里清PTE中断(或关闭PTE通道)。

\textbf{1.} 配置触发外设和被触发外设以及相应中断(被触发外设中断一定要有)

\textbf{2.} 配置要使用的PTE通道寄存器以及中断(建议不使用PTE中断)

\textbf{3.} 使能触发外设,等待PTE中断(如定义)和被触发外设来中断

\textbf{4.} 在PTE中断中清PTE mask(如定义)

\textbf{5.} 在被触发外设中断中清PTE中断(如定义),如果只触发一次则直接关闭PTE通道

\hypertarget{ux6ce8ux610fux70b9-4}{%
\subsection{注意点}\label{ux6ce8ux610fux70b9-4}}

\begin{itemize}
\item
  不清src中断会循环通过PTE触发dst外设,使程序陷入死循环
\item
  不清PTE中断会循环触发dst外设,使程序陷入死循环
\item
  PTE中断优先级低容易被打断,在极端情况下如果dst外设来中断非常快会出问题(一般不会)
\item
  使用PTE中断会更多占用CPU资源并增加触发过程操作复杂度、增加出错风险,中断处理程序完全可以在src和dst中断中完成,所以强烈建议不要使用PTE中断
\end{itemize}

\hypertarget{ux7f16ux7a0bux6307ux5357-4}{%
\section{编程指南}\label{ux7f16ux7a0bux6307ux5357-4}}

\hypertarget{srcdstux5916ux8bbe}{%
\subsection{src\&dst外设}\label{srcdstux5916ux8bbe}}

当前PTE支持的src外设定义在SYSCTRL\_PTE\_SRC\_INT中:

\begin{Shaded}
\begin{Highlighting}[]
\KeywordTok{typedef} \KeywordTok{enum}
\NormalTok{\{}
\NormalTok{    SYSCTRL_PTE_I2C0_INT       = }\DecValTok{0}\NormalTok{,}
\NormalTok{    SYSCTRL_PTE_I2C1_INT       = }\DecValTok{1}\NormalTok{,}
\NormalTok{    SYSCTRL_PTE_SARADC_INT     = }\DecValTok{2}\NormalTok{,}
\NormalTok{    SYSCTRL_PTE_I2S_INT        = }\DecValTok{3}\NormalTok{,}
\NormalTok{    SYSCTRL_PTE_DMA_INT        = }\DecValTok{4}\NormalTok{,}
\NormalTok{    SYSCTRL_PTE_IR_INT         = }\DecValTok{5}\NormalTok{,}
\NormalTok{    SYSCTRL_PTE_KEYSCANNER_INT = }\DecValTok{6}\NormalTok{,}
\NormalTok{    SYSCTRL_PTE_PWMC0_INT      = }\DecValTok{7}\NormalTok{,}
\NormalTok{    SYSCTRL_PTE_PWMC1_INT      = }\DecValTok{8}\NormalTok{,}
\NormalTok{    SYSCTRL_PTE_PWMC2_INT      = }\DecValTok{9}\NormalTok{,}
\NormalTok{    SYSCTRL_PTE_TIMER0_INT     = }\DecValTok{10}\NormalTok{,}
\NormalTok{    SYSCTRL_PTE_TIMER1_INT     = }\DecValTok{11}\NormalTok{,}
\NormalTok{    SYSCTRL_PTE_TIMER2_INT     = }\DecValTok{12}\NormalTok{,}
\NormalTok{    SYSCTRL_PTE_GPIO0_INT      = }\DecValTok{13}\NormalTok{,}
\NormalTok{    SYSCTRL_PTE_GPIO1_INT      = }\DecValTok{14}\NormalTok{,}
\NormalTok{    SYSCTRL_PTE_UART0_INT      = }\DecValTok{15}\NormalTok{,}
\NormalTok{    SYSCTRL_PTE_UART1_INT      = }\DecValTok{16}\NormalTok{,}
\NormalTok{    SYSCTRL_PTE_SPI0_INT       = }\DecValTok{17}\NormalTok{,}
\NormalTok{    SYSCTRL_PTE_SPI1_INT       = }\DecValTok{18}\NormalTok{,}
\NormalTok{    SYSCTRL_PTE_SPIFLASH       = }\DecValTok{19}\NormalTok{,}
\NormalTok{    SYSCTRL_PTE_RCT_CNT        = }\DecValTok{20}\NormalTok{,}
\NormalTok{    SYSCTRL_PTE_IR_WAKEUP      = }\DecValTok{21}\NormalTok{,}
\NormalTok{    SYSCTRL_PTE_USB_INT        = }\DecValTok{22}\NormalTok{,}
\NormalTok{    SYSCTRL_PTE_QDEC_INT       = }\DecValTok{23}\NormalTok{,}

\NormalTok{    SYSCTRL_PTE_SRC_INT_MAX    = }\DecValTok{24}\NormalTok{,}
\NormalTok{\} SYSCTRL_PTE_SRC_INT;}
\end{Highlighting}
\end{Shaded}

dst外设定义在SYSCTRL\_PTE\_DST\_EN中:

\begin{Shaded}
\begin{Highlighting}[]
\KeywordTok{typedef} \KeywordTok{enum}
\NormalTok{\{}
\NormalTok{    SYSCTRL_PTE_I2C0_EN        = }\DecValTok{0}\NormalTok{,}
\NormalTok{    SYSCTRL_PTE_I2C1_EN        = }\DecValTok{1}\NormalTok{,}
\NormalTok{    SYSCTRL_PTE_SARADC_EN      = }\DecValTok{2}\NormalTok{,}
\NormalTok{    SYSCTRL_PTE_I2S_TX_EN      = }\DecValTok{3}\NormalTok{,}
\NormalTok{    SYSCTRL_PTE_I2S_RX_EN      = }\DecValTok{4}\NormalTok{,}
\NormalTok{    SYSCTRL_PTE_IR_EN          = }\DecValTok{5}\NormalTok{,}
\NormalTok{    SYSCTRL_PTE_KEYSCANNER_EN  = }\DecValTok{6}\NormalTok{,}
\NormalTok{    SYSCTRL_PTE_PWMC0_EN       = }\DecValTok{7}\NormalTok{,}
\NormalTok{    SYSCTRL_PTE_PWMC1_EN       = }\DecValTok{8}\NormalTok{,}
\NormalTok{    SYSCTRL_PTE_PWMC2_EN       = }\DecValTok{9}\NormalTok{,}
\NormalTok{    SYSCTRL_PTE_TIMER0_CH0_EN  = }\DecValTok{10}\NormalTok{,}
\NormalTok{    SYSCTRL_PTE_TIMER0_CH1_EN  = }\DecValTok{11}\NormalTok{,}
\NormalTok{    SYSCTRL_PTE_TIMER1_CH0_EN  = }\DecValTok{12}\NormalTok{,}
\NormalTok{    SYSCTRL_PTE_TIMER1_CH1_EN  = }\DecValTok{13}\NormalTok{,}
\NormalTok{    SYSCTRL_PTE_TIMER2_CH0_EN  = }\DecValTok{14}\NormalTok{,}
\NormalTok{    SYSCTRL_PTE_TIMER2_CH1_EN  = }\DecValTok{15}\NormalTok{,}

\NormalTok{    SYSCTRL_PTE_DST_EN_MAX     = }\DecValTok{16}\NormalTok{,}
\NormalTok{\} SYSCTRL_PTE_DST_EN;}
\end{Highlighting}
\end{Shaded}

通过PTE连接的src外设和dst外设需要在已注册枚举中选取。

\hypertarget{ux9a71ux52a8ux63a5ux53e3-4}{%
\subsection{驱动接口}\label{ux9a71ux52a8ux63a5ux53e3-4}}

\begin{itemize}
\item
  PTE\_ConnectPeripheral:PTE外设连接接口
\item
  PTE\_EnableChennel:PTE通道使能接口
\item
  PTE\_ChennelClose:PTE通道关闭接口
\item
  PTE\_IrqProcess:PTE标准中断程序接口
\item
  PTE\_OutPeripheralContinueProcess:dst外设中断标准PTE中继触发接口
\item
  PTE\_OutPeripheralEndProcess:dst外设中断标准PTE结束接口
\end{itemize}

\hypertarget{ux4ee3ux7801ux793aux4f8b-4}{%
\subsection{代码示例}\label{ux4ee3ux7801ux793aux4f8b-4}}

下面以Timer0通过PTE通道0触发Timer1为例展示PTE的具体使用方法。

src外设和dst外设配置方法不在本文档介绍范围内,我们默认Timer0和Timer1已经配置好并注册好中断。

\begin{Shaded}
\begin{Highlighting}[]
\DataTypeTok{uint32_t}\NormalTok{ Timer0Isr(}\DataTypeTok{void}\NormalTok{ *user_data)}
\NormalTok{\{}
\NormalTok{    TMR_IntClr(APB_TMR0);}
    \ControlFlowTok{return} \DecValTok{0}\NormalTok{;}
\NormalTok{\}}

\DataTypeTok{uint32_t}\NormalTok{ Timer1Isr(}\DataTypeTok{void}\NormalTok{ *user_data)}
\NormalTok{\{}
\NormalTok{    TMR_IntClr(APB_TMR1);}
\NormalTok{    PTE_OutPeripheralContinueProcess(}\DecValTok{0}\NormalTok{);}
    \ControlFlowTok{return} \DecValTok{0}\NormalTok{;}
\NormalTok{\}}

\CommentTok{// 仅供参考,不建议注册PTE中断}
\DataTypeTok{uint32_t}\NormalTok{ PTE0Isr(}\DataTypeTok{void}\NormalTok{ *user_data)}
\NormalTok{\{}
\NormalTok{    PTE_IrqProcess(}\DecValTok{0}\NormalTok{);}
    \ControlFlowTok{return} \DecValTok{0}\NormalTok{;}
\NormalTok{\}}

\DataTypeTok{void}\NormalTok{ PTE_Test(}\DataTypeTok{void}\NormalTok{)}
\NormalTok{\{}
\NormalTok{    PTE_ConnectPeripheral(SYSCTRL_PTE_CHENNEL_0, }
\NormalTok{                          SYSCTRL_PTE_TIMER0_INT, }
\NormalTok{                          SYSCTRL_PTE_TIMER1_CH0_EN);}
\NormalTok{    TMR_Enable(APB_TMR0);}
\NormalTok{\}}
\end{Highlighting}
\end{Shaded}

上面示例会保留PTE通道0并等待下一次触发。如果想要触发之后直接关闭通道代码如下:

\begin{Shaded}
\begin{Highlighting}[]
\DataTypeTok{uint32_t}\NormalTok{ Timer1Isr(}\DataTypeTok{void}\NormalTok{ *user_data)}
\NormalTok{\{}
\NormalTok{    TMR_IntClr(APB_TMR1);}
\NormalTok{    PTE_OutPeripheralEndProcess(}\DecValTok{0}\NormalTok{);}
    \ControlFlowTok{return} \DecValTok{0}\NormalTok{;}
\NormalTok{\}}
\end{Highlighting}
\end{Shaded}

关闭通道会断开Timer0和Timer1的连接,再次触发需要重新建立连接。

\hypertarget{ch-pwm}{%
\chapter{增强型脉宽调制发生器(PWM)}\label{ch-pwm}}

增强型脉宽调制发生器具有两大功能:生成脉宽调制信号(PWM),捕捉外部脉冲输入(PCAP)。
增强型脉宽调制发生器具备 3 个通道,每个通道都可以单独配置为 PWM 或者 PCAP 模式。
每个通道拥有独立的 FIFO。FIFO 里的每个存储单元为 2 个 20bit 数据。
FIFO 深度为 4,即最多存储 4 个单元,共 \(8 \times 20bit\) 数据。
这里的 20bit 位宽是因为本硬件模块内部 PWM 使用的各计数器都是 20 比特。
可根据 FIFO 内的数据量触发中断或者 DMA 传输。

说明:\protect\hyperlink{ch-timer}{TIMER} 也支持生成脉宽调制信号,但是可配置的参数较简单,不支持死区等。

PWM 特性:

\begin{itemize}
\tightlist
\item
  最多支持 3 个 PWM 通道,每一个通道包含 A、B 两个输出
\item
  每个通道参数独立
\item
  支持死区
\item
  支持通过 DMA 更新 PWM 配置
\end{itemize}

PCAP 特性:

\begin{itemize}
\tightlist
\item
  支持 3 个 PCAP 通道,每一个通道包含两个输入
\item
  支持捕捉上升沿、下降沿
\item
  支持通过 DMA 读取数据
\end{itemize}

\hypertarget{pwm-ux5de5ux4f5cux6a21ux5f0f}{%
\section{PWM 工作模式}\label{pwm-ux5de5ux4f5cux6a21ux5f0f}}

PWM 使用的时钟频率可配置,请参考 \protect\hyperlink{ch-sysctrl}{SYSCTRL}。

每个 PWM 通道支持以下多种工作模式:

\begin{Shaded}
\begin{Highlighting}[]
\KeywordTok{typedef} \KeywordTok{enum}
\NormalTok{\{}
\NormalTok{    ..._UP_WITHOUT_DIED_ZONE          = ...,}
\NormalTok{    ..._UP_WITH_DIED_ZONE             = ...,}
\NormalTok{    ..._UPDOWN_WITHOUT_DIED_ZONE      = ...,}
\NormalTok{    ..._UPDOWN_WITH_DIED_ZONE         = ...,}
\NormalTok{    ..._SINGLE_WITHOUT_DIED_ZONE      = ...,}
\NormalTok{    ..._DMA                           = ...,}
\NormalTok{\} PWM_WorkMode_t;}
\end{Highlighting}
\end{Shaded}

\hypertarget{ux6700ux7b80ux5355ux7684ux6a21ux5f0fup_without_died_zone}{%
\subsection{最简单的模式:UP\_WITHOUT\_DIED\_ZONE}\label{ux6700ux7b80ux5355ux7684ux6a21ux5f0fup_without_died_zone}}

此模式需要配置两个门限:计数器回零门限 PERA\_TH、高门限 HIGH\_TH,HIGH\_TH
必须小于 HIGH\_TH。以伪代码描述 A、B 输出如下:

\begin{Shaded}
\begin{Highlighting}[]
\NormalTok{cnt = }\DecValTok{0}\NormalTok{;}
\NormalTok{on_clock_rising_edge()}
\NormalTok{\{}
\NormalTok{    cnt = cnt < PERA_TH ? cnt + }\DecValTok{1}\NormalTok{ : }\DecValTok{0}\NormalTok{;}
\NormalTok{    A = HIGH_TH <= cnt;}
\NormalTok{    B = !A;}
\NormalTok{\}}
\end{Highlighting}
\end{Shaded}

\hypertarget{up_with_died_zone}{%
\subsection{UP\_WITH\_DIED\_ZONE}\label{up_with_died_zone}}

与 UP\_WITHOUT\_DIED\_ZONE 相比,此模式需要一个新的死区门限 DZONE\_TH,DZONE\_TH
必须小于 HIGH\_TH。以伪代码描述 A、B 输出如下:

cnt = 0;
on\_clock\_rising\_edge()
\{
cnt = cnt \textless{} PERA\_TH ? cnt + 1 : 0;
A = HIGH\_TH + DZONE\_TH \textless= cnt;
B = DZONE\_TH \textless= cnt \textless{} HIGH\_TH);
\}

\hypertarget{updown_without_died_zone}{%
\subsection{UPDOWN\_WITHOUT\_DIED\_ZONE}\label{updown_without_died_zone}}

此模式需要的门限参数与 UP\_WITHOUT\_DIED\_ZONE 相同。以伪代码描述 A、B 输出如下:

\begin{Shaded}
\begin{Highlighting}[]
\NormalTok{cnt = }\DecValTok{0}\NormalTok{;}
\NormalTok{on_clock_rising_edge()}
\NormalTok{\{}
\NormalTok{    cnt = cnt < }\DecValTok{2}\NormalTok{ * PERA_TH ? cnt + }\DecValTok{1}\NormalTok{ : }\DecValTok{0}\NormalTok{;}
\NormalTok{    A = PERA_TH - HIGH_TH <= cnt <= PERA_TH + HIGH_TH;}
\NormalTok{    B = !A;}
\NormalTok{\}}
\end{Highlighting}
\end{Shaded}

\hypertarget{updown_with_died_zone}{%
\subsection{UPDOWN\_WITH\_DIED\_ZONE}\label{updown_with_died_zone}}

与 UP\_WITHOUT\_DIED\_ZONE 相比,此模式需要一个新的死区门限 DZONE\_TH。
以伪代码描述 A、B 输出如下:

\begin{Shaded}
\begin{Highlighting}[]
\NormalTok{cnt = }\DecValTok{0}\NormalTok{;}
\NormalTok{on_clock_rising_edge()}
\NormalTok{\{}
\NormalTok{    cnt = cnt < }\DecValTok{2}\NormalTok{ * PERA_TH ? cnt + }\DecValTok{1}\NormalTok{ : }\DecValTok{0}\NormalTok{;}
\NormalTok{    A = PERA_TH - HIGH_TH + DZONE_TH <= cnt <= PERA_TH + HIGH_TH;}
\NormalTok{    B = (cnt < PERA_TH - HIGH_TH) || (cnt > PERA_TH + HIGH_TH + DZONE_TH);}
\NormalTok{\}}
\end{Highlighting}
\end{Shaded}

\hypertarget{single_without_died_zone}{%
\subsection{SINGLE\_WITHOUT\_DIED\_ZONE}\label{single_without_died_zone}}

此模式需要配置两个门限:计数器回零门限 PERA\_TH、高门限 HIGH\_TH,HIGH\_TH
必须小于 HIGH\_TH。此模式只产生一个脉冲,以伪代码描述 A、B 输出如下:

\begin{Shaded}
\begin{Highlighting}[]
\NormalTok{cnt = }\DecValTok{0}\NormalTok{;}
\NormalTok{on_clock_rising_edge()}
\NormalTok{\{}
\NormalTok{    cnt++;}
\NormalTok{    A = HIGH_TH <= cnt < PERA_TH;}
\NormalTok{    B = !A;}
\NormalTok{\}}
\end{Highlighting}
\end{Shaded}

\begin{rmdcaution}
以上伪代码仅用于辅助描述硬件行为,与实际行为可以存在微小差异。
\end{rmdcaution}

\hypertarget{dma-ux6a21ux5f0f}{%
\subsection{DMA 模式}\label{dma-ux6a21ux5f0f}}

此模式支持通过 DMA 实时更新门限。

\hypertarget{ux8f93ux51faux63a7ux5236}{%
\subsection{输出控制}\label{ux8f93ux51faux63a7ux5236}}

对于每个通道的每一路输出,另有 3 个参数控制最终的两路输出:掩膜、停机输出值、反相。
最终的输出以伪代码描述如下:

\begin{Shaded}
\begin{Highlighting}[]
\NormalTok{output_control(v)}
\NormalTok{\{}
    \ControlFlowTok{if}\NormalTok{ (掩膜 == }\DecValTok{1}\NormalTok{) }\ControlFlowTok{return}\NormalTok{ A 路输出 }\DecValTok{0}\NormalTok{、B 路输出 }\DecValTok{1}\NormalTok{;}
    \ControlFlowTok{if}\NormalTok{ (本通道已停机) }\ControlFlowTok{return}\NormalTok{ 停机输出值;}
    \ControlFlowTok{if}\NormalTok{ (反相) v = !v;}
    \ControlFlowTok{return}\NormalTok{ v;}
\NormalTok{\}}
\end{Highlighting}
\end{Shaded}

\hypertarget{pcap}{%
\section{PCAP}\label{pcap}}

PCAP 每个通道包含两路输入。PCAP 内部有一个单独的 32 比特计数器\footnote{所有 6 路输入共有此计数器。},
当检测到输入信号变化(包含上升沿和下降沿)时,PCAP 将计数器的值及边沿变化信息作为一个存储单元压入
FIFO:

\begin{Shaded}
\begin{Highlighting}[]
\KeywordTok{struct}\NormalTok{ data0}
\NormalTok{\{}
    \DataTypeTok{uint32_t}\NormalTok{ cnt_high:}\DecValTok{12}\NormalTok{;}
    \DataTypeTok{uint32_t}\NormalTok{ p_cap_0_p:}\DecValTok{1}\NormalTok{; }\CommentTok{// A 路出现上升沿}
    \DataTypeTok{uint32_t}\NormalTok{ p_cap_0_n:}\DecValTok{1}\NormalTok{; }\CommentTok{// A 路出现下降沿}
    \DataTypeTok{uint32_t}\NormalTok{ p_cap_1_p:}\DecValTok{1}\NormalTok{; }\CommentTok{// B 路出现上升沿}
    \DataTypeTok{uint32_t}\NormalTok{ p_cap_1_n:}\DecValTok{1}\NormalTok{; }\CommentTok{// B 路出现下降沿}
    \DataTypeTok{uint32_t}\NormalTok{ tag:}\DecValTok{4}\NormalTok{;}
    \DataTypeTok{uint32_t}\NormalTok{ padding:}\DecValTok{12}\NormalTok{;}
\NormalTok{\};}
\KeywordTok{struct}\NormalTok{ data1}
\NormalTok{\{}
    \DataTypeTok{uint32_t}\NormalTok{ cnt_low:}\DecValTok{20}\NormalTok{;}
    \DataTypeTok{uint32_t}\NormalTok{ padding:}\DecValTok{12}\NormalTok{;}
\NormalTok{\};}
\end{Highlighting}
\end{Shaded}

通过复位整个模块可以清零 PCAP 计数器。

\hypertarget{pwm-ux4f7fux7528ux8bf4ux660e}{%
\section{PWM 使用说明}\label{pwm-ux4f7fux7528ux8bf4ux660e}}

\hypertarget{ux542fux52a8ux4e0eux505cux6b62}{%
\subsection{启动与停止}\label{ux542fux52a8ux4e0eux505cux6b62}}

共有两个开关与 PWM 的启动和停止有关:使能(Enable)、停机控制(HaltCtrl)。只有当 \texttt{Enable} 为 1,
\texttt{HaltCtrl} 为 0 时,PWM 才真正开始工作。

相关的 API 为:

\begin{Shaded}
\begin{Highlighting}[]
\CommentTok{// 使能 PWM 通道}
\DataTypeTok{void}\NormalTok{ PWM_Enable(}
    \DataTypeTok{const} \DataTypeTok{uint8_t}\NormalTok{ channel_index,    }\CommentTok{// 通道号}
    \DataTypeTok{const} \DataTypeTok{uint8_t}\NormalTok{ enable            }\CommentTok{// 使能或禁用}
\NormalTok{    );}

\CommentTok{// PWM 通道停机控制}
\DataTypeTok{void}\NormalTok{ PWM_HaltCtrlEnable(}
    \DataTypeTok{const} \DataTypeTok{uint8_t}\NormalTok{ channel_index,    }\CommentTok{// 通道号}
    \DataTypeTok{const} \DataTypeTok{uint8_t}\NormalTok{ enable            }\CommentTok{// 停机(1) 或运转(0)}
\NormalTok{    );}
\end{Highlighting}
\end{Shaded}

\hypertarget{ux914dux7f6eux5de5ux4f5cux6a21ux5f0f}{%
\subsection{配置工作模式}\label{ux914dux7f6eux5de5ux4f5cux6a21ux5f0f}}

\begin{Shaded}
\begin{Highlighting}[]
\DataTypeTok{void}\NormalTok{ PWM_SetMode(}
    \DataTypeTok{const} \DataTypeTok{uint8_t}\NormalTok{ channel_index,    }\CommentTok{// 通道号}
    \DataTypeTok{const}\NormalTok{ PWM_WorkMode_t mode       }\CommentTok{// 模式}
\NormalTok{    );}
\end{Highlighting}
\end{Shaded}

\hypertarget{ux914dux7f6eux95e8ux9650}{%
\subsection{配置门限}\label{ux914dux7f6eux95e8ux9650}}

\begin{Shaded}
\begin{Highlighting}[]
\CommentTok{// 配置 PERA_TH}
\DataTypeTok{void}\NormalTok{ PWM_SetPeraThreshold(}
    \DataTypeTok{const} \DataTypeTok{uint8_t}\NormalTok{ channel_index,}
    \DataTypeTok{const} \DataTypeTok{uint32_t}\NormalTok{ threshold);}
\end{Highlighting}
\end{Shaded}

\begin{Shaded}
\begin{Highlighting}[]
\CommentTok{// 配置 DZONE_TH}
\DataTypeTok{void}\NormalTok{ PWM_SetDiedZoneThreshold(}
    \DataTypeTok{const} \DataTypeTok{uint8_t}\NormalTok{ channel_index,}
    \DataTypeTok{const} \DataTypeTok{uint32_t}\NormalTok{ threshold);}
\end{Highlighting}
\end{Shaded}

\begin{Shaded}
\begin{Highlighting}[]
\CommentTok{// 配置 HIGH_TH}
\DataTypeTok{void}\NormalTok{ PWM_SetHighThreshold(}
    \DataTypeTok{const} \DataTypeTok{uint8_t}\NormalTok{ channel_index,}
    \DataTypeTok{const} \DataTypeTok{uint8_t}\NormalTok{ multi_duty_index, }\CommentTok{// 对于 ING916XX,此参数无效}
    \DataTypeTok{const} \DataTypeTok{uint32_t}\NormalTok{ threshold);}
\end{Highlighting}
\end{Shaded}

各门限值最大支持 0xFFFFF,共 20 个比特。

\hypertarget{ux8f93ux51faux63a7ux5236-1}{%
\subsection{输出控制}\label{ux8f93ux51faux63a7ux5236-1}}

\begin{Shaded}
\begin{Highlighting}[]
\CommentTok{// 掩膜控制}
\DataTypeTok{void}\NormalTok{ PWM_SetMask(}
    \DataTypeTok{const} \DataTypeTok{uint8_t}\NormalTok{ channel_index,    }\CommentTok{// 通道号}
    \DataTypeTok{const} \DataTypeTok{uint8_t}\NormalTok{ mask_a,           }\CommentTok{// A 路掩膜}
    \DataTypeTok{const} \DataTypeTok{uint8_t}\NormalTok{ mask_b            }\CommentTok{// B 路掩膜}
\NormalTok{    );}
\end{Highlighting}
\end{Shaded}

\begin{Shaded}
\begin{Highlighting}[]
\CommentTok{// 配置停机输出值}
\DataTypeTok{void}\NormalTok{ PWM_HaltCtrlCfg(}
    \DataTypeTok{const} \DataTypeTok{uint8_t}\NormalTok{ channel_index,    }\CommentTok{// 通道号}
    \DataTypeTok{const} \DataTypeTok{uint8_t}\NormalTok{ out_a,            }\CommentTok{// A 路停机输出值}
    \DataTypeTok{const} \DataTypeTok{uint8_t}\NormalTok{ out_b             }\CommentTok{// B 路停机输出值}
\NormalTok{    );}
\end{Highlighting}
\end{Shaded}

\begin{Shaded}
\begin{Highlighting}[]
\CommentTok{// 反相}
\DataTypeTok{void}\NormalTok{ PWM_SetInvertOutput(}
    \DataTypeTok{const} \DataTypeTok{uint8_t}\NormalTok{ channel_index,    }\CommentTok{// 通道号}
    \DataTypeTok{const} \DataTypeTok{uint8_t}\NormalTok{ inv_a,            }\CommentTok{// A 路是否反相}
    \DataTypeTok{const} \DataTypeTok{uint8_t}\NormalTok{ inv_b             }\CommentTok{// B 路是否反相}
\NormalTok{    );}
\end{Highlighting}
\end{Shaded}

\hypertarget{ux7efcux5408ux793aux4f8b} 的方波,
涉及 3 个关键参数:

\begin{itemize}
\item
  生成这种最简单的 PWM 信号需要的模式为 UP\_WITHOUT\_DIED\_ZONE;
\item
  PERA\_TH 控制输出信号的频率,设置为 \texttt{PWM\_CLOCK\_FREQ\ /\ frequency};
\item
  HIGH\_TH 控制信号的占空比,设置为 \texttt{PERA\_TH\ *\ (100\ -\ on\_duty)\ \%}
\end{itemize}

\begin{Shaded}
\begin{Highlighting}[]
\DataTypeTok{void}\NormalTok{ PWM_SetupSimple(}
    \DataTypeTok{const} \DataTypeTok{uint8_t}\NormalTok{ channel_index,}
    \DataTypeTok{const} \DataTypeTok{uint32_t}\NormalTok{ frequency,}
    \DataTypeTok{const} \DataTypeTok{uint16_t}\NormalTok{ on_duty)}
\NormalTok{\{}
    \DataTypeTok{uint32_t}\NormalTok{ pera = PWM_CLOCK_FREQ / frequency;}
    \DataTypeTok{uint32_t}\NormalTok{ high = pera > }\DecValTok{1000}\NormalTok{ ?}
\NormalTok{          pera / }\DecValTok{100}\NormalTok{ * (}\DecValTok{100}\NormalTok{ - on_duty)}
\NormalTok{        : pera * (}\DecValTok{100}\NormalTok{ - on_duty) / }\DecValTok{100}\NormalTok{;}
\NormalTok{    PWM_HaltCtrlEnable(channel_index, }\DecValTok{1}\NormalTok{);}
\NormalTok{    PWM_Enable(channel_index, }\DecValTok{0}\NormalTok{);}
\NormalTok{    PWM_SetPeraThreshold(channel_index, pera);}
\NormalTok{    PWM_SetHighThreshold(channel_index, }\DecValTok{0}\NormalTok{, high);}
\NormalTok{    PWM_SetMode(channel_index, PWM_WORK_MODE_UP_WITHOUT_DIED_ZONE);}
\NormalTok{    PWM_SetMask(channel_index, }\DecValTok{0}\NormalTok{, }\DecValTok{0}\NormalTok{);}
\NormalTok{    PWM_Enable(channel_index, }\DecValTok{1}\NormalTok{);}
\NormalTok{    PWM_HaltCtrlEnable(channel_index, }\DecValTok{0}\NormalTok{);}
\NormalTok{\}}
\end{Highlighting}
\end{Shaded}

\hypertarget{ux4f7fux7528-dma-ux5b9eux65f6ux66f4ux65b0ux914dux7f6e}{%
\subsection{使用 DMA 实时更新配置}\label{ux4f7fux7528-dma-ux5b9eux65f6ux66f4ux65b0ux914dux7f6e}}

使用 DMA 能够实时更新配置(相当于工作在 UP\_WITHOUT\_DIED\_ZONE,但是每个循环使用不同的参数):
每当 PWM 计数器计完一圈回零时,自动使用来自 DMA 的数据更新配置。
这些数据以 2 个 \texttt{uint32\_t} 为一组,依次表示 HIGH\_TH 和 PERA\_TH。

\begin{Shaded}
\begin{Highlighting}[]
\DataTypeTok{void}\NormalTok{ PWM_DmaEnable(}
    \DataTypeTok{const} \DataTypeTok{uint8_t}\NormalTok{ channel_index, }\CommentTok{// 通道号}
    \DataTypeTok{uint8_t}\NormalTok{ trig_cfg,            }\CommentTok{// DMA 请求触发门限}
    \DataTypeTok{uint8_t}\NormalTok{ enable               }\CommentTok{// 使能}
\NormalTok{    );}
\end{Highlighting}
\end{Shaded}

当 PWM 内部 FIFO 数据少于 \texttt{trig\_cfg},PWM 请求 DMA 传输数据。PWM FIFO 深度为 4(指可以存储 4 组 PWM 配置),
所以 \texttt{trig\_cfg} 的取值范围为 \(1..4\)。

\hypertarget{pcap-ux4f7fux7528ux8bf4ux660e}{%
\section{PCAP 使用说明}\label{pcap-ux4f7fux7528ux8bf4ux660e}}

\hypertarget{ux914dux7f6e-pcap-ux6a21ux5f0f}{%
\subsection{配置 PCAP 模式}\label{ux914dux7f6e-pcap-ux6a21ux5f0f}}

要启用 PCAP 模式,需要 5 个步骤:

\begin{enumerate}
\def\labelenumi{\arabic{enumi}.}
\item
  关闭整个模块的时钟(参考 \protect\hyperlink{ch-sysctrl}{SYSCTRL})
\item
  使用 \texttt{PCAP\_Enable} 使能 PCAP 模式

\begin{Shaded}
\begin{Highlighting}[]
\DataTypeTok{void}\NormalTok{ PCAP_Enable(}
    \DataTypeTok{const} \DataTypeTok{uint8_t}\NormalTok{ channel_index     }\CommentTok{// 通道号}
\NormalTok{);}
\end{Highlighting}
\end{Shaded}
\item
  使用 \texttt{PCAP\_EnableEvents} 选择要检测的事件

\begin{Shaded}
\begin{Highlighting}[]
\DataTypeTok{void}\NormalTok{ PCAP_EnableEvents(}
    \DataTypeTok{const} \DataTypeTok{uint8_t}\NormalTok{ channel_index,}
    \DataTypeTok{uint8_t}\NormalTok{ events_on_0,}
    \DataTypeTok{uint8_t}\NormalTok{ events_on_1);}
\end{Highlighting}
\end{Shaded}

  \texttt{events} 为下面两个事件的组合:

\begin{Shaded}
\begin{Highlighting}[]
\KeywordTok{enum}\NormalTok{ PCAP_PULSE_EVENT}
\NormalTok{\{}
\NormalTok{    PCAP_PULSE_RISING_EDGE  = }\BaseNTok{0x1}\NormalTok{,}
\NormalTok{    PCAP_PULSE_FALLING_EDGE = }\BaseNTok{0x2}\NormalTok{,}
\NormalTok{\};}
\end{Highlighting}
\end{Shaded}

  比如在通道 1 的 A 路输入上同时检测、上报上升沿和下降沿:

\begin{Shaded}
\begin{Highlighting}[]
\NormalTok{PCAP_EnableEvents(}\DecValTok{1}\NormalTok{,}
\NormalTok{    PCAP_PULSE_RISING_EDGE}
\NormalTok{    | PCAP_PULSE_FALLING_EDGE,}
\NormalTok{    ...);}
\end{Highlighting}
\end{Shaded}
\item
  打开整个模块的时钟(参考 \protect\hyperlink{ch-sysctrl}{SYSCTRL})
\item
  配置 DMA 传输

  当 PCAP 通道 FIFO 内存储的数据多于 \texttt{trig\_cfg},请求 DMA 传输数据。\texttt{trig\_cfg} 的取值范围为 \(0..4\)。

\begin{Shaded}
\begin{Highlighting}[]
\DataTypeTok{void}\NormalTok{ PCAP_DmaEnable(}
    \DataTypeTok{const} \DataTypeTok{uint8_t}\NormalTok{ channel_index, }\CommentTok{// 通道号}
    \DataTypeTok{uint8_t}\NormalTok{ trig_cfg,            }\CommentTok{// DMA 请求触发门限}
    \DataTypeTok{uint8_t}\NormalTok{ enable               }\CommentTok{// 使能}
\NormalTok{    );}
\end{Highlighting}
\end{Shaded}
\item
  使能计数器

\begin{Shaded}
\begin{Highlighting}[]
\DataTypeTok{void}\NormalTok{ PCAP_CounterEnable(}
    \DataTypeTok{uint8_t}\NormalTok{ enable              }\CommentTok{// 使能(1)/禁用(0)}
\NormalTok{    );}
\end{Highlighting}
\end{Shaded}
\end{enumerate}

\hypertarget{ux8bfbux53d6ux8ba1ux6570ux5668}{%
\subsection{读取计数器}\label{ux8bfbux53d6ux8ba1ux6570ux5668}}

\begin{Shaded}
\begin{Highlighting}[]
\DataTypeTok{uint32_t}\NormalTok{ PCAP_ReadCounter(}\DataTypeTok{void}\NormalTok{);}
\end{Highlighting}
\end{Shaded}

\hypertarget{ch-rtc}{%
\chapter{实时时钟(RTC)}\label{ch-rtc}}

\hypertarget{ux529fux80fdux63cfux8ff0-5}{%
\section{功能描述}\label{ux529fux80fdux63cfux8ff0-5}}

实时时钟是一个独立的定时器。RTC模块拥有一组连续计数的计数器,在相应的软件配置下,
可提供时钟日历的功能。修改计数器的值可以重新设置系统当前的时间和日期。

\hypertarget{ux4f7fux7528ux8bf4ux660e-4}{%
\section{使用说明}\label{ux4f7fux7528ux8bf4ux660e-4}}

\hypertarget{rtcux4f7fux80fd}{%
\subsection{RTC使能}\label{rtcux4f7fux80fd}}

使用 \texttt{RTC\_Enable} 使能RTC。

\begin{Shaded}
\begin{Highlighting}[]
\DataTypeTok{void}\NormalTok{ RTC_Enable(}\DataTypeTok{uint8_t}\NormalTok{ enable);}
\end{Highlighting}
\end{Shaded}

\hypertarget{ux83b7ux53d6ux5f53ux524dux65e5ux671f}{%
\subsection{获取当前日期}\label{ux83b7ux53d6ux5f53ux524dux65e5ux671f}}

使用 \texttt{RTC\_GetTime} 获取当前时间(包括时分秒)。

\begin{Shaded}
\begin{Highlighting}[]
\DataTypeTok{uint16_t}\NormalTok{ RTC_GetTime(}
  \DataTypeTok{uint8_t}\NormalTok{ *hour, }
  \DataTypeTok{uint8_t}\NormalTok{ *minute, }
  \DataTypeTok{uint8_t}\NormalTok{ *second}
\NormalTok{  );}
\end{Highlighting}
\end{Shaded}

\hypertarget{ux4feeux6539ux65e5ux671f}{%
\subsection{修改日期}\label{ux4feeux6539ux65e5ux671f}}

使用 \texttt{RTC\_ModifyTime} 修改当前时间。

\begin{Shaded}
\begin{Highlighting}[]
\DataTypeTok{void}\NormalTok{ RTC_ModifyTime(}
  \DataTypeTok{uint16_t}\NormalTok{ day, }
  \DataTypeTok{uint8_t}\NormalTok{ hour, }
  \DataTypeTok{uint8_t}\NormalTok{ minute, }
  \DataTypeTok{uint8_t}\NormalTok{ second}
\NormalTok{  );}
\end{Highlighting}
\end{Shaded}

\hypertarget{ux914dux7f6eux95f9ux949f}{%
\subsection{配置闹钟}\label{ux914dux7f6eux95f9ux949f}}

使用 \texttt{RTC\_ConfigAlarm}

\begin{Shaded}
\begin{Highlighting}[]
\DataTypeTok{void}\NormalTok{ RTC_ConfigAlarm(}
  \DataTypeTok{uint8_t}\NormalTok{ hour, }
  \DataTypeTok{uint8_t}\NormalTok{ minute, }
  \DataTypeTok{uint8_t}\NormalTok{ second}
\NormalTok{  );}
\end{Highlighting}
\end{Shaded}

\hypertarget{ux914dux7f6eux4e2dux65adux8bf7ux6c42-1}{%
\subsection{配置中断请求}\label{ux914dux7f6eux4e2dux65adux8bf7ux6c42-1}}

使用 \texttt{RTC\_EnableIRQ} 配置并使能RTC中断请求。

\begin{Shaded}
\begin{Highlighting}[]
\DataTypeTok{void}\NormalTok{ RTC_EnableIRQ(}\DataTypeTok{uint32_t}\NormalTok{ mask);}
\end{Highlighting}
\end{Shaded}

其中的 \texttt{mask} 为RTC中断类型,一共有六种:

\begin{Shaded}
\begin{Highlighting}[]
\KeywordTok{typedef} \KeywordTok{enum}
\NormalTok{\{}
\NormalTok{    RTC_IRQ_ALARM = }\BaseNTok{0x04}\NormalTok{,}
\NormalTok{    RTC_IRQ_DAY = }\BaseNTok{0x08}\NormalTok{,}
\NormalTok{    RTC_IRQ_HOUR = }\BaseNTok{0x10}\NormalTok{,}
\NormalTok{    RTC_IRQ_MINUTE = }\BaseNTok{0x20}\NormalTok{,}
\NormalTok{    RTC_IRQ_SECOND = }\BaseNTok{0x40}\NormalTok{,}
\NormalTok{    RTC_IRQ_HALF_SECOND = }\BaseNTok{0x80}\NormalTok{,}
\NormalTok{\} rtc_irq_t;}
\end{Highlighting}
\end{Shaded}

\begin{itemize}
\item
  例如将 RTC 设置为alarm中断

\begin{Shaded}
\begin{Highlighting}[]
\NormalTok{RTC_EnableIRQ(RTC_IRQ_ALARM);}
\end{Highlighting}
\end{Shaded}
\item
  例如将 RTC 设置为day中断

\begin{Shaded}
\begin{Highlighting}[]
\NormalTok{RTC_EnableIRQ(RTC_IRQ_DAY);}
\end{Highlighting}
\end{Shaded}
\item
  例如将 RTC 设置为hour中断

\begin{Shaded}
\begin{Highlighting}[]
\NormalTok{RTC_EnableIRQ(RTC_IRQ_HOUR);}
\end{Highlighting}
\end{Shaded}
\item
  例如将 RTC 设置为minute中断

\begin{Shaded}
\begin{Highlighting}[]
\NormalTok{RTC_EnableIRQ(RTC_IRQ_MINUTE);}
\end{Highlighting}
\end{Shaded}
\item
  例如将 RTC 设置为second中断

\begin{Shaded}
\begin{Highlighting}[]
\NormalTok{RTC_EnableIRQ(RTC_IRQ_SECOND);}
\end{Highlighting}
\end{Shaded}
\item
  例如将 RTC 设置为half-second中断

\begin{Shaded}
\begin{Highlighting}[]
\NormalTok{RTC_EnableIRQ(RTC_IRQ_HALF_SECOND);}
\end{Highlighting}
\end{Shaded}
\end{itemize}

\hypertarget{ux83b7ux53d6ux5f53ux524dux4e2dux65adux72b6ux6001}{%
\subsection{获取当前中断状态}\label{ux83b7ux53d6ux5f53ux524dux4e2dux65adux72b6ux6001}}

使用 \texttt{RTC\_GetIntState} 获取当前RTC的中断状态。

\begin{Shaded}
\begin{Highlighting}[]
\DataTypeTok{uint32_t}\NormalTok{ RTC_GetIntState(}\DataTypeTok{void}\NormalTok{);}
\end{Highlighting}
\end{Shaded}

\hypertarget{ux6e05ux9664ux4e2dux65ad}{%
\subsection{清除中断}\label{ux6e05ux9664ux4e2dux65ad}}

使用 \texttt{RTC\_ClearIntState} 清除当前RTC的中断状态。

\begin{Shaded}
\begin{Highlighting}[]
\DataTypeTok{void}\NormalTok{ RTC_ClearIntState(}\DataTypeTok{uint32_t}\NormalTok{ state);}
\end{Highlighting}
\end{Shaded}

\hypertarget{ux5904ux7406ux4e2dux65adux72b6ux6001-1}{%
\subsection{处理中断状态}\label{ux5904ux7406ux4e2dux65adux72b6ux6001-1}}

用 \texttt{RTC\_GetIntState} 获取RTC上的中断触发,返回非 0 值表示RTC上产生了中断请求;RTC产生中断后,
需要消除中断状态方可再次触发。利用 \texttt{RTC\_ClearIntState}可清除RTC的中断触发状态。

\hypertarget{spiux529fux80fdux6982ux8ff0}{%
\chapter{SPI功能概述}\label{spiux529fux80fdux6982ux8ff0}}

\begin{itemize}
\tightlist
\item
  两个SPI模块
\item
  支持SPI主\&从模式
\item
  支持Quad SPI,可以从外挂Flash执行代码
\item
  独立的RX\&TX FIFO,深度为8个word
\item
  支持DMA
\end{itemize}

\hypertarget{spiux4f7fux7528ux8bf4ux660e}{%
\section{SPI使用说明}\label{spiux4f7fux7528ux8bf4ux660e}}

以下场景中均以SPI1为例,如果需要SPI0则可以根据情况修改

\hypertarget{ux573aux666f1ux53eaux8bfbux53eaux5199ux4e0dux5e26dma}{%
\section{场景1:只读只写不带DMA}\label{ux573aux666f1ux53eaux8bfbux53eaux5199ux4e0dux5e26dma}}

其中SPI主配置为只写模式,SPI从配置为只读模式,CPU操作读写,没有使用DMA 配置之前需要决定使用的GPIO,如果是普通模式,则不需要SPI\_MIC\_WP和SPI\_MIC\_HOLD

\begin{Shaded}
\begin{Highlighting}[]
\PreprocessorTok{#define SPI_MIC_CLK         GIO_GPIO_10}
\PreprocessorTok{#define SPI_MIC_MOSI        GIO_GPIO_11}
\PreprocessorTok{#define SPI_MIC_MISO        GIO_GPIO_12}
\PreprocessorTok{#define SPI_MIC_CS          GIO_GPIO_13}
\PreprocessorTok{#define SPI_MIC_WP          GIO_GPIO_14}
\PreprocessorTok{#define SPI_MIC_HOLD        GIO_GPIO_15}
\end{Highlighting}
\end{Shaded}

\hypertarget{spiux4e3bux914dux7f6e}{%
\subsection{SPI主配置}\label{spiux4e3bux914dux7f6e}}

\hypertarget{ux63a5ux53e3ux914dux7f6e}{%
\subsubsection{接口配置}\label{ux63a5ux53e3ux914dux7f6e}}

\begin{Shaded}
\begin{Highlighting}[]
\DataTypeTok{static} \DataTypeTok{void}\NormalTok{ setup_peripherals_spi_pin(}\DataTypeTok{void}\NormalTok{)}
\NormalTok{\{}
    \CommentTok{// 打开SPI模块时钟}
    \CommentTok{// 此处是以SPI1为例,如果是SPI0则需要更改}
    \CommentTok{// 根据使用的GPIO,选择打开GPIO0或者GPIO1的时钟}
\NormalTok{    SYSCTRL_ClearClkGateMulti(    (}\DecValTok{1}\NormalTok{ << SYSCTRL_ITEM_APB_SPI1)}
\NormalTok{                                | (}\DecValTok{1}\NormalTok{ << SYSCTRL_ITEM_APB_SysCtrl)}
\NormalTok{                                | (}\DecValTok{1}\NormalTok{ << SYSCTRL_ITEM_APB_PinCtrl)}
\NormalTok{                                | (}\DecValTok{1}\NormalTok{ << SYSCTRL_ITEM_APB_GPIO1)}
\NormalTok{                                | (}\DecValTok{1}\NormalTok{ << SYSCTRL_ITEM_APB_GPIO0));}

    \CommentTok{// 设置IO MUX,将GPIO映射成SPI功能,不需要的pin可以使用IO_NOT_A_PIN替代}
\NormalTok{    PINCTRL_SelSpiIn(SPI_PORT_1, SPI_MIC_CLK, SPI_MIC_CS, SPI_MIC_HOLD, SPI_MIC_WP, SPI_MIC_MISO, SPI_MIC_MOSI);}
\NormalTok{    PINCTRL_SetPadMux(SPI_MIC_CLK, IO_SOURCE_SPI1_CLK_OUT);}
\NormalTok{    PINCTRL_SetPadMux(SPI_MIC_CS, IO_SOURCE_SPI1_CSN_OUT);}
\NormalTok{    PINCTRL_SetPadMux(SPI_MIC_MOSI, IO_SOURCE_SPI1_MOSI_OUT);}
    
    \CommentTok{// 设置SPI的中断}
\NormalTok{    platform_set_irq_callback(PLATFORM_CB_IRQ_APBSPI, peripherals_spi_isr, NULL);}
\end{Highlighting}
\end{Shaded}

\hypertarget{spiux6a21ux5757ux521dux59cbux5316}{%
\subsubsection{SPI模块初始化}\label{spiux6a21ux5757ux521dux59cbux5316}}

常用设置项用注释标出,详细定义请参考``peripheral\_ssp.h''

\begin{Shaded}
\begin{Highlighting}[]
\CommentTok{// 示例,每次传输大小是8个word}
\PreprocessorTok{#define DATA_LEN (SPI_FIFO_DEPTH)}
\DataTypeTok{static} \DataTypeTok{void}\NormalTok{ setup_peripherals_spi_module(}\DataTypeTok{void}\NormalTok{)}
\NormalTok{\{}
\NormalTok{    apSSP_sDeviceControlBlock pParam;}
\NormalTok{    pParam.eSclkDiv = SPI_INTERFACETIMINGSCLKDIV_DEFAULT_2M;}\CommentTok{// SPI 时钟设置}
\NormalTok{    pParam.eSCLKPolarity = SPI_CPOL_SCLK_LOW_IN_IDLE_STATES;}\CommentTok{// SPI 模式设置}
\NormalTok{    pParam.eSCLKPhase = SPI_CPHA_ODD_SCLK_EDGES;}\CommentTok{// SPI 模式设置}
\NormalTok{    pParam.eLsbMsbOrder = SPI_LSB_MOST_SIGNIFICANT_BIT_FIRST;}
\NormalTok{    pParam.eDataSize = SPI_DATALEN_32_BITS;}\CommentTok{// SPI 每个传输单位的大小}
\NormalTok{    pParam.eMasterSlaveMode = SPI_SLVMODE_MASTER_MODE;}\CommentTok{// SPI主模式}
\NormalTok{    pParam.eReadWriteMode = SPI_TRANSMODE_WRITE_ONLY;}\CommentTok{// SPI只写}
\NormalTok{    pParam.eQuadMode = SPI_DUALQUAD_REGULAR_MODE;}
\NormalTok{    pParam.eWriteTransCnt = DATA_LEN;}\CommentTok{// SPI每次传输多少个单位(每个单位的大小是pParam.eDataSize)}
\NormalTok{    pParam.eReadTransCnt = DATA_LEN;}\CommentTok{// SPI每次接收多少个单位(每个单位的大小是pParam.eDataSize)}
\NormalTok{    pParam.eAddrEn = SPI_ADDREN_DISABLE;}
\NormalTok{    pParam.eCmdEn = SPI_CMDEN_DISABLE;}
\NormalTok{    pParam.RxThres = DATA_LEN/}\DecValTok{2}\NormalTok{;}
\NormalTok{    pParam.TxThres = DATA_LEN/}\DecValTok{2}\NormalTok{;}
\NormalTok{    pParam.SlaveDataOnly = SPI_SLVDATAONLY_ENABLE;}
\NormalTok{    pParam.eAddrLen = SPI_ADDRLEN_1_BYTE;}
\NormalTok{    pParam.eInterruptMask = (}\DecValTok{1}\NormalTok{ << bsSPI_INTREN_ENDINTEN);}\CommentTok{// 打开SPI中断(传输结束后触发)}
  
\NormalTok{    apSSP_DeviceParametersSet(APB_SSP1, &pParam);}
\NormalTok{\}}
\end{Highlighting}
\end{Shaded}

\hypertarget{spi-ux4e2dux65ad}{%
\subsubsection{SPI 中断}\label{spi-ux4e2dux65ad}}

\begin{Shaded}
\begin{Highlighting}[]
\CommentTok{// SPI ENDINT中断触发标志传输结束,清除中断状态}
\DataTypeTok{static} \DataTypeTok{uint32_t}\NormalTok{ peripherals_spi_isr(}\DataTypeTok{void}\NormalTok{ *user_data)}
\NormalTok{\{}
  \DataTypeTok{uint32_t}\NormalTok{ stat = apSSP_GetIntRawStatus(APB_SSP1);}
  
  \ControlFlowTok{if}\NormalTok{(stat & (}\DecValTok{1}\NormalTok{ << bsSPI_INTREN_ENDINTEN))}
\NormalTok{  \{}
\NormalTok{    apSSP_ClearIntStatus(APB_SSP1, }\DecValTok{1}\NormalTok{ << bsSPI_INTREN_ENDINTEN);}
\NormalTok{  \}  }
\NormalTok{\}}
\end{Highlighting}
\end{Shaded}

\hypertarget{spi-ux53d1ux9001ux6570ux636e}{%
\subsubsection{SPI 发送数据}\label{spi-ux53d1ux9001ux6570ux636e}}

\begin{Shaded}
\begin{Highlighting}[]
\DataTypeTok{uint32_t}\NormalTok{ write_data[DATA_LEN];}\CommentTok{//数据大小等于pParam.eDataSize,数组大小等于pParam.eWriteTransCnt}
\DataTypeTok{void}\NormalTok{ peripherals_spi_send_data(}\DataTypeTok{void}\NormalTok{)}
\NormalTok{\{}
  \CommentTok{// 写入命令,触发SPI传输}
\NormalTok{  apSSP_WriteCmd(APB_SSP1, }\BaseNTok{0x00}\NormalTok{, }\BaseNTok{0x00}\NormalTok{);}\CommentTok{//trigger transfer}
  
  \CommentTok{// 填写数据到TX FIFO,这个例子中DATA_LEN等于FIFO的深度(8),如果大于8,可以分为多次发送}
  \ControlFlowTok{for}\NormalTok{(i = }\DecValTok{0}\NormalTok{; i < DATA_LEN; i++)}
\NormalTok{  \{}
\NormalTok{    apSSP_WriteFIFO(APB_SSP1, write_data[i]);}
\NormalTok{  \}}

  \CommentTok{// 等待发送结束}
  \ControlFlowTok{while}\NormalTok{(apSSP_GetSPIActiveStatus(APB_SSP1));}
\NormalTok{\}}
\end{Highlighting}
\end{Shaded}

\hypertarget{ux4f7fux7528ux6d41ux7a0b-10}{%
\subsubsection{使用流程}\label{ux4f7fux7528ux6d41ux7a0b-10}}

\begin{itemize}
\tightlist
\item
  设置GPIO,setup\_peripherals\_spi\_pin()
\item
  初始化SPI,setup\_peripherals\_spi\_module()
\item
  在需要时候发送SPI数据,peripherals\_spi\_send\_data()
\item
  检查中断状态
\end{itemize}

\hypertarget{spiux4eceux914dux7f6e}{%
\subsection{SPI从配置}\label{spiux4eceux914dux7f6e}}

\hypertarget{ux63a5ux53e3ux914dux7f6e-1}{%
\subsubsection{接口配置}\label{ux63a5ux53e3ux914dux7f6e-1}}

\begin{Shaded}
\begin{Highlighting}[]
\DataTypeTok{static} \DataTypeTok{void}\NormalTok{ setup_peripherals_spi_pin(}\DataTypeTok{void}\NormalTok{)}
\NormalTok{\{}
    \CommentTok{// 打开SPI模块时钟}
    \CommentTok{// 此处是以SPI1为例,如果是SPI0则需要更改}
    \CommentTok{// 根据使用的GPIO,选择打开GPIO0或者GPIO1的时钟}
\NormalTok{    SYSCTRL_ClearClkGateMulti(    (}\DecValTok{1}\NormalTok{ << SYSCTRL_ITEM_APB_SPI1)}
\NormalTok{                                | (}\DecValTok{1}\NormalTok{ << SYSCTRL_ITEM_APB_SysCtrl)}
\NormalTok{                                | (}\DecValTok{1}\NormalTok{ << SYSCTRL_ITEM_APB_PinCtrl)}
\NormalTok{                                | (}\DecValTok{1}\NormalTok{ << SYSCTRL_ITEM_APB_GPIO1)}
\NormalTok{                                | (}\DecValTok{1}\NormalTok{ << SYSCTRL_ITEM_APB_GPIO0));}

    \CommentTok{// 设置IO MUX,将GPIO映射成SPI功能,不需要的pin可以使用IO_NOT_A_PIN替代}
\NormalTok{    PINCTRL_Pull(IO_SOURCE_SPI1_CLK_IN,PINCTRL_PULL_DOWN);}
\NormalTok{    PINCTRL_Pull(IO_SOURCE_SPI1_CSN_IN,PINCTRL_PULL_UP);}\CommentTok{// CS 需要默认上拉}
\NormalTok{    PINCTRL_SelSpiIn(SPI_PORT_1, SPI_MIC_CLK, SPI_MIC_CS, SPI_MIC_HOLD, SPI_MIC_WP, SPI_MIC_MISO, SPI_MIC_MOSI);}
\NormalTok{    PINCTRL_SetPadMux(SPI_MIC_CLK, IO_SOURCE_SPI1_CLK_OUT);}
\NormalTok{    PINCTRL_SetPadMux(SPI_MIC_MISO, IO_SOURCE_SPI1_MISO_OUT);}
    
    \CommentTok{// 设置SPI的中断}
\NormalTok{    platform_set_irq_callback(PLATFORM_CB_IRQ_APBSPI, peripherals_spi_isr, NULL);}
\end{Highlighting}
\end{Shaded}

\hypertarget{spiux6a21ux5757ux521dux59cbux5316-1}{%
\subsubsection{SPI模块初始化}\label{spiux6a21ux5757ux521dux59cbux5316-1}}

常用设置项用注释标出,详细定义请参考``peripheral\_ssp.h''

\begin{Shaded}
\begin{Highlighting}[]
\CommentTok{// 示例,每次传输大小是8个word}
\PreprocessorTok{#define DATA_LEN (SPI_FIFO_DEPTH)}
\DataTypeTok{static} \DataTypeTok{void}\NormalTok{ setup_peripherals_spi_module(}\DataTypeTok{void}\NormalTok{)}
\NormalTok{\{}
\NormalTok{    apSSP_sDeviceControlBlock pParam;}
\NormalTok{    pParam.eSclkDiv = SPI_INTERFACETIMINGSCLKDIV_DEFAULT_2M;}\CommentTok{// SPI 时钟设置}
\NormalTok{    pParam.eSCLKPolarity = SPI_CPOL_SCLK_LOW_IN_IDLE_STATES;}\CommentTok{// SPI 模式设置}
\NormalTok{    pParam.eSCLKPhase = SPI_CPHA_ODD_SCLK_EDGES;}\CommentTok{// SPI 模式设置}
\NormalTok{    pParam.eLsbMsbOrder = SPI_LSB_MOST_SIGNIFICANT_BIT_FIRST;}
\NormalTok{    pParam.eDataSize = SPI_DATALEN_32_BITS;}\CommentTok{// SPI 每个传输单位的大小}
\NormalTok{    pParam.eMasterSlaveMode = SPI_SLVMODE_SLAVE_MODE;}\CommentTok{// SPI 从模式}
\NormalTok{    pParam.eReadWriteMode = SPI_TRANSMODE_READ_ONLY;}\CommentTok{// SPI 只读}
\NormalTok{    pParam.eQuadMode = SPI_DUALQUAD_REGULAR_MODE;}
\NormalTok{    pParam.eWriteTransCnt = DATA_LEN;}\CommentTok{// SPI每次传输多少个单位(每个单位的大小是pParam.eDataSize)}
\NormalTok{    pParam.eReadTransCnt = DATA_LEN;}\CommentTok{// SPI每次接收多少个单位(每个单位的大小是pParam.eDataSize)}
\NormalTok{    pParam.eAddrEn = SPI_ADDREN_DISABLE;}
\NormalTok{    pParam.eCmdEn = SPI_CMDEN_DISABLE;}
\NormalTok{    pParam.RxThres = DATA_LEN/}\DecValTok{2}\NormalTok{;}
\NormalTok{    pParam.TxThres = DATA_LEN/}\DecValTok{2}\NormalTok{;}
\NormalTok{    pParam.SlaveDataOnly = SPI_SLVDATAONLY_ENABLE;}
\NormalTok{    pParam.eAddrLen = SPI_ADDRLEN_1_BYTE;}
\NormalTok{    pParam.eInterruptMask = (}\DecValTok{1}\NormalTok{ << bsSPI_INTREN_ENDINTEN);}\CommentTok{// 打开SPI中断(传输结束后触发)}
  
\NormalTok{    apSSP_DeviceParametersSet(APB_SSP1, &pParam);}
\NormalTok{\}}
\end{Highlighting}
\end{Shaded}

\hypertarget{spi-ux63a5ux6536ux6570ux636e}{%
\subsubsection{SPI 接收数据}\label{spi-ux63a5ux6536ux6570ux636e}}

\begin{Shaded}
\begin{Highlighting}[]
\DataTypeTok{uint32_t}\NormalTok{ read_data[DATA_LEN];}\CommentTok{//数据大小等于pParam.eDataSize,数组大小等于pParam.eReadTransCnt}
\DataTypeTok{static} \DataTypeTok{uint32_t}\NormalTok{ peripherals_spi_isr(}\DataTypeTok{void}\NormalTok{ *user_data)}
\NormalTok{\{}
  \DataTypeTok{uint32_t}\NormalTok{ stat = apSSP_GetIntRawStatus(APB_SSP1), i;}
  \ControlFlowTok{if}\NormalTok{(stat & (}\DecValTok{1}\NormalTok{ << bsSPI_INTREN_ENDINTEN))}
\NormalTok{  \{}
    \CommentTok{/* check if rx fifo still have some left data */}
    \CommentTok{// 检查当前RX FIFO中有效值的个数,根据个数读取RX FIFO}
    \DataTypeTok{uint32_t}\NormalTok{ num = apSSP_GetDataNumInRxFifo(APB_SSP1);}
    \ControlFlowTok{for}\NormalTok{(i = }\DecValTok{0}\NormalTok{; i < num; i++)}
\NormalTok{    \{}
\NormalTok{      apSSP_ReadFIFO(APB_SSP1, &read_data[i]);}
\NormalTok{    \}}

\NormalTok{    apSSP_ClearIntStatus(APB_SSP1, }\DecValTok{1}\NormalTok{ << bsSPI_INTREN_ENDINTEN);}
    
\NormalTok{  \}}
\NormalTok{\}}
\end{Highlighting}
\end{Shaded}

\hypertarget{ux4f7fux7528ux6d41ux7a0b-11}{%
\subsubsection{使用流程}\label{ux4f7fux7528ux6d41ux7a0b-11}}

\begin{itemize}
\tightlist
\item
  设置GPIO,setup\_peripherals\_spi\_pin()
\item
  初始化SPI,setup\_peripherals\_spi\_module()
\item
  观察SPI中断,中断触发代表当前传输结束
\end{itemize}

\begin{center}\rule{0.5\linewidth}{0.5pt}\end{center}

\hypertarget{ux573aux666f2ux53eaux8bfbux53eaux5199ux5e76ux4e14ux4f7fux7528dma}{%
\section{场景2:只读只写并且使用DMA}\label{ux573aux666f2ux53eaux8bfbux53eaux5199ux5e76ux4e14ux4f7fux7528dma}}

其中SPI主配置为只写模式,SPI从配置为只读模式,同时使用DMA进行读写 配置之前需要决定使用的GPIO,如果是普通模式,则不需要SPI\_MIC\_WP和SPI\_MIC\_HOLD

\begin{Shaded}
\begin{Highlighting}[]
\PreprocessorTok{#define SPI_MIC_CLK         GIO_GPIO_10}
\PreprocessorTok{#define SPI_MIC_MOSI        GIO_GPIO_11}
\PreprocessorTok{#define SPI_MIC_MISO        GIO_GPIO_12}
\PreprocessorTok{#define SPI_MIC_CS          GIO_GPIO_13}
\PreprocessorTok{#define SPI_MIC_WP          GIO_GPIO_14}
\PreprocessorTok{#define SPI_MIC_HOLD        GIO_GPIO_15}
\end{Highlighting}
\end{Shaded}

\hypertarget{spiux4e3bux914dux7f6e-1}{%
\subsection{SPI主配置}\label{spiux4e3bux914dux7f6e-1}}

\hypertarget{ux63a5ux53e3ux914dux7f6e-2}{%
\subsubsection{接口配置}\label{ux63a5ux53e3ux914dux7f6e-2}}

\begin{Shaded}
\begin{Highlighting}[]
\DataTypeTok{static} \DataTypeTok{void}\NormalTok{ setup_peripherals_spi_pin(}\DataTypeTok{void}\NormalTok{)}
\NormalTok{\{}
    \CommentTok{// 打开SPI模块时钟}
    \CommentTok{// 此处是以SPI1为例,如果是SPI0则需要更改}
    \CommentTok{// 根据使用的GPIO,选择打开GPIO0或者GPIO1的时钟}
\NormalTok{    SYSCTRL_ClearClkGateMulti(    (}\DecValTok{1}\NormalTok{ << SYSCTRL_ITEM_APB_SPI1)}
\NormalTok{                                | (}\DecValTok{1}\NormalTok{ << SYSCTRL_ITEM_APB_SysCtrl)}
\NormalTok{                                | (}\DecValTok{1}\NormalTok{ << SYSCTRL_ITEM_APB_PinCtrl)}
\NormalTok{                                | (}\DecValTok{1}\NormalTok{ << SYSCTRL_ITEM_APB_GPIO1)}
\NormalTok{                                | (}\DecValTok{1}\NormalTok{ << SYSCTRL_ITEM_APB_GPIO0));}

    \CommentTok{// 设置IO MUX,将GPIO映射成SPI功能,不需要的pin可以使用IO_NOT_A_PIN替代}
\NormalTok{    PINCTRL_SelSpiIn(SPI_PORT_1, SPI_MIC_CLK, SPI_MIC_CS, SPI_MIC_HOLD, SPI_MIC_WP, SPI_MIC_MISO, SPI_MIC_MOSI);}
\NormalTok{    PINCTRL_SetPadMux(SPI_MIC_CLK, IO_SOURCE_SPI1_CLK_OUT);}
\NormalTok{    PINCTRL_SetPadMux(SPI_MIC_CS, IO_SOURCE_SPI1_CSN_OUT);}
\NormalTok{    PINCTRL_SetPadMux(SPI_MIC_MOSI, IO_SOURCE_SPI1_MOSI_OUT);}
    
    \CommentTok{// 设置SPI的中断}
\NormalTok{    platform_set_irq_callback(PLATFORM_CB_IRQ_APBSPI, peripherals_spi_isr, NULL);}
\end{Highlighting}
\end{Shaded}

\hypertarget{spiux6a21ux5757ux521dux59cbux5316-2}{%
\subsubsection{SPI模块初始化}\label{spiux6a21ux5757ux521dux59cbux5316-2}}

常用设置项用注释标出,详细定义请参考``peripheral\_ssp.h''

\begin{Shaded}
\begin{Highlighting}[]
\CommentTok{// 示例,每次传输大小是8个word}
\PreprocessorTok{#define DATA_LEN (SPI_FIFO_DEPTH)}
\DataTypeTok{static} \DataTypeTok{void}\NormalTok{ setup_peripherals_spi_module(}\DataTypeTok{void}\NormalTok{)}
\NormalTok{\{}
\NormalTok{    apSSP_sDeviceControlBlock pParam;}
\NormalTok{    pParam.eSclkDiv = SPI_INTERFACETIMINGSCLKDIV_DEFAULT_2M;}\CommentTok{// SPI 时钟设置}
\NormalTok{    pParam.eSCLKPolarity = SPI_CPOL_SCLK_LOW_IN_IDLE_STATES;}\CommentTok{// SPI 模式设置}
\NormalTok{    pParam.eSCLKPhase = SPI_CPHA_ODD_SCLK_EDGES;}\CommentTok{// SPI 模式设置}
\NormalTok{    pParam.eLsbMsbOrder = SPI_LSB_MOST_SIGNIFICANT_BIT_FIRST;}
\NormalTok{    pParam.eDataSize = SPI_DATALEN_32_BITS;}\CommentTok{// SPI 每个传输单位的大小}
\NormalTok{    pParam.eMasterSlaveMode = SPI_SLVMODE_MASTER_MODE;}\CommentTok{// SPI主模式}
\NormalTok{    pParam.eReadWriteMode = SPI_TRANSMODE_WRITE_ONLY;}\CommentTok{// SPI只写}
\NormalTok{    pParam.eQuadMode = SPI_DUALQUAD_REGULAR_MODE;}
\NormalTok{    pParam.eWriteTransCnt = DATA_LEN;}\CommentTok{// SPI每次传输多少个单位(每个单位的大小是pParam.eDataSize)}
\NormalTok{    pParam.eReadTransCnt = DATA_LEN;}\CommentTok{// SPI每次接收多少个单位(每个单位的大小是pParam.eDataSize)}
\NormalTok{    pParam.eAddrEn = SPI_ADDREN_DISABLE;}
\NormalTok{    pParam.eCmdEn = SPI_CMDEN_DISABLE;}
\NormalTok{    pParam.RxThres = DATA_LEN/}\DecValTok{2}\NormalTok{;}
\NormalTok{    pParam.TxThres = DATA_LEN/}\DecValTok{2}\NormalTok{;}
\NormalTok{    pParam.SlaveDataOnly = SPI_SLVDATAONLY_ENABLE;}
\NormalTok{    pParam.eAddrLen = SPI_ADDRLEN_1_BYTE;}
\NormalTok{    pParam.eInterruptMask = (}\DecValTok{1}\NormalTok{ << bsSPI_INTREN_ENDINTEN);}\CommentTok{// 打开SPI中断(传输结束后触发)}
  
\NormalTok{    apSSP_DeviceParametersSet(APB_SSP1, &pParam);}
    
\NormalTok{\}}
\end{Highlighting}
\end{Shaded}

\hypertarget{spi-dmaux521dux59cbux5316}{%
\subsubsection{SPI DMA初始化}\label{spi-dmaux521dux59cbux5316}}

// 初始化DMA模块

\begin{Shaded}
\begin{Highlighting}[]
\DataTypeTok{static} \DataTypeTok{void}\NormalTok{ setup_peripherals_dma_module(}\DataTypeTok{void}\NormalTok{)}
\NormalTok{\{}
\NormalTok{    SYSCTRL_ClearClkGateMulti(}\DecValTok{1}\NormalTok{ << SYSCTRL_ClkGate_APB_DMA);}
\NormalTok{    DMA_Reset(}\DecValTok{1}\NormalTok{);}
\NormalTok{    DMA_Reset(}\DecValTok{0}\NormalTok{);}
\NormalTok{\}}
\end{Highlighting}
\end{Shaded}

\hypertarget{spi-dmaux8bbeux7f6e}{%
\subsubsection{SPI DMA设置}\label{spi-dmaux8bbeux7f6e}}

// 此处是以SPI1为例

\begin{Shaded}
\begin{Highlighting}[]
\DataTypeTok{void}\NormalTok{ peripherals_spi_dma_to_txfifo(}\DataTypeTok{int}\NormalTok{ channel_id, }\DataTypeTok{void}\NormalTok{ *src, }\DataTypeTok{int}\NormalTok{ size)}
\NormalTok{\{}
\NormalTok{    DMA_Descriptor descriptor __attribute__((aligned (}\DecValTok{8}\NormalTok{)));}

\NormalTok{    descriptor.Next = (DMA_Descriptor *)}\DecValTok{0}\NormalTok{;}
\NormalTok{    DMA_PrepareMem2Peripheral(&descriptor,SYSCTRL_DMA_SPI1_TX,src,size,DMA_ADDRESS_INC,}\DecValTok{0}\NormalTok{);}

\NormalTok{    DMA_EnableChannel(channel_id, &descriptor);}
\NormalTok{\}}
\end{Highlighting}
\end{Shaded}

\hypertarget{spi-ux4e2dux65ad-1}{%
\subsubsection{SPI 中断}\label{spi-ux4e2dux65ad-1}}

\begin{Shaded}
\begin{Highlighting}[]
\CommentTok{// SPI ENDINT中断触发标志传输结束,清除中断状态}
\DataTypeTok{static} \DataTypeTok{uint32_t}\NormalTok{ peripherals_spi_isr(}\DataTypeTok{void}\NormalTok{ *user_data)}
\NormalTok{\{}
  \DataTypeTok{uint32_t}\NormalTok{ stat = apSSP_GetIntRawStatus(APB_SSP1);}
  
  \ControlFlowTok{if}\NormalTok{(stat & (}\DecValTok{1}\NormalTok{ << bsSPI_INTREN_ENDINTEN))}
\NormalTok{  \{}
\NormalTok{    apSSP_ClearIntStatus(APB_SSP1, }\DecValTok{1}\NormalTok{ << bsSPI_INTREN_ENDINTEN);}
\NormalTok{  \}  }
\NormalTok{\}}
\end{Highlighting}
\end{Shaded}

\hypertarget{spi-ux53d1ux9001ux6570ux636e-1}{%
\subsubsection{SPI 发送数据}\label{spi-ux53d1ux9001ux6570ux636e-1}}

\begin{Shaded}
\begin{Highlighting}[]
\DataTypeTok{uint32_t}\NormalTok{ write_data[DATA_LEN];}\CommentTok{//数据大小等于pParam.eDataSize,数组大小等于pParam.eWriteTransCnt}

\DataTypeTok{void}\NormalTok{ peripherals_spi_send_data(}\DataTypeTok{void}\NormalTok{)}
\NormalTok{\{}
  \CommentTok{// 首先需要打开 SPI 模块中的DMA功能}
\NormalTok{  apSSP_SetTxDmaEn(APB_SSP1,}\DecValTok{1}\NormalTok{);}
  \CommentTok{// 初始化中已经设置了pParam.eWriteTransCnt,如果需要调整则可以调用这个API}
\NormalTok{  apSSP_SetTransferControlWrTranCnt(APB_SSP1,DATA_LEN);}
  
  \CommentTok{// DMA共有8个channel}
  \PreprocessorTok{#define SPI_DMA_TX_CHANNEL   (0)}\CommentTok{//DMA channel 0}
  \CommentTok{// 配置DMA,指向需要发送的数据}
\NormalTok{  peripherals_spi_dma_to_txfifo(SPI_DMA_TX_CHANNEL, write_data, }\KeywordTok{sizeof}\NormalTok{(write_data));}
  
  \CommentTok{// 写入命令,触发SPI传输}
\NormalTok{  apSSP_WriteCmd(APB_SSP1, }\BaseNTok{0x00}\NormalTok{, }\BaseNTok{0x00}\NormalTok{);}\CommentTok{//trigger transfer}

  \CommentTok{// 等待发送结束}
  \ControlFlowTok{while}\NormalTok{(apSSP_GetSPIActiveStatus(APB_SSP1));}
  
  \CommentTok{// 关闭 SPI 模块中的DMA功能}
\NormalTok{  apSSP_SetTxDmaEn(APB_SSP1,}\DecValTok{0}\NormalTok{);}
\NormalTok{\}}
\end{Highlighting}
\end{Shaded}

\hypertarget{ux4f7fux7528ux6d41ux7a0b-12}{%
\subsubsection{使用流程}\label{ux4f7fux7528ux6d41ux7a0b-12}}

\begin{itemize}
\tightlist
\item
  设置GPIO,setup\_peripherals\_spi\_pin()
\item
  初始化SPI,setup\_peripherals\_spi\_module()
\item
  初始化DMA,setup\_peripherals\_dma\_module()
\item
  在需要时候发送SPI数据,peripherals\_spi\_send\_data()
\item
  检查中断状态
\end{itemize}

\hypertarget{spiux4eceux914dux7f6e-1}{%
\subsection{SPI从配置}\label{spiux4eceux914dux7f6e-1}}

\hypertarget{ux63a5ux53e3ux914dux7f6e-3}{%
\subsubsection{接口配置}\label{ux63a5ux53e3ux914dux7f6e-3}}

\begin{Shaded}
\begin{Highlighting}[]
\DataTypeTok{static} \DataTypeTok{void}\NormalTok{ setup_peripherals_spi_pin(}\DataTypeTok{void}\NormalTok{)}
\NormalTok{\{}
    \CommentTok{// 打开SPI模块时钟}
    \CommentTok{// 此处是以SPI1为例,如果是SPI0则需要更改}
    \CommentTok{// 根据使用的GPIO,选择打开GPIO0或者GPIO1的时钟}
\NormalTok{    SYSCTRL_ClearClkGateMulti(    (}\DecValTok{1}\NormalTok{ << SYSCTRL_ITEM_APB_SPI1)}
\NormalTok{                                | (}\DecValTok{1}\NormalTok{ << SYSCTRL_ITEM_APB_SysCtrl)}
\NormalTok{                                | (}\DecValTok{1}\NormalTok{ << SYSCTRL_ITEM_APB_PinCtrl)}
\NormalTok{                                | (}\DecValTok{1}\NormalTok{ << SYSCTRL_ITEM_APB_GPIO1)}
\NormalTok{                                | (}\DecValTok{1}\NormalTok{ << SYSCTRL_ITEM_APB_GPIO0));}

    \CommentTok{// 设置IO MUX,将GPIO映射成SPI功能,不需要的pin可以使用IO_NOT_A_PIN替代}
\NormalTok{    PINCTRL_Pull(IO_SOURCE_SPI1_CLK_IN,PINCTRL_PULL_DOWN);}
\NormalTok{    PINCTRL_Pull(IO_SOURCE_SPI1_CSN_IN,PINCTRL_PULL_UP);}\CommentTok{// CS 需要默认上拉}
\NormalTok{    PINCTRL_SelSpiIn(SPI_PORT_1, SPI_MIC_CLK, SPI_MIC_CS, SPI_MIC_HOLD, SPI_MIC_WP, SPI_MIC_MISO, SPI_MIC_MOSI);}
\NormalTok{    PINCTRL_SetPadMux(SPI_MIC_CLK, IO_SOURCE_SPI1_CLK_OUT);}
\NormalTok{    PINCTRL_SetPadMux(SPI_MIC_MISO, IO_SOURCE_SPI1_MISO_OUT);}
    
    \CommentTok{// 设置SPI的中断}
\NormalTok{    platform_set_irq_callback(PLATFORM_CB_IRQ_APBSPI, peripherals_spi_isr, NULL);}
\end{Highlighting}
\end{Shaded}

\hypertarget{spiux6a21ux5757ux521dux59cbux5316-3}{%
\subsubsection{SPI模块初始化}\label{spiux6a21ux5757ux521dux59cbux5316-3}}

常用设置项用注释标出,详细定义请参考``peripheral\_ssp.h''

\begin{Shaded}
\begin{Highlighting}[]
\CommentTok{// 示例,每次传输大小是8个word}
\PreprocessorTok{#define DATA_LEN (SPI_FIFO_DEPTH)}
\DataTypeTok{static} \DataTypeTok{void}\NormalTok{ setup_peripherals_spi_module(}\DataTypeTok{void}\NormalTok{)}
\NormalTok{\{}
\NormalTok{    apSSP_sDeviceControlBlock pParam;}
\NormalTok{    pParam.eSclkDiv = SPI_INTERFACETIMINGSCLKDIV_DEFAULT_2M;}\CommentTok{// SPI 时钟设置}
\NormalTok{    pParam.eSCLKPolarity = SPI_CPOL_SCLK_LOW_IN_IDLE_STATES;}\CommentTok{// SPI 模式设置}
\NormalTok{    pParam.eSCLKPhase = SPI_CPHA_ODD_SCLK_EDGES;}\CommentTok{// SPI 模式设置}
\NormalTok{    pParam.eLsbMsbOrder = SPI_LSB_MOST_SIGNIFICANT_BIT_FIRST;}
\NormalTok{    pParam.eDataSize = SPI_DATALEN_32_BITS;}\CommentTok{// SPI 每个传输单位的大小}
\NormalTok{    pParam.eMasterSlaveMode = SPI_SLVMODE_SLAVE_MODE;}\CommentTok{// SPI 从模式}
\NormalTok{    pParam.eReadWriteMode = SPI_TRANSMODE_READ_ONLY;}\CommentTok{// SPI 只读}
\NormalTok{    pParam.eQuadMode = SPI_DUALQUAD_REGULAR_MODE;}
\NormalTok{    pParam.eWriteTransCnt = DATA_LEN;}\CommentTok{// SPI每次传输多少个单位(每个单位的大小是pParam.eDataSize)}
\NormalTok{    pParam.eReadTransCnt = DATA_LEN;}\CommentTok{// SPI每次接收多少个单位(每个单位的大小是pParam.eDataSize)}
\NormalTok{    pParam.eAddrEn = SPI_ADDREN_DISABLE;}
\NormalTok{    pParam.eCmdEn = SPI_CMDEN_DISABLE;}
\NormalTok{    pParam.RxThres = }\DecValTok{0}\NormalTok{;}
\NormalTok{    pParam.TxThres = }\DecValTok{0}\NormalTok{;}
\NormalTok{    pParam.SlaveDataOnly = SPI_SLVDATAONLY_ENABLE;}
\NormalTok{    pParam.eAddrLen = SPI_ADDRLEN_1_BYTE;}
\NormalTok{    pParam.eInterruptMask = (}\DecValTok{1}\NormalTok{ << bsSPI_INTREN_ENDINTEN);}\CommentTok{// 打开SPI中断(传输结束后触发)}
  
\NormalTok{    apSSP_DeviceParametersSet(APB_SSP1, &pParam);}
\NormalTok{\}}
\end{Highlighting}
\end{Shaded}

\hypertarget{spi-dmaux521dux59cbux5316-1}{%
\subsubsection{SPI DMA初始化}\label{spi-dmaux521dux59cbux5316-1}}

// 初始化DMA模块

\begin{Shaded}
\begin{Highlighting}[]
\DataTypeTok{static} \DataTypeTok{void}\NormalTok{ setup_peripherals_dma_module(}\DataTypeTok{void}\NormalTok{)}
\NormalTok{\{}
\NormalTok{    SYSCTRL_ClearClkGateMulti(}\DecValTok{1}\NormalTok{ << SYSCTRL_ClkGate_APB_DMA);}
\NormalTok{    DMA_Reset(}\DecValTok{1}\NormalTok{);}
\NormalTok{    DMA_Reset(}\DecValTok{0}\NormalTok{);}
\NormalTok{\}}
\end{Highlighting}
\end{Shaded}

\hypertarget{spi-dmaux8bbeux7f6e-1}{%
\subsubsection{SPI DMA设置}\label{spi-dmaux8bbeux7f6e-1}}

// 此处是以SPI1为例

\begin{Shaded}
\begin{Highlighting}[]
\DataTypeTok{void}\NormalTok{ peripherals_spi_rxfifo_to_dma(}\DataTypeTok{int}\NormalTok{ channel_id, }\DataTypeTok{void}\NormalTok{ *dst, }\DataTypeTok{int}\NormalTok{ size)}
\NormalTok{\{}
\NormalTok{    DMA_Descriptor descriptor __attribute__((aligned (}\DecValTok{8}\NormalTok{)));}

\NormalTok{    descriptor.Next = (DMA_Descriptor *)}\DecValTok{0}\NormalTok{;}
\NormalTok{    DMA_PreparePeripheral2Mem(&descriptor,dst,SYSCTRL_DMA_SPI1_RX,size,DMA_ADDRESS_INC,}\DecValTok{0}\NormalTok{);}

\NormalTok{    DMA_EnableChannel(channel_id, &descriptor);}
\NormalTok{\}}
\end{Highlighting}
\end{Shaded}

\hypertarget{spi-ux63a5ux6536ux6570ux636e-1}{%
\subsubsection{SPI 接收数据}\label{spi-ux63a5ux6536ux6570ux636e-1}}

\begin{Shaded}
\begin{Highlighting}[]
\DataTypeTok{uint32_t}\NormalTok{ read_data[DATA_LEN];}\CommentTok{//数据大小等于pParam.eDataSize,数组大小等于pParam.eReadTransCnt}
\DataTypeTok{void}\NormalTok{ peripherals_spi_read_data(}\DataTypeTok{void}\NormalTok{)}
\NormalTok{\{}
    \CommentTok{// 打开SPI DMA功能}
\NormalTok{    apSSP_SetRxDmaEn(APB_SSP1,}\DecValTok{1}\NormalTok{);}
    \CommentTok{// 功能等同于重新设置pParam.eReadTransCnt,代表一次传输的单位个数}
\NormalTok{    apSSP_SetTransferControlRdTranCnt(APB_SSP1,DATA_LEN);}
    
    \PreprocessorTok{#define SPI_DMA_RX_CHANNEL   (0)}\CommentTok{//DMA channel 0}
\NormalTok{    peripherals_spi_rxfifo_to_dma(SPI_DMA_RX_CHANNEL, read_data, }\KeywordTok{sizeof}\NormalTok{(read_data));}
\NormalTok{\}}
\end{Highlighting}
\end{Shaded}

\hypertarget{spi-ux4e2dux65ad-2}{%
\subsubsection{SPI 中断}\label{spi-ux4e2dux65ad-2}}

\begin{Shaded}
\begin{Highlighting}[]
\CommentTok{// SPI ENDINT中断触发标志传输结束,清除中断状态}
\DataTypeTok{static} \DataTypeTok{uint32_t}\NormalTok{ peripherals_spi_isr(}\DataTypeTok{void}\NormalTok{ *user_data)}
\NormalTok{\{}
  \DataTypeTok{uint32_t}\NormalTok{ stat = apSSP_GetIntRawStatus(APB_SSP1);}
  
  \ControlFlowTok{if}\NormalTok{(stat & (}\DecValTok{1}\NormalTok{ << bsSPI_INTREN_ENDINTEN))}
\NormalTok{  \{}
\NormalTok{    peripherals_spi_read_data();}
\NormalTok{    apSSP_ClearIntStatus(APB_SSP1, }\DecValTok{1}\NormalTok{ << bsSPI_INTREN_ENDINTEN);}
\NormalTok{  \}  }
\NormalTok{\}}
\end{Highlighting}
\end{Shaded}

\hypertarget{ux4f7fux7528ux6d41ux7a0b-13}{%
\subsubsection{使用流程}\label{ux4f7fux7528ux6d41ux7a0b-13}}

\begin{itemize}
\tightlist
\item
  设置GPIO,setup\_peripherals\_spi\_pin()
\item
  初始化SPI,setup\_peripherals\_spi\_module()
\item
  初始化DMA,setup\_peripherals\_dma\_module()
\item
  设置接收DMA,peripherals\_spi\_read\_data()
\item
  观察SPI中断,中断触发代表当前接收结束
\end{itemize}

\begin{center}\rule{0.5\linewidth}{0.5pt}\end{center}

\hypertarget{ux573aux666f3ux540cux65f6ux8bfbux5199ux4e0dux5e26dma}{%
\section{场景3:同时读写不带DMA}\label{ux573aux666f3ux540cux65f6ux8bfbux5199ux4e0dux5e26dma}}

其中SPI主从都配置为同时读写模式,CPU操作读写,没有使用DMA 配置之前需要决定使用的GPIO,如果是普通模式,则不需要SPI\_MIC\_WP和SPI\_MIC\_HOLD

\begin{Shaded}
\begin{Highlighting}[]
\PreprocessorTok{#define SPI_MIC_CLK         GIO_GPIO_10}
\PreprocessorTok{#define SPI_MIC_MOSI        GIO_GPIO_11}
\PreprocessorTok{#define SPI_MIC_MISO        GIO_GPIO_12}
\PreprocessorTok{#define SPI_MIC_CS          GIO_GPIO_13}
\PreprocessorTok{#define SPI_MIC_WP          GIO_GPIO_14}
\PreprocessorTok{#define SPI_MIC_HOLD        GIO_GPIO_15}
\end{Highlighting}
\end{Shaded}

\hypertarget{spiux4e3bux914dux7f6e-2}{%
\subsection{SPI主配置}\label{spiux4e3bux914dux7f6e-2}}

\hypertarget{ux63a5ux53e3ux914dux7f6e-4}{%
\subsubsection{接口配置}\label{ux63a5ux53e3ux914dux7f6e-4}}

\begin{Shaded}
\begin{Highlighting}[]
\DataTypeTok{static} \DataTypeTok{void}\NormalTok{ setup_peripherals_spi_pin(}\DataTypeTok{void}\NormalTok{)}
\NormalTok{\{}
    \CommentTok{// 打开SPI模块时钟}
    \CommentTok{// 此处是以SPI1为例,如果是SPI0则需要更改}
    \CommentTok{// 根据使用的GPIO,选择打开GPIO0或者GPIO1的时钟}
\NormalTok{    SYSCTRL_ClearClkGateMulti(    (}\DecValTok{1}\NormalTok{ << SYSCTRL_ITEM_APB_SPI1)}
\NormalTok{                                | (}\DecValTok{1}\NormalTok{ << SYSCTRL_ITEM_APB_SysCtrl)}
\NormalTok{                                | (}\DecValTok{1}\NormalTok{ << SYSCTRL_ITEM_APB_PinCtrl)}
\NormalTok{                                | (}\DecValTok{1}\NormalTok{ << SYSCTRL_ITEM_APB_GPIO1)}
\NormalTok{                                | (}\DecValTok{1}\NormalTok{ << SYSCTRL_ITEM_APB_GPIO0));}

    \CommentTok{// 设置IO MUX,将GPIO映射成SPI功能,不需要的pin可以使用IO_NOT_A_PIN替代}
\NormalTok{    PINCTRL_SelSpiIn(SPI_PORT_1, SPI_MIC_CLK, SPI_MIC_CS, SPI_MIC_HOLD, SPI_MIC_WP, SPI_MIC_MISO, SPI_MIC_MOSI);}
\NormalTok{    PINCTRL_SetPadMux(SPI_MIC_CLK, IO_SOURCE_SPI1_CLK_OUT);}
\NormalTok{    PINCTRL_SetPadMux(SPI_MIC_CS, IO_SOURCE_SPI1_CSN_OUT);}
\NormalTok{    PINCTRL_SetPadMux(SPI_MIC_MOSI, IO_SOURCE_SPI1_MOSI_OUT);}
    
    \CommentTok{// 设置SPI的中断}
\NormalTok{    platform_set_irq_callback(PLATFORM_CB_IRQ_APBSPI, peripherals_spi_isr, NULL);}
\end{Highlighting}
\end{Shaded}

\hypertarget{spiux6a21ux5757ux521dux59cbux5316-4}{%
\subsubsection{SPI模块初始化}\label{spiux6a21ux5757ux521dux59cbux5316-4}}

常用设置项用注释标出,详细定义请参考``peripheral\_ssp.h''

\begin{Shaded}
\begin{Highlighting}[]
\CommentTok{// 示例,每次传输大小是8个word}
\PreprocessorTok{#define DATA_LEN (SPI_FIFO_DEPTH)}
\DataTypeTok{static} \DataTypeTok{void}\NormalTok{ setup_peripherals_spi_module(}\DataTypeTok{void}\NormalTok{)}
\NormalTok{\{}
\NormalTok{    apSSP_sDeviceControlBlock pParam;}
\NormalTok{    pParam.eSclkDiv = SPI_INTERFACETIMINGSCLKDIV_DEFAULT_2M;}\CommentTok{// SPI 时钟设置}
\NormalTok{    pParam.eSCLKPolarity = SPI_CPOL_SCLK_LOW_IN_IDLE_STATES;}\CommentTok{// SPI 模式设置}
\NormalTok{    pParam.eSCLKPhase = SPI_CPHA_ODD_SCLK_EDGES;}\CommentTok{// SPI 模式设置}
\NormalTok{    pParam.eLsbMsbOrder = SPI_LSB_MOST_SIGNIFICANT_BIT_FIRST;}
\NormalTok{    pParam.eDataSize = SPI_DATALEN_32_BITS;}\CommentTok{// SPI 每个传输单位的大小}
\NormalTok{    pParam.eMasterSlaveMode = SPI_SLVMODE_MASTER_MODE;}\CommentTok{// SPI主模式}
\NormalTok{    pParam.eReadWriteMode = SPI_TRANSMODE_WRITE_READ_SAME_TIME;}\CommentTok{// SPI同时读写}
\NormalTok{    pParam.eQuadMode = SPI_DUALQUAD_REGULAR_MODE;}
\NormalTok{    pParam.eWriteTransCnt = DATA_LEN;}\CommentTok{// SPI每次传输多少个单位(每个单位的大小是pParam.eDataSize)}
\NormalTok{    pParam.eReadTransCnt = DATA_LEN;}\CommentTok{// SPI每次接收多少个单位(每个单位的大小是pParam.eDataSize)}
\NormalTok{    pParam.eAddrEn = SPI_ADDREN_DISABLE;}
\NormalTok{    pParam.eCmdEn = SPI_CMDEN_DISABLE;}
\NormalTok{    pParam.RxThres = DATA_LEN/}\DecValTok{2}\NormalTok{;}
\NormalTok{    pParam.TxThres = DATA_LEN/}\DecValTok{2}\NormalTok{;}
\NormalTok{    pParam.SlaveDataOnly = SPI_SLVDATAONLY_ENABLE;}
\NormalTok{    pParam.eAddrLen = SPI_ADDRLEN_1_BYTE;}
\NormalTok{    pParam.eInterruptMask = (}\DecValTok{1}\NormalTok{ << bsSPI_INTREN_ENDINTEN);}\CommentTok{// 打开SPI中断(传输结束后触发)}
  
\NormalTok{    apSSP_DeviceParametersSet(APB_SSP1, &pParam);}
\NormalTok{\}}
\end{Highlighting}
\end{Shaded}

\hypertarget{spi-ux4e2dux65ad-3}{%
\subsubsection{SPI 中断}\label{spi-ux4e2dux65ad-3}}

\begin{Shaded}
\begin{Highlighting}[]
\CommentTok{// SPI ENDINT中断触发标志传输结束,清除中断状态}
\DataTypeTok{static} \DataTypeTok{uint32_t}\NormalTok{ peripherals_spi_isr(}\DataTypeTok{void}\NormalTok{ *user_data)}
\NormalTok{\{}
  \DataTypeTok{uint32_t}\NormalTok{ stat = apSSP_GetIntRawStatus(APB_SSP1);}
  
  \ControlFlowTok{if}\NormalTok{(stat & (}\DecValTok{1}\NormalTok{ << bsSPI_INTREN_ENDINTEN))}
\NormalTok{  \{}
\NormalTok{    apSSP_ClearIntStatus(APB_SSP1, }\DecValTok{1}\NormalTok{ << bsSPI_INTREN_ENDINTEN);}
\NormalTok{  \}  }
\NormalTok{\}}
\end{Highlighting}
\end{Shaded}

\hypertarget{spi-ux53d1ux9001ux6570ux636e-2}{%
\subsubsection{SPI 发送数据}\label{spi-ux53d1ux9001ux6570ux636e-2}}

\begin{Shaded}
\begin{Highlighting}[]
\DataTypeTok{uint32_t}\NormalTok{ write_data[DATA_LEN];}\CommentTok{//数据大小等于pParam.eDataSize,数组大小等于pParam.eWriteTransCnt}
\DataTypeTok{void}\NormalTok{ peripherals_spi_send_data(}\DataTypeTok{void}\NormalTok{)}
\NormalTok{\{}
  \CommentTok{// 写入命令,触发SPI传输}
\NormalTok{  apSSP_WriteCmd(APB_SSP1, }\BaseNTok{0x00}\NormalTok{, }\BaseNTok{0x00}\NormalTok{);}\CommentTok{//trigger transfer}
  
  \CommentTok{// 填写数据到TX FIFO,这个例子中DATA_LEN等于FIFO的深度(8),如果大于8,可以分为多次发送接收,}
  \CommentTok{// 每次发送完8个单位,需要读取RX FIFO中的数据}
  \ControlFlowTok{for}\NormalTok{(i = }\DecValTok{0}\NormalTok{; i < DATA_LEN; i++)}
\NormalTok{  \{}
\NormalTok{    apSSP_WriteFIFO(APB_SSP1, write_data[i]);}
\NormalTok{  \}}

  \CommentTok{// 等待发送结束}
  \ControlFlowTok{while}\NormalTok{(apSSP_GetSPIActiveStatus(APB_SSP1));}
  
  \CommentTok{// 读取当前RX FIFO中有效值的个数,然后从RX FIFO中读取返回值}
  \DataTypeTok{uint32_t}\NormalTok{ num = apSSP_GetDataNumInRxFifo(APB_SSP1);}
  \ControlFlowTok{for}\NormalTok{(i = }\DecValTok{0}\NormalTok{; i < num; i++)}
\NormalTok{  \{}
\NormalTok{    apSSP_ReadFIFO(APB_SSP1, &read_data[i]);}
\NormalTok{  \}}
\NormalTok{\}}
\end{Highlighting}
\end{Shaded}

\hypertarget{ux4f7fux7528ux6d41ux7a0b-14}{%
\subsubsection{使用流程}\label{ux4f7fux7528ux6d41ux7a0b-14}}

\begin{itemize}
\tightlist
\item
  设置GPIO,setup\_peripherals\_spi\_pin()
\item
  初始化SPI,setup\_peripherals\_spi\_module()
\item
  在需要时候发送SPI数据,peripherals\_spi\_send\_data()
\item
  检查中断状态
\end{itemize}

\hypertarget{spiux4eceux914dux7f6e-2}{%
\subsection{SPI从配置}\label{spiux4eceux914dux7f6e-2}}

\hypertarget{ux63a5ux53e3ux914dux7f6e-5}{%
\subsubsection{接口配置}\label{ux63a5ux53e3ux914dux7f6e-5}}

\begin{Shaded}
\begin{Highlighting}[]
\DataTypeTok{static} \DataTypeTok{void}\NormalTok{ setup_peripherals_spi_pin(}\DataTypeTok{void}\NormalTok{)}
\NormalTok{\{}
    \CommentTok{// 打开SPI模块时钟}
    \CommentTok{// 此处是以SPI1为例,如果是SPI0则需要更改}
    \CommentTok{// 根据使用的GPIO,选择打开GPIO0或者GPIO1的时钟}
\NormalTok{    SYSCTRL_ClearClkGateMulti(    (}\DecValTok{1}\NormalTok{ << SYSCTRL_ITEM_APB_SPI1)}
\NormalTok{                                | (}\DecValTok{1}\NormalTok{ << SYSCTRL_ITEM_APB_SysCtrl)}
\NormalTok{                                | (}\DecValTok{1}\NormalTok{ << SYSCTRL_ITEM_APB_PinCtrl)}
\NormalTok{                                | (}\DecValTok{1}\NormalTok{ << SYSCTRL_ITEM_APB_GPIO1)}
\NormalTok{                                | (}\DecValTok{1}\NormalTok{ << SYSCTRL_ITEM_APB_GPIO0));}

    \CommentTok{// 设置IO MUX,将GPIO映射成SPI功能,不需要的pin可以使用IO_NOT_A_PIN替代}
\NormalTok{    PINCTRL_Pull(IO_SOURCE_SPI1_CLK_IN,PINCTRL_PULL_DOWN);}
\NormalTok{    PINCTRL_Pull(IO_SOURCE_SPI1_CSN_IN,PINCTRL_PULL_UP);}\CommentTok{// CS 需要默认上拉}
\NormalTok{    PINCTRL_SelSpiIn(SPI_PORT_1, SPI_MIC_CLK, SPI_MIC_CS, SPI_MIC_HOLD, SPI_MIC_WP, SPI_MIC_MISO, SPI_MIC_MOSI);}
\NormalTok{    PINCTRL_SetPadMux(SPI_MIC_CLK, IO_SOURCE_SPI1_CLK_OUT);}
\NormalTok{    PINCTRL_SetPadMux(SPI_MIC_MISO, IO_SOURCE_SPI1_MISO_OUT);}
    
    \CommentTok{// 设置SPI的中断}
\NormalTok{    platform_set_irq_callback(PLATFORM_CB_IRQ_APBSPI, peripherals_spi_isr, NULL);}
\end{Highlighting}
\end{Shaded}

\hypertarget{spiux6a21ux5757ux521dux59cbux5316-5}{%
\subsubsection{SPI模块初始化}\label{spiux6a21ux5757ux521dux59cbux5316-5}}

常用设置项用注释标出,详细定义请参考``peripheral\_ssp.h''

\begin{Shaded}
\begin{Highlighting}[]
\CommentTok{// 示例,每次传输大小是8个word}
\PreprocessorTok{#define DATA_LEN (SPI_FIFO_DEPTH)}
\DataTypeTok{static} \DataTypeTok{void}\NormalTok{ setup_peripherals_spi_module(}\DataTypeTok{void}\NormalTok{)}
\NormalTok{\{}
\NormalTok{    apSSP_sDeviceControlBlock pParam;}
\NormalTok{    pParam.eSclkDiv = SPI_INTERFACETIMINGSCLKDIV_DEFAULT_2M;}\CommentTok{// SPI 时钟设置}
\NormalTok{    pParam.eSCLKPolarity = SPI_CPOL_SCLK_LOW_IN_IDLE_STATES;}\CommentTok{// SPI 模式设置}
\NormalTok{    pParam.eSCLKPhase = SPI_CPHA_ODD_SCLK_EDGES;}\CommentTok{// SPI 模式设置}
\NormalTok{    pParam.eLsbMsbOrder = SPI_LSB_MOST_SIGNIFICANT_BIT_FIRST;}
\NormalTok{    pParam.eDataSize = SPI_DATALEN_32_BITS;}\CommentTok{// SPI 每个传输单位的大小}
\NormalTok{    pParam.eMasterSlaveMode = SPI_SLVMODE_SLAVE_MODE;}\CommentTok{// SPI 从模式}
\NormalTok{    pParam.eReadWriteMode = SPI_TRANSMODE_WRITE_READ_SAME_TIME;}\CommentTok{// SPI 同时读写}
\NormalTok{    pParam.eQuadMode = SPI_DUALQUAD_REGULAR_MODE;}
\NormalTok{    pParam.eWriteTransCnt = DATA_LEN;}\CommentTok{// SPI每次传输多少个单位(每个单位的大小是pParam.eDataSize)}
\NormalTok{    pParam.eReadTransCnt = DATA_LEN;}\CommentTok{// SPI每次接收多少个单位(每个单位的大小是pParam.eDataSize)}
\NormalTok{    pParam.eAddrEn = SPI_ADDREN_DISABLE;}
\NormalTok{    pParam.eCmdEn = SPI_CMDEN_DISABLE;}
\NormalTok{    pParam.RxThres = DATA_LEN/}\DecValTok{2}\NormalTok{;}
\NormalTok{    pParam.TxThres = DATA_LEN/}\DecValTok{2}\NormalTok{;}
\NormalTok{    pParam.SlaveDataOnly = SPI_SLVDATAONLY_ENABLE;}
\NormalTok{    pParam.eAddrLen = SPI_ADDRLEN_1_BYTE;}
\NormalTok{    pParam.eInterruptMask = (}\DecValTok{1}\NormalTok{ << bsSPI_INTREN_ENDINTEN);}\CommentTok{// 打开SPI中断(传输结束后触发)}
  
\NormalTok{    apSSP_DeviceParametersSet(APB_SSP1, &pParam);}
\NormalTok{\}}
\end{Highlighting}
\end{Shaded}

\hypertarget{spi-ux63a5ux6536ux6570ux636e-2}{%
\subsubsection{SPI 接收数据}\label{spi-ux63a5ux6536ux6570ux636e-2}}

\begin{Shaded}
\begin{Highlighting}[]
\DataTypeTok{void}\NormalTok{ peripherals_spi_push_data(}\DataTypeTok{void}\NormalTok{)}
\NormalTok{\{}
    \ControlFlowTok{for}\NormalTok{(i = }\DecValTok{0}\NormalTok{; i < DATA_LEN; i++)}
\NormalTok{    \{}
\NormalTok{      apSSP_WriteFIFO(APB_SSP1, write_data[i]);}
\NormalTok{    \}}
\NormalTok{\}}

\DataTypeTok{uint32_t}\NormalTok{ read_data[DATA_LEN];}\CommentTok{//数据大小等于pParam.eDataSize,数组大小等于pParam.eReadTransCnt}
\DataTypeTok{static} \DataTypeTok{uint32_t}\NormalTok{ peripherals_spi_isr(}\DataTypeTok{void}\NormalTok{ *user_data)}
\NormalTok{\{}
  \DataTypeTok{uint32_t}\NormalTok{ stat = apSSP_GetIntRawStatus(APB_SSP1), i;}
  \ControlFlowTok{if}\NormalTok{(stat & (}\DecValTok{1}\NormalTok{ << bsSPI_INTREN_ENDINTEN))}
\NormalTok{  \{}
    \CommentTok{/* check if rx fifo still have some left data */}
    \CommentTok{// 检查当前RX FIFO中有效值的个数,根据个数读取RX FIFO}
    \DataTypeTok{uint32_t}\NormalTok{ num = apSSP_GetDataNumInRxFifo(APB_SSP1);}
    \ControlFlowTok{for}\NormalTok{(i = }\DecValTok{0}\NormalTok{; i < num; i++)}
\NormalTok{    \{}
\NormalTok{      apSSP_ReadFIFO(APB_SSP1, &read_data[i]);}
\NormalTok{    \}}

    \CommentTok{// 根据需要填充下一次发送的SPI数据}
\NormalTok{    peripherals_spi_push_data();}
    
\NormalTok{    apSSP_ClearIntStatus(APB_SSP1, }\DecValTok{1}\NormalTok{ << bsSPI_INTREN_ENDINTEN);}
    
\NormalTok{  \}}
\NormalTok{\}}
\end{Highlighting}
\end{Shaded}

\hypertarget{ux4f7fux7528ux6d41ux7a0b-15}{%
\subsubsection{使用流程}\label{ux4f7fux7528ux6d41ux7a0b-15}}

\begin{itemize}
\tightlist
\item
  设置GPIO,setup\_peripherals\_spi\_pin()
\item
  初始化SPI,setup\_peripherals\_spi\_module()
\item
  根据需要填充TX FIFO,peripherals\_spi\_push\_data()
\item
  观察SPI中断,中断触发代表当前传输结束
\end{itemize}

\begin{center}\rule{0.5\linewidth}{0.5pt}\end{center}

\hypertarget{ux573aux666f4ux540cux65f6ux8bfbux5199ux5e76ux4e14ux4f7fux7528dma}{%
\section{场景4:同时读写并且使用DMA}\label{ux573aux666f4ux540cux65f6ux8bfbux5199ux5e76ux4e14ux4f7fux7528dma}}

其中SPI主从配置为同时读写模式,同时使用DMA进行读写 配置之前需要决定使用的GPIO,如果是普通模式,则不需要SPI\_MIC\_WP和SPI\_MIC\_HOLD

\begin{Shaded}
\begin{Highlighting}[]
\PreprocessorTok{#define SPI_MIC_CLK         GIO_GPIO_10}
\PreprocessorTok{#define SPI_MIC_MOSI        GIO_GPIO_11}
\PreprocessorTok{#define SPI_MIC_MISO        GIO_GPIO_12}
\PreprocessorTok{#define SPI_MIC_CS          GIO_GPIO_13}
\PreprocessorTok{#define SPI_MIC_WP          GIO_GPIO_14}
\PreprocessorTok{#define SPI_MIC_HOLD        GIO_GPIO_15}

\CommentTok{// RX FIFO 和 TX FIFO 使用两个DMA channel}
\PreprocessorTok{#define SPI_DMA_TX_CHANNEL   (0)}\CommentTok{//DMA channel 0}
\PreprocessorTok{#define SPI_DMA_RX_CHANNEL   (1)}\CommentTok{//DMA channel 1}
\end{Highlighting}
\end{Shaded}

\hypertarget{spiux4e3bux914dux7f6e-3}{%
\subsection{SPI主配置}\label{spiux4e3bux914dux7f6e-3}}

\hypertarget{ux63a5ux53e3ux914dux7f6e-6}{%
\subsubsection{接口配置}\label{ux63a5ux53e3ux914dux7f6e-6}}

\begin{Shaded}
\begin{Highlighting}[]
\DataTypeTok{static} \DataTypeTok{void}\NormalTok{ setup_peripherals_spi_pin(}\DataTypeTok{void}\NormalTok{)}
\NormalTok{\{}
    \CommentTok{// 打开SPI模块时钟}
    \CommentTok{// 此处是以SPI1为例,如果是SPI0则需要更改}
    \CommentTok{// 根据使用的GPIO,选择打开GPIO0或者GPIO1的时钟}
\NormalTok{    SYSCTRL_ClearClkGateMulti(    (}\DecValTok{1}\NormalTok{ << SYSCTRL_ITEM_APB_SPI1)}
\NormalTok{                                | (}\DecValTok{1}\NormalTok{ << SYSCTRL_ITEM_APB_SysCtrl)}
\NormalTok{                                | (}\DecValTok{1}\NormalTok{ << SYSCTRL_ITEM_APB_PinCtrl)}
\NormalTok{                                | (}\DecValTok{1}\NormalTok{ << SYSCTRL_ITEM_APB_GPIO1)}
\NormalTok{                                | (}\DecValTok{1}\NormalTok{ << SYSCTRL_ITEM_APB_GPIO0));}

    \CommentTok{// 设置IO MUX,将GPIO映射成SPI功能,不需要的pin可以使用IO_NOT_A_PIN替代}
\NormalTok{    PINCTRL_SelSpiIn(SPI_PORT_1, SPI_MIC_CLK, SPI_MIC_CS, SPI_MIC_HOLD, SPI_MIC_WP, SPI_MIC_MISO, SPI_MIC_MOSI);}
\NormalTok{    PINCTRL_SetPadMux(SPI_MIC_CLK, IO_SOURCE_SPI1_CLK_OUT);}
\NormalTok{    PINCTRL_SetPadMux(SPI_MIC_CS, IO_SOURCE_SPI1_CSN_OUT);}
\NormalTok{    PINCTRL_SetPadMux(SPI_MIC_MOSI, IO_SOURCE_SPI1_MOSI_OUT);}
    
    \CommentTok{// 设置SPI的中断}
\NormalTok{    platform_set_irq_callback(PLATFORM_CB_IRQ_APBSPI, peripherals_spi_isr, NULL);}
\end{Highlighting}
\end{Shaded}

\hypertarget{spiux6a21ux5757ux521dux59cbux5316-6}{%
\subsubsection{SPI模块初始化}\label{spiux6a21ux5757ux521dux59cbux5316-6}}

常用设置项用注释标出,详细定义请参考``peripheral\_ssp.h''

\begin{Shaded}
\begin{Highlighting}[]
\CommentTok{// 示例,每次传输大小是8个word}
\PreprocessorTok{#define DATA_LEN (SPI_FIFO_DEPTH)}
\DataTypeTok{static} \DataTypeTok{void}\NormalTok{ setup_peripherals_spi_module(}\DataTypeTok{void}\NormalTok{)}
\NormalTok{\{}
\NormalTok{    apSSP_sDeviceControlBlock pParam;}
\NormalTok{    pParam.eSclkDiv = SPI_INTERFACETIMINGSCLKDIV_DEFAULT_2M;}\CommentTok{// SPI 时钟设置}
\NormalTok{    pParam.eSCLKPolarity = SPI_CPOL_SCLK_LOW_IN_IDLE_STATES;}\CommentTok{// SPI 模式设置}
\NormalTok{    pParam.eSCLKPhase = SPI_CPHA_ODD_SCLK_EDGES;}\CommentTok{// SPI 模式设置}
\NormalTok{    pParam.eLsbMsbOrder = SPI_LSB_MOST_SIGNIFICANT_BIT_FIRST;}
\NormalTok{    pParam.eDataSize = SPI_DATALEN_32_BITS;}\CommentTok{// SPI 每个传输单位的大小}
\NormalTok{    pParam.eMasterSlaveMode = SPI_SLVMODE_MASTER_MODE;}\CommentTok{// SPI主模式}
\NormalTok{    pParam.eReadWriteMode = SPI_TRANSMODE_WRITE_READ_SAME_TIME;}\CommentTok{// SPI 同时读写}
\NormalTok{    pParam.eQuadMode = SPI_DUALQUAD_REGULAR_MODE;}
\NormalTok{    pParam.eWriteTransCnt = DATA_LEN;}\CommentTok{// SPI每次传输多少个单位(每个单位的大小是pParam.eDataSize)}
\NormalTok{    pParam.eReadTransCnt = DATA_LEN;}\CommentTok{// SPI每次接收多少个单位(每个单位的大小是pParam.eDataSize)}
\NormalTok{    pParam.eAddrEn = SPI_ADDREN_DISABLE;}
\NormalTok{    pParam.eCmdEn = SPI_CMDEN_DISABLE;}
\NormalTok{    pParam.RxThres = DATA_LEN/}\DecValTok{2}\NormalTok{;}
\NormalTok{    pParam.TxThres = DATA_LEN/}\DecValTok{2}\NormalTok{;}
\NormalTok{    pParam.SlaveDataOnly = SPI_SLVDATAONLY_ENABLE;}
\NormalTok{    pParam.eAddrLen = SPI_ADDRLEN_1_BYTE;}
\NormalTok{    pParam.eInterruptMask = (}\DecValTok{1}\NormalTok{ << bsSPI_INTREN_ENDINTEN);}\CommentTok{// 打开SPI中断(传输结束后触发)}
  
\NormalTok{    apSSP_DeviceParametersSet(APB_SSP1, &pParam);}
    
\NormalTok{\}}
\end{Highlighting}
\end{Shaded}

\hypertarget{spi-dmaux521dux59cbux5316-2}{%
\subsubsection{SPI DMA初始化}\label{spi-dmaux521dux59cbux5316-2}}

// 初始化DMA模块

\begin{Shaded}
\begin{Highlighting}[]
\DataTypeTok{static} \DataTypeTok{void}\NormalTok{ setup_peripherals_dma_module(}\DataTypeTok{void}\NormalTok{)}
\NormalTok{\{}
\NormalTok{    SYSCTRL_ClearClkGateMulti(}\DecValTok{1}\NormalTok{ << SYSCTRL_ClkGate_APB_DMA);}
\NormalTok{    DMA_Reset(}\DecValTok{1}\NormalTok{);}
\NormalTok{    DMA_Reset(}\DecValTok{0}\NormalTok{);}
\NormalTok{\}}
\end{Highlighting}
\end{Shaded}

\hypertarget{spi-dmaux8bbeux7f6e-2}{%
\subsubsection{SPI DMA设置}\label{spi-dmaux8bbeux7f6e-2}}

// 此处是以SPI1为例

\begin{Shaded}
\begin{Highlighting}[]
\CommentTok{// 分别设置RX FIFO 和 TX FIFO的DMA}
\DataTypeTok{void}\NormalTok{ peripherals_spi_dma_to_txfifo(}\DataTypeTok{int}\NormalTok{ channel_id, }\DataTypeTok{void}\NormalTok{ *src, }\DataTypeTok{int}\NormalTok{ size)}
\NormalTok{\{}
\NormalTok{    DMA_Descriptor descriptor __attribute__((aligned (}\DecValTok{8}\NormalTok{)));}

\NormalTok{    descriptor.Next = (DMA_Descriptor *)}\DecValTok{0}\NormalTok{;}
\NormalTok{    DMA_PrepareMem2Peripheral(&descriptor,SYSCTRL_DMA_SPI1_TX,src,size,DMA_ADDRESS_INC,}\DecValTok{0}\NormalTok{);}

\NormalTok{    DMA_EnableChannel(channel_id, &descriptor);}
\NormalTok{\}}

\DataTypeTok{void}\NormalTok{ peripherals_spi_rxfifo_to_dma(}\DataTypeTok{int}\NormalTok{ channel_id, }\DataTypeTok{void}\NormalTok{ *dst, }\DataTypeTok{int}\NormalTok{ size)}
\NormalTok{\{}
\NormalTok{    DMA_Descriptor descriptor __attribute__((aligned (}\DecValTok{8}\NormalTok{)));}

\NormalTok{    descriptor.Next = (DMA_Descriptor *)}\DecValTok{0}\NormalTok{;}
\NormalTok{    DMA_PreparePeripheral2Mem(&descriptor,dst,SYSCTRL_DMA_SPI1_RX,size,DMA_ADDRESS_INC,}\DecValTok{0}\NormalTok{);}

\NormalTok{    DMA_EnableChannel(channel_id, &descriptor);}
\NormalTok{\}}
\end{Highlighting}
\end{Shaded}

\hypertarget{spi-ux4e2dux65ad-4}{%
\subsubsection{SPI 中断}\label{spi-ux4e2dux65ad-4}}

\begin{Shaded}
\begin{Highlighting}[]
\CommentTok{// SPI ENDINT中断触发标志传输结束,清除中断状态}
\DataTypeTok{static} \DataTypeTok{uint32_t}\NormalTok{ peripherals_spi_isr(}\DataTypeTok{void}\NormalTok{ *user_data)}
\NormalTok{\{}
  \DataTypeTok{uint32_t}\NormalTok{ stat = apSSP_GetIntRawStatus(APB_SSP1);}
  
  \ControlFlowTok{if}\NormalTok{(stat & (}\DecValTok{1}\NormalTok{ << bsSPI_INTREN_ENDINTEN))}
\NormalTok{  \{}
\NormalTok{    apSSP_ClearIntStatus(APB_SSP1, }\DecValTok{1}\NormalTok{ << bsSPI_INTREN_ENDINTEN);}
\NormalTok{  \}  }
\NormalTok{\}}
\end{Highlighting}
\end{Shaded}

\hypertarget{spi-ux63a5ux6536ux6570ux636e-3}{%
\subsubsection{SPI 接收数据}\label{spi-ux63a5ux6536ux6570ux636e-3}}

\begin{Shaded}
\begin{Highlighting}[]
\DataTypeTok{uint32_t}\NormalTok{ read_data[DATA_LEN] = \{}\DecValTok{0}\NormalTok{,\};}\CommentTok{//数据大小等于pParam.eDataSize,数组大小等于pParam.eReadTransCnt}

\DataTypeTok{void}\NormalTok{ peripherals_spi_read_data(}\DataTypeTok{void}\NormalTok{)}
\NormalTok{\{}
  \CommentTok{// 首先需要打开 SPI 模块中的DMA功能}
\NormalTok{  apSSP_SetRxDmaEn(APB_SSP1,}\DecValTok{1}\NormalTok{);}
\NormalTok{  apSSP_SetTransferControlRdTranCnt(APB_SSP1,DATA_LEN);}
  \CommentTok{// 配置DMA,指向存储接收数据的地址}
\NormalTok{  peripherals_spi_rxfifo_to_dma(SPI_DMA_RX_CHANNEL, read_data, }\KeywordTok{sizeof}\NormalTok{(read_data));}
\NormalTok{\}}
\end{Highlighting}
\end{Shaded}

\hypertarget{spi-ux53d1ux9001ux6570ux636e-3}{%
\subsubsection{SPI 发送数据}\label{spi-ux53d1ux9001ux6570ux636e-3}}

\begin{Shaded}
\begin{Highlighting}[]
\DataTypeTok{uint32_t}\NormalTok{ write_data[DATA_LEN];}\CommentTok{//数据大小等于pParam.eDataSize,数组大小等于pParam.eWriteTransCnt}

\DataTypeTok{void}\NormalTok{ peripherals_spi_push_data(}\DataTypeTok{void}\NormalTok{)}
\NormalTok{\{}
  \CommentTok{// 首先需要打开 SPI 模块中的DMA功能}
\NormalTok{  apSSP_SetTxDmaEn(APB_SSP1,}\DecValTok{1}\NormalTok{);}
\NormalTok{  apSSP_SetTransferControlWrTranCnt(APB_SSP1,DATA_LEN);}
  \CommentTok{// 配置DMA,指向需要发送的数据}
\NormalTok{  peripherals_spi_dma_to_txfifo(SPI_DMA_TX_CHANNEL, write_data, }\KeywordTok{sizeof}\NormalTok{(write_data));}

\NormalTok{\}}

\DataTypeTok{void}\NormalTok{ peripherals_spi_send_data(}\DataTypeTok{void}\NormalTok{)}
\NormalTok{\{}
  \CommentTok{// 分别设置接收和发射的DMA}
\NormalTok{  peripherals_spi_read_data();}
\NormalTok{  peripherals_spi_push_data();}
  
  \CommentTok{// 写入命令,触发SPI传输}
\NormalTok{  apSSP_WriteCmd(APB_SSP1, }\BaseNTok{0x00}\NormalTok{, }\BaseNTok{0x00}\NormalTok{);}\CommentTok{//trigger transfer}

  \CommentTok{// 等待发送结束}
  \ControlFlowTok{while}\NormalTok{(apSSP_GetSPIActiveStatus(APB_SSP1));}
  
  \CommentTok{// 关闭 SPI 模块中的DMA功能}
\NormalTok{  apSSP_SetTxDmaEn(APB_SSP1,}\DecValTok{0}\NormalTok{);}
\NormalTok{\}}
\end{Highlighting}
\end{Shaded}

\hypertarget{ux4f7fux7528ux6d41ux7a0b-16}{%
\subsubsection{使用流程}\label{ux4f7fux7528ux6d41ux7a0b-16}}

\begin{itemize}
\tightlist
\item
  设置GPIO,setup\_peripherals\_spi\_pin()
\item
  初始化SPI,setup\_peripherals\_spi\_module()
\item
  初始化DMA,setup\_peripherals\_dma\_module()
\item
  在需要时候发送SPI数据,peripherals\_spi\_send\_data()
\item
  检查中断状态
\end{itemize}

\hypertarget{spiux4eceux914dux7f6e-3}{%
\subsection{SPI从配置}\label{spiux4eceux914dux7f6e-3}}

\hypertarget{ux63a5ux53e3ux914dux7f6e-7}{%
\subsubsection{接口配置}\label{ux63a5ux53e3ux914dux7f6e-7}}

\begin{Shaded}
\begin{Highlighting}[]
\DataTypeTok{static} \DataTypeTok{void}\NormalTok{ setup_peripherals_spi_pin(}\DataTypeTok{void}\NormalTok{)}
\NormalTok{\{}
    \CommentTok{// 打开SPI模块时钟}
    \CommentTok{// 此处是以SPI1为例,如果是SPI0则需要更改}
    \CommentTok{// 根据使用的GPIO,选择打开GPIO0或者GPIO1的时钟}
\NormalTok{    SYSCTRL_ClearClkGateMulti(    (}\DecValTok{1}\NormalTok{ << SYSCTRL_ITEM_APB_SPI1)}
\NormalTok{                                | (}\DecValTok{1}\NormalTok{ << SYSCTRL_ITEM_APB_SysCtrl)}
\NormalTok{                                | (}\DecValTok{1}\NormalTok{ << SYSCTRL_ITEM_APB_PinCtrl)}
\NormalTok{                                | (}\DecValTok{1}\NormalTok{ << SYSCTRL_ITEM_APB_GPIO1)}
\NormalTok{                                | (}\DecValTok{1}\NormalTok{ << SYSCTRL_ITEM_APB_GPIO0));}

    \CommentTok{// 设置IO MUX,将GPIO映射成SPI功能,不需要的pin可以使用IO_NOT_A_PIN替代}
\NormalTok{    PINCTRL_Pull(IO_SOURCE_SPI1_CLK_IN,PINCTRL_PULL_DOWN);}
\NormalTok{    PINCTRL_Pull(IO_SOURCE_SPI1_CSN_IN,PINCTRL_PULL_UP);}\CommentTok{// CS 需要默认上拉}
\NormalTok{    PINCTRL_SelSpiIn(SPI_PORT_1, SPI_MIC_CLK, SPI_MIC_CS, SPI_MIC_HOLD, SPI_MIC_WP, SPI_MIC_MISO, SPI_MIC_MOSI);}
\NormalTok{    PINCTRL_SetPadMux(SPI_MIC_CLK, IO_SOURCE_SPI1_CLK_OUT);}
\NormalTok{    PINCTRL_SetPadMux(SPI_MIC_MISO, IO_SOURCE_SPI1_MISO_OUT);}
    
    \CommentTok{// 设置SPI的中断}
\NormalTok{    platform_set_irq_callback(PLATFORM_CB_IRQ_APBSPI, peripherals_spi_isr, NULL);}
\end{Highlighting}
\end{Shaded}

\hypertarget{spiux6a21ux5757ux521dux59cbux5316-7}{%
\subsubsection{SPI模块初始化}\label{spiux6a21ux5757ux521dux59cbux5316-7}}

常用设置项用注释标出,详细定义请参考``peripheral\_ssp.h''

\begin{Shaded}
\begin{Highlighting}[]
\CommentTok{// 示例,每次传输大小是8个word}
\PreprocessorTok{#define DATA_LEN (SPI_FIFO_DEPTH)}
\DataTypeTok{static} \DataTypeTok{void}\NormalTok{ setup_peripherals_spi_module(}\DataTypeTok{void}\NormalTok{)}
\NormalTok{\{}
\NormalTok{    apSSP_sDeviceControlBlock pParam;}
\NormalTok{    pParam.eSclkDiv = SPI_INTERFACETIMINGSCLKDIV_DEFAULT_2M;}\CommentTok{// SPI 时钟设置}
\NormalTok{    pParam.eSCLKPolarity = SPI_CPOL_SCLK_LOW_IN_IDLE_STATES;}\CommentTok{// SPI 模式设置}
\NormalTok{    pParam.eSCLKPhase = SPI_CPHA_ODD_SCLK_EDGES;}\CommentTok{// SPI 模式设置}
\NormalTok{    pParam.eLsbMsbOrder = SPI_LSB_MOST_SIGNIFICANT_BIT_FIRST;}
\NormalTok{    pParam.eDataSize = SPI_DATALEN_32_BITS;}\CommentTok{// SPI 每个传输单位的大小}
\NormalTok{    pParam.eMasterSlaveMode = SPI_SLVMODE_SLAVE_MODE;}\CommentTok{// SPI 从模式}
\NormalTok{    pParam.eReadWriteMode = SPI_TRANSMODE_WRITE_READ_SAME_TIME;}\CommentTok{// SPI 同时读写}
\NormalTok{    pParam.eQuadMode = SPI_DUALQUAD_REGULAR_MODE;}
\NormalTok{    pParam.eWriteTransCnt = DATA_LEN;}\CommentTok{// SPI每次传输多少个单位(每个单位的大小是pParam.eDataSize)}
\NormalTok{    pParam.eReadTransCnt = DATA_LEN;}\CommentTok{// SPI每次接收多少个单位(每个单位的大小是pParam.eDataSize)}
\NormalTok{    pParam.eAddrEn = SPI_ADDREN_DISABLE;}
\NormalTok{    pParam.eCmdEn = SPI_CMDEN_DISABLE;}
\NormalTok{    pParam.RxThres = }\DecValTok{0}\NormalTok{;}
\NormalTok{    pParam.TxThres = }\DecValTok{0}\NormalTok{;}
\NormalTok{    pParam.SlaveDataOnly = SPI_SLVDATAONLY_ENABLE;}
\NormalTok{    pParam.eAddrLen = SPI_ADDRLEN_1_BYTE;}
\NormalTok{    pParam.eInterruptMask = (}\DecValTok{1}\NormalTok{ << bsSPI_INTREN_ENDINTEN);}\CommentTok{// 打开SPI中断(传输结束后触发)}
  
\NormalTok{    apSSP_DeviceParametersSet(APB_SSP1, &pParam);}
\NormalTok{\}}
\end{Highlighting}
\end{Shaded}

\hypertarget{spi-dmaux521dux59cbux5316-3}{%
\subsubsection{SPI DMA初始化}\label{spi-dmaux521dux59cbux5316-3}}

// 初始化DMA模块

\begin{Shaded}
\begin{Highlighting}[]
\DataTypeTok{static} \DataTypeTok{void}\NormalTok{ setup_peripherals_dma_module(}\DataTypeTok{void}\NormalTok{)}
\NormalTok{\{}
\NormalTok{    SYSCTRL_ClearClkGateMulti(}\DecValTok{1}\NormalTok{ << SYSCTRL_ClkGate_APB_DMA);}
\NormalTok{    DMA_Reset(}\DecValTok{1}\NormalTok{);}
\NormalTok{    DMA_Reset(}\DecValTok{0}\NormalTok{);}
\NormalTok{\}}
\end{Highlighting}
\end{Shaded}

\hypertarget{spi-dmaux8bbeux7f6e-3}{%
\subsubsection{SPI DMA设置}\label{spi-dmaux8bbeux7f6e-3}}

// 此处是以SPI1为例

\begin{Shaded}
\begin{Highlighting}[]
\CommentTok{// 分别设置RX FIFO 和 TX FIFO的DMA}
\DataTypeTok{void}\NormalTok{ peripherals_spi_dma_to_txfifo(}\DataTypeTok{int}\NormalTok{ channel_id, }\DataTypeTok{void}\NormalTok{ *src, }\DataTypeTok{int}\NormalTok{ size)}
\NormalTok{\{}
\NormalTok{    DMA_Descriptor descriptor __attribute__((aligned (}\DecValTok{8}\NormalTok{)));}

\NormalTok{    descriptor.Next = (DMA_Descriptor *)}\DecValTok{0}\NormalTok{;}
\NormalTok{    DMA_PrepareMem2Peripheral(&descriptor,SYSCTRL_DMA_SPI1_TX,src,size,DMA_ADDRESS_INC,}\DecValTok{0}\NormalTok{);}

\NormalTok{    DMA_EnableChannel(channel_id, &descriptor);}
\NormalTok{\}}

\DataTypeTok{void}\NormalTok{ peripherals_spi_rxfifo_to_dma(}\DataTypeTok{int}\NormalTok{ channel_id, }\DataTypeTok{void}\NormalTok{ *dst, }\DataTypeTok{int}\NormalTok{ size)}
\NormalTok{\{}
\NormalTok{    DMA_Descriptor descriptor __attribute__((aligned (}\DecValTok{8}\NormalTok{)));}

\NormalTok{    descriptor.Next = (DMA_Descriptor *)}\DecValTok{0}\NormalTok{;}
\NormalTok{    DMA_PreparePeripheral2Mem(&descriptor,dst,SYSCTRL_DMA_SPI1_RX,size,DMA_ADDRESS_INC,}\DecValTok{0}\NormalTok{);}

\NormalTok{    DMA_EnableChannel(channel_id, &descriptor);}
\NormalTok{\}}
\end{Highlighting}
\end{Shaded}

\hypertarget{spi-ux63a5ux6536ux6570ux636e-4}{%
\subsubsection{SPI 接收数据}\label{spi-ux63a5ux6536ux6570ux636e-4}}

\begin{Shaded}
\begin{Highlighting}[]
\DataTypeTok{uint32_t}\NormalTok{ read_data[DATA_LEN] = \{}\DecValTok{0}\NormalTok{,\};}\CommentTok{//数据大小等于pParam.eDataSize,数组大小等于pParam.eReadTransCnt}

\DataTypeTok{void}\NormalTok{ peripherals_spi_read_data(}\DataTypeTok{void}\NormalTok{)}
\NormalTok{\{}
  \CommentTok{// 首先需要打开 SPI 模块中的DMA功能}
\NormalTok{  apSSP_SetRxDmaEn(APB_SSP1,}\DecValTok{1}\NormalTok{);}
\NormalTok{  apSSP_SetTransferControlRdTranCnt(APB_SSP1,DATA_LEN);}
  \CommentTok{// 配置DMA,指向存储接收数据的地址}
\NormalTok{  peripherals_spi_rxfifo_to_dma(SPI_DMA_RX_CHANNEL, read_data, }\KeywordTok{sizeof}\NormalTok{(read_data));}
\NormalTok{\}}
\end{Highlighting}
\end{Shaded}

\hypertarget{spi-ux53d1ux9001ux6570ux636e-4}{%
\subsubsection{SPI 发送数据}\label{spi-ux53d1ux9001ux6570ux636e-4}}

\begin{Shaded}
\begin{Highlighting}[]
\DataTypeTok{uint32_t}\NormalTok{ write_data[DATA_LEN];}\CommentTok{//数据大小等于pParam.eDataSize,数组大小等于pParam.eWriteTransCnt}

\DataTypeTok{void}\NormalTok{ peripherals_spi_push_data(}\DataTypeTok{void}\NormalTok{)}
\NormalTok{\{}
  \CommentTok{// 首先需要打开 SPI 模块中的DMA功能}
\NormalTok{  apSSP_SetTxDmaEn(APB_SSP1,}\DecValTok{1}\NormalTok{);}
\NormalTok{  apSSP_SetTransferControlWrTranCnt(APB_SSP1,DATA_LEN);}
  \CommentTok{// 配置DMA,指向需要发送的数据}
\NormalTok{  peripherals_spi_dma_to_txfifo(SPI_DMA_TX_CHANNEL, write_data, }\KeywordTok{sizeof}\NormalTok{(write_data));}

\NormalTok{\}}
\end{Highlighting}
\end{Shaded}

\hypertarget{spi-ux4e2dux65ad-5}{%
\subsubsection{SPI 中断}\label{spi-ux4e2dux65ad-5}}

\begin{Shaded}
\begin{Highlighting}[]
\CommentTok{// SPI ENDINT中断触发标志传输结束,清除中断状态}
\DataTypeTok{static} \DataTypeTok{uint32_t}\NormalTok{ peripherals_spi_isr(}\DataTypeTok{void}\NormalTok{ *user_data)}
\NormalTok{\{}
  \DataTypeTok{uint32_t}\NormalTok{ stat = apSSP_GetIntRawStatus(APB_SSP1);}
  
  \ControlFlowTok{if}\NormalTok{(stat & (}\DecValTok{1}\NormalTok{ << bsSPI_INTREN_ENDINTEN))}
\NormalTok{  \{}
    \CommentTok{// 根据情况决定是否需要准备下一次的接收和发射}
\NormalTok{    peripherals_spi_read_data();}
\NormalTok{    peripherals_spi_push_data();}
\NormalTok{    apSSP_ClearIntStatus(APB_SSP1, }\DecValTok{1}\NormalTok{ << bsSPI_INTREN_ENDINTEN);}
\NormalTok{  \}  }
\NormalTok{\}}
\end{Highlighting}
\end{Shaded}

\hypertarget{ux4f7fux7528ux6d41ux7a0b-17}{%
\subsubsection{使用流程}\label{ux4f7fux7528ux6d41ux7a0b-17}}

\begin{itemize}
\tightlist
\item
  设置GPIO,setup\_peripherals\_spi\_pin()
\item
  初始化SPI,setup\_peripherals\_spi\_module()
\item
  初始化DMA,setup\_peripherals\_dma\_module()
\item
  设置接收DMA,peripherals\_spi\_read\_data();
\item
  设置发射DMA,peripherals\_spi\_push\_data();
\item
  观察SPI中断,中断触发代表当前接收结束
\end{itemize}

\begin{center}\rule{0.5\linewidth}{0.5pt}\end{center}

\hypertarget{spiux4f7fux7528ux5176ux4ed6ux914dux7f6e}{%
\section{SPI使用其他配置}\label{spiux4f7fux7528ux5176ux4ed6ux914dux7f6e}}

\hypertarget{spi-clockux914dux7f6e}{%
\subsection{SPI clock配置}\label{spi-clockux914dux7f6e}}

\hypertarget{ux9ed8ux8ba4ux914dux7f6e}{%
\subsubsection{默认配置}\label{ux9ed8ux8ba4ux914dux7f6e}}

对于默认配置,spi时钟的配置通过 ``pParam.eSclkDiv'' 来实现,计算公式为: ``spi interface clock / (2 * (eSclkDiv + 1))'',其中默认配置下,spi interface clock 为24M,因此可以得到不同时钟下的eSclkDiv:

\begin{Shaded}
\begin{Highlighting}[]
\PreprocessorTok{#define SPI_INTERFACETIMINGSCLKDIV_DEFAULT_6M    (1)}
\PreprocessorTok{#define SPI_INTERFACETIMINGSCLKDIV_DEFAULT_4M    (2)}
\PreprocessorTok{#define SPI_INTERFACETIMINGSCLKDIV_DEFAULT_3M    (3)}
\PreprocessorTok{#define SPI_INTERFACETIMINGSCLKDIV_DEFAULT_2M4   (4)}
\PreprocessorTok{#define SPI_INTERFACETIMINGSCLKDIV_DEFAULT_2M    (5)}
\end{Highlighting}
\end{Shaded}

详细请参考``peripheral\_ssp.h''

\hypertarget{hclkux914dux7f6e}{%
\subsubsection{HCLK配置}\label{hclkux914dux7f6e}}

对于更高的时钟,则需要在配置``pParam.eSclkDiv''之前打开HCLK配置,打开方式如下:

\begin{Shaded}
\begin{Highlighting}[]
\NormalTok{    SYSCTRL_SelectHClk(SYSCTRL_CLK_PLL_DIV_1+}\DecValTok{3}\NormalTok{);}
\NormalTok{    SYSCTRL_SelectSpiClk(SPI_PORT_1,SYSCTRL_CLK_HCLK);}
\end{Highlighting}
\end{Shaded}

其中PLL为384M,``SYSCTRL\_CLK\_PLL\_DIV\_1+3''代表4分频(推荐配置),可以将spi interface clock 提高到384M/4 = 96M。在此基础上通过计算公式``spi interface clock / (2 * (eSclkDiv + 1))'' 得到不同时钟下的eSclkDiv:

\begin{Shaded}
\begin{Highlighting}[]
\PreprocessorTok{#define SPI_INTERFACETIMINGSCLKDIV_HCLK_24M    (1)}
\PreprocessorTok{#define SPI_INTERFACETIMINGSCLKDIV_HCLK_12M    (3)}
\end{Highlighting}
\end{Shaded}

详细请参考``peripheral\_ssp.h'' 其中spi interface clock的大小可以在配置后,通过``SYSCTRL\_GetClk()''来获取

ATTENTION : 此处的API使用均以SPI1(SPI\_PORT\_1)为例,具体使用请根据需要调整

\hypertarget{qspi-ux4f7fux7528}{%
\subsection{QSPI 使用}\label{qspi-ux4f7fux7528}}

QSPI的使用需要在以上的配置的基础上,做一些额外修改 \#\#\#\# pin 配置 QSPI用到了CLK,CS,MOSI,MISO,HOLD,WP,主从都需要配置为输入输出

\begin{Shaded}
\begin{Highlighting}[]
\NormalTok{    PINCTRL_SelSpiIn(SPI_PORT_1, SPI_MIC_CLK, SPI_MIC_CS, SPI_MIC_HOLD, SPI_MIC_WP, SPI_MIC_MISO, SPI_MIC_MOSI);}
\NormalTok{    PINCTRL_SetPadMux(SPI_MIC_CLK, IO_SOURCE_SPI1_CLK_OUT);}
\NormalTok{    PINCTRL_SetPadMux(SPI_MIC_CS, IO_SOURCE_SPI1_CSN_OUT);}
\NormalTok{    PINCTRL_SetPadMux(SPI_MIC_MOSI, IO_SOURCE_SPI1_MOSI_OUT);}
\NormalTok{    PINCTRL_SetPadMux(SPI_MIC_MISO, IO_SOURCE_SPI1_MISO_OUT);}
\NormalTok{    PINCTRL_SetPadMux(SPI_MIC_WP, IO_SOURCE_SPI1_WP_OUT);}
\NormalTok{    PINCTRL_SetPadMux(SPI_MIC_HOLD, IO_SOURCE_SPI1_HOLD_OUT);}
\end{Highlighting}
\end{Shaded}

\hypertarget{pparam-ux914dux7f6e}{%
\subsubsection{pParam 配置}\label{pparam-ux914dux7f6e}}

QSPI的部分pParam参数需要修改为如下,同时Addr和Cmd需要打开

\begin{Shaded}
\begin{Highlighting}[]
\NormalTok{    pParam.eQuadMode = SPI_DUALQUAD_QUAD_IO_MODE;}
\NormalTok{    pParam.SlaveDataOnly = SPI_SLVDATAONLY_DISABLE;}
\NormalTok{    pParam.eAddrEn = SPI_ADDREN_ENABLE;}
\NormalTok{    pParam.eCmdEn = SPI_CMDEN_ENABLE;}
\end{Highlighting}
\end{Shaded}

此外,主还需要切换读写Mode(读和写顺序执行,比如首先利用四线完成写操作,然后再读)

\begin{Shaded}
\begin{Highlighting}[]
\NormalTok{    pParam.eReadWriteMode = SPI_TRANSMODE_WRITE_READ;}
\end{Highlighting}
\end{Shaded}

或者(在Write和Read之间添加dummy(默认为8个clk cycle))

\begin{Shaded}
\begin{Highlighting}[]
\NormalTok{    pParam.eReadWriteMode = SPI_TRANSMODE_WRITE_DUMMY_READ;}
\end{Highlighting}
\end{Shaded}

从则添加

\begin{Shaded}
\begin{Highlighting}[]
\NormalTok{    pParam.eReadWriteMode = SPI_TRANSMODE_READ_WRITE;}
\end{Highlighting}
\end{Shaded}

或者(在Read和Write之间添加dummy(默认为8个clk cycle))

\begin{Shaded}
\begin{Highlighting}[]
\NormalTok{    pParam.eReadWriteMode = SPI_TRANSMODE_READ_DUMMY_WRITE;}
\end{Highlighting}
\end{Shaded}

\begin{center}\rule{0.5\linewidth}{0.5pt}\end{center}

\hypertarget{ch-sysctrl}{%
\chapter{系统控制(SYSCTRL)}\label{ch-sysctrl}}

\hypertarget{ux529fux80fdux6982ux8ff0-4}{%
\section{功能概述}\label{ux529fux80fdux6982ux8ff0-4}}

SYSCTRL 负责管理、控制各种片上外设,主要功能有:

\begin{itemize}
\tightlist
\item
  外设的复位
\item
  外设的时钟管理,包括时钟源、频率设置、门控等
\item
  DMA 规划
\item
  其它功能
\end{itemize}

\hypertarget{ux5916ux8bbeux6807ux8bc6}{%
\subsection{外设标识}\label{ux5916ux8bbeux6807ux8bc6}}

SYSCTRL 为外设定义了几种不同的标识。最常见的一种标识为:

\begin{Shaded}
\begin{Highlighting}[]
\KeywordTok{typedef} \KeywordTok{enum}
\NormalTok{\{}
\NormalTok{    SYSCTRL_ITEM_APB_GPIO0     ,}
\NormalTok{    SYSCTRL_ITEM_APB_GPIO1     ,}
    \CommentTok{// ...}
\NormalTok{    SYSCTRL_ITEM_NUMBER,}
\NormalTok{\} SYSCTRL_Item;}
\end{Highlighting}
\end{Shaded}

这种标识用于外设的复位、时钟门控等。\texttt{SYSCTRL\_ResetItem} 和 \texttt{SYSCTRL\_ClkGateItem} 是
\texttt{SYSCTRL\_Item} 的两个别名。

下面这种标识用于 DMA 规划:

\begin{Shaded}
\begin{Highlighting}[]
\KeywordTok{typedef} \KeywordTok{enum}
\NormalTok{\{}
\NormalTok{    SYSCTRL_DMA_UART0_RX = }\DecValTok{0}\NormalTok{,}
\NormalTok{    SYSCTRL_DMA_UART1_RX = }\DecValTok{1}\NormalTok{,}
    \CommentTok{//...}
\NormalTok{\} SYSCTRL_DMA;}
\end{Highlighting}
\end{Shaded}

\hypertarget{ux65f6ux949fux6811}{%
\subsection{时钟树}\label{ux65f6ux949fux6811}}

\begin{enumerate}
\def\labelenumi{\arabic{enumi}.}
\item
  32KiHz 时钟(\emph{clk\_32k})

  32k 时钟有两个来源:内部 RC 电路,外部 32768Hz 晶体。
\item
  PLL 输入的 24MHz 时钟(\emph{clk\_pll\_in})

  24MHz 时钟有两个来源:内部 RC 电路,外部 24MHz 晶体。
\item
  PLL 输出(\emph{clk\_pll})

  \emph{clk\_pll} 的频率 \(f_{pll}\) 可配置,受 \(loop\)、\(div_{pre}\) 和 \(div_{output}\) 等 3 个参数控制:

  \[f_{vco}=\frac{f_{osc}\times loop}{div_{pre}}\]

  \[f_{pll}=\frac{f_{vco}}{div_{output}}\]

  这里,\(f_{osc}\) 即 \emph{clk\_pll\_in} 的频率。要求 \(f_{vco} \in [60,600]MHz\), \(f_{osc}/div_{pre} \in [2,24]MHz\)。
\item
  \emph{sclk\_fast} 与 \emph{sclk\_slow}

  \emph{clk\_pll} 经过门控后的时钟称为 \emph{sclk\_fast},24MHz 时钟 \emph{clk\_pll\_in} 经过门控后称为 \emph{sclk\_slow}。
\item
  \emph{hclk}

  \emph{sclk\_fast} 经过分频后得到 \emph{hclk}。下列外设(包括 MCU)固定使用这个时钟\footnote{每个外设可单独对 \emph{hclk} 门控。}:

  \begin{itemize}
  \tightlist
  \item
    DMA
  \item
    片内 Flash
  \item
    QSPI
  \item
    USB\footnote{仅高速时钟。}
  \item
    其它内部模块如 AES、Cache 等
  \end{itemize}

  \emph{hclk} 经过分频后得到 \emph{pclk}。\emph{pclk} 主要用于硬件内部接口。
\item
  \emph{sclk\_slow} 的进一步分频

  \emph{sclk\_slow} 经过若干独立的分频器得到以下多种时钟:

  \begin{itemize}
  \tightlist
  \item
    \emph{sclk\_slow\_pwm\_div}:专供 PWM 选择使用
  \item
    \emph{sclk\_slow\_timer\_div}:供 TIMER0、TIMER1、TIMER2 选择使用
  \item
    \emph{sclk\_slow\_ks\_div}:专供 KeyScan 选择使用
  \item
    \emph{sclk\_slow\_adc\_div}:供 EFUSE、ADC、IR 选择使用
  \item
    \emph{sclk\_slow\_pdm\_div}:专供 PDM 选择使用
  \end{itemize}
\item
  \emph{sclk\_fast} 的进一步分频:

  \emph{sclk\_fast} 经过若干独立的分频器得到以下多种时钟:

  \begin{itemize}
  \tightlist
  \item
    \emph{sclk\_fast\_i2s\_div}:专供 I2S 选择使用
  \item
    \emph{sclk\_fast\_qspi\_div}:专供 QSPI 选择使用
  \item
    \emph{sclk\_fast\_flash\_div}:专供片内 Flash 选择使用
  \item
    \emph{sclk\_fast\_usb\_div}:专供 USB 选择使用
  \end{itemize}
\end{enumerate}

各硬件外设可配置的时钟源汇总如表 \ref{tab:ch-sysctrl-tab-clk}。

\begin{longtable}[]{@{}ll@{}}
\caption{\label{tab:ch-sysctrl-tab-clk} 各硬件外设的时钟源}\tabularnewline
\toprule
外设 & 时钟源\tabularnewline
\midrule
\endfirsthead
\toprule
外设 & 时钟源\tabularnewline
\midrule
\endhead
GPIO0、GPIO1 & 选择 \emph{sclk\_slow} 或者 \emph{clk\_32k}\tabularnewline
TMR0、TMR1、TMR2 & 独立配置 \emph{sclk\_slow\_timer\_div} 或者 \emph{clk\_32k}\tabularnewline
WDT & \emph{clk\_32k}\tabularnewline
PWM & \emph{sclk\_slow\_pwm\_div} 或者 \emph{clk\_32k}\tabularnewline
PDM & \emph{sclk\_slow\_pdm\_div}\tabularnewline
QDEC & 对 \emph{hclk} 或者 \emph{sclk\_slow} 分频\tabularnewline
KeyScan & \emph{sclk\_slow\_ks\_div} 或者 \emph{clk\_32k}\tabularnewline
IR、ADC、EFUSE & 独立配置 \emph{sclk\_slow\_adc\_div} 或者 \emph{sclk\_slow}\tabularnewline
DMA & \emph{hclk}\tabularnewline
SPI0 & \emph{sclk\_fast\_qspi\_div} 或者 \emph{sclk\_slow}\tabularnewline
I2S & \emph{sclk\_fast\_i2s\_div} 或者 \emph{sclk\_slow}\tabularnewline
UART0、UART1、SPI1 & 独立配置 \emph{hclk} 或者 \emph{sclk\_slow}\tabularnewline
I2C0、I2C1 & \emph{sclk\_slow}\tabularnewline
\bottomrule
\end{longtable}

\hypertarget{dma-ux89c4ux5212}{%
\subsection{DMA 规划}\label{dma-ux89c4ux5212}}

由于 \protect\hyperlink{ch-dma}{DMA} 支持的硬件握手信号只有 16 种,无法同时支持所有外设。
因此需要事先确定将要的外设握手信号,并通过 \texttt{SYSCTRL\_SelectUsedDmaItems} 接口声明。

一个外设可能具备一个以上的握手信号,需要注意区分。比如 UART0 有两个握手信号 UART0\_RX 和
UART0\_TX,分别用于触发 DMA 发送请求(通过 DMA 传输接收到的数据)和读取请求(向 DMA 请
求新的待发送数据)。外设握手信号定义在 \texttt{SYSCTRL\_DMA} 内:

\begin{Shaded}
\begin{Highlighting}[]
\KeywordTok{typedef} \KeywordTok{enum}
\NormalTok{\{}
\NormalTok{    SYSCTRL_DMA_UART0_RX = }\DecValTok{0}\NormalTok{,}
\NormalTok{    SYSCTRL_DMA_UART1_RX = }\DecValTok{1}\NormalTok{,}
    \CommentTok{// ...}
\NormalTok{\} SYSCTRL_DMA;}
\end{Highlighting}
\end{Shaded}

\hypertarget{ux4f7fux7528ux8bf4ux660e-5}{%
\section{使用说明}\label{ux4f7fux7528ux8bf4ux660e-5}}

\hypertarget{ux5916ux8bbeux590dux4f4d}{%
\subsection{外设复位}\label{ux5916ux8bbeux590dux4f4d}}

通过 \texttt{SYSCTRL\_ResetBlock} 复位外设,通过 \texttt{SYSCTRL\_ReleaseBlock} 释放复位。

\begin{Shaded}
\begin{Highlighting}[]
\DataTypeTok{void}\NormalTok{ SYSCTRL_ResetBlock(SYSCTRL_ResetItem item);}
\DataTypeTok{void}\NormalTok{ SYSCTRL_ReleaseBlock(SYSCTRL_ResetItem item);}
\end{Highlighting}
\end{Shaded}

\hypertarget{ux65f6ux949fux95e8ux63a7}{%
\subsection{时钟门控}\label{ux65f6ux949fux95e8ux63a7}}

通过 \texttt{SYSCTRL\_SetClkGate} 设置门控(即关闭时钟),通过 \texttt{SYSCTRL\_ClearClkGate}
消除门控(即恢复时钟)。

\begin{Shaded}
\begin{Highlighting}[]
\DataTypeTok{void}\NormalTok{ SYSCTRL_SetClkGate(SYSCTRL_ClkGateItem item);}
\DataTypeTok{void}\NormalTok{ SYSCTRL_ClearClkGate(SYSCTRL_ClkGateItem item);}
\end{Highlighting}
\end{Shaded}

\texttt{SYSCTRL\_SetClkGateMulti} 和 \texttt{SYSCTRL\_ClearClkGateMulti} 可以同时控制多个外设的门控。
\texttt{items} 参数里的各个比特与 \texttt{SYSCTRL\_ClkGateItem} 里的各个外设一一对应。

\begin{Shaded}
\begin{Highlighting}[]
\DataTypeTok{void}\NormalTok{ SYSCTRL_SetClkGateMulti(}\DataTypeTok{uint32_t}\NormalTok{ items);}
\DataTypeTok{void}\NormalTok{ SYSCTRL_ClearClkGateMulti(}\DataTypeTok{uint32_t}\NormalTok{ items);}
\end{Highlighting}
\end{Shaded}

\hypertarget{ux65f6ux949fux914dux7f6e-1}{%
\subsection{时钟配置}\label{ux65f6ux949fux914dux7f6e-1}}

举例如下。

\begin{enumerate}
\def\labelenumi{\arabic{enumi}.}
\item
  \emph{clk\_pll} 与 \emph{hclk}

  使用 \texttt{SYSCTRL\_ConfigPLLClk} 配置 \emph{clk\_pll}:

\begin{Shaded}
\begin{Highlighting}[]
\DataTypeTok{int}\NormalTok{ SYSCTRL_ConfigPLLClk(}
\DataTypeTok{uint32_t}\NormalTok{ div_pre,}
\DataTypeTok{uint32_t}\NormalTok{ loop,}
\DataTypeTok{uint32_t}\NormalTok{ div_output);}
\end{Highlighting}
\end{Shaded}

  例如,将 \emph{hclk} 配置为 220MHz 并读取到变量:

\begin{Shaded}
\begin{Highlighting}[]
\NormalTok{SYSCTRL_ConfigPLLClk(}\DecValTok{6}\NormalTok{, }\DecValTok{110}\NormalTok{, }\DecValTok{1}\NormalTok{);}
\NormalTok{SYSCTRL_SelectHClk(SYSCTRL_CLK_PLL_DIV_1 + }\DecValTok{1}\NormalTok{);}
\DataTypeTok{uint32_t}\NormalTok{ SystemCoreClock = SYSCTRL_GetHClk();}
\end{Highlighting}
\end{Shaded}
\item
  为硬件 I2S 配置时钟

  使用 \texttt{SYSCTRL\_SelectI2sClk} 为 I2S 配置时钟:

\begin{Shaded}
\begin{Highlighting}[]
\DataTypeTok{void}\NormalTok{ SYSCTRL_SelectI2sClk(SYSCTRL_ClkMode mode);}
\end{Highlighting}
\end{Shaded}

  \texttt{SYSCTRL\_ClkMode} 的定为为:

\begin{Shaded}
\begin{Highlighting}[]
\KeywordTok{typedef} \KeywordTok{enum}
\NormalTok{\{}
\NormalTok{    SYSCTRL_CLK_OSC,            }\CommentTok{// 使用 sclk_slow}
\NormalTok{    SYSCTRL_CLK_HCLK,           }\CommentTok{// 使用 hclk}
\NormalTok{    SYSCTRL_CLK_ADC_DIV = ...,  }\CommentTok{// 使用 sclk_slow_adc_div}
\NormalTok{    SYSCTRL_CLK_PLL_DIV_1 = ...,}\CommentTok{// 对 sclk_fast 分频}
\NormalTok{\} SYSCTRL_ClkMode;}
\end{Highlighting}
\end{Shaded}

  根据表 \ref{tab:ch-sysctrl-tab-clk} 可知,I2S 可使用 \_slk\_slow:

\begin{Shaded}
\begin{Highlighting}[]
\NormalTok{SYSCTRL_SelectI2sClk(SYSCTRL_CLK_OSC);}
\end{Highlighting}
\end{Shaded}

  或者独占一个分频器,对 \emph{sclk\_fast} 分频得到 \emph{sclk\_fast\_i2s\_div},比如使用 \emph{sclk\_fast}
  的 5\footnote{\(5=1 + 4\)} 分频:

\begin{Shaded}
\begin{Highlighting}[]
\NormalTok{SYSCTRL_SelectI2sClk(SYSCTRL_CLK_PLL_DIV_1 + }\DecValTok{4}\NormalTok{);}
\end{Highlighting}
\end{Shaded}
\item
  读取时钟频率

  使用 \texttt{SYSCTRL\_GetClk} 可获得指定外设的时钟频率:

\begin{Shaded}
\begin{Highlighting}[]
\DataTypeTok{uint32_t}\NormalTok{ SYSCTRL_GetClk(SYSCTRL_Item item);}
\end{Highlighting}
\end{Shaded}

  比如,

\begin{Shaded}
\begin{Highlighting}[]
\CommentTok{// I2S 使用 PLL 的 5 分频}
\NormalTok{SYSCTRL_SelectI2sClk(SYSCTRL_CLK_PLL_DIV_1 + }\DecValTok{4}\NormalTok{);}
\CommentTok{// freq = sclk_fast 的频率 / 5}
\DataTypeTok{uint32_t}\NormalTok{ freq = SYSCTRL_GetClk(SYSCTRL_ITEM_APB_I2S);}
\end{Highlighting}
\end{Shaded}
\end{enumerate}

\hypertarget{dma-ux89c4ux5212-1}{%
\subsection{DMA 规划}\label{dma-ux89c4ux5212-1}}

使用 \texttt{SYSCTRL\_SelectUsedDmaItems} 配置要使用的 DMA 握手信号:

\begin{Shaded}
\begin{Highlighting}[]
\DataTypeTok{int}\NormalTok{ SYSCTRL_SelectUsedDmaItems(}
  \DataTypeTok{uint32_t}\NormalTok{ items }\CommentTok{// 各比特与 SYSCTRL_DMA 一一对应}
\NormalTok{  );}
\end{Highlighting}
\end{Shaded}

使用 \texttt{SYSCTRL\_GetDmaId} 可获取为某外设握手信号的 DMA 信号 ID,如果返回 -1,
说明没有规划该外设握手信号\footnote{\texttt{SYSCTRL\_SelectUsedDmaItems} 的 \texttt{items} 参数里对应的比特为 0}:

\begin{Shaded}
\begin{Highlighting}[]
\DataTypeTok{int}\NormalTok{ SYSCTRL_GetDmaId(SYSCTRL_DMA item);}
\end{Highlighting}
\end{Shaded}

\hypertarget{ch-timer}{%
\chapter{定时器(TIMER)}\label{ch-timer}}

\hypertarget{ux529fux80fdux6982ux8ff0-5}{%
\section{功能概述}\label{ux529fux80fdux6982ux8ff0-5}}

定时器(TIMER)是由两个寄存器组成的,其中一个寄存器用来确定计数器的工作形式和功能,
另外一个计时器是用来控制单片机的启动和停止的,同时也设置了一个

\hypertarget{ux4f7fux7528ux8bf4ux660e-6}{%
\section{使用说明}\label{ux4f7fux7528ux8bf4ux660e-6}}

\hypertarget{ux8bbeux7f6eux5de5ux4f5cux6a21ux5f0f}{%
\subsection{设置工作模式}\label{ux8bbeux7f6eux5de5ux4f5cux6a21ux5f0f}}

在使用 TIMER 之前,需要使用 \texttt{TMR\_SetOpMode} 按需设置TIMER的工作模式。

\begin{Shaded}
\begin{Highlighting}[]
\DataTypeTok{void}\NormalTok{ TMR_SetOpMode(}
\NormalTok{  TMR_TypeDef *pTMR, }
  \DataTypeTok{uint8_t}\NormalTok{ ch_id, }
  \DataTypeTok{uint8_t}\NormalTok{ op_mode, }
  \DataTypeTok{uint8_t}\NormalTok{ clk_mode, }
  \DataTypeTok{uint8_t}\NormalTok{ pwm_park_value}
\NormalTok{  );}
\end{Highlighting}
\end{Shaded}

TIMER的工作模式有以下6种,分别三类:定时器、PWM、复用。
* 定时器功能包括:32bit单定时器、16bit双定时器、8bit四定时器
```c
\#define TMR\_CTL\_OP\_MODE\_32BIT\_TIMER\_x1 1 // one 32bit timer
\#define TMR\_CTL\_OP\_MODE\_16BIT\_TIMER\_x2 2 // dual 16bit timers
\#define TMR\_CTL\_OP\_MODE\_8BIT\_TIMER\_x4 3 // four 8bit timers

\begin{verbatim}

* PWM功能:16bit双计数器
```c
  #define TMR_CTL_OP_MODE_16BIT_PWM                 4           // PWM with two 16bit counters
\end{verbatim}

\begin{itemize}
\item
  定时器+PWM复用功能:8bit双计数器+16bit单定时器、 8bit双计数器+8bit双定时器

\begin{Shaded}
\begin{Highlighting}[]
  \PreprocessorTok{#define TMR_CTL_OP_MODE_8BIT_PWM_16BIT_TIMER_x1   6           }\CommentTok{// MIXED: PWM with two 8bit counters + one 16bit timer}
  \PreprocessorTok{#define TMR_CTL_OP_MODE_8BIT_PWM_8BIT_TIMER_x2    7           }\CommentTok{// MIXED: PWM with two 8bit counters + dual 8bit timer}
\end{Highlighting}
\end{Shaded}
\item
  例如将TIMER1的通道0设置为\texttt{TMR\_CTL\_OP\_MODE\_32BIT\_TIMER\_x1}模式:

\begin{Shaded}
\begin{Highlighting}[]
\NormalTok{  TMR_SetOpMode(APB_TMR1, }\DecValTok{0}\NormalTok{, TMR_CTL_OP_MODE_32BIT_TIMER_x1, TMR_CLK_MODE_APB, }\DecValTok{0}\NormalTok{);}
\NormalTok{  TMR_SetReload(APB_TMR1, }\DecValTok{0}\NormalTok{, TMR_GetClk(APB_TMR1, }\DecValTok{0}\NormalTok{) / }\DecValTok{10}\NormalTok{);}
\NormalTok{  TMR_Enable(APB_TMR1, }\DecValTok{0}\NormalTok{, }\BaseNTok{0xf}\NormalTok{);}
\NormalTok{  TMR_IntEnable(APB_TMR1, }\DecValTok{0}\NormalTok{, }\BaseNTok{0xf}\NormalTok{);}
\end{Highlighting}
\end{Shaded}
\end{itemize}

\hypertarget{ux65f6ux949fux9891ux7387}{%
\subsection{时钟频率}\label{ux65f6ux949fux9891ux7387}}

使用 \texttt{TMR\_GetClk} 获取TIMER某个通道的时钟频率。

\begin{Shaded}
\begin{Highlighting}[]
\DataTypeTok{uint32_t}\NormalTok{ TMR_GetClk(}
\NormalTok{  TMR_TypeDef *pTMR, }
  \DataTypeTok{uint8_t}\NormalTok{ ch_id}
\NormalTok{  );}
\end{Highlighting}
\end{Shaded}

\hypertarget{ux91cdux8f7dux503c}{%
\subsection{重载值}\label{ux91cdux8f7dux503c}}

使用 \texttt{TMR\_SetReload} 设置TIMER某个通道的重载值。

\begin{Shaded}
\begin{Highlighting}[]
\DataTypeTok{void}\NormalTok{ TMR_SetReload(}
\NormalTok{  TMR_TypeDef *pTMR, }
  \DataTypeTok{uint8_t}\NormalTok{ ch_id, }
  \DataTypeTok{uint32_t}\NormalTok{ value}
\NormalTok{  );}
\end{Highlighting}
\end{Shaded}

在不同的TIMER模式中, \texttt{value} 的值分配如下表。
Table: \label{tab:ch0-abbreviations}

\begin{longtable}[]{@{}ccccc@{}}
\toprule
TIMER模式 & bits{[}0:7{]} & bits{[}8:15{]} & bits{[}16:23{]} & bits{[}24:31{]}\tabularnewline
\midrule
\endhead
TMR\_CTL\_OP\_MODE\_32BIT\_TIMER\_x1 & Timer 0 & Timer 0 & Timer 0 & Timer 0\tabularnewline
TMR\_CTL\_OP\_MODE\_16BIT\_TIMER\_x2 & Timer 0 & Timer 0 & Timer 1 & Timer 1\tabularnewline
TMR\_CTL\_OP\_MODE\_8BIT\_TIMER\_x4 & Timer 0 & Timer 1 & Timer 2 & Timer 3\tabularnewline
TMR\_CTL\_OP\_MODE\_16BIT\_PWM & PWM low period & PWM low period & PWM high period & PWM high period\tabularnewline
TMR\_CTL\_OP\_MODE\_8BIT\_PWM\_16BIT\_TIMER\_x1 & Timer 0 & Timer 0 & PWM low period & PWM low period\tabularnewline
TMR\_CTL\_OP\_MODE\_8BIT\_PWM\_8BIT\_TIMER\_x2 & Timer 0 & Timer 1 & PWM low period & PWM low period\tabularnewline
\bottomrule
\end{longtable}

\hypertarget{ux4f7fux80fd}{%
\subsection{使能}\label{ux4f7fux80fd}}

使用 \texttt{TMR\_Enable} TIMER的某个通道。

\begin{Shaded}
\begin{Highlighting}[]
\DataTypeTok{void}\NormalTok{ TMR_Enable(}
\NormalTok{  TMR_TypeDef *pTMR, }
  \DataTypeTok{uint8_t}\NormalTok{ ch_id, }
  \DataTypeTok{uint8_t}\NormalTok{ mask}
\NormalTok{  );}
\end{Highlighting}
\end{Shaded}

\hypertarget{ux6bd4ux8f83}{%
\subsection{比较}\label{ux6bd4ux8f83}}

使用 \texttt{TMR\_GetCMP} 。

\begin{Shaded}
\begin{Highlighting}[]
\DataTypeTok{uint32_t}\NormalTok{ TMR_GetCMP(}
\NormalTok{  TMR_TypeDef *pTMR, }
  \DataTypeTok{uint8_t}\NormalTok{ ch_id}
\NormalTok{  );}
\end{Highlighting}
\end{Shaded}

\hypertarget{ux914dux7f6eux4e2dux65adux8bf7ux6c42-2}{%
\subsection{配置中断请求}\label{ux914dux7f6eux4e2dux65adux8bf7ux6c42-2}}

使用 \texttt{TMR\_IntEnable} 配置并使能TIMER中断。

\begin{Shaded}
\begin{Highlighting}[]
\DataTypeTok{void}\NormalTok{ TMR_IntEnable(}
\NormalTok{  TMR_TypeDef *pTMR, }
  \DataTypeTok{uint8_t}\NormalTok{ ch_id, }
  \DataTypeTok{uint8_t}\NormalTok{ mask}
\NormalTok{  );}
\end{Highlighting}
\end{Shaded}

\hypertarget{ux4e2dux65adux6e05ux9664}{%
\subsection{中断清除}\label{ux4e2dux65adux6e05ux9664}}

使用 \texttt{TMR\_IntClr} 清除某个TIMER的中断状态。

\begin{Shaded}
\begin{Highlighting}[]
\DataTypeTok{void}\NormalTok{ TMR_IntClr(}
\NormalTok{  TMR_TypeDef *pTMR, }
  \DataTypeTok{uint8_t}\NormalTok{ ch_id, }
  \DataTypeTok{uint8_t}\NormalTok{ mask}
\NormalTok{  );}
\end{Highlighting}
\end{Shaded}

\hypertarget{ux83b7ux53d6ux4e2dux65ad}{%
\subsection{获取中断}\label{ux83b7ux53d6ux4e2dux65ad}}

使用 \texttt{TMR\_IntHappened} 获取某个TIMER的中断状态。

\begin{Shaded}
\begin{Highlighting}[]
\DataTypeTok{uint8_t}\NormalTok{ TMR_IntHappened(}
\NormalTok{  TMR_TypeDef *pTMR, }
  \DataTypeTok{uint8_t}\NormalTok{ ch_id}
\NormalTok{  );}
\end{Highlighting}
\end{Shaded}

\hypertarget{ch-uart}{%
\chapter{通用异步收发传输器(UART)}\label{ch-uart}}

\hypertarget{ux529fux80fdux6982ux8ff0-6}{%
\section{功能概述}\label{ux529fux80fdux6982ux8ff0-6}}

UART负责处理数据总线和串行口之间的串/并、并/串转换,并规定了相应的帧格式,通信双方只要采用相同的帧格式和波特率,
就能在未共享时钟信号的情况下,仅用两根信号线(RX和TX)完成通信过程。

特性:

\begin{itemize}
\tightlist
\item
  异步串行通信,可为全双工、半双工、单发送(TX)或单接收(RX)模式;
\item
  支持5\textasciitilde8位数据位的配置,波特率几百bps至几百Kbbps;
\item
  可配置奇校验、偶校验或无校验位;可配置1、1.5或2位停止位;
\item
  将并行数据写入内存缓冲区,再通过FIFO逐位发送,接收时同理;
\item
  输出传输时,从低位到高位传输。
\end{itemize}

\hypertarget{ux4f7fux7528ux8bf4ux660e-7}{%
\section{使用说明}\label{ux4f7fux7528ux8bf4ux660e-7}}

\hypertarget{ux8bbeux7f6eux6ce2ux7279ux7387}{%
\subsection{设置波特率}\label{ux8bbeux7f6eux6ce2ux7279ux7387}}

使用 \texttt{apUART\_BaudRateSet} 设置对应UART设备的波特率。

\begin{Shaded}
\begin{Highlighting}[]
\DataTypeTok{void}\NormalTok{ apUART_BaudRateSet(}
\NormalTok{  UART_TypeDef* pBase, }
  \DataTypeTok{uint32_t}\NormalTok{ ClockFrequency, }
  \DataTypeTok{uint32_t}\NormalTok{ BaudRate}
\NormalTok{  );}
\end{Highlighting}
\end{Shaded}

\hypertarget{ux83b7ux53d6ux6ce2ux7279ux7387}{%
\subsection{获取波特率}\label{ux83b7ux53d6ux6ce2ux7279ux7387}}

使用 \texttt{apUART\_BaudRateGet} 获取对应UART设备的波特率。

\begin{Shaded}
\begin{Highlighting}[]
\DataTypeTok{uint32_t}\NormalTok{ apUART_BaudRateGet (}
\NormalTok{  UART_TypeDef* pBase, }
  \DataTypeTok{uint32_t}\NormalTok{ ClockFrequency}
\NormalTok{  );}
\end{Highlighting}
\end{Shaded}

\hypertarget{ux63a5ux6536ux9519ux8befux67e5ux8be2}{%
\subsection{接收错误查询}\label{ux63a5ux6536ux9519ux8befux67e5ux8be2}}

使用 \texttt{apUART\_Check\_Rece\_ERROR} 查询接收产生的错误。

\begin{Shaded}
\begin{Highlighting}[]
\DataTypeTok{uint8_t}\NormalTok{ apUART_Check_Rece_ERROR(}
\NormalTok{  UART_TypeDef* pBase}
\NormalTok{  );}
\end{Highlighting}
\end{Shaded}

\hypertarget{fifoux8f6eux8be2ux6a21ux5f0f}{%
\subsection{FIFO轮询模式}\label{fifoux8f6eux8be2ux6a21ux5f0f}}

在轮询模式下,CPU通过检查线路状态寄存器中的位来检测事件:

\begin{itemize}
\item
  使用 \texttt{apUART\_Check\_Rece\_ERROR} 查询接收产生的错误字。

\begin{Shaded}
\begin{Highlighting}[]
  \DataTypeTok{uint8_t}\NormalTok{ apUART_Check_Rece_ERROR(}
\NormalTok{    UART_TypeDef* pBase}
\NormalTok{    );}
\end{Highlighting}
\end{Shaded}
\item
  用 \texttt{apUART\_Check\_RXFIFO\_EMPTY}查询RXFIFO是否为空。

\begin{Shaded}
\begin{Highlighting}[]
  \DataTypeTok{uint8_t}\NormalTok{ apUART_Check_RXFIFO_EMPTY(}
\NormalTok{    UART_TypeDef* pBase}
\NormalTok{    );}
\end{Highlighting}
\end{Shaded}
\item
  使用 \texttt{apUART\_Check\_RXFIFO\_FULL}查询RXFIFO是否已满。

\begin{Shaded}
\begin{Highlighting}[]
  \DataTypeTok{uint8_t}\NormalTok{ apUART_Check_RXFIFO_FULL(}
\NormalTok{    UART_TypeDef* pBase}
\NormalTok{    );}
\end{Highlighting}
\end{Shaded}
\item
  使用 \texttt{apUART\_Check\_TXFIFO\_EMPTY} 查询TXFIFO是否为空。

\begin{Shaded}
\begin{Highlighting}[]
  \DataTypeTok{uint8_t}\NormalTok{ apUART_Check_TXFIFO_EMPTY(}
\NormalTok{    UART_TypeDef* pBase}
\NormalTok{    );}
\end{Highlighting}
\end{Shaded}
\item
  使用 \texttt{apUART\_Check\_TXFIFO\_FULL} 查询TXFIFO是否已满。

\begin{Shaded}
\begin{Highlighting}[]
  \DataTypeTok{uint8_t}\NormalTok{ apUART_Check_TXFIFO_FULL(}
\NormalTok{    UART_TypeDef* pBase}
\NormalTok{    ); }
\end{Highlighting}
\end{Shaded}
\end{itemize}

\hypertarget{ux53d1ux9001ux6570ux636e}{%
\subsection{发送数据}\label{ux53d1ux9001ux6570ux636e}}

使用 \texttt{UART\_SendData} 发送8bits数据。

\begin{Shaded}
\begin{Highlighting}[]
\DataTypeTok{void}\NormalTok{ UART_SendData(}
\NormalTok{  UART_TypeDef* pBase, }
  \DataTypeTok{uint8_t}\NormalTok{ Data}
\NormalTok{  );}
\end{Highlighting}
\end{Shaded}

\hypertarget{ux63a5ux6536ux6570ux636e}{%
\subsection{接收数据}\label{ux63a5ux6536ux6570ux636e}}

使用 \texttt{UART\_ReceData} 接收8bits数据。

\begin{Shaded}
\begin{Highlighting}[]
\DataTypeTok{uint8_t}\NormalTok{ UART_ReceData(}
\NormalTok{  UART_TypeDef* pBase}
\NormalTok{  );}
\end{Highlighting}
\end{Shaded}

\hypertarget{ux914dux7f6eux4e2dux65adux8bf7ux6c42-3}{%
\subsection{配置中断请求}\label{ux914dux7f6eux4e2dux65adux8bf7ux6c42-3}}

使用 \texttt{apUART\_Enable\_TRANSMIT\_INT} 使能发送中断状态。

\begin{Shaded}
\begin{Highlighting}[]
\DataTypeTok{void}\NormalTok{ apUART_Enable_TRANSMIT_INT(}
\NormalTok{  UART_TypeDef* pBase}
\NormalTok{  );}
\end{Highlighting}
\end{Shaded}

使用 \texttt{apUART\_Disable\_TRANSMIT\_INT} 禁用发送中断状态。

\begin{Shaded}
\begin{Highlighting}[]
\DataTypeTok{void}\NormalTok{ apUART_Disable_TRANSMIT_INT(}
\NormalTok{  UART_TypeDef* pBase}
\NormalTok{  );}
\end{Highlighting}
\end{Shaded}

使用 \texttt{apUART\_Enable\_RECEIVE\_INT} 使能接收中断状态。

\begin{Shaded}
\begin{Highlighting}[]
\DataTypeTok{void}\NormalTok{ apUART_Enable_RECEIVE_INT(}
\NormalTok{  UART_TypeDef* pBase}
\NormalTok{  );}
\end{Highlighting}
\end{Shaded}

使用 \texttt{apUART\_Disable\_RECEIVE\_INT} 禁用接收中断状态。

\begin{Shaded}
\begin{Highlighting}[]
\DataTypeTok{void}\NormalTok{ apUART_Disable_RECEIVE_INT(}
\NormalTok{  UART_TypeDef* pBase}
\NormalTok{  );}
\end{Highlighting}
\end{Shaded}

\hypertarget{ux5904ux7406ux4e2dux65adux72b6ux6001-2}{%
\subsection{处理中断状态}\label{ux5904ux7406ux4e2dux65adux72b6ux6001-2}}

uint8\_t apUART\_Get\_ITStatus(UART\_TypeDef* pBase,uint8\_t UART\_IT);
uint32\_t apUART\_Get\_all\_raw\_int\_stat(UART\_TypeDef* pBase);

void apUART\_Clr\_RECEIVE\_INT(UART\_TypeDef* pBase);
void apUART\_Clr\_TX\_INT(UART\_TypeDef* pBase);
void apUART\_Clr\_NonRx\_INT(UART\_TypeDef* pBase);

\hypertarget{uartux521dux59cbux5316}{%
\subsection{UART初始化}\label{uartux521dux59cbux5316}}

两个设备使用UART通讯时,必须先约定好传输速率和一些数据位。

\begin{Shaded}
\begin{Highlighting}[]
\KeywordTok{typedef} \KeywordTok{struct}\NormalTok{ UART_xStateStruct}
\NormalTok{\{}
    \CommentTok{// Line Control Register, UARTLCR_H}
\NormalTok{  UART_eWLEN      word_length;   }\CommentTok{// WLEN}
\NormalTok{  UART_ePARITY    parity;        }\CommentTok{// PEN, EPS, SPS}
  \DataTypeTok{uint8_t}\NormalTok{         fifo_enable;   }\CommentTok{// FEN}
  \DataTypeTok{uint8_t}\NormalTok{         two_stop_bits; }\CommentTok{// STP2}
  \CommentTok{// Control Register, UARTCR}
  \DataTypeTok{uint8_t}\NormalTok{         receive_en;        }\CommentTok{// RXE}
  \DataTypeTok{uint8_t}\NormalTok{         transmit_en;       }\CommentTok{// TXE}
  \DataTypeTok{uint8_t}\NormalTok{         UART_en;           }\CommentTok{// UARTEN}
  \DataTypeTok{uint8_t}\NormalTok{         cts_en;            }\CommentTok{//CTSEN}
  \DataTypeTok{uint8_t}\NormalTok{         rts_en;            }\CommentTok{//RTSEN}
  \CommentTok{// Interrupt FIFO Level Select Register, UARTIFLS}
  \DataTypeTok{uint8_t}\NormalTok{         rxfifo_waterlevel; }\CommentTok{// RXIFLSEL}
  \DataTypeTok{uint8_t}\NormalTok{         txfifo_waterlevel; }\CommentTok{// TXIFLSEL}
  \CommentTok{//UART_eFIFO_WATERLEVEL    rxfifo_waterlevel; // RXIFLSEL}
  \CommentTok{//UART_eFIFO_WATERLEVEL    txfifo_watchlevel; // TXIFLSEL}

  \CommentTok{// UART Clock Frequency}
  \DataTypeTok{uint32_t}\NormalTok{        ClockFrequency;}
  \DataTypeTok{uint32_t}\NormalTok{        BaudRate;}

\NormalTok{\} UART_sStateStruct;}
\end{Highlighting}
\end{Shaded}

定义函数 \texttt{config\_uart} ,

\begin{Shaded}
\begin{Highlighting}[]
\DataTypeTok{void}\NormalTok{ config_uart(}
  \DataTypeTok{uint32_t}\NormalTok{ freq, }
  \DataTypeTok{uint32_t}\NormalTok{ baud}
\NormalTok{  );}
\end{Highlighting}
\end{Shaded}

在函数中,对UART\_sStateStruct的各项参数初始化,并调用 \texttt{apUART\_Initialize} 对UART进行初始化。

\begin{Shaded}
\begin{Highlighting}[]
\DataTypeTok{void}\NormalTok{ config_uart(}\DataTypeTok{uint32_t}\NormalTok{ freq, }\DataTypeTok{uint32_t}\NormalTok{ baud)}
\NormalTok{\{}
\NormalTok{    UART_sStateStruct config;}

\NormalTok{    config.word_length       = UART_WLEN_8_BITS;}
\NormalTok{    config.parity            = UART_PARITY_NOT_CHECK;}
\NormalTok{    config.fifo_enable       = }\DecValTok{1}\NormalTok{;}
\NormalTok{    config.two_stop_bits     = }\DecValTok{0}\NormalTok{;}
\NormalTok{    config.receive_en        = }\DecValTok{1}\NormalTok{;}
\NormalTok{    config.transmit_en       = }\DecValTok{1}\NormalTok{;}
\NormalTok{    config.UART_en           = }\DecValTok{1}\NormalTok{;}
\NormalTok{    config.cts_en            = }\DecValTok{0}\NormalTok{;}
\NormalTok{    config.rts_en            = }\DecValTok{0}\NormalTok{;}
\NormalTok{    config.rxfifo_waterlevel = }\DecValTok{1}\NormalTok{;}
\NormalTok{    config.txfifo_waterlevel = }\DecValTok{1}\NormalTok{;}
\NormalTok{    config.ClockFrequency    = freq;}
\NormalTok{    config.BaudRate          = baud;}

\NormalTok{    apUART_Initialize(PRINT_PORT, &config, }\DecValTok{0}\NormalTok{);}
\NormalTok{\}}
\end{Highlighting}
\end{Shaded}

\hypertarget{ux53d1ux9001ux6570ux636e-1}{%
\subsection{发送数据}\label{ux53d1ux9001ux6570ux636e-1}}

使用 \texttt{UART\_SendData} 发送数据。

\begin{Shaded}
\begin{Highlighting}[]
\DataTypeTok{void}\NormalTok{ UART_SendData(}
\NormalTok{  UART_TypeDef* pBase, }
  \DataTypeTok{uint8_t}\NormalTok{ Data}
\NormalTok{  );}
\end{Highlighting}
\end{Shaded}

\hypertarget{ux63a5ux6536ux6570ux636e-1}{%
\subsection{接收数据}\label{ux63a5ux6536ux6570ux636e-1}}

使用 \texttt{UART\_ReceData} 接收数据。

\begin{Shaded}
\begin{Highlighting}[]
\DataTypeTok{uint8_t}\NormalTok{ UART_ReceData(}
\NormalTok{  UART_TypeDef* pBase}
\NormalTok{  );}
\end{Highlighting}
\end{Shaded}

\hypertarget{ux6e05ux7a7afifo}{%
\subsection{清空FIFO}\label{ux6e05ux7a7afifo}}

使用 \texttt{uart\_empty\_fifo} 清空UART的FIFO。

\begin{Shaded}
\begin{Highlighting}[]
\DataTypeTok{static} \DataTypeTok{void}\NormalTok{ uart_empty_fifo(}
\NormalTok{  UART_TypeDef* pBase}
\NormalTok{  );}
\end{Highlighting}
\end{Shaded}

\hypertarget{ux4f7fux80fdfifo}{%
\subsection{使能FIFO}\label{ux4f7fux80fdfifo}}

使用 \texttt{uart\_enable\_fifo} 使能UART的FIFO。

\begin{Shaded}
\begin{Highlighting}[]
\DataTypeTok{static} \DataTypeTok{void}\NormalTok{ uart_enable_fifo(}
\NormalTok{  UART_TypeDef* pBase}
\NormalTok{  );}
\end{Highlighting}
\end{Shaded}

\hypertarget{ux5904ux7406ux4e2dux65adux72b6ux6001-3}{%
\subsection{处理中断状态}\label{ux5904ux7406ux4e2dux65adux72b6ux6001-3}}

用 \texttt{apUART\_Get\_ITStatus} 获取某个UART上的中断触发状态,返回非 0 值表示该 UART
上产生了中断请求;用 \texttt{apUART\_Get\_all\_raw\_int\_stat} 一次性获取所有 UART 的中断触发状态,
第 \(n\) 比特(第 0 比特为最低比特)对应 UART \(n\) 上的中断触发状态。

UART产生中断后,需要消除中断状态方可再次触发。用 \texttt{apUART\_Clr\_RECEIVE\_INT} 消除某个 UART上接收中断的状态,
用 \texttt{apUART\_Clr\_TX\_INT} 消除某个 UART上发送中断的状态。用 \texttt{apUART\_Clr\_NonRx\_INT} 消除某个 UART上除接收以外的中断状态。

\hypertarget{dmaux4f20ux8f93ux6a21ux5f0fux4f7fux80fd}{%
\subsection{DMA传输模式使能}\label{dmaux4f20ux8f93ux6a21ux5f0fux4f7fux80fd}}

DMA向CPU发出总线请求,CPU将总线交给DMA之后,由DMA控制数据的收发工作。

使用 \texttt{UART\_DmaEnable}使能DMA工作模式。

\begin{Shaded}
\begin{Highlighting}[]
\DataTypeTok{void}\NormalTok{ UART_DmaEnable(}
\NormalTok{  UART_TypeDef *pBase, }
  \DataTypeTok{uint8_t}\NormalTok{ tx_enable, }
  \DataTypeTok{uint8_t}\NormalTok{ rx_enable, }
  \DataTypeTok{uint8_t}\NormalTok{ dma_on_err}
\NormalTok{  );}
\end{Highlighting}
\end{Shaded}

\hypertarget{universal-serial-bus-deviceusb}{%
\chapter{Universal serial bus device(USB)}\label{universal-serial-bus-deviceusb}}

\hypertarget{ux529fux80fdux6982ux8ff0-7}{%
\section{功能概述}\label{ux529fux80fdux6982ux8ff0-7}}

\begin{itemize}
\tightlist
\item
  支持full-speed (12 Mbps) device模式
\item
  集成PHY Transceiver,内置上拉,软件可控
\item
  Endpoints:

  \begin{itemize}
  \tightlist
  \item
    Endpoints 0: control endpoint
  \item
    Endpoints 1-5: 可以配置为in/out,以及control/isochronous/bulk/interrupt
  \end{itemize}
\item
  支持UBS suspend, resume, remote-wakeup
\item
  内置DMA方便数据传输
\end{itemize}

\hypertarget{ux4f7fux7528ux8bf4ux660e-8}{%
\section{使用说明}\label{ux4f7fux7528ux8bf4ux660e-8}}

\hypertarget{usbux8f6fux4ef6ux7ed3ux6784}{%
\subsection{USB软件结构}\label{usbux8f6fux4ef6ux7ed3ux6784}}

\begin{itemize}
\tightlist
\item
  driver layer,USB的底层处理,不建议用户修改。

  \begin{itemize}
  \tightlist
  \item
    处理了大部分和应用场景无关的流程,提供了USB\_IrqHandler,调用event handler。
  \item
    位置:\texttt{\textbackslash{}ING\textbackslash{}\_SDK\textbackslash{}sdk\textbackslash{}src\textbackslash{}FWlib\textbackslash{}peripheral\textbackslash{}\_usb.c}
  \end{itemize}
\item
  bsp layer,处理场景相关的流程,需要用户提供event handler,并实现control和transfer相关处理。

  \begin{itemize}
  \tightlist
  \item
    位置:\texttt{\textbackslash{}ING\textbackslash{}\_SDK\textbackslash{}sdk\textbackslash{}src\textbackslash{}BSP\textbackslash{}bsp\textbackslash{}\_usb\_xxx.c}
  \end{itemize}
\end{itemize}

\hypertarget{usb-device-ux72b6ux6001}{%
\subsection{USB Device 状态}\label{usb-device-ux72b6ux6001}}

\begin{itemize}
\tightlist
\item
  USB的使用首先需要配置USB CLK,USB IO以及PHY,并且初始化USB模块,此时USB为``NONE''状态,等待USB的reset中断(USB\_IrqHandler)。
\item
  reset中断的触发代表USB cable已经连接,而且host已经检测到了device,在reset中断中,USB模块完成相关的USB初始化。并继续等待中断。
\item
  enumeration中断的触发代表device可以开始接收SOF以及control传输,device需要配置并打开endpoint 0,进入``DEFAULT''状态。
\item
  out中断的触发代表收到了host的get descriptor,用户需要准备好相应的descriptor,并配置相关的in endpoint。
\item
  out中断中的set address request会将device的状态切换为``ADDRESS''。
\item
  out中断中的set configuration会将device的状态切换为``CONFIGURED''。此时device可以开始在配置的endpoint上传输数据。
\item
  bus上的idle会自动触发suspend中断(用户需要在初始化中使能suspend中断),此时切换为``SUSPEND''状态。
\item
  idle之后任何bus上的活动将会触发resume中断(用户需要在初始化中使能resume中断),用户也可以选择使用remote wakeup主动唤醒。
\item
  唤醒之后的usb将重新进入``CONFIGURED''状态,每1m(full-speed)将会收到1个SOF中断(用户需要在初始化中使能sof中断)。
\end{itemize}

\begin{Shaded}
\begin{Highlighting}[]
\KeywordTok{typedef} \KeywordTok{enum}
\NormalTok{\{}
\NormalTok{  USB_DEVICE_NONE,}
  \CommentTok{/* A USB device may be attached or detached from the USB */}
\NormalTok{  USB_DEVICE_ATTACHED,}
  \CommentTok{/*USB devices may obtain power from an external source  */}
\NormalTok{  USB_DEVICE_POWERED,}
  \CommentTok{/*  After the device has been powered, and reset is done */}
\NormalTok{  USB_DEVICE_DEFAULT,}
  \CommentTok{/*  All USB devices use the default address when initially powered or after the device has been reset. Each}
\CommentTok{      USB device is assigned a unique address by the host after attachment or after reset. */}
\NormalTok{  USB_DEVICE_ADDRESS,}
  \CommentTok{/* Before a USB device function may be used, the device must be configured. */}
\NormalTok{  USB_DEVICE_CONFIGURED,}
  \CommentTok{/* In order to conserve power, USB devices automatically enter the Suspended state when the device has}
\CommentTok{     observed no bus traffic for a specified period */}
\NormalTok{  USB_DEVICE_SUSPENDED,}
\NormalTok{  USB_DEVICE_TEST_RESET_DONE}
\NormalTok{\}USB_DEVICE_STATE_E;}
\end{Highlighting}
\end{Shaded}

\hypertarget{ux8bbeux7f6e-io}{%
\subsection{设置 IO}\label{ux8bbeux7f6e-io}}

对于Ing91682,USB的DP/DM固定在GPIO16/17,IO初始化细节请参考\texttt{ING\_SDK\textbackslash{}sdk\textbackslash{}src\textbackslash{}BSP\textbackslash{}bsp\textbackslash{}\_usb.c}中的\texttt{bsp\_usb\_init()}

\begin{Shaded}
\begin{Highlighting}[]
\CommentTok{// }\AlertTok{ATTENTION}\CommentTok{ ! FIXED IO FOR USB on 916 series}
\PreprocessorTok{#define USB_PIN_DP GIO_GPIO_16}
\PreprocessorTok{#define USB_PIN_DM GIO_GPIO_17}
\end{Highlighting}
\end{Shaded}

\hypertarget{ux8bbeux7f6e-phy}{%
\subsection{设置 PHY}\label{ux8bbeux7f6e-phy}}

使用\texttt{SYSCTRL\_USBPhyConfig()}初始化PHY, 细节请参考\texttt{ING\_SDK\textbackslash{}sdk\textbackslash{}src\textbackslash{}BSP\textbackslash{}bsp\textbackslash{}\_usb.c}中的\texttt{bsp\_usb\_init()}

\begin{Shaded}
\begin{Highlighting}[]
\CommentTok{/**}
\CommentTok{ * }\AnnotationTok{@brief}\CommentTok{ Config USB PHY functionality}
\CommentTok{ *}
\CommentTok{ * }\AnnotationTok{@param[in]}\CommentTok{ }\CommentVarTok{enable}\CommentTok{            Enable(1)/Disable(0) usb phy module}
\CommentTok{ * }\AnnotationTok{@param[in]}\CommentTok{ }\CommentVarTok{pull_sel}\CommentTok{          DP pull up(0x1)/DM pull up(0x2)/DP&DM pull down(0x3)}
\CommentTok{ */}
\DataTypeTok{void}\NormalTok{ SYSCTRL_USBPhyConfig(}\DataTypeTok{uint8_t}\NormalTok{ enable, }\DataTypeTok{uint8_t}\NormalTok{ pull_sel);}
\end{Highlighting}
\end{Shaded}

\hypertarget{usb-ux6a21ux5757ux521dux59cbux5316}{%
\subsection{USB 模块初始化}\label{usb-ux6a21ux5757ux521dux59cbux5316}}

细节请参考\texttt{ING\_SDK\textbackslash{}sdk\textbackslash{}src\textbackslash{}BSP\textbackslash{}bsp\textbackslash{}\_usb.c}中的\texttt{bsp\_usb\_init()}。

\begin{itemize}
\tightlist
\item
  USB模块首选需要打开USB中断并配置相应接口,其中USB\_IrqHandler由driver提供不需要用户修改。
\end{itemize}

\begin{Shaded}
\begin{Highlighting}[]
\NormalTok{  platform_set_irq_callback(PLATFORM_CB_IRQ_USB, USB_IrqHandler, NULL);}
\end{Highlighting}
\end{Shaded}

\begin{itemize}
\tightlist
\item
  其次需要初始化USB模块以及相关状态信息,入参结构体中用户需要提供event handler,其余为可选项
\end{itemize}

\begin{Shaded}
\begin{Highlighting}[]
\CommentTok{/**}
\CommentTok{ * }\AnnotationTok{@brief}\CommentTok{ interface API. initilize usb module and related variables, must be called before any usb usage }
\CommentTok{ *}
\CommentTok{ * }\AnnotationTok{@param[in]}\CommentTok{ }\CommentVarTok{device}\CommentTok{ callback function with structure USB_INIT_CONFIG_T. }
\CommentTok{ *            When this function has been called your device is ready to be enumerated by the USB host.}
\CommentTok{ * }\AnnotationTok{@param[out]}\CommentTok{ }\CommentVarTok{null}\CommentTok{. }
\CommentTok{ */}
\KeywordTok{extern}\NormalTok{ USB_ERROR_TYPE_E USB_InitConfig(USB_INIT_CONFIG_T *config);}
\end{Highlighting}
\end{Shaded}

\hypertarget{event-handler}{%
\subsection{event handler}\label{event-handler}}

USB的用户层调用通过event handler来实现, 细节请参考\texttt{ING\_SDK\textbackslash{}sdk\textbackslash{}src\textbackslash{}BSP\textbackslash{}bsp\textbackslash{}\_usb.c}中的\texttt{bsp\_usb\_event\_handler()}。

event handler需要包含对以下event事件的处理:

\begin{itemize}
\item
  USB\_EVENT\_EP0\_SETUP

  \begin{itemize}
  \tightlist
  \item
    该event包含EPO(control endpoint)上的所有request,包括读取/设置 descriptor,设置address,set/clear feature等request,按照USB协议,device需要支持所有协议中的标准request。
  \item
    descriptor需要按照协议格式准备,并且放置在4bytes对齐的全局地址,并通过USB\_SendData()发送给host,在整个过程中,该全局地址和数据需要保持。(4bytes对齐是内部DMA搬运的要求,否则可能出现错误)。
  \item
    对于没有data stage的request,event handler中不需要使用USB\_SendData()/USB\_RecvData()'。
  \item
    对于不支持的request,需要设置status为:
  \end{itemize}

\begin{Shaded}
\begin{Highlighting}[]
\NormalTok{status = USB_ERROR_REQUEST_NOT_SUPPORT;}
\end{Highlighting}
\end{Shaded}

  \begin{itemize}
  \tightlist
  \item
    根据返回的status,driver判断当前request是否支持,否则按照协议发送stall给host。
  \item
    对于包含data stage的out传输,driver将继续接收数据,数据将在USB\_EVENT\_EP\_DATA\_TRANSFER的ep0通知用户。
  \item
    setup/data/status stage的切换将在driver内进行。
  \end{itemize}
\item
  USB\_EVENT\_EP\_DATA\_TRANSFER

  数据相关的处理,接收和发射数据。

  \begin{itemize}
  \tightlist
  \item
    参数包含ep number
  \end{itemize}

\begin{Shaded}
\begin{Highlighting}[]
\DataTypeTok{uint8_t}\NormalTok{  ep;}
\end{Highlighting}
\end{Shaded}

  以及数据处理类型,分别代表发送和接收transfer的结束。

\begin{Shaded}
\begin{Highlighting}[]
\KeywordTok{typedef} \KeywordTok{enum}
\NormalTok{\{}
    \CommentTok{/// Event send when a receive transfert is finish}
\NormalTok{    USB_CALLBACK_TYPE_RECEIVE_END,}
    \CommentTok{/// Event send when a transmit transfert is finish}
\NormalTok{    USB_CALLBACK_TYPE_TRANSMIT_END}
\NormalTok{\} USB_CALLBACK_EP_TYPE_T;}
\end{Highlighting}
\end{Shaded}
\item
  USB\_EVENT\_DEVICE\_RESET

  USB reset 中断的event,代表枚举的开始。
\item
  USB\_EVENT\_DEVICE\_SOF

  SOF中断,每1m(full-speed)将会收到1个SOF中断(用户需要在初始化中使能sof中断)。
\item
  USB\_EVENT\_DEVICE\_SUSPEND

  bus 进入idle状态后触发suspend,此时总线上没有USB活动。driver会关闭phy clock。
\item
  USB\_EVENT\_DEVICE\_RESUME

  bus 上的任何USB活动将会触发wakeup中断。resume后driver打开phy clock,USB恢复到正常状态。
\end{itemize}

\hypertarget{usb_event_ep0_setupux7684ux5b9eux73b0}{%
\subsubsection{USB\_EVENT\_EP0\_SETUP的实现}\label{usb_event_ep0_setupux7684ux5b9eux73b0}}

control以及枚举相关的流程实现需要通过USB\_EVENT\_EP0\_SETUP event来进行。

\begin{itemize}
\item
  默认的control endpoint是ep0, 所有request都会触发该event
\item
  如果场景不支持某个request,则需要设置status == USB\_ERROR\_REQUEST\_NOT\_SUPPORT。driver会据此发送stall。
\item
  如果场景需要处理某个request,则需要将status设置为非USB\_ERROR\_REQUEST\_NOT\_SUPPORT状态。
\item
  用户需要将所有descriptor保存在4bytes对齐的全局地址。以device descriptor为例

\begin{Shaded}
\begin{Highlighting}[]
\ControlFlowTok{case}\NormalTok{ USB_REQUEST_DEVICE_DESCRIPTOR_DEVICE:}
\NormalTok{\{}
\NormalTok{  size = }\KeywordTok{sizeof}\NormalTok{(USB_DEVICE_DESCRIPTOR_REAL_T);}
\NormalTok{  size = (setup->wLength < size) ? (setup->wLength) : size;}

\NormalTok{  status |= USB_SendData(}\DecValTok{0}\NormalTok{, (}\DataTypeTok{void}\NormalTok{*)&DeviceDescriptor, size, }\DecValTok{0}\NormalTok{);}
\NormalTok{\}}
\ControlFlowTok{break}\NormalTok{;}
\end{Highlighting}
\end{Shaded}

  首先判断size,确保数据没有超出request要求,然后使用USB\_SendData发送in transfer数据。其中DeviceDescriptor为device descriptor地址。
\item
  set address request需要配置device地址,因此在driver layer实现。
\end{itemize}

\hypertarget{suspend-ux7684ux5904ux7406}{%
\subsubsection{SUSPEND 的处理}\label{suspend-ux7684ux5904ux7406}}

SUSPEND状态下可以根据需求进行power saving,默认配置只关闭了phy clock,其余的USB power/clock处理需要根据场景在应用层中的low power mode中来实现。

\hypertarget{remote-wakeup}{%
\subsubsection{remote wakeup}\label{remote-wakeup}}

进入suspend的device可以选择主动唤醒,唤醒通过bsp\_usb\_device\_remote\_wakeup()连续发送10ms的resume signal来实现。

\begin{Shaded}
\begin{Highlighting}[]
\DataTypeTok{void}\NormalTok{ bsp_usb_device_remote_wakeup(}\DataTypeTok{void}\NormalTok{)}
\NormalTok{\{}
\NormalTok{  USB_DeviceSetRemoteWakeupBit(U_TRUE);}
\NormalTok{  platform_set_timer(internal_bsp_usb_device_remote_wakeup_stop,}\DecValTok{16}\NormalTok{);}\CommentTok{// setup timer for 10ms, then disable resume signal}
\NormalTok{\}}
\end{Highlighting}
\end{Shaded}

\hypertarget{ux5e38ux7528driver-api}{%
\subsection{常用driver API}\label{ux5e38ux7528driver-api}}

\hypertarget{send-usb-data}{%
\subsubsection{send usb data}\label{send-usb-data}}

使用该API发送USB数据(包括setup data和应用数据),但需要在set config(打开endpoint)之后使用。

\begin{Shaded}
\begin{Highlighting}[]
\CommentTok{/**}
\CommentTok{ * }\AnnotationTok{@brief}\CommentTok{ interface API. send usb device data packet.}
\CommentTok{ *}
\CommentTok{ * }\AnnotationTok{@param[in]}\CommentTok{ }\CommentVarTok{ep}\CommentTok{, endpoint number, the highest bit represent direction, use USB_EP_DIRECTION_IN/OUT}
\CommentTok{ * }\AnnotationTok{@param[in]}\CommentTok{ }\CommentVarTok{buffer}\CommentTok{, global buffer to hold data of the packet, it must be a DWORD-aligned address. }
\CommentTok{ * }\AnnotationTok{@param[in]}\CommentTok{ }\CommentVarTok{size}\CommentTok{. it should be a value smaller than 512*mps(eg, for EP0, mps is 64byte, maximum packet number is 512, MPS is 64).}
\CommentTok{ * }\AnnotationTok{@param[in]}\CommentTok{ }\CommentVarTok{flag}\CommentTok{. null}
\CommentTok{ * }\AnnotationTok{@param[out]}\CommentTok{ }\CommentVarTok{return}\CommentTok{ U_TRUE if successful, otherwise U_FALSE. }
\CommentTok{ */}
\KeywordTok{extern}\NormalTok{ USB_ERROR_TYPE_E USB_SendData(}\DataTypeTok{uint8_t}\NormalTok{ ep, }\DataTypeTok{void}\NormalTok{* buffer, }\DataTypeTok{uint16_t}\NormalTok{ size, }\DataTypeTok{uint32_t}\NormalTok{ flag);}
\end{Highlighting}
\end{Shaded}

\hypertarget{receive-usb-data}{%
\subsubsection{receive usb data}\label{receive-usb-data}}

使用该API接收USB数据(包括setup data和应用数据),但需要在set config(打开endpoint)之后使用。

\begin{Shaded}
\begin{Highlighting}[]
\CommentTok{/**}
\CommentTok{ * }\AnnotationTok{@brief}\CommentTok{ interface API. receive usb device data packet.}
\CommentTok{ *}
\CommentTok{ * }\AnnotationTok{@param[in]}\CommentTok{ }\CommentVarTok{ep}\CommentTok{, endpoint number, the highest bit represent direction, use USB_EP_DIRECTION_IN/OUT}
\CommentTok{ * }\AnnotationTok{@param[in]}\CommentTok{ }\CommentVarTok{buffer}\CommentTok{, global buffer to hold data of the packet, it must be a DWORD-aligned address. }
\CommentTok{ * }\AnnotationTok{@param[in]}\CommentTok{ }\CommentVarTok{size}\CommentTok{. For OUT transfers, the Transfer Size field in the endpoint Transfer Size register must be a multiple}
\CommentTok{ *            of the maximum packet size of the endpoint(eg, EP0 is 64byte), adjusted to the DWORD boundary}
\CommentTok{ * }\AnnotationTok{@param[in]}\CommentTok{ }\CommentVarTok{flag}\CommentTok{. null}
\CommentTok{ * }\AnnotationTok{@param[out]}\CommentTok{ }\CommentVarTok{return}\CommentTok{ U_TRUE if successful, otherwise U_FALSE. }
\CommentTok{ */}
\KeywordTok{extern}\NormalTok{ USB_ERROR_TYPE_E USB_RecvData(}\DataTypeTok{uint8_t}\NormalTok{ ep, }\DataTypeTok{void}\NormalTok{* buffer, }\DataTypeTok{uint16_t}\NormalTok{ size, }\DataTypeTok{uint32_t}\NormalTok{ flag);}
\end{Highlighting}
\end{Shaded}

\hypertarget{enabledisable-ep}{%
\subsubsection{enable/disable ep}\label{enabledisable-ep}}

正常处理中不需要使用该API,特殊情况下可以根据需求打开关闭某个特定的endpoint

\begin{Shaded}
\begin{Highlighting}[]
\CommentTok{/**}
\CommentTok{ * }\AnnotationTok{@brief}\CommentTok{ interface APIs. use this pair for enable/disable certain ep.}
\CommentTok{ *}
\CommentTok{ * }\AnnotationTok{@param[in]}\CommentTok{ }\CommentVarTok{ep}\CommentTok{ number with USB_EP_DIRECTION_IN/OUT. }
\CommentTok{ * }\AnnotationTok{@param[out]}\CommentTok{ }\CommentVarTok{null}\CommentTok{ }
\CommentTok{ */}
\KeywordTok{extern} \DataTypeTok{void}\NormalTok{ USB_EnableEp(}\DataTypeTok{uint8_t}\NormalTok{ ep, USB_EP_TYPE_T type);}
\KeywordTok{extern} \DataTypeTok{void}\NormalTok{ USB_DisableEp(}\DataTypeTok{uint8_t}\NormalTok{ ep);}
\end{Highlighting}
\end{Shaded}

\hypertarget{usb-close}{%
\subsubsection{usb close}\label{usb-close}}

USB的disable请使用bsp layer中的bsp\_usb\_disable()

\begin{Shaded}
\begin{Highlighting}[]
\CommentTok{/**}
\CommentTok{ * }\AnnotationTok{@brief}\CommentTok{ interface API. shutdown usb module and reset all status data.}
\CommentTok{ *}
\CommentTok{ * }\AnnotationTok{@param[in]}\CommentTok{ }\CommentVarTok{null}\CommentTok{. }
\CommentTok{ * }\AnnotationTok{@param[out]}\CommentTok{ }\CommentVarTok{null}\CommentTok{. }
\CommentTok{ */}
\KeywordTok{extern} \DataTypeTok{void}\NormalTok{ USB_Close(}\DataTypeTok{void}\NormalTok{);}
\end{Highlighting}
\end{Shaded}

\hypertarget{usb-stall}{%
\subsubsection{usb stall}\label{usb-stall}}

\begin{Shaded}
\begin{Highlighting}[]
\CommentTok{/**}
\CommentTok{ * }\AnnotationTok{@brief}\CommentTok{ interface API. set ep stall pid for current transfer}
\CommentTok{ *}
\CommentTok{ * }\AnnotationTok{@param[in]}\CommentTok{ }\CommentVarTok{ep}\CommentTok{ num with direction. }
\CommentTok{ * }\AnnotationTok{@param[in]}\CommentTok{ }\CommentVarTok{U_TRUE:}\CommentTok{ stall, U_FALSE: set back to normal }
\CommentTok{ * }\AnnotationTok{@param[out]}\CommentTok{ }\CommentVarTok{null}\CommentTok{. }
\CommentTok{ */}
\KeywordTok{extern} \DataTypeTok{void}\NormalTok{ USB_SetStallEp(}\DataTypeTok{uint8_t}\NormalTok{ ep, }\DataTypeTok{uint8_t}\NormalTok{ stall);}
\end{Highlighting}
\end{Shaded}

\hypertarget{usb-in-endpoint-nak}{%
\subsubsection{usb In endpoint NAK}\label{usb-in-endpoint-nak}}

\begin{Shaded}
\begin{Highlighting}[]
\CommentTok{/**}
\CommentTok{ * }\AnnotationTok{@brief}\CommentTok{ interface API. use this api to set NAK on a specific IN ep}
\CommentTok{ *}
\CommentTok{ * }\AnnotationTok{@param[in]}\CommentTok{ }\CommentVarTok{U_TRUE:}\CommentTok{ enable NAK on required IN ep. U_FALSE: stop NAK}
\CommentTok{ * }\AnnotationTok{@param[in]}\CommentTok{ }\CommentVarTok{ep:}\CommentTok{ ep number with USB_EP_DIRECTION_IN/OUT. }
\CommentTok{ * }\AnnotationTok{@param[out]}\CommentTok{ }\CommentVarTok{null}\CommentTok{. }
\CommentTok{ */}
\KeywordTok{extern} \DataTypeTok{void}\NormalTok{ USB_SetInEndpointNak(}\DataTypeTok{uint8_t}\NormalTok{ ep, }\DataTypeTok{uint8_t}\NormalTok{ enable);}
\end{Highlighting}
\end{Shaded}

\hypertarget{example-0-winusb}{%
\subsection{example 0: WINUSB}\label{example-0-winusb}}

WinUSB 是适用于 USB 设备的通用驱动程序,随附在自 Windows Vista 起的所有 Windows 版本中。 对于某些通用串行总线 (USB) 设备(例如只有单个应用程序访问的设备),可以将 WinUSB (Winusb.sys) 安装在 设备的内核模式堆栈中作为 USB 设备的函数驱动程序而不是实现驱动程序。如果已将设备定义为 WinUSB 设备 ,Windows会自动加载Winusb.sys。

\begin{itemize}
\tightlist
\item
  参考:\texttt{\textbackslash{}ING\textbackslash{}\_SDK\textbackslash{}sdk\textbackslash{}src\textbackslash{}BSP\textbackslash{}bsp\_usb.c}
\end{itemize}

首先调用\texttt{bsp\_usb\_init()}初始化USB模块,之后的USB活动则全部在\texttt{bsp\_usb\_event\_handler()}中处理。

\begin{Shaded}
\begin{Highlighting}[]
\PreprocessorTok{#define FEATURE_WCID_SUPPORT}
\end{Highlighting}
\end{Shaded}

\begin{itemize}
\tightlist
\item
  device需要在enumeration阶段提供WCID标识和相关descriptor。示例中的descriptor实现如下:
\end{itemize}

\begin{Shaded}
\begin{Highlighting}[]
\PreprocessorTok{#define USB_WCID_DESCRIPTOR_INDEX_4 \textbackslash{}}
\PreprocessorTok{\{ \textbackslash{}}

\PreprocessorTok{#define USB_WCID_DESCRIPTOR_INDEX_5 \textbackslash{}}
\PreprocessorTok{\{ \textbackslash{}}
\end{Highlighting}
\end{Shaded}

\begin{itemize}
\tightlist
\item
  通过修改\texttt{USB\_STRING\_PRODUCT}来改变产品名称iproduct
\end{itemize}

\begin{Shaded}
\begin{Highlighting}[]
\PreprocessorTok{#define USB_STRING_PRODUCT \{16,0x3,'w',0,'i',0,'n',0,'-',0,'d',0,'e',0,'v',0\}}
\end{Highlighting}
\end{Shaded}

第一个值是整个数组的长度,第二个值不变,之后是16bit unicode字符串 (每个符号占用两个字节),该示例中iproduct为'win-dev'。

\begin{itemize}
\tightlist
\item
  该示例中打开了两个bulk endpoint,endpoint 1 为input,endpoint 2为output,最大包长为64:
\end{itemize}

\begin{Shaded}
\begin{Highlighting}[]
\PreprocessorTok{#define USB_EP_1_DESCRIPTOR \textbackslash{}}
\PreprocessorTok{\{ \textbackslash{}}
\PreprocessorTok{  .size = sizeof(USB_EP_DESCRIPTOR_REAL_T), \textbackslash{}}
\PreprocessorTok{  .type = 5, \textbackslash{}}
\PreprocessorTok{  .ep = USB_EP_DIRECTION_IN(EP_IN), \textbackslash{}}
\PreprocessorTok{  .attributes = USB_EP_TYPE_BULK, \textbackslash{}}
\PreprocessorTok{  .mps = EP_X_MPS_BYTES, \textbackslash{}}
\PreprocessorTok{  .interval = 0 \textbackslash{}}
\PreprocessorTok{\}}

\PreprocessorTok{#define USB_EP_2_DESCRIPTOR \textbackslash{}}
\PreprocessorTok{\{ \textbackslash{}}
\PreprocessorTok{  .size = sizeof(USB_EP_DESCRIPTOR_REAL_T), \textbackslash{}}
\PreprocessorTok{  .type = 5, \textbackslash{}}
\PreprocessorTok{  .ep = USB_EP_DIRECTION_OUT(EP_OUT), \textbackslash{}}
\PreprocessorTok{  .attributes = USB_EP_TYPE_BULK, \textbackslash{}}
\PreprocessorTok{  .mps = EP_X_MPS_BYTES, \textbackslash{}}
\PreprocessorTok{  .interval = 0 \textbackslash{}}
\PreprocessorTok{\}}
\end{Highlighting}
\end{Shaded}

\begin{itemize}
\item
  在set configuration(USB\_REQUEST\_DEVICE\_SET\_CONFIGURATION)之后,In/out endpoint 可以使用,通过\texttt{USB\_RecvData()}配置out endpoint接收host发送的数据。 在收到host的数据后(USB\_CALLBACK\_TYPE\_RECEIVE\_END),通过\texttt{USB\_SendData()}将数据 发送给host(通过in endpoint)。
\item
  在WIN10及以上的系统上,该设备会自动加载winusb.sys并枚举成WinUsb Device, 设备名称为``win-dev''。 \includegraphics{img/USB/usb_win_dev.PNG}
\item
  通过ing\_usb.exe可以对该设备进行一些简单数据测试(ing\_usb.exe VID:PID -w(write command) 2(transfer type(2==bulk)) 以及需要传输的数据(默认包长度为endpoint的mps),数据会通过out endpoint发送给USB Device并通过in endpoint回环并打印出来):\includegraphics{img/USB/usb_ing_usb_exe.PNG}
\item
  使用ing\_usb.exe可以读取in endpoint的数据((ing\_usb.exe VID:PID -r(read command) 2(transfer type(2==bulk))), 但是bsp layer中需要做相应的修改(使用in endpoint发送数据给host)。
\end{itemize}

\hypertarget{example-1-hid-composite}{%
\subsection{example 1: HID composite}\label{example-1-hid-composite}}

该示例实现了一个mouse + keyboard的复合设备, 使用了两个独立的interface,每个interface包含一个In Endpoint。

\begin{itemize}
\tightlist
\item
  参考:\texttt{\textbackslash{}ING\_SDK\textbackslash{}sdk\textbackslash{}src\textbackslash{}BSP\textbackslash{}bsp\_usb\_hid.c}
\end{itemize}

首先调用\texttt{bsp\_usb\_init()}初始化USB模块, 之后的USB活动则全部在\texttt{bsp\_usb\_event\_handler()}中处理,report的发送参考\texttt{report\ data\ 发送}。

\hypertarget{ux6807ux51c6ux63cfux8ff0ux7b26}{%
\subsubsection{标准描述符}\label{ux6807ux51c6ux63cfux8ff0ux7b26}}

其标准描述符结构如下,configuration descriptor之后分别是interface descriptor, hid descriptor, endpoint descriptor。

\begin{Shaded}
\begin{Highlighting}[]
\KeywordTok{typedef} \KeywordTok{struct}\NormalTok{ __attribute__((packed))}
\NormalTok{\{}
\NormalTok{  USB_CONFIG_DESCRIPTOR_REAL_T config;}
\NormalTok{  USB_INTERFACE_DESCRIPTOR_REAL_T interface_kb;}
\NormalTok{  BSP_USB_HID_DESCRIPTOR_T hid_kb;}
\NormalTok{  USB_EP_DESCRIPTOR_REAL_T ep_kb[bNUM_EP_KB];}
\NormalTok{  USB_INTERFACE_DESCRIPTOR_REAL_T interface_mo;}
\NormalTok{  BSP_USB_HID_DESCRIPTOR_T hid_mo;}
\NormalTok{  USB_EP_DESCRIPTOR_REAL_T ep_mo[bNUM_EP_MO];}
\NormalTok{\}BSP_USB_DESC_STRUCTURE_T;}
\end{Highlighting}
\end{Shaded}

上述描述符的示例在路径\texttt{\textbackslash{}ING\_SDK\textbackslash{}sdk\textbackslash{}src\textbackslash{}BSP\textbackslash{}bsp\_usb\_hid.h},以keyboard interface为例:

\begin{Shaded}
\begin{Highlighting}[]
\PreprocessorTok{#define USB_INTERFACE_DESCRIPTOR_KB \textbackslash{}}
\PreprocessorTok{\{ \textbackslash{}}
\PreprocessorTok{  .size = sizeof(USB_INTERFACE_DESCRIPTOR_REAL_T), \textbackslash{}}
\PreprocessorTok{  .type = 4, \textbackslash{}}
\PreprocessorTok{  .interfaceIndex = 0x00, \textbackslash{}}
\PreprocessorTok{  .alternateSetting = 0x00, \textbackslash{}}
\PreprocessorTok{  .nbEp = bNUM_EP_KB,  \textbackslash{}}
\PreprocessorTok{  .usbClass = 0x03, \textbackslash{}}
\PreprocessorTok{  /* 0: no subclass, 1: boot interface */ \textbackslash{}}
\PreprocessorTok{  .usbSubClass = 0x00, \textbackslash{}}
\PreprocessorTok{  /* 0: none, 1: keyboard, 2: mouse */ \textbackslash{}}
\PreprocessorTok{  .usbProto = 0x00, \textbackslash{}}
\PreprocessorTok{  .iDescription = 0x00 \textbackslash{}}
\PreprocessorTok{\}}
\end{Highlighting}
\end{Shaded}

其中可能需要根据场景修改的变量使用宏定义,其他则使用常量。请注意:该实现中没有打开boot function。

\hypertarget{ux62a5ux544aux63cfux8ff0ux7b26}{%
\subsubsection{报告描述符}\label{ux62a5ux544aux63cfux8ff0ux7b26}}

报告描述符的示例在路径\texttt{\textbackslash{}ING\_SDK\textbackslash{}sdk\textbackslash{}src\textbackslash{}BSP\textbackslash{}bsp\_usb\_hid.h}

\begin{itemize}
\tightlist
\item
  keyboard的report描述符为:
\end{itemize}

\begin{Shaded}
\begin{Highlighting}[]
\PreprocessorTok{#define USB_HID_KB_REPORT_DESCRIPTOR \{  \textbackslash{}}
\PreprocessorTok{    0x05, 0x01, /* USAGE_PAGE (Generic Desktop)                       */  \textbackslash{}}
\PreprocessorTok{    0x09, 0x06, /* USAGE (Keyboard)                                   */  \textbackslash{}}
\PreprocessorTok{    0xa1, 0x01, /* COLLECTION (Application)                           */  \textbackslash{}}
\PreprocessorTok{    0x05, 0x07, /*   USAGE_PAGE (Keyboard)                            */  \textbackslash{}}
\PreprocessorTok{    ...\textbackslash{}ING_SDK\textbackslash{}sdk\textbackslash{}src\textbackslash{}BSP\textbackslash{}bsp_usb_hid.h}
\end{Highlighting}
\end{Shaded}

keyboard report descriptor包含8bit modifier input,8bit reserve, 5bit led output, 3bit reserve, 6bytes key usage id 。report本地结构为:

\begin{Shaded}
\begin{Highlighting}[]
\PreprocessorTok{#define KEY_TABLE_LEN (6)}
\KeywordTok{typedef} \KeywordTok{struct}\NormalTok{ __attribute__((packed))}
\NormalTok{\{}
  \DataTypeTok{uint8_t}\NormalTok{ modifier;}
  \DataTypeTok{uint8_t}\NormalTok{ reserved;}
  \DataTypeTok{uint8_t}\NormalTok{ key_table[KEY_TABLE_LEN];}
\NormalTok{\}BSP_KEYB_REPORT_s;}
\end{Highlighting}
\end{Shaded}

report中的usage id data的实现分别在以下enum中:

\begin{Shaded}
\begin{Highlighting}[]
\NormalTok{BSP_KEYB_KEYB_USAGE_ID_e}
\NormalTok{BSP_KEYB_KEYB_MODIFIER_e}
\NormalTok{BSP_KEYB_KEYB_LED_e}
\end{Highlighting}
\end{Shaded}

\begin{itemize}
\tightlist
\item
  mouse的report描述符为:
\end{itemize}

\begin{Shaded}
\begin{Highlighting}[]
\PreprocessorTok{#define USB_HID_MOUSE_REPORT_DESCRIPTOR_SIZE (50)}
\PreprocessorTok{#define USB_HID_MOUSE_REPORT_DESCRIPTOR \{ \textbackslash{}}
\PreprocessorTok{    0x05, 0x01, /* USAGE_PAGE (Generic Desktop)       */   \textbackslash{}}
\PreprocessorTok{    0x09, 0x02, /* USAGE (Mouse)                      */   \textbackslash{}}
\PreprocessorTok{    0xa1, 0x01, /* COLLECTION (Application)           */   \textbackslash{}}
\PreprocessorTok{    ...\textbackslash{}ING_SDK\textbackslash{}sdk\textbackslash{}src\textbackslash{}BSP\textbackslash{}bsp_usb_hid.h}
\end{Highlighting}
\end{Shaded}

report包含一个3bit button(button 1 \textasciitilde{} button 3), 5bit reserve, 8bit x value, 8bit y value 结构为:

\begin{Shaded}
\begin{Highlighting}[]
\KeywordTok{typedef} \KeywordTok{struct}\NormalTok{ __attribute__((packed))}
\NormalTok{\{}
  \DataTypeTok{uint8_t}\NormalTok{ button;}\CommentTok{/* 1 ~ 3 */}
  \DataTypeTok{int8_t}\NormalTok{ pos_x;}\CommentTok{/* -127 ~ 127 */}
  \DataTypeTok{int8_t}\NormalTok{ pos_y;}\CommentTok{/* -127 ~ 127 */}
\NormalTok{\}BSP_MOUSE_REPORT_s;}
\end{Highlighting}
\end{Shaded}

\hypertarget{standardclass-request}{%
\subsubsection{standard/class request}\label{standardclass-request}}

ep0的request处理在event:USB\_EVENT\_EP0\_SETUP。 HID Class相关的处理在interface destination下:USB\_REQUEST\_DESTINATION\_INTERFACE。

以其中keyboard report 描述符的获取为例:

\begin{itemize}
\tightlist
\item
  setup-\textgreater wIndex代表了interface num,其中0为keyboard interface(参考BSP\_USB\_DESC\_STRUCTURE\_T)
\item
  使用USB\_SendData发送report数据
\end{itemize}

\begin{Shaded}
\begin{Highlighting}[]
\ControlFlowTok{case}\NormalTok{ USB_REQUEST_DEVICE_GET_DESCRIPTOR:}
\NormalTok{\{}
  \ControlFlowTok{switch}\NormalTok{(((setup->wValue)>>}\DecValTok{8}\NormalTok{)&}\BaseNTok{0xFF}\NormalTok{)}
\NormalTok{  \{}
    \ControlFlowTok{case}\NormalTok{ USB_REQUEST_HID_CLASS_DESCRIPTOR_REPORT:}
\NormalTok{    \{}
      \ControlFlowTok{switch}\NormalTok{(setup->wIndex)}
\NormalTok{      \{}
        \ControlFlowTok{case} \DecValTok{0}\NormalTok{:}
\NormalTok{        \{}
\NormalTok{          size = }\KeywordTok{sizeof}\NormalTok{(ReportKeybDescriptor);}
\NormalTok{          size = (setup->wLength < size) ? (setup->wLength) : size;}

\NormalTok{          status |= USB_SendData(}\DecValTok{0}\NormalTok{, (}\DataTypeTok{void}\NormalTok{*)&ReportKeybDescriptor, size, }\DecValTok{0}\NormalTok{);}
\NormalTok{          KeybReport.pending = U_FALSE;}
\NormalTok{        \}}\ControlFlowTok{break}\NormalTok{;}
\NormalTok{        ...\textbackslash{}ING_SDK\textbackslash{}sdk\textbackslash{}src\textbackslash{}BSP\textbackslash{}bsp_usb_hid.c}
\end{Highlighting}
\end{Shaded}

\hypertarget{report-data-ux53d1ux9001}{%
\subsubsection{report data 发送}\label{report-data-ux53d1ux9001}}

\begin{itemize}
\tightlist
\item
  keyboard key的发送,入参为一个键值以及其是否处于按下的状态,如果该键按下,则将其添加到report中并发送出去。
\end{itemize}

\begin{Shaded}
\begin{Highlighting}[]
\CommentTok{/**}
\CommentTok{ * }\AnnotationTok{@brief}\CommentTok{ interface API. send keyboard key report}
\CommentTok{ *}
\CommentTok{ * }\AnnotationTok{@param[in]}\CommentTok{ }\CommentVarTok{key:}\CommentTok{ value comes from BSP_KEYB_KEYB_USAGE_ID_e}
\CommentTok{ * }\AnnotationTok{@param[in]}\CommentTok{ }\CommentVarTok{press:}\CommentTok{ 1: pressed, 0: released}
\CommentTok{ * }\AnnotationTok{@param[out]}\CommentTok{ }\CommentVarTok{null}\CommentTok{. }
\CommentTok{ */}
\KeywordTok{extern} \DataTypeTok{void}\NormalTok{ bsp_usb_handle_hid_keyb_key_report(}\DataTypeTok{uint8_t}\NormalTok{ key, }\DataTypeTok{uint8_t}\NormalTok{ press);}
\end{Highlighting}
\end{Shaded}

\begin{itemize}
\tightlist
\item
  keyboard modifier的发送,与key类似,区别是modifier是bitmap data。
\end{itemize}

\begin{Shaded}
\begin{Highlighting}[]
\KeywordTok{extern} \DataTypeTok{void}\NormalTok{ bsp_usb_handle_hid_keyb_key_report(}\DataTypeTok{uint8_t}\NormalTok{ key, }\DataTypeTok{uint8_t}\NormalTok{ press);}
\CommentTok{/**}
\CommentTok{ * }\AnnotationTok{@brief}\CommentTok{ interface API. send keyboard modifier report}
\CommentTok{ *}
\CommentTok{ * }\AnnotationTok{@param[in]}\CommentTok{ }\CommentVarTok{modifier:}\CommentTok{ value comes from BSP_KEYB_KEYB_MODIFIER_e}
\CommentTok{ * }\AnnotationTok{@param[in]}\CommentTok{ }\CommentVarTok{press:}\CommentTok{ 1: pressed, 0: released}
\CommentTok{ * }\AnnotationTok{@param[out]}\CommentTok{ }\CommentVarTok{null}\CommentTok{. }
\CommentTok{ */}
\KeywordTok{extern} \DataTypeTok{void}\NormalTok{ bsp_usb_handle_hid_keyb_modifier_report(BSP_KEYB_KEYB_MODIFIER_e modifier, }\DataTypeTok{uint8_t}\NormalTok{ press);}
\end{Highlighting}
\end{Shaded}

\begin{itemize}
\item
  keyboard led的获取,该示例中没有out endpoint,因此led report可能是用ep0的set report得到, 参考:USB\_REQUEST\_HID\_CLASS\_REQUEST\_SET\_REPORT。
\item
  mouse report的发送,入参分别为x,y的相对值和button的组合(按下为1,释放为0)。
\end{itemize}

\begin{Shaded}
\begin{Highlighting}[]
\CommentTok{/**}
\CommentTok{ * }\AnnotationTok{@brief}\CommentTok{ interface API. send mouse report}
\CommentTok{ *}
\CommentTok{ * }\AnnotationTok{@param[in]}\CommentTok{ }\CommentVarTok{x:}\CommentTok{ 8bit int x axis value, relative,}
\CommentTok{ * }\AnnotationTok{@param[in]}\CommentTok{ }\CommentVarTok{y:}\CommentTok{ 8bit int y axis value, relative,}
\CommentTok{ * }\AnnotationTok{@param[in]}\CommentTok{ }\CommentVarTok{btn:}\CommentTok{ 8bit value, button 1 to 3,}
\CommentTok{ * }\AnnotationTok{@param[out]}\CommentTok{ }\CommentVarTok{null}\CommentTok{. }
\CommentTok{ */}
\KeywordTok{extern} \DataTypeTok{void}\NormalTok{ bsp_usb_handle_hid_mouse_report(}\DataTypeTok{int8_t}\NormalTok{ x, }\DataTypeTok{int8_t}\NormalTok{ y, }\DataTypeTok{uint8_t}\NormalTok{ btn);}
\end{Highlighting}
\end{Shaded}

\hypertarget{ch-watchdog}{%
\chapter{看门狗(WATCHDOG)}\label{ch-watchdog}}

\hypertarget{ux529fux80fdux6982ux8ff0-8}{%
\section{功能概述}\label{ux529fux80fdux6982ux8ff0-8}}

看门狗就是定期的查看芯片内部情况,一旦发生错误就向芯片内部发出重启信号。
看门狗指令在程序的中断中拥有最高的优先级。防止程序跑飞,同时也能防止程序在线运行时出现死循环

\hypertarget{ux4f7fux7528ux8bf4ux660e-9}{%
\section{使用说明}\label{ux4f7fux7528ux8bf4ux660e-9}}

\hypertarget{ux914dux7f6eux770bux95e8ux72d7}{%
\subsection{配置看门狗}\label{ux914dux7f6eux770bux95e8ux72d7}}

使用 \texttt{TMR\_WatchDogEnable3} 配置并启用看门狗。

\begin{Shaded}
\begin{Highlighting}[]
\DataTypeTok{void}\NormalTok{ TMR_WatchDogEnable3(}
  \DataTypeTok{uint32_t}\NormalTok{ int_timeout_ms, }
  \DataTypeTok{uint32_t}\NormalTok{ reset_timeout_ms, }
  \DataTypeTok{uint8_t}\NormalTok{ enable_int}
\NormalTok{  );}
\end{Highlighting}
\end{Shaded}

为了ING916xx和ING918xx统一接口,定义宏 \texttt{TMR\_WatchDogEnable} 代替 \texttt{TMR\_WatchDogEnable3} 。

\begin{Shaded}
\begin{Highlighting}[]
\PreprocessorTok{#define TMR_WatchDogEnable(timeout) do \{ }
  \DataTypeTok{uint32_t}\NormalTok{ TMR_CLK_FREQ = OSC_CLK_FREQ; }
  \DataTypeTok{uint32_t}\NormalTok{ cnt = ((}\DataTypeTok{uint64_t}\NormalTok{)(timeout) * }\DecValTok{1000}\NormalTok{ / OSC_CLK_FREQ); }
\NormalTok{  TMR_WatchDogEnable3(cnt, cnt, }\DecValTok{0}\NormalTok{); }
\NormalTok{  \} }\ControlFlowTok{while}\NormalTok{ (}\DecValTok{0}\NormalTok{)}
\end{Highlighting}
\end{Shaded}

例如:

\hypertarget{ux91cdux542fux770bux95e8ux72d7}{%
\subsection{重启看门狗}\label{ux91cdux542fux770bux95e8ux72d7}}

使用 \texttt{TMR\_WatchDogRestart} 重启看门狗。

\begin{Shaded}
\begin{Highlighting}[]
\DataTypeTok{void}\NormalTok{ TMR_WatchDogRestart(}\DataTypeTok{void}\NormalTok{);}
\end{Highlighting}
\end{Shaded}

\hypertarget{ux6e05ux9664ux4e2dux65ad-1}{%
\subsection{清除中断}\label{ux6e05ux9664ux4e2dux65ad-1}}

使用 \texttt{TMR\_WatchDogClearInt} 清除看门狗的中断。

\begin{Shaded}
\begin{Highlighting}[]
\DataTypeTok{void}\NormalTok{ TMR_WatchDogClearInt(}\DataTypeTok{void}\NormalTok{);}
\end{Highlighting}
\end{Shaded}

\hypertarget{ux7981ux7528ux770bux95e8ux72d7}{%
\subsection{禁用看门狗}\label{ux7981ux7528ux770bux95e8ux72d7}}

在使用\texttt{TMR\_WatchDogEnable} 启用或 \texttt{TMR\_WatchDogRestart} 重启看门狗之后,
需要使用 \texttt{TMR\_WatchDogDisable} 对其禁用。

\begin{Shaded}
\begin{Highlighting}[]
\DataTypeTok{void}\NormalTok{ TMR_WatchDogDisable(}\DataTypeTok{void}\NormalTok{);}
\end{Highlighting}
\end{Shaded}

\hypertarget{ch-eflash}{%
\chapter{内置 Flash(EFlash)}\label{ch-eflash}}

\hypertarget{ux529fux80fdux6982ux8ff0-9}{%
\section{功能概述}\label{ux529fux80fdux6982ux8ff0-9}}

芯片内置一定容量的 Flash,可编程擦写。擦除时以扇区(sector)为单位进行,
写入时以 32bit 为单位。

\hypertarget{ux4f7fux7528ux8bf4ux660e-10}{%
\section{使用说明}\label{ux4f7fux7528ux8bf4ux660e-10}}

\hypertarget{ux64e6ux9664ux5e76ux5199ux5165ux65b0ux6570ux636e}{%
\subsection{擦除并写入新数据}\label{ux64e6ux9664ux5e76ux5199ux5165ux65b0ux6570ux636e}}

通过 \texttt{program\_flash} 擦除并写入一段数据。

\begin{Shaded}
\begin{Highlighting}[]
\DataTypeTok{int}\NormalTok{ program_flash(}
    \CommentTok{// 待写入的地址}
    \DataTypeTok{const} \DataTypeTok{uint32_t}\NormalTok{ dest_addr,}
    \CommentTok{// 数据源的地址}
    \DataTypeTok{const} \DataTypeTok{uint8_t}\NormalTok{ *buffer,}
    \CommentTok{// 数据长度(以字节为单位,必须是 4 的倍数)}
    \DataTypeTok{uint32_t}\NormalTok{ size);}
\end{Highlighting}
\end{Shaded}

\texttt{dest\_addr} 为统一编址后的地址,而非 Flash 内部从 0 开始的地址。\texttt{dest\_addr}
必须对应于某个扇区的起始地址。数据源不可位于 Flash 内。

\texttt{program\_flash} 将根据 \texttt{size} 自动擦除一个或多个扇区并写入数据。

本函数如果成功,则返回 0,否则返回非 0。

\hypertarget{ux4e0dux64e6ux9664ux76f4ux63a5ux5199ux5165ux6570ux636e}{%
\subsection{不擦除直接写入数据}\label{ux4e0dux64e6ux9664ux76f4ux63a5ux5199ux5165ux6570ux636e}}

通过 \texttt{write\_flash} 不擦除直接写入数据。

\begin{Shaded}
\begin{Highlighting}[]
\DataTypeTok{int}\NormalTok{ write_flash(}
    \CommentTok{// 待写入的地址}
    \DataTypeTok{const} \DataTypeTok{uint32_t}\NormalTok{ dest_addr,}
    \CommentTok{// 数据源的地址}
    \DataTypeTok{const} \DataTypeTok{uint8_t}\NormalTok{ *buffer,}
    \CommentTok{// 数据长度(以字节为单位,必须是 4 的倍数)}
    \DataTypeTok{uint32_t}\NormalTok{ size);}
\end{Highlighting}
\end{Shaded}

\texttt{dest\_addr} 为统一编址后的地址,必须 32bit 对齐。\texttt{write\_data} 不擦除 Flash,而是直接写入。
数据源不可位于 Flash 内。如果对应的 Flash 空间未被擦除,将无法写入。

本函数如果成功,则返回 0,否则返回非 0。

\hypertarget{ux5355ux72ecux64e6ux9664}{%
\subsection{单独擦除}\label{ux5355ux72ecux64e6ux9664}}

通过 \texttt{erase\_flash\_sector} 擦除一个指定的扇区。

\begin{Shaded}
\begin{Highlighting}[]
\DataTypeTok{int}\NormalTok{ erase_flash_sector(}
    \CommentTok{// 待擦除的地址}
    \DataTypeTok{const} \DataTypeTok{uint32_t}\NormalTok{ addr);}
\end{Highlighting}
\end{Shaded}

\texttt{addr} 必须对应于某个扇区的起始地址。本函数如果成功,则返回 0,否则返回非 0。

\hypertarget{flash-ux6570ux636eux5347ux7ea7}{%
\subsection{Flash 数据升级}\label{flash-ux6570ux636eux5347ux7ea7}}

通过 \texttt{flash\_do\_update} 可以升级 Flash 里的数据。这个函数可用于 FOTA 升级。

\begin{Shaded}
\begin{Highlighting}[]
\DataTypeTok{int}\NormalTok{ flash_do_update(}
    \CommentTok{// 数据块数目}
    \DataTypeTok{const} \DataTypeTok{int}\NormalTok{ block_num,}
    \CommentTok{// 每个数据块的信息}
    \DataTypeTok{const}\NormalTok{ fota_update_block_t *blocks,}
    \CommentTok{// 用于缓存一个扇区的内存}
    \DataTypeTok{uint8_t}\NormalTok{ *ram_buffer);}
\end{Highlighting}
\end{Shaded}

每个数据块的定为为:

\begin{Shaded}
\begin{Highlighting}[]
\KeywordTok{typedef} \KeywordTok{struct}\NormalTok{ fota_update_block}
\NormalTok{\{}
    \DataTypeTok{uint32_t}\NormalTok{ src;}
    \DataTypeTok{uint32_t}\NormalTok{ dest;}
    \DataTypeTok{uint32_t}\NormalTok{ size;}
\NormalTok{\} fota_update_block_t;}
\end{Highlighting}
\end{Shaded}

这个函数的行为大致如下:

\begin{Shaded}
\begin{Highlighting}[]
\NormalTok{flash_do_update() \{}
    \ControlFlowTok{for}\NormalTok{ (block in blocks) \{}
\NormalTok{        flash_copy(block.dest,}
\NormalTok{                   block.src,}
\NormalTok{                   block.size);}
\NormalTok{    \}}
\NormalTok{\}}
\end{Highlighting}
\end{Shaded}

如前所述,\texttt{program\_flash} 的数据源不能位于 Flash,所以 \texttt{flash\_copy} 需要把各扇区逐个读入
\texttt{ram\_buffer},然后使用 \texttt{program\_flash} 擦除、写入。

这个函数如果成功,将自动重启系统,否则返回非 0。

  \bibliography{book.bib,packages.bib}

\backmatter
\printindex

\end{document}
